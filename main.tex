 \documentclass[11pt,a4paper,twoside]{tesis}
% SI NO PENSAS IMPRIMIRLO EN FORMATO LIBRO PODES USAR
%\documentclass[11pt,a4paper]{tesis}

\usepackage{graphicx}
\usepackage[utf8]{inputenc}
\usepackage[spanish]{babel}
\usepackage[left=3cm,right=3cm,bottom=3.5cm,top=3.5cm]{geometry}
\usepackage[square,sort,comma,numbers]{natbib}
\usepackage{hyperref}
\usepackage{booktabs}
\usepackage{multirow}
\usepackage{amsmath}
\usepackage{tabularx}
\usepackage{todonotes}

\setlength{\marginparwidth}{2cm}

\bibliographystyle{plainnat}

\begin{document}

%%%% CARATULA

\def\autor{Juan Manuel Pérez}
\def\tituloTesis{Técnicas y recursos para la detección automática de lenguaje discriminatorio en redes sociales}
\def\runtitulo{Técnicas y recursos para la detección automática de lenguaje discriminatorio en redes sociales}
\def\runtitle{Techniques and resources for the automatic detection of hate speech in social networks}
\def\director{Franco Luque}
\def\codirector{Agustín Gravano}
\def\lugar{Buenos Aires, 2021}
\newcommand{\HRule}{\rule{\linewidth}{0.2mm}}
%
\thispagestyle{empty}

\begin{center}\leavevmode

\vspace{-2cm}

\begin{tabular}{l}
\includegraphics[width=2.6cm]{img/logofcen.png}
\end{tabular}


{\large \sc Universidad de Buenos Aires

Facultad de Ciencias Exactas y Naturales

Departamento de Computaci\'on}

\vspace{6.0cm}

%\vspace{3.0cm}
%{
%\Large \color{red}
%\begin{tabular}{|p{2cm}cp{2cm}|}
%\hline
%& Pre-Final Version: \today &\\
%\hline
%\end{tabular}
%}
%\vspace{2.5cm}

\begin{huge}
\textbf{\tituloTesis}
\end{huge}

\vspace{2cm}

{\large Tesis presentada para optar al título de Doctor de la Universidad de Buenos Aires en
el área de Ciencias de la Computaci\'on}

\vspace{2cm}

{\Large \autor}

\end{center}

\vfill

{\large

\begin{tabular}{l l}
    \vspace{.3cm} Director:               & Dr. Franco Luque \\
    \vspace{.3cm} Director Asistente:     & Dr. Agustín Gravano \\
    \vspace{.3cm} Consejero de Estudios:  & Dr. Diego Fernández Slezak \\
    Lugar de Trabajo:          & Departamento de Computación \\
                               & Facultad de Cs. Exactas y Naturales \\
    \vspace{1cm}               & Universidad de Buenos Aires \\
    Buenos Aires, 2021         & \\
\end{tabular}

\vspace{.2cm}



\vspace{.2cm}


}

\newpage\thispagestyle{empty}


%%%% ABSTRACTS, AGRADECIMIENTOS Y DEDICATORIA
\frontmatter
\pagestyle{empty}
%\begin{center}
%\large \bf \runtitulo
%\end{center}
%\vspace{1cm}
\chapter*{\runtitulo}

\noindent El discurso discriminatorio (también conocido como discurso de odio) puede describirse como aquel discurso en clave de intenso aborrecimiento, denigración y enemistad que ataca a un individuo o un grupo de individuos por poseer –o aparentar poseer– cierta característica protegida por tratados internacionales como el sexo, el género, la etnia, etc. En los últimos años, este tipo de discurso ha tomado gran relevancia en redes sociales y otros medios virtuales debido a su intensidad y a su relación con actos violentos contra miembros de estos grupos. A raíz de esto, estados y organizaciones supranacionales como la Unión Europea han sancionado legislación que insta a las empresas de redes sociales a moderar y eliminar contenido discriminatorio, con particular foco en aquel que insta a la violencia física.

Debido a la enorme cantidad de contenido generado por usuarios en las redes sociales, es necesario contar con cierta automatización en esta tarea, bien para su análisis o para su moderación. Desde la óptica del procesamiento de lenguaje natural, la detección de discriminación puede entenderse como un problema de clasificación de texto: dado un texto generado por un usuario, predecir si es o no contenido discriminatorio. Así mismo, puede ser de interés predecir otras características: por ejemplo, si el texto contiene un llamado a la acción violenta, si está dirigido contra un individuo o un grupo, o el tipo de característica ofendida, entre otras.

Una de las limitaciones de los enfoques actuales para la detección del lenguaje discriminatorio es la falta de contexto en el mensaje. La mayoría de los estudios y recursos están hechos sobre datos fuera de contexto; es decir, mensajes aislados sin ningún tipo de contexto conversacional o del tema del cual se habla. Esto restringe la información disponible –tanto para un humano como para un sistema– para poder discernir si un texto social es discriminatorio. Otra información usualmente faltante es la característica atacada: es común que los datasets estén anotados de manera poco granular, no brindando información acerca de si la agresión es por motivos de sexo, género, clase social, etc. Por último, una limitación puntual del español es la poca disponibilidad de recursos para esta tarea.

En esta tesis pretendemos abordar algunas de las limitaciones marcadas. Por un lado, analizamos el impacto de agregar contexto a la detección de lenguaje discriminatorio en redes sociales. Para ello, construimos un conjunt de datos de tweets en base a las respuestas de los usuarios a los posteos de medios periodísticos en Twitter. Esto nos permite obtener dos tipos de contextos: uno “conversacional” al tener una respuesta a un tweet anterior, y otro más extenso al obtener el texto de la noticia en cuestión. El corpus fue recolectado sobre noticias relacionadas a la pandemia de COVID-19, en idioma español mayormente en su variedad dialectal rioplatense y anotado por hablantes nativos de ese dialecto con un nuevo modelo de etiquetado, que es granular respecto de las características ofendidas.

Sobre los comentarios de este dataset realizamos experimentos de detección de discurso de odio planteando dos tareas: detección “plana” del lenguaje discriminatorio, donde sólo predecimos una etiqueta binaria indicando presencia de lenguaje discriminatorio; y detección “granular”, donde predecimos las características ofendidas. Usando técnicas del estado del arte, obtuvimos mejoras significativas en ambas tareas al agregar contexto como entrada de cada instancia, tanto en su forma corta (sólo el titular/tweet de la noticia) como en su forma larga (titular y cuerpo de la noticia). Así mismo, observamos que un clasificador entrenado para la tarea “granular” mejora levemente su performance al ser evaluado para la tarea “plana”, obviando los posibles errores de motivos discriminatorios. Combinando la adición de contexto y granularidad, un clasificador para la detección de lenguaje discriminatorio obtiene mejoras considerables sobre un BERT en español que sólo consume el texto del comentario.

Considerando la detección de discurso de odio dentro del área más abarcativa de clasificación de documentos en dominios sociales, analizamos también algunos aspectos generales de tareas relacionadas como el análisis de sentimiento y la detección de emociones, entre otras. En particular, analizamos el desempeño de varias técnicas modernas de representación al ser entrenadas en dominios sociales. Comúnmente, los modelos de representación son entrenados a partir de textos de dominios “formales”, como pueden ser Wikipedia u otras fuentes similares. En esta tesis observamos que –desde los word embeddings hasta los modelos pre-entrenados basados en transformers– las representaciones generadas son robustas y mejoran la performance en un conjunto de tareas de clasificación en textos sociales. Sobre los modelos pre-entrenados, estudiamos el impacto de entrenarlos desde cero en textos sociales o efectuar una adaptación a este dominio.

Todos los estudios y recursos presentados en esta tesis fueron realizados en el idioma español. Como un objetivo secundario, pretendemos contribuir a mitigar la enorme asimetría de recursos existente en el área del procesamiento del lenguaje natural.





\bigskip

\noindent\textbf{Palabras claves:} Guerra, Rebelión, Wookie, Jedi, Fuerza, Imperio (no menos de 5).

\cleardoublepage
%\begin{center}
%\large \bf \runtitle
%\end{center}
%\vspace{1cm}
\chapter*{\runtitle}

\noindent




{
    \setstretch{1.5}
Hate speech can be described as speech containing intense hatred, denigration, and enmity that attacks an individual or a group of individuals because of possessing –or pretending to possess– any characteristic protected by international treaties such as gender, ethnicity, religion, language, among others. In recent years, this type of discourse has gained great relevance in social networks and other virtual media due to its intensity and its relationship with violent acts against members of these groups. As a result, states and supranational organizations –such as the European Union– have enacted legislation that urges social media companies to moderate and remove discriminatory content, with particular focus on that which promotes physical violence.

Due to the enormous amount of user-generated content on social media, it is necessary to have some degree of automation in this task, either for analysis or for moderation. From a natural language processing (NLP) perspective, hate speech detection can be understood as a text classification problem: given a text generated by a user, predict whether it is discriminatory content. Likewise, it may be of interest to predict other features: for example, if the text contains a call to violent action; if it is directed against an individual or a group; or the offended characteristic, among others.

One of the limitations of current approaches to hate speech detection is the lack of context. Most studies and resources are performed on data without context; that is, isolated messages without any type of conversational context or the topic being discussed. This restricts the information available –both for a human and for an automated system– to discern if a social text is hateful or not. Other information usually lacking is the offended characteristic: datasets are usually annotated with a low level of granularity, failing to provide information about whether the offending message attacks the individual or group due to their gender, social class, race, or whatsoever. Finally, a specific limitation of Spanish is the limited availability of resources for this task.

In this thesis, we intend to address some of the marked limitations. On the one hand, we analyze the impact of adding context to hate speech detection in social networks. To do this, we built a tweet dataset based on user responses to news media posts on Twitter. This provided us two types of contexts: a conversational context, given by the tweet and its answer, and another context given by the text of the news in question. This dataset was collected on news related to the COVID-19 pandemic, in the Spanish language in its Rioplatense dialectal variety. Native speakers of this dialect annotated the comments with a novel labeling model that is granular regarding the offended characteristics.

Using this dataset, we carried out hate speech detection experiments, proposing two tasks: ``binary'' detection of discriminatory language, where we only predict a binary label indicating the presence of discriminatory language; and ``granular'' detection, where we predict the attacked characteristics (n-binary classification tasks at the same time). Using state-of-the-art techniques, we obtained significant improvements in both tasks by adding context as input for each instance, both in its short form (only the headline/tweet of the news article) and in its long-form (headline and body of the news article). We also observed that a classifier trained for the ``granular'' task slightly improves its performance when being evaluated for the ``flat'' task, ignoring possible errors of discriminatory motives. Combining the addition of context and granularity, a classifier for the detection of discriminatory language obtained considerable improvements over a BERT in Spanish that only consumes the text of the comment.

Considering hate speech detection within the most comprehensive area of ​​document classification in social domains, we further explored  some general aspects of related tasks such as sentiment analysis and emotion detection, among others. In particular, we analyzed the performance of various modern representation techniques when trained in social domains. Commonly, NLP researchers train representation models on texts from ``formal'' domains, such as Wikipedia or other similar sources. We observed that –from word embeddings to pre-trained models based on transformers– the representations generated are robust and improve performance in a set of classification tasks in social texts. On the pre-trained models, we studied the impact of training them from scratch in social texts versus performing domain-adaptation on the language models.

All of the studies and resources presented in this thesis were carried out in the Spanish language. As a secondary objective, we aim to mitigate the enormous asymmetry of resources in the area of NLP.
}
\bigskip

\noindent\textbf{Keywords:} Hate Speech, Natural Language Processing, Abusive Language Detection, Domain Adaptation, Social NLP.

\cleardoublepage
\chapter*{Agradecimientos}


En primer lugar, quiero agradecer a mis directores Franco y Agustín, que guiaron mi trabajo y me formaron como investigador. Realizar un doctorado es una labor bastante dura, en el cual uno se encuentra muchas veces perdido en el camino. La función que ambos cumplieron guiándome en esos momentos de desorientación ---pero dándome la libertad de elección en cada momento--- ha sido fundamental para llegar hasta acá.

Quiero agradecer a todes mis compañeres del Laboratorio de Inteligencia Artificial Aplicada (LIAA) y del Departamento de Computación de Exactas UBA quienes me ayudaron a transitar este doctorado, compartiendo conocimiento, charlas --- a veces simplemente catarsis. Edgar Altszyler, Pablo Brusco, Ramiro Gálvez, Bruno Bianchi, Damián Furman, Lara Gauder, Jazmín Vidal, y todos los que me falten en esta lista. A Viviana Cotik, que me ayudó de gran manera en los momentos más críticos de este trabajo.

A todes les integrantes del Proyecto Interdisciplinario de la UBA sobre marginaciones sociales (PIUBAMAS), que fueron fundamentales en los segmentos más importantes de esta tesis.

A mis compañeres de activismo y militancia, particularmente a los compañeros de la Asociación Gremial de Docentes de la UBA (AGD-UBA) y Jóvenes Científicxs Precarizados (JCP). Luchar por nuestros derechos y reconocimiento como trabajadores ha sido sin dudas parte de mi formación.

A mis amigos que me vieron poco estos años. A Víctor, Pablo, Tamara, Silvina, Andrés, Nico, Chudi, Tomás, Pigre, Joe. A Mariela Rajngewerc, con quien atravesamos paralelamente las dificultades de la academia. A Nina Pardal, con quien compartimos caminatas y charlas en Exactas.

A mi familia. A mi hermano Fer, a Graciela, a Julio. A mis primos Nico, Héctor y Meli. A mis viejos, dondequiera que estén, por impulsar mi curiosidad desde pequeño y siempre apoyarme en el estudio. Cada uno, a su manera, me fue llevando por este camino.

Finalmente quiero agradecer a Valeria, mi compañera de vida, que me apoyó en todo momento y soportó el estado de desborde emocional permanente que atraviesa todo doctorando. Realmente hubiera sido imposible sin vos.

 % OPCIONAL: comentar si no se quiere

\cleardoublepage
\hfill

\textit{A Valeria}

\textit{A mis viejos}

\textit{A quienes luchan cada día por hacer este mundo más justo}
  % OPCIONAL: comentar si no se quiere

\cleardoublepage
\tableofcontents

\mainmatter
\pagestyle{headings}

%%%% ACA VA EL CONTENIDO DE LA TESIS


\chapter{Intro}

\section{Discursos de odio}


El discurso discriminatorio  puede describirse como aquel discurso en clave de intenso aborrecimiento, denigración y enemistad que ataca a un individuo o un grupo de individuos por poseer --o aparentar poseer-- cierta característica protegida por tratados internacionales como el sexo, el género, la etnia, etc. Si bien no hay un consenso generalizado sobre qué configura exactamente lenguaje de odio o discriminación\cite{article192015}, un posible punto de contacto entre las distintas definiciones apuntan hacia


En los últimos años, este tipo de discurso ha tomado gran relevancia en redes sociales y otros medios virtuales debido a su intensidad y a su relación con actos violentos contra miembros de estos grupos.

A raíz de esto, estados y organizaciones supranacionales como la Unión Europea han sancionado legislación que insta a las empresas de redes sociales a moderar y eliminar contenido discriminatorio, con particular foco de aquel que insta a la violencia física.

Debido a la enorme cantidad de contenido generado por usuarios en las redes sociales, es necesario contar con cierta automatización en esta tarea bien para su análisis o para su moderación. Desde el procesamiento de lenguaje natural, la detección de discriminación puede entenderse como una clasificación de texto: dado un texto generado por un usuario, predecir si es o no contenido discriminatorio. Así mismo, puede ser de interés predecir otras características: por ejemplo, si el texto contiene un llamado a la acción violenta, si está dirigido contra un individuo o un grupo, el tipo de característica ofendida, entre otras.

Una de las limitaciones de los enfoques actuales para la detección del lenguaje discriminatorio es la falta de contexto en el mensaje. La mayoría de los estudios y recursos están hechos sobre datos fuera de contexto; es decir, mensajes aislados sin ningún tipo de contexto conversacional o del tema del cual se habla. Esto restringe la información disponible –tanto para un humano como para un sistema– para poder discernir si un texto social es discriminatorio. Otra información usualmente faltante es la característica atacada: es común que los datasets estén anotados de manera poco granular, no brindando información acerca de si la agresión es por motivos de sexo, género, clase social, etc. Por último, una limitación puntual del español es la poca disponibilidad de recursos para esta tarea. Más aún, los datasets suelen estar anotados por anotadores que no son hablantes de las variedades dialectales de los textos utilizados, lo cual genera un déficit en su calidad al ser el lenguaje discriminatorio altamente dependiente de la jerga específica de cada región.

En esta tesis pretendemos abordar algunas de las limitaciones marcadas. Por un lado, analizamos el impacto de agregar contexto a la detección de lenguaje discriminatorio en redes sociales. Para ello, construimos un dataset de tweets en base a las respuestas de los usuarios a los posteos de medios periodísticos en Twitter. Esto nos permite obtener dos tipos de contextos: uno “conversacional” al tener una respuesta a un tweet anterior, y uno más extenso al obtener el texto de la noticia en cuestión. El corpus fue recolectado sobre noticias relacionadas a la pandemia del COVID-19, en idioma español mayormente en su variedad dialectal rioplatense y anotado por hablantes nativos de ese dialecto con un modelo de etiquetado granular respecto a las características ofendidas.

Sobre los comentarios de este dataset realizamos experimentos de detección de discurso de odio planteando dos tareas: detección “plana” del lenguaje discriminatorio, donde sólo predecimos una etiqueta binaria indicando presencia de lenguaje discriminatorio; y detección “granular”, donde predecimos las características ofendidas. Usando técnicas del estado del arte, obtuvimos mejoras significativas en ambas tareas al agregar contexto como entrada de cada instancia, tanto en su forma corta (sólo el titular/tweet de la noticia) como en su forma larga (titular + cuerpo de la noticia). Así mismo, observamos que un clasificador entrenado para la tarea “granular” mejora levemente su performance al ser evaluado para la tarea “plana”, obviando los posibles errores de motivos discriminatorios. Combinando la adición de contexto y granularidad, un clasificador para la detección de lenguaje discriminatorio obtiene mejoras considerables sobre un BERT en español que sólo consume el texto del comentario.

Considerando la detección de discurso de odio dentro del área más abarcativa de clasificación de documentos en dominios sociales, analizamos algunos aspectos generales para tareas relacionadas como el análisis de sentimiento y la detección de emociones, entre otras. En particular, analizamos el desempeño de las técnicas de representación al ser entrenadas en distintos dominios. En general, los modelos de representación son entrenados a partir de textos de dominios “formales”, como pueden ser Wikipedia u otras fuentes similares. En esta tesis analizamos el efecto de generar estas representaciones desde textos informales. Observamos que –desde los word embeddings hasta los modelos pre-entrenados basados en transformers– las representaciones generadas son robustas y mejoran la performance en un conjunto de tareas de clasificación en textos sociales. Sobre los modelos pre-entrenados, estudiamos el impacto de entrenarlos desde cero en textos sociales o efectuar una adaptación sobre este dominio

Todos los estudios y recursos de esta tesis fueron realizados en español. Como un objetivo secundario, pretendemos mitigar la enorme asimetría de recursos existente en el área del procesamiento del lenguaje natural.



\subsection{Atentados en Charlottesville}

En Agosto del 2017, una gran movilización organizada por varios movimientos de ultraderecha y supremacistas blancos tuvo lugar en la ciudad de Charlottesville, Virginia, Estados Unidos, y particularmente centrada en la Universidad de dicho Estado. Esta concentración fue llamada en el medio del intento de universitarios y el movimiento Black Lives Matter (BLM) de remover estatuas de militares conferados pro-esclavitud de la Guerra de Secesión; en el caso de la Universidad de Virginia, sobre la estatua de Robert Lee. Más aún, tuvo lugar durante los primeros meses del mandato de Donald Trump.

Numerosos grupos de ultraderecha, neonazis, neo-confederados, entre otros, convocaron a la marcha ``Unite the Right'', diseñada como una campaña militar y organizada hasta 3 meses antes de su concreción. \citet{blout2020white} describen la experiencia de Charlottesville como la de un ``terrorismo inmersivo'', ya que generaron un ámbito de terror en varios ``teatros'' (como lo llaman los autores, usando jerga militar). Principalmente, el teatro físico, con la marcha y enfrentamientos con las contra-movilizaciones, la intimidante marcha de antorchas, y el asesinato de Heather Heyer atropellada por un manifestante neo-nazi. Así mismo, el teatro ``virtual'', que sirvió para generar un clima de intimidación en la previa, durante, y luego del evento, de alguna manera. \citet{blout2020white} citan particularmente una campaña judeófoba contra el mayor de Charlottesville (de ascendencia judía) y el vicemayor (afroamericano).

En el trabajo anteriormente mencionado, los autores llegan a la conclusión de que el evento fue organizado de manera centralizada, tanto en su planificación como despliegue en un intento de ejercicio militar. También concluyen que, la propaganda y la información propagada por los organizadores sirvió para publicitar y reclutar a simpatizantes y también para aterrorizar a la población. Esta propaganda tuvo lugar tanto en medios impresos (por ejemplo, posters pegados en las calles) como por medios virtuales y redes sociales como Facebook, Twitter o Discord. \citet{klein2019twitter} analiza los intercambios en Twitter entre los dos bandos (manifestantes de ultraderecha y los contramanifestantes) y muestra que, en el caso de quienes se encontraban del lado de la marcha de UtR, se identifica como enemigos a los musulmanes, liberales o izquierdistas,a miembros de la comunidad LGBTQ, judíos, entre otros.


\subsection{Matanza en Sinagoga de Pittsburgh}

En Octubre de 2018, un hombre fuertemente armado entró a la sinagoga ``El Árbol de la Vida'' en Pittsburgh, Pensilvania, Estados Unidos. Luego de gritar ``muerte a los judíos'', abrió fuego contra la multitud matando 11 personas y dejando decenas de heridos. El tirador era usuario activo de Gab\footnote{\url{https://gab.com/}}, una red social que nació en 2016 bajo la égida de la ``libertad de expresión'' con motivo de la creciente moderación de Twitter a discursos discriminatorios. El asesino en cuestión posteaba frecuentemente contenido antisemita en dicha red social \cite{mcilroy2019welcome}, la cual ha sido descrita como el ``Twitter racista''.

A raíz de esto, Gab fue dado de baja durante cierto tiempo al serle negado alojamiento web. Desde entonces, diversos trabajos han estudiado y recopilado el contenido discriminatorio en esta red social \cite{mcilroy2019welcome,kennedy2018gab}.


\begin{figure}[htbp]
    \centering
    \includegraphics[height=6cm, keepaspectratio]{img/gab-pittsburgh-post.jpg}
    \caption{Último post de Robert Bowers, tirador en la masacre de Pittsburgh, en la red social Gab.}
    \label{fig:gab_post}
\end{figure}
















\section{Revolución de NLP en los últimos años}
\section{NLP aplicado a redes sociales}

\section{Aportes de este trabajo}

Agrego algunas referencias para ir teniendo en cuenta:

\begin{itemize}
    \item Manifesto of Computational Social Sciences \cite{conte2012manifesto}
    \item Text Analysis in Python for Social Scientists: Discovery and Exploration \cite{hovy2020text} (Leer intro nada más)
    \item Computational social science and sociology (2020): \cite{edelmann2020computational}
\end{itemize}

\chapter{Preliminares}
\section{Redes Neuronales}

\subsection{Multi-layer perceptron}

Una red neuronal puede pensarse simplemente como una función $f: \mathbb{R}^m \rightarrow O$ que intenta aproximar una función $f^*$ con misma aridad. Podremos notar $ y = f(x; \Theta)$, siendo $\Theta$ los parámetros de dicha función

El perceptrón, desarrollado en 1958 por Frank Rosenblatt \cite{rosenblatt1958perceptron}, intenta ajustar

\begin{equation*}
    y = H(\Theta_1 x + \Theta_0)
\end{equation*}

donde $H$ es la función de activación, y $\Theta = (\Theta_0, \Theta_1)$ son los parámetros de la función. Este modelo es el primero cuyos pesos se encontraban mediante un algoritmo, considerando que $H$ es una función derivable. Este puede considerarse como el primer modelo de una neurona, junto al modelo de McCulloch-Pitts.

\citet{minsky1969perceptrons} demostraron que este tipo de modelos sólo pueden ajustarse a datos linealmente separables, provocando el primer ``invierno'' de las redes neuronales.

Una forma de sortear estas dificultades planteadas es ``apilar'' (stack en inglés) varias de estas funciones para poder ajustar a más tipos de funciones. En términos matemáticos, esto es tan sólo una composición de funciones, tomando ahora $f = f_3 \circ f_2 \circ f_1$, donde $f_1$ es la primer ``capa'' de nuestra función correspondiente a la entrada, $f_2$ es la capa intermedia u oculta, y $f_3$ es la capa de salida. Si bien este ejemplo consta de 3 capas, se puede generalizar a arbitrarias capas ocultas. Este modelo es el que conocemos como \textbf{Perceptrón Multicapa} o \textbf{Multi-Layer Perceptron} (MLP por sus siglas en inglés)

\subsection{Redes neuronales recurrentes}

Los problemas descriptos de NLP suelen constar de procesar una secuencia de palabras o tokens $x_1, x_2, \ldots, x_k$ de longitud variable, de manera de ajustar a una función

\begin{equation*}
    y = f([x_1, \ldots, x_k])
\end{equation*}

Una manera de ajustar una función de este tipo (usando una entrada de largo fijo) es convertir este problema a ajustar una función autorregresiva

\begin{equation*}
    y_k = f(x_k, y_{k-1})
\end{equation*}

donde tenemos una salida para cada paso $k$ de tiempo. Si $f$ es una red neuronal, llamamos a este tipo de redes neuronales \textbf{recurrentes}, ya que la salida a cada paso ($y_k$) depende de la salida del paso anterior, $y_{k-1}$\footnote{No confundir con las redes neuronales recursivas}.

Una primer aproximación a este problema es la red recurrente de Elman \cite{elman1990finding} definida por las siguientes ecuaciones

\begin{align}
h_t &= \sigma(W_h x_t + U_h h_{t-1} + b_h) \\
y_t &= \sigma(W_y h_t + b_y)
\label{eq:elman}
\end{align}

$h_t$ es normalmente llamado el \textbf{estado oculto} en las redes neuronales recurrentes. Los parámetros a ajustar son $W_h, U_h$ (matrices) y $b_h, b_y$ (escalares). Podemos ver que, a grandes rasgos, este tipo de red recurrente no es nada más que un perceptrón multicapa cuya entrada consta de $x_t$, la entrada original en el tiempo actual $t$, y el estado oculto anterior, $h_{t-1}$.

Para entrenar este tipo de redes recurrentes utilizamos back-propagation through time (BPTT), que consta en desplegar la relación recurrente y aplicar back-propagation de manera normal. \todo{explicar un poco mejor esto}

Este tipo de redes recurrentes sufren de varios problemas: entre ellos, \textbf{vanishing gradient} y \textbf{exploding gradient}. Estos problemas pueden observarse ya que el cálculo del gradiente de las ecuaciones \ref{eq:elman} usando BPTT induce la potencia a la $n$ (donde $n$ es el largo de la secuencia) de las matrices $W_h$ y $U_h$. Usando la descomposición de Jordan de estas matrices, podemos ver que sus elementos en la diagonal que sean distintos de 1, o bien tienden a infinito o a cero.

El problema de \textbf{exploding gradient} puede solucionarse mediante la técnica de \textbf{gradient clipping}\todo{citation needed}, que consta de reajustar la norma del gradiente. Sin embargo, nos queda aún el problema de \textbf{vanishing gradient}. Para ello, se han propuesto otras arquitecturas recurrentes.

\citet{hochreiter1997long} propusieron las \textbf{Long Short-Term Memory} (LSTM) como solución a estos problemas. Para solucionar los problemas mencionados, proponen una arquitectura basadas en compuertas (\textbf{gates}) que regulan los cambios en el estado oculto y en la salida. Concretamente, la arquitectura de las LSTMs está regida por las siguientes ecuaciones\footnote{Para una muy buena explicación de las redes recurrentes, sugerimos este artículo: \url{https://colah.github.io/posts/2015-08-Understanding-LSTMs/}}:


\begin{align}
    h_t &= \sigma(W_h x_t + U_h h_{t-1} + b_h) \\
    y_t &= \sigma(W_y h_t + b_y)
    \label{eq:lstm}
\end{align}



\subsection{Optimización}


\section{Técnicas de representación}

Una de las necesidades que tienen las redes neuronales para poder trabajar con textos es el de tener representaciones continuas de cada token o palabra. Las representaciones utilizadas en la época previa de los modelos lineales --bolsas de palabras/caracteres ponderadas con algún esquema similar a TF/IDF-- adolecen de varios problemas: tienen una altísima dimensionalidad; no tienen representación semántica de la similaridad de las palabras; están concentradas en una o pocas dimensiones y suelen ser discretas.

Trabajo previo ha demostrado que obtener representaciones continuas y distribuidas (de manera opuesta a discretas y concentradas)

Latent Semantic Analysis (LSA) \cite{landauer1997solution} es una de las primeras técnicas de representación continua utilizadas para esta tarea. Plantean el problema de obtener representaciones continuas como el de factorizar una matriz de co-ocurrencia entre tokens y documentos (o contextos). LDA (Latent Dirichlet Allocation) \cite{blei2003latent} es otra técnica basada en modelos gráficos entrenados mediante métodos variacionales, muy utilizada aún en la actualidad ya que genera representaciones latentes de los tópicos de los textos.

\begin{figure}
    \centering
    \includegraphics[width=0.65\textwidth]{img/02/bengio_neural_language_model.pdf}
    \caption{Ilustración del modelo de lenguaje neuronal de \citet{bengio2003neural}. La entrada consta de $n-1$ palabras, que primero pasan por una lookup table (o capa de embeddings), una función de activación y son colapsadas para luego ser utilizadas como entrada de una función softmax.}
    \label{fig:bengio_neural_language_model}

\end{figure}

Dentro de los métodos neuronales, uno de los más populares ha sido el de \citet{bengio2003neural} que propone una arquitectura neuronal para un modelo de lenguaje markoviano. La arquitectura de esta red está ilustrada en la figura \ref{fig:bengio_neural_language_model}. En la capa intermedia contiene una tabla de lookup de vectores de las diferentes palabras (también conocido como capa de embeddings) donde se generan las representaciones de las palabras. Trabajo posterior (con diferentes variaciones de esta misma idea) como el de \citet{collobert2011natural} ha demostrado que la utilización de este tipo de representaciones es útil para diversas tareas de NLP como POS Tagging, NER, y otras.

Uno de los problemas de los métodos vistos hasta el momento es que sufrían problemas de eficiencia, sólo pudiéndose entrenar con pocos millones de palabras y con dimensiones reducidas. La técnica \emph{word2vec} \cite{mikolov2013efficient} permite entrenar representaciones de palabras de mayor dimensión y sobre grandes cantidades de textos de manera eficiente. Los vectores de palabras guardan cierta estructura lineal y semántica, como ilustran los autores con algunos ejemplos, como el ya clásico $v(\text{rey}) - v(\text{hombre}) + v(\text{mujer}) \approx v(\text{reina})$.

Para generar los vectores de \emph{word2vec}, los autores plantean una relajación del problema de modelado de lenguaje mediante dos alternativas: Continuous Bag of Words (CBOW) y Skip-Gram. En CBOW intentamos predecir la palabra faltante dada una bolsa de palabras del contexto, y en skip-gram intentamos predecir las palabras del contexto dada la palabra central. Para ambos problemas, se generan representaciones intermedias ricas para las distintas palabras. \citet{mikolov2013efficient} extiende la idea del anterior trabajo proponiendo plantear el problema de skip-gram como uno de distinguir ruido de palabras efectivamente del contexto, haciendo mucho más eficiente el cálculo de estas representaciones. GloVe \cite{pennington2014glove} es otra técnica de representación de palabras que combina las ideas de factorización de matrices de LSA  mediante un problema de optimización distinto y generando representaciones que superan ligeramente en algunos benchmarks de tareas a los de \emph{word2vec}.

Uno de los problemas que tienen estos métodos es que cada representación se calcula sobre las distintas palabras. En español, por ejemplo, las palabras gato, gata, gatito, gatuno, todas tienen representaciones independientes en \emph{word2vec}, a pesar de tener información morfológica en común. Esto es un problema en varios escenarios: idiomas con muchas inflexiones o aglutinantes (como el turco, alemán o finés) o --lo que es de nuestro interés-- texto altamente desnormalizado como el de redes sociales. La técnica \fasttext{} \cite{bojanowski16} extiende la idea de \emph{word2vec} mediante la asignación de vectores a secuencias de 3 caracteres (subpalabras), capturando así cierta información morfológica. La representación de una palabra se obtiene mediante una combinación lineal de los vectores de las subpalabras que la componen.

\subsection{Tweet Embeddings}
\label{sec:02_tweet_embeddings}

%%
%%
%%
%%  https://docs.google.com/drawings/d/1BU3ulBiqU0NojpW6Fkb4xFlMCDigSWwfjN7z9smO6nY/edit
%%
%%

\begin{figure}[t]
    \centering
    \includegraphics[width=0.80\textwidth]{img/tweet_embeddings.pdf}
    \caption{Representación contínua de un tweet mediante combinación lineal de las representaciones de cada palabra.}
    \label{fig:tweet_embeddings}
\end{figure}

Una forma relativamente simple de obtener una representación de un documento u oración (en nuestra tesis, esto será casi siempre un tweet) es realizar una combinación lineal de las representaciones obtenidas para cada palabra. Es decir, dada una oración $s = w_1 w_2 \ldots w_n$, y representaciones $\overline{w_1}, \overline{w_2}, \ldots, \overline{w_n} \in \mathbb{R}^m$, podemos obtener una representación

\begin{equation}
    \overline{s} = \sum\limits_{i=1}^{n} \alpha_i \overline{w_i}
\end{equation}

con $\alpha_1, \ldots, \alpha_n \in \mathbb{R}$ escalares (dependientes de la oración). De esta manera, obtenemos de $n$ representaciones independientes del contexto una representación para el tweet, sin tener en cuenta posibles interacciones entre los distintos componentes. La figura \ref{fig:tweet_embeddings} ilustra esta metodología simple para obtener representaciones de oraciones.

Tenemos entonces dos posibilidades para determinar la combinación lineal: la forma de obtener las representaciones, y la forma de calcular los coeficientes. Para las representaciones, podemos usar varias de las técnicas que ya vimos como word2vec, GloVe, o \fasttext{}. Para calcular los coeficientes, consideramos en nuestro trabajo dos formas. La primera, la forma canónica, calculando un promedio de las representaciones, es decir, tomando $\alpha_i = \frac{1}{n}$

Se utilizaron combinaciones lineales para calcular una representación de un solo tweet.
Seguimos dos enfoques simples: promedio simple y promedio ponderado. En el segundo caso, utilizamos un esquema que se asemeja a la frecuencia inversa suave (SIF) \cite{arora17}, inspirado TF-IDF.
Cada palabra $ w $ se pondera con $ \frac {a} {a + p (w)} $, donde $ p (w) $ es la palabra probabilidad unigrama y $ a $ es un hiperparámetro de suavizado.
Los valores altos de $ a $ significan más suavizado hacia el promedio simple.



\section{Transfer Learning y modelos pre-entrenados}

\subsection{ELMo y ULMFiT}
\label{subsec:elmo}

Hasta cerca de 2018, la forma canónica de abordar un problema de NLP era entrenar una red neuronal recurrente que consumiera embeddings no contextualizados de los tokens de entrada. Esta arquitectura tiene algunas limitaciones; una de ellas es que, dados dos o más problemas distintos (por ejemplo, análisis de sentimientos e inferencia de lenguaje natural --NLI--) lo único compartido por ambas arquitecturas es la capa más baja de la red -- la capa de embeddings -- teniendo que entrenar desde cero todo el resto de los parámetros del modelo. En términos coloquiales, cada red debe ``aprender a leer'' sobre cada tarea, ignorando muchas construcciones sintácticas y semánticas comunes del lenguaje.

Uno de los primeros esfuerzos exitosos en sobrepasar los embeddings no contextualizados es \elmo{} \cite{peters2018}. Este modelo aprende embeddings ya no sobre una única palabra como \emph{word2vec} y sus variantes, sino sobre todo una oración, generando representaciones contextualizadas de cada palabra.  Para aprenderlas, \elmo{} se entrena sobre modelo de lenguaje bidireccional \footnote{En realidad no es estrictamente bidireccional, sino dos modelos de lenguaje concatenados} recurrente de varias capas sobre grandes cantidades de texto. En dicho trabajo, utilizan luego una combinación lineal de la salida de cada capa para obtener representaciones contextualizadas de cada token. Esta misma idea es una continuación \citet{peters2017semi}, y también parcialmente de \citet{mccann2017learned}; en este último trabajo abordan la construcción de representaciones contextualizadas mediante la tarea de traducción automática.

\begin{figure}[t]
    \centering
    \includegraphics[width=\textwidth]{img/02/ulmfit.pdf}
    \caption{Universal Language Modeling for Text Classification (ULMFiT). Esquema del método planteado:}
    \label{fig:ulmfit}
\end{figure}

Alrededor de 2018, este paradigma comenzó a cambiar hacia un esquema donde se entrena una red neuronal sobre una tarea genérica para luego ajustarla a la tarea específica, algo muy común en el área de Computer Vision. \citet{howard-ruder-2018-universal} introdujeron la técnica de ULMFiT(Universal Language Modeling for Fine-tuning for text classification), uno de los trabajos fundamentales de este nuevo paradigma. La idea propuesta se puede resumir en entrenar un modelo de lenguaje sobre un gran dataset no etiquetado, y luego utilizar esa misma red (cambiándole la última capa) ajustándola a una tarea específica.

Los autores proponen 3 etapas: primero, el pre-entrenamiento sobre la tarea de modelado de lenguaje en un gran dataset de texto (e.g. Wikipedia o Common Crawl); segundo, un ajuste de la tarea de modelado de lenguaje sobre el texto de la tarea en cuestión (LM fine-tuning); y finalmente, el entrenamiento sobre las etiquetas de la tarea (Classifier fine-tuning). La figura \ref{fig:ulmfit} ilustra las 3 etapas para el problema de clasificación de sentimientos. Entre varias técnicas que utilizan para entrenar estos modelos y evitar el olvido catastrófico, vale destacar el uso de \emph{slanted triangular learning rates}, donde el learning rate tiene una etapa de \emph{warmup} donde sube hasta el pico y luego una etapa de \emph{annealing} donde se reduce linealmente hasta 0 por el resto del entrenamiento. Esta técnica será utilizada por \bert{} y otros modelos de lenguaje basados en transformers.

El modelo de lenguaje utilizado por los autores de ULMFiT utiliza una arquitectura AWD-LSTM \cite{merity2018regularizing}. Estas arquitecturas recurrentes fueron el estado del arte hasta el momento, pero fueron sobrepasados por las basadas en \emph{transformers}, que pasaremos a detallar.


\subsection{Modelos basados en atención}
\label{sec:02_transformers}

Una de las limitaciones de los modelos basados en redes recurrentes es que sufren un \tbf{sesgo de localidad o secuencialidad}(locality bias) \cite{battaglia2018relational}. En palabras coloquiales, las redes recurrentes tienen problemas para aprender dependencias de largo rango en las oraciones, siendo esto producto de su arquitectura autorregresiva donde se construye la salida $y_t$ en base a $y_{t-1}$. Este sesgo es particularmente dañino en tareas de transducción de sequencias con la arquitectura encoder-decoder básica ya que a esto se le suma un cuello de botella forzoso por la compresión de toda la secuencia de entrada en un vector de longitud fija. \footnote{\citet{sutskever2014sequence} de hecho en su trabajo observa que invertir la oración de entrada obtiene mejores resultados para la tarea de traducción automática}.


\begin{figure}[t]
    \centering
    \includegraphics[width=0.5\textwidth]{img/02/attention_model.pdf}
    \caption{Mecanismo general de atención. En azul, la salida del encoder recurrente de la entrada. En rojo, la salida del decoder recurrente. Fuente: \citet{luong2015effective}}
    \label{fig:attention_mechanism}
\end{figure}


Una de las formas de mitigar este sesgo es la utilización de mecanismos de \emph{atención} \cite{bahdanau2014neural}. Suponiendo una arquitectura recurrente de encoder y decoder, y siguiendo la notación de \citet{luong2015effective}, para la tarea de traducción automática de una secuencia $(x_1, \ldots , x_n)$ a $(y_1, \ldots , y_m)$, con estados ocultos $(\overline{h_1}, \ldots , \overline{h_n})$ para la entrada y $(h_1, \ldots , h_m)$ y para la salida, el mecanismo de atención \footnote{global, en \citet{luong2015effective} se menciona el mecanismo local que no consideramos} consta de calcular para cada paso $t$ un vector de contexto

\begin{equation*}
    c_t = \sum_{i=1}^n \alpha_i^{(t)} \overline{h_i}
\end{equation*}

donde $\alpha^{(t)}$ es el vector de alineamiento, calculado como

\newcommand{\score}[0]{\text{score}}

\begin{equation*}
    \alpha^{(t)} = \softmax(\score(\overline{h_1}, h_t), \score(\overline{h_2}, h_t) \ldots , \score(\overline{h_n}, h_t))
\end{equation*}

Cada $\score(\overline{h_i}, h_t)$ marca una similaridad no normalizada entre sus argumentos. Siguiendo las alternativas planteadas en \citet{luong2015effective}, tenemos algunas posibles opciones para esto:

\begin{equation}
    score(\overline{h_i}, h_t) =  \begin{cases}
        \overline{h_i}^T h_t   & \text{dot} \\
        \overline{h_i}^T W h_t & \text{general} \\
        v^T\text{tanh}(W [\overline{h_i}^T; h_t]) & concat
     \end{cases}
\end{equation}

% copypasting random stuff from the Internetz
% https://tex.stackexchange.com/questions/66537/making-hats-and-other-accents-bold
%
\newcommand{\thicktilde}[1]{\mathbf{\tilde{\text{$#1$}}}}

con $W, v$ parámetros adicionales. En el caso de la atención producto interno (dot) podemos reescribir todas las ecuaciones como:

\begin{equation}
    C = \softmax(H \widehat{H}^T ) \widehat{H}
    \label{eq:attention_product}
\end{equation}

donde $\widehat{H}, H$ son los vectores que tienen $(\overline{h_1}, \ldots , \overline{h_n})$  y $(h_1, \ldots , h_m)$ como filas respectivamente, y $\softmax$ se calcula fila a fila.

Finalmente, el vector $\widetilde{h_t}$ es calculado como una transformación del estado oculto del decoder $h_t$ y el vector contextual $c_t$:


\begin{equation*}
    \widetilde{h_t} = \tanh(W_h [h_t; c_t])
\end{equation*}


El vector $\widetilde{h_t}$ codifica información de manera global de todos los estados ocultos del codificador, mitigando este problema del sesgo de localidad. Esta técnica se convirtió en parte integral de los modelos seq2seq como ser traducción automática, sumarización, entre otras. La figura \ref{fig:attention_mechanism} ilustra esta arquitectura.

La técnica de auto-atención o intra-atención \cite{parikh-etal-2016-decomposable} consiste en aproximadamente la misma idea que la atención sólo que teniendo una única secuencia; podemos asumir ecuaciones similares con $\overline{h_i} = h_i$. La auto-atención genera representaciones de los distintos vectores de entrada observando la totalidad de la secuencia, a diferencia de las redes recurrentes que sólo construyen una representación en base al paso anterior. Esta capa se utiliza en arquitecturas para clasificación de texto encima de una capa recurrente para generar representaciones con dependencias sin distinción de la distancia entre los distintos tokens.

\section{Transformers}

\begin{figure}[t]
    \centering
    \begin{subfigure}[]{0.55\textwidth}
        \centering
        \includegraphics[width=0.55\textwidth]{img/02/transformer_architecture.png}
        \caption{Arquitectura de modelos Transformer}
        \label{fig:transformer_architecture}
    \end{subfigure}
    \begin{subfigure}[]{0.40\textwidth}
        \centering
        \includegraphics[width=0.40\textwidth]{img/02/scaled_self_attention.png}
        \caption{Arquitectura de modelos Transformer}
        \label{fig:scaled_self_attention}
    \end{subfigure}
    \caption{Modelo de transformador y su versión de auto-atención. La subfigura \ref{fig:transformer_architecture} muestra la arquitectura de los codificadores y decodificadores. Fuente: \citet{vaswani2017attention}}
    \label{fig:transformer_mechanism}
\end{figure}

Mencionamos el sesgo de la secuencialidad como uno de los problemas de los modelos recurrentes. Otro de los grandes obstáculos para las arquitecturas autorregresivas es la paralelización. El modelo de cómputo secuencial donde $h_t$ se calcula en base a $h_{t-1}$ inhibe un cálculo paralelo, donde las diferentes representaciones puedan ser generadas simultáneamente. \citet{parikh-etal-2016-decomposable} es uno de los primeros trabajos que proponen una arquitectura para el problema de inferencia (NLI) enteramente basada en modelos de atención, sin modelos recurrentes.

\citet{vaswani2017attention} introdujeron la arquitectura \emph{Transformer} para la tarea de traducción automática. Esta arquitectura no utiliza capas recurrentes ni convolucionales, basándose enteramente en el mecanismo de auto-atención. La figura \ref{fig:transformer_architecture} muestra la arquitectura de los modelos basados en Transformer, organizado en forma de encoder-decoder, con 6 capas de cada uno.

Cada capa del encoder utiliza un mecanismo de auto-atención múltiple seguido de una capa feed-forward punto a punto. Las dos capas de auto-atención o feed-forward están sucedidas por una conexión residual \cite{he2016deep} \footnote{El fin de estas conexiones residuales es facilitar el flujo del gradiente en arquitecturas profundas} y una normalización, de manera que la salida se expresa como:

\begin{equation*}
    \text{Layer}(x) = \text{Norm}(x + \text{subLayer}(x))
\end{equation*}

Las capas decodificadoras son similares, salvo que se les agrega una capa extra de auto-atención donde se combinan las salidas del encoder con las representaciones que genera el decoder. A su vez, las capas de multi-atención están enmascaradas para no poder ``ver'' las representaciones que se generan en pasos posteriores para guardar su naturaleza secuencial en la tarea.

El cálculo de atención utilizado en este trabajo es similar al visto en la ecuación \ref{eq:attention_product}, aunque normalizado por $\sqrt{d_k}$, donde $d_k$ es la dimensión de los vectores de entrada:

\begin{equation*}
    Attention(Q, K, V) = \text{softmax}(\frac{Q^T K}{\sqrt{d_k}}) V
\end{equation*}

Cada capa utiliza varias cabezas de auto-atención, las cuales concatena y proyecta en su salida en su salida. La salida de cada una de capa pasa por una regularización de tipo dropout \cite{srivastava2014dropout}.

Un punto no menor es que el modelo Transformer, siendo que no tiene ningún tipo de recurrencia y convolución, carece de cualquier ordenamiento de la secuencia de tokens. Para inyectar ese conocimiento en la red, utilizan \emph{vectores de posicionamiento} (positional embeddings) que se suman a los vectores de entrada de la capa de embeddings, como se ilustra en la figura \ref{fig:transformer_architecture}. Estos vectores no son parámetros entrenados (como sí lo son en \bert{}) sino que se calculan mediante funciones sinusoidales.


No nos extenderemos más en la explicación de esta arquitectura, y referimos para más información a los excelentes artículos \emph{Transformers from Scratch} \footnote{\url{http://peterbloem.nl/blog/transformers}}, \emph{Annotated Transformer} \footnote{\url{https://nlp.seas.harvard.edu/2018/04/03/attention.html}} y \emph{The Illustrated Transformer} \footnote{\url{https://jalammar.github.io/illustrated-transformer/}}.


\section{GPT, BERT, y modelos pre-entrenados basados en Transformers}

Combinando las ideas de ULMFit --entrenaje semi-supervisado sobre la tarea de modelado de lenguaje-- y la arquitectura Transformer --removiendo las redes recurrentes y facilitando la paralelización del cálculo-- en \citet{radford2018improving} se introduce GPT (\emph{generative pre-training}). Esta técnica consiste de un pre-entrenamiento sobre un gran corpus no etiquetado y luego un fine-tuning discriminativo para cada tarea, muy en la línea de \citet{howard-ruder-2018-universal}, introduciendo parámetros específicos únicamente para cada una de estas. El modelo que usa esta tarea es el de \tbf{modelado de lenguaje causal} -- es decir, de izquierda a derecha. Este modelo obtuvo para muchas tareas como el benchmark GLUE \cite{wang-etal-2018-glue} el estado del arte.


\bert{} \cite{devlin2018bert} (Bidirectional Encoder Representations from Transformers) plantea una modificación sobre GPT: en lugar de pre-entrenar el modelo sobre la tarea de modelado de lenguaje \tbf{causal} --de izquierda a derecha-- entrenar sobre la tarea de modelado de lenguaje \tbf{enmascarado}. Esta tarea (usualmente llamada \emph{Cloze task} \cite{taylor1953cloze}) consta de enmascarar una cierta cantidad de palabras de una frase, y luego intentar predecir las palabras faltantes. Por ejemplo, en la siguiente frase, consta de reemplazar los dos tokens \verb|[MASK]|:

\begin{center}
    El \verb|[MASK]| es celeste y el pasto \verb|[MASK]|
\end{center}


\begin{figure}
    \centering
    \includegraphics[width=\textwidth]{img/02/gpt_vs_bert.pdf}
    \caption{Comparación entre ELMo, GPT y BERT. }
\end{figure}

A diferencia de la tarea de modelado de lenguaje causal, los autores argumentan que esta tarea permite generar representaciones bidireccionales ricas. La figura


A modo de comparación entre las distintas estrategias de pre-entrenamiento, la figura




\part{Trabajos con datasets existentes}

\chapter{Análisis de Sentimientos}
\label{chap:03_social_text_classification}

La extracción de opiniones en distintos espacios virtuales ha atraído mucho interés desde los comienzos de la World Wide Web. Inicialmente motivados por fines puramente comerciales, diferentes motivaciones han surgido debido al desarrollo de la técnica y la proliferación de las redes sociales: desde intereses sociológicos (como el análisis de discurso de odio o las reacciones a la pandemia) hasta políticos (como observar cuál es la opinión general sobre tal o cual candidato o sobre un tema candente). Desde principios de los años 2000, y debido a la combinación del desarrollo de métodos de aprendizaje estadístico y la cantidad creciente de datos disponibles generados por usuarios en Internet, numerosos trabajos han analizado este tipo de textos para poder extraer conocimiento \textbf{subjetivo} de estos.

Debido a la inmensa cantidad de contenido generado en diversos sitios y redes sociales (se estima que en el mundo se generan XXX tweets por segundo), hace ya muchos años esta tarea es difícil de realizar sin algún tipo de automatización. Para ello, muchísimo esfuerzo se ha volcado en utilizar técnicas de aprendizaje automático para atacarla. El avance de las técnicas de NLP --como hemos descrito en el capítulo anterior-- han permitido avanzar sobre este terreno; sin embargo, muchas de las limitaciones actuales del área \todo{citar paper Climbing towards NLU} en conjunto a las dificultades particulares de las interacciones en medios sociales hacen esta tarea difícil.

En este capítulo haremos una breve introducción a clasificación de textos sociales. Esto es, dado un texto generado por un usuario (un post en Facebook, Instagram, un tweet, etc) predecir alguna característica discreta de éste, como por ejemplo si es un texto positivo o negativo, si tiene algún tipo de emoción de ira, alegría, u otra; si contiene discurso de odio contra algún grupo o no; si es irónico; entre otras. En base a datasets en español para distintas tareas, presentaremos modelos de clasificación basados en técnicas del estado del arte.

Finalmente, analizaremos algunas cuestiones relacionadas a la adaptación de dominio y representaciones generadas sobre dominios sociales. Analizaremos para técnicas de representación no contextualizadas \footnote{Que al día de la fecha, en pocos años, han quedado obsoletas} y algunas técnicas más recientes el impacto de entrenar desde cero o realizar cierta adaptación sobre la performance de las técnicas de clasificación.


\section{Motivación}

Las motivaciones para extraer opiniones subjetivas de usuarios en Internet son múltiples, aunque intentaremos categorizarlas en algunos grupos de notable interés. Dado el aumento considerable de contenido generado por usuarios desde el comienzo de la WWW --y subsiguientemente con la explosión de las Redes Sociales-- una de las motivaciones es netamente comercial: ¿qué opinan los usuarios sobre este nuevo producto? ¿cuáles creen que son sus falencias? ¿qué tal es el servicio en el Restaurant X? Desde ya más de 20 años, numerosos sitios de e-commerce brindan la posibilidad de que los clientes vuelquen sus opiniones al respecto de los productos que consumen en sus plataformas, como así también pueden incorporarse en otras aplicaciones que brindan esta posibilidad de expresar comentarios sobre productos, servicios u otros lugares. Para citar unos ejemplos, IMDb permite agregar comentarios sobre películas, Google Maps sobre distintos sitios --tanto turísticos como locales comerciales--, o los distintos sitios de venta minorista como MercadoLibre, eBay, o Amazon.

Con la explosión de las redes sociales, otros horizontes de preguntas se abrieron\footnote{Si bien algunas preguntas de carácter sociológico tuvieron lugar con anterioridad, podemos marcar el uso intensivo de Facebook y Twitter como el comienzo de un estudio más sistemático de ellas}. Uno de estos horizontes, que es de interés particular para esta tesis, es el de las preguntas de carácter sociológico. Preguntas que pueden suscitar interés dentro de este punto pueden ser:

\begin{itemize}
    \item ¿cuál es la opinión de los usuarios acerca de la legalización del aborto?
    \item ¿cuál es el sentimiento que tienen ciertos usuarios hacia los inmigrantes subsaharianos en España?
    \item ¿cómo se ha modificado el ``humor social'' de acuerdo a crisis económicas o pandemias como la del COVID-19?
    \item ¿quiénes generan discurso de odio contra la comunidad LGBTI en Argentina?
    \item ¿qué artículos periodísticos suscitan la mayor cantidad de discurso discriminatorio en las redes sociales?
    \item ¿cuáles son las principales preocupaciones de ciertos sectores de la población?
\end{itemize}

entre otras. Estos tópicos son de gran interés para investigadores y políticos. Usualmente, la forma más estandarizada de acceder a la opinión de distintos actores sociales ha sido la de encuestas; sin embargo, la recolección y extracción automática de opiniones de medios virtuales brinda una alternativa (a veces) más económica y masiva aunque con un sesgo poblacional distinto al de otras metodologías.

\section{Cómo atacamos este problema desde Procesamiento de Lenguaje Natural}



\section{Trabajo previo}

Talleres
datasets


Describir data augmentation como otra técnica de regularización. Comentar backtranslation

Español
- Citar nuestro trabajo

\section{Tareas analizadas}
\subsection{Análisis de Sentimiento}

\subsection{Análisis de Emociones}


\section{Preprocesamiento}

\section{Técnicas de clasificación}

\subsection{Embeddings}

\section{Resultados}

\section{Discusión}

\section{Librería de análisis de sentimientos}

\newcommand{\pysentimiento}[0]{\textbf{pysentimiento}}

Algo que suele obstaculizar la utilización de herramientas de extracción de opinión (como las que acabamos de ver en este capítulo pero así mismo las que veremos más adelante) con fines de investigación es la dificultad a su acceso. O bien estos servicios están detrás de APIs pagas con precios demasiado altos para los presupuestos académicos o están disponibles pero no en español (u otro idioma de ``bajos recursos''). En otros casos, estos recursos están disponibles pero no para ser usados de forma de ``caja negra'', lo cual para alguien que no es un experto en NLP suele complicar su utilización.

Como una pequeñísima contribución de esta tesis y con el objetivo de facilitar el acceso de estos recursos para la investigación, creamos la librería \textbf{pysentimiento}\footnote{\url{https://github.com/pysentimiento/pysentimiento}}. Este paquete provee modelos pre-entrenados y herramientas de preprocesado para textos sociales. Si bien tiene soporte multilingual tanto en español como inglés, su eje original es el de proveer recursos para el español que tiene una disparidad importante en recursos.

La figura XXX muestra la arquitectura de \pysentimiento{}. Utiliza el model hub de \emph{huggingface}\footnote{\url{https://huggingface.co/models}}, un repositorio de modelos pre-entrenados basados en transformers. Allí es donde colocamos todos los modelos que entrenamos, tanto de sentimientos, emociones, y los que mostraremos más adelante como detección de discurso de odio. Cada tweet que es analizado por la librería pasa primero por una etapa de preprocesamiento (siguiendo el proceso explicado en la sección zzz), y luego procesado por el modelo, quien nos brinda un output. Dependiendo el problema, puede haber una etapa de post-procesamiento.

\todo{Completar las cosas que quedaron acá sin referencias}



%
% Pysentimiento architecture
% https://www.canva.com/design/DAEufPDskMI/Gg_phzjuXgFihF1g3x9L-A/edit#
%
%

\section{Conclusiones}

\section{Notas adicionales}

Comentar acá nuestro trabajo en TASS 2020

\chapter{Discurso de odio}
Los Discursos de odio contra mujeres, inmigrantes y muchos otros grupos es un fenómeno generalizado en la Internet. En los primeros días de la World Wide Web, algunos académicos se aventuraron a decir a que los prejuicios y el odio sería eliminado en este espacio por la disolución de identidades \cite{levy2001cyberculture, rheingold1993virtual}. Veinte años después de esta hipótesis, podemos
decir que no ha sido el caso. La prevalencia del racismo en la ``World White Web'' se ha estudiado en una serie de trabajos \cite{adams2005white, kettrey2014staking}, como así también la misoginia en el mundo virtual \cite{filipovic2007blogging, mantilla2013gendertrolling}.

El discurso racista y sexista es una constante en las redes sociales, pero los picos se documentan después de eventos ``detonantes'', como asesinatos con motivos religiosos o políticos \cite{burnap2015cyber}. Las empresas de redes sociales están preocupadas por esto y toman acciones en su contra; sin embargo, la mayoría de los esfuerzos todavía necesitan la intervención humana, lo que hace que esta tarea sea muy costosa. Reducir la intervención humana es vital para tener herramientas efectivas para evitar la escalada del discurso de odio.


En este capítulo haremos una introducción a este problema, que a su vez trataremos en los capítulos subsiguientes. Definiremos el discurso de odio y haremos una breve reseña de este fenómeno desde un marco legal y de tratados internacionales para luego centrarnos en este problema desde una perspectiva del procesamiento de lenguaje natural. Comentaremos parte de nuestro trabajo en \citet{atalaya_tass2018} como parte de la competencia hatEval\cite{hateval2019semeval}, a la vez que marcaremos algunos problemas actuales en los enfoques actuales del discurso discriminatorio.


\section{Definición de discurso de odio}

\label{sec:hate_speech_definitions}

No existe una definición universalmente aceptada de lo que configura discurso de odio. En esta sección haremos un repaso muy breve de algunos tratados internacionales sobre la materia para intentar aproximarnos a este concepto, a la vez que también haremos un racconto de las definiciones utilizadas en trabajos dedicados a la construcción de datasets.

Un derecho que suele estar protegido por constituciones nacionales y tratados internacionales es el del derecho a la expresión. Por ejemplo, el Pacto de San José de Costa Rica (a la cual Argentina adhiere)\cite{humanos2018convencion} dice en su Artículo 13:

\begin{displayquote}[CADH, Artículo 13][]

    1. Toda persona tiene derecho a la libertad de pensamiento y de expresión.  Este derecho comprende la libertad de buscar, recibir y difundir informaciones e ideas de toda índole, sin consideración de fronteras, ya sea oralmente, por escrito o en forma impresa o artística, o por cualquier otro procedimiento de su elección.

    2. El ejercicio del derecho previsto en el inciso precedente no puede estar sujeto a previa censura sino a responsabilidades ulteriores, las que deben estar expresamente fijadas por la ley y ser necesarias para asegurar:

    a)  el respeto a los derechos o a la reputación de los demás, o

    b) la protección de la seguridad nacional, el orden público o la salud o la moral públicas.
\end{displayquote}

En Estados Unidos, la primer enmienda protege este derecho humano, mientras que en la Unión Europea, legislación similar ofrece protección a la libertad de expresión. Finalmente, la declaración universal de los derechos humanos de la ONU \todo{citation needed} menciona tanto en su preámbulo como en el artículo 19

\begin{displayquote}[Declaración Universal de los Derechos Humanos][ONU]
    Todo individuo tiene derecho a la libertad de opinión y de expresión; este derecho incluye el de no ser molestado a causa de sus opiniones, el de investigar y recibir informaciones y opiniones, y el de difundirlas, sin limitación de fronteras, por cualquier medio de expresión.
\end{displayquote}

Otro documento conocido como el Pacto Internacional de Derechos Civiles y Políticos (ICCPR por sus siglas en inglés) menciona

\begin{displayquote}[Artículo 19 de la ICCPR]
1. Nadie podrá ser molestado a causa de sus opiniones.

2. Toda persona tiene derecho a la libertad de expresión; este derecho comprende la libertad de buscar, recibir y difundir informaciones e ideas de toda índole, sin consideración de fronteras, ya sea oralmente, por escrito o en forma impresa o artística, o por cualquier otro procedimiento de su elección.

3. El ejercicio del derecho previsto en el párrafo 2 de este artículo entraña deberes y responsabilidades especiales. Por consiguiente, puede estar sujeto a ciertas restricciones, que deberán, sin embargo, estar expresamente fijadas por la ley y ser necesarias para:

a) Asegurar el respeto a los derechos o a la reputación de los demás;

b) La protección de la seguridad nacional, el orden público o la salud o la moral públicas.
\end{displayquote}

Sin embargo, y como mencionan estos dos últimos apartados, la libertad de expresión tiene un límite: el ejercicio de los derechos e igualdad ante la ley. Por ejemplo, el Artículo 1 del Pacto de San José de Costa Rica dice lo siguiente:

\begin{displayquote}[Pacto San José de Costa Rica, CADH][Artículo 1]
    1. Los Estados Partes en esta Convención se comprometen a respetar los derechos y libertades reconocidos en ella y a garantizar su libre y pleno ejercicio a toda persona que esté sujeta a su jurisdicción, sin discriminación alguna por motivos de raza, color, sexo, idioma, religión, opiniones políticas o de cualquier otra índole, origen nacional o social, posición económica, nacimiento o cualquier otra condición social.
\end{displayquote}

A su vez, la Declaración Universal de los Derechos Humanos en su Artículo 1:

\begin{displayquote}
    Todos los seres humanos nacen libres e iguales en dignidad y derechos y, dotados como están de razón y conciencia, deben comportarse fraternalmente los unos con los otros.
\end{displayquote}

Entonces, los Estados y otros organismos deben tomar medidas para poder asegurar el libre ejercicio de los derechos y la igualdad de todos sus miembros, aún cuando esto pueda significar una restricción en la libertad de expresión \cite{article192015}.


¿Qué es el discurso de odio entonces? Como hemos mencionado, no hay una definición universalmente aceptada. Repasemos algunas clasificaciones hechas en estos tratados para acercarnos un poco más a las características comunes que comparten.

Como vimos, el Pacto de San José de Costa Rica en su Artículo 1 habla del ejercicio de derechos sin discriminación alguna por varias razones, entre las que menciona raza, sexo, idioma, religión, política, nacionalidad, posición económica, entre otras. La Observación General 35 del Comité por la Eliminación de la Discriminación Racial de la ONU (CERD) considera que será discurso de odio, y debe ser tipificado penalmente:


\begin{displayquote}[Recomendación 35 del Comité por la Eliminación de la Discriminación Racial, CERD]

    a) Toda difusión de ideas basada en la superioridad o en el odio racial o étnico, por cualquier medio;

    b) La incitación al odio, el desprecio o la discriminación contra los miembros de un grupo por motivos de su raza, color, linaje, u origen nacional o étnico;

    c) Las amenazas o la incitación a la violencia contra personas o grupos por los motivos señalados en el apartado anterior;

    d) La expresión de insultos, burlas o calumnias contra personas o grupos, o la justificación del odio, el desprecio o la discriminación por los motivos señalados en el apartado b) anterior, cuando constituyan claramente incitación al odio o a la discriminación;

    e) La participación en organizaciones y actividades que promuevan e inciten a la discriminación racial.
\end{displayquote}

\citet{gagliardone2015countering} presenta un análisis de diversos organismos y sus definiciones de discurso de odio. En líneas generales, como se menciona en \citet{CIDH2015}, el concepto usualmente es referido a expresiones que incitan a tomar algún tipo de medida hostil contra una víctima o un grupo de personas, siendo esta perteneciente a un determinado grupo social definido por laguna característica. Dicho esto, podría delimitarse el discurso discriminatorio del discurso de odio por la componente de la promoción e instigación de la violencia; sin embargo, para los fines de este trabajo utilizaremos los términos indistintamente. Como se menciona también en \citet{CIDH2015}, aún cuando el discurso no contenga arengas ni incitaciones a cometer actos violentos, puede entenderse ese discurso como generador de un ambiente hostil y de intolerancia que termine promoviendo estos ataques físicos.


\citet{article192015} condensa muchas de estas definiciones de una manera succinta, desglosando esto en ``odio'' y ``discurso'':

\begin{displayquote}[Article 19: Hate Speech Toolkit]

    – Odio: emoción intensa e irracional de oprobio, enemistad y aborrecimiento hacia una persona o grupo de personas, por tener determinadas características protegidas (reconocidas en el derecho internacional), reales o percibidas. El “odio” es más que un mero prejuicio y debe ser discriminatorio. El odio es una muestra de un estado emocional u opinión y, por lo tanto, se diferencia de cualquier acto o acción que se haya llevado a cabo.
    – Discurso: cualquier expresión que vierta opiniones o ideas, que comparte una
    opinión o una idea interna con un público externo. Puede adoptar muchas
    formas: escrita, no-verbal, visual o artística y puede ser difundida en los
    medios, incluyendo Internet, material impreso, radio o televisión.
\end{displayquote}

%%
%%
%% Link
%% https://docs.google.com/drawings/d/149dpb2nrvmFgWZJYcrToAxO4M5n7JNQInfWd62kw3jc/edit
%%
%%

\begin{figure}[t]
    \centering
    \includegraphics[width=\textwidth]{img/discurso_de_odio.pdf}
    \caption{Definición de discurso de odio de acuerdo al Toolkit de Article 19}
    \label{fig:hate_speech_definition_article_19}
\end{figure}


Entonces, puede entenderse como un discurso de cierta intensidad e irracionalidad que ataca a una persona o un grupo de personas por alguna característica históricamente vulnerada: por ser mujer, por su etnia, nacionalidad, religión, idioma, etc. La clave está en la combinación: un discurso irracional e intenso contra alguien que no posea una característica protegida no configura discurso de odio; por ejemplo, ataques a ciertas personas por ser periodistas. La figura \ref{fig:hate_speech_definition_article_19} ilustra esta definición.

No todo ataque a un individuo o una persona de algún colectivo discriminado es discurso de odio. En particular, en \citet{CIDH2015} se menciona en base al informe de la UNESCO sobre discurso de odio \citet{gagliardone2015countering} que:

\begin{displayquote}[]
    (...) el discurso de odio no puede abarcar ideas amplias y abstractas, tales como las visiones e ideologías políticas, la fe o las creencias personales. Tampoco se refiere simplemente a un insulto, expresión injuriosa o provocadora respecto de una persona. Así definido, el discurso de odio puede ser manipulado fácilmente para abarcar expresiones que puedan ser consideradas ofensivas por otras personas, particularmente por quienes están en el poder, lo que conduce a la indebida aplicación de la ley para restringir las expresiones críticas y disidentes. Asimismo, el discurso de odio tiene que distinguirse de aquellos “crímenes de odio” que se basan en conductas expresivas, como las amenazas y la violencia sexual, y que se encuentran fuera de cualquier protección del derecho a la libertad de expresión
\end{displayquote}

Como vemos, no sólamente es difusa la frontera fijada la característica sobre qué es discurso de odio o insultos, sino que incluso también es difícil definir qué característica es protegida o no. En el siguiente capítulo hablaremos más de esto al relatar cuáles fueron usadas a la hora de anotar nuestro dataset.

Si bien, como mencionamos, en cierta legislación se diferencia entre discurso discriminatorio y discurso de odio, para los fines de este trabajo utilizaremos ambas acepciones indistintamente. Cuando haya un llamado o una incitación a la violencia o algún tipo de represalia se hará explícita esta cuestión.


\section{Trabajo previo}

En esta sección haremos una breve reseña de algunos trabajos destacados del área. Un análisis extensivo de esta disciplina escapa totalmente al alcance de este trabajo. Referimos a quien esté interesado a \citet{schmidt2017survey} y a \citet{fortuna2018survey}. Más recientemente, \citet{poletto2021resources} hace un análisis pormenorizado y actualizado de los recursos para la tarea de detección de discurso de odio.

La detección del discurso del odio es una tarea de clasificación de oraciones bastante relacionada con el análisis de sentimientos y ha sido estudiada para varias redes sociales \cite{thelwall2008social, pak2010twitter, saleem2017web}. Uno de los primeros trabajos al respecto es \citet{greevy2004classifying} usando bolsas de palabras y SMVs para detectar contenido racista en páginas web. Construyeron su dataset de manera semi-supervisada buscando sitios mediante keywords y sus links en motores de búsqueda. Siguiendo un enfoque similar, \citet{warner2012detecting} usó unigrams y clusters Brown con SVM para detectar mensajes antisemitas en Twitter.

\citet{waseem2016hateful} anotó un corpus y usó n-gramas de caracteres para detectar comentarios de odio, y \citet{badjatiya2017deep} usó el mismo conjunto de datos para entrenar modelos de aprendizaje profundo e incrustaciones ajustadas junto con Gradient Boosted Trees. \citet {zhang2018detecting} entrenó una red neuronal profunda que combina CNN con unidades recurrentes cerradas \cite{cho2014learning}, superando a los sistemas anteriores en varios conjuntos de datos.

\citet{anzovino2018automatic} recopiló un corpus de tweets misóginos y propuso una taxonomía para distinguirlos en diferentes categorías. Los autores propusieron una serie de técnicas diferentes para clasificarlos, mostrando que enfoques simples (como el uso de modelos lineales junto con n-gramas de token) logran un rendimiento competitivo en conjuntos de datos de pequeño tamaño.

En cuanto a las tareas compartidas, \citet{fersini2018overview} presentó un desafío en la detección de misoginia en Twitter, tanto en español como en inglés, mientras que \citet{fersini2018evalitaoverview} planteó un desafío similar pero en italiano e inglés. \citet{bosco2018overview} propuso un concurso de detección automática sobre publicaciones de Twitter y comentarios de Facebook, que incluía discursos de odio en general.



\section{Datasets}

En esta sección haremos un recuento de los conjuntos de datos para la detección de lenguaje odioso.
\citet{nobata2016abusive} fue uno de los primeros datasets => mencionar sus criterios, y su experimento (muy interesante) de Turkers vs anotadores in-house

% \begin{table*}
%     \centering
%     \begin{tabular*}

%     \end{tabular*}
% \end{table*}

\citet{gao2018detecting}

\subsection{Criterios de anotación}

\begin{table*}[t]
    \centering
    \begin{tabular}{p{0.20\linewidth}  p{0.80\linewidth}}
        \textbf{Hate Speech} & Language which attacks or demeans a group based on race, ethnic origin, religion, disability, gender, age, disability,
        or sexual orientation/gender identity. Some examples:
        \begin{itemize}
            \item The JEWZ are Market Manipulators Hate them freaking Jews !! To bad Hitler did not finish them off !!!
            \item  I hate the mall because there are too many unemployed apefricans and mudsharks.
            \item 300 missiles are cool! Love to see um launched into TelAviv! Kill all the gays there!
            \item EGYPTIANS ARE DESCENDED FROM APES AND PIGS. NUKE EGYPT NOW Ah me tu se ku sh
        \end{itemize} \\

        \hline

        \textbf{Derogatory} & Language which attacks an individual or a group, but which is not hate speech. Some examples:
        \begin{itemize}
            \item All you perverts (other than me) who posted today, needs to leave the O Board. Dfasdfdasfadfs
            \item yikes...another republiCUNT weighs in....
        \end{itemize} \\

        \hline

        \textbf{Profanity} & Language which contains sexual remarks or profanity. Some examples:

        \begin{itemize}
            \item T.Boone Pickens needs to take a minimum wage job in FL for a week. I guarantee he shuts the f up after that.
            \item Somebody told me that Little Debbie likes to take it up the A.\$.\$.
            \item So if the pre market is any indication Kind of like the bloody red tampons that you to suck on all day??
        \end{itemize}
         \\
    \end{tabular}
    \caption{Annotation guidelines used in \cite{nobata2016abusive}}

    \label{tab:nobata_guidelines}
\end{table*}


\subsubsection{}

\section{Método}

\subsection {Preprocesamiento}


\newcommand{\elmo}[0]{ELMo}
\newcommand{\elmomodel}[0]{\emph{LSTM-\elmo{}}}
\newcommand{\bow}[0]{BoW}
\newcommand{\boc}[0]{BoC}
\newcommand{\elmobowmodel}[0]{\emph{LSTM-\elmo{}+\bow{}}}
\newcommand{\svmmodel}[0]{$\mathrm{SVM}_0$}
\newcommand{\hateval}[0]{HatEval}
\newcommand{\semeval}[0]{SemEval-2019}
\newcommand{\fasttext}[0]{\emph{fastText}}

El preprocesamiento es crucial en las aplicaciones de PNL, especialmente cuando se trabaja con datos ruidosos generados por el usuario. Aquí, seguimos \citet{atalaya_tass2018}, definiendo dos niveles de preprocesamiento: preprocesamiento básico y orientado a sentimientos. Usamos uno u otro, dependiendo de la configuración.

El preprocesamiento básico de tweets incluye tokenización, reemplazo de identificadores, URL y correos electrónicos, y acortamiento de letras repetidas.

El preprocesamiento orientado a sentimientos incluye minúsculas, eliminación de puntuación, palabras vacías y números, lematización (usando TreeTagger \cite{schmid95}) y manejo de negación.
Para el manejo de la negación, seguimos un enfoque simple:
% \cite {das01, pang02}:
Buscamos palabras de negación y agregamos el prefijo 'NOT \_' a los siguientes tokens. Se niegan hasta tres tokens, o menos si se encuentra un token que no sea una palabra.

\section{Técnicas de clasificación}

Para capturar esta información, consideramos una representación de bolsa de caracteres que codifica recuentos de caracteres $n$ -gramas para algunos valores de $ n $. Estos vectores se calculan a partir de textos originales de tweets, sin ningún procesamiento previo. \boc {} s tienen las mismas variantes y parámetros que \bow {} s.


\subsection {Word-embeddings}

Usamos \fasttext {}, una biblioteca de incrustaciones consciente de subpalabras \cite{bojanowski16} para obtener representaciones de palabras independientes del contexto.
En lugar de usar vectores previamente entrenados disponibles públicamente, entrenamos nuestras propias incrustaciones en un conjunto de datos de $ \sim90 $ millones de tweets de varios países de habla hispana.
Preparamos dos versiones de los datos: una usando solo preprocesamiento básico y la otra usando preprocesamiento orientado a sentimientos (con la excepción de la lematización). Para estos dos conjuntos de datos, las incrustaciones de omisión de gramática se entrenaron utilizando diferentes configuraciones de parámetros, incluyendo una serie de dimensiones, tamaño de n-gramas de palabras y subpalabras, y tamaño de la ventana de contexto.

\subsection{Tweet Embeddings}
\label{sec:sif}

% Hay varias formas de usar incrustaciones de palabras para el análisis de sentimientos en tweets: los enfoques van desde el simple promedio de vectores para cada palabra en el tweet hasta el uso de arquitecturas más complejas como CNN o RNN. En este trabajo,
Se utilizaron combinaciones lineales para calcular una representación de un solo tweet.
Seguimos dos enfoques simples: promedio simple y promedio ponderado. En el segundo caso, utilizamos un esquema que se asemeja a la frecuencia inversa suave (SIF) \cite {arora17}, inspirado en la reponderación de TF-IDF.
Cada palabra $ w $ se pondera con $ \frac {a} {a + p (w)} $, donde $ p (w) $ es la palabra probabilidad unigrama y $ a $ es un hiperparámetro de suavizado.
Los valores altos de $ a $ significan más suavizado hacia el promedio simple.

% También consideramos dos opciones que afectan las incrustaciones de tweets: binarización, que ignora las repeticiones de tokens en los tweets; y normalización, que escalas dando como resultado que los vectores de tweets tengan una norma unitaria.


\subsection{Embeddings contextualizados}
\label{subsec:elmo}

Después del gran salto adelante que representó las incrustaciones de palabras independientes del contexto, llegó una nueva ola en los últimos años. En lugar de tener vectores entrenados para cada palabra, se generan representaciones dependientes del contexto para cada token dada una oración. Por ejemplo, \citet{mccann2017learned} usó un codificador LSTM profundo para traducción automática para generar vectores sensibles al contexto.

\elmo{} \cite{peters2018} es uno de estos enfoques dependientes del contexto y se basa en un modelo de lenguaje bidireccional profundo (biLM). La arquitectura del modelo de lenguaje consta de L capas de LSTM bidireccionales, además de una representación de token independiente del contexto. Por lo tanto, para cada token en una secuencia, obtenemos representaciones vectoriales de $ 2L + 1 $.
% Estas representaciones se consideran profundas ya que utilizan la salida de cada capa LSTM.
Para obtener un vector final para cada token, los autores sugieren colapsar las capas en vectores mediante una combinación lineal.

% Sea $ t_1, \ldots, t_n $ una secuencia de tokens, y sea $ h_ {k, j} $ el vector que representa la salida de la capa $ j $ cuando se consume el token $ t_k $. Entonces, el vector contextualizado para el token $ k $ es:
%
% \begin {ecuación}
% ELMo_k ^ {tarea} = \gamma ^ {tarea} \sum_ {j = 0} ^ {L} s_j h_ {k, j} \label {eq: elmo}
% \end {ecuación}

En este trabajo, usamos la implementación y los modelos entrenados previamente de \cite{che-EtAl:2018:K18-2}. El modelo español se entrenó con $L = 2 $ capas y 1024 dimensiones, y la combinación lineal se realizó utilizando un promedio simple.

\section{Resultados de clasificación}

\subsection{Análisis de Error}

\subsection{Intepretabilidad}

\section{Limitaciones}

\subsection{Problemas de anotación}

\subsection{Falta de contexto}

\subsection{Interpretabilidad y fragilidad de clasificadores}

% \chapter{Limitaciones de todo: metodológicos, de los corpus, de los modelos.}
% \section{Anotación y sus limitaciones}

\section{Falta de contexto}

Citar paper

\section{Interpretabilidad y fragilidad de clasificadores}
\part{Creación del dataset}

\chapter{Metodología y definiciones}
\label{chap:dataset_creation}

En este capítulo contaremos el diseño del proceso de recolección, selección y anotación de datos. Por lo marcado en anteriores secciones, consideramos interesante el problema de hacer una detección de lenguaje discriminatorio bajo un contexto. Es decir, no es lo mismo considerar el mensaje ``sos un hombre'' en solitario que si ese mismo mensaje está dirigido hacia una mujer trans.

Nos abocamos a la decisión de crear un dataset que no sólo contenga un mensaje/comentario, sino que provea un contexto en el cual se da este mensaje. Un ámbito natural para esta tarea son las notas periodísticas, donde disponemos de una nota y comentarios realizados sobre esta. Un ejemplo puede verse .

Muchos sitios de noticias disponen de sistemas embebidos de comentarios, pero vista la dificultad para la recolección a la vez que los limitados datos provistos por estos sitios nos llevaron a buscar otro medio: Twitter. Twitter provee una sencilla API para descargar datos, a la vez

Algo a tener en cuenta es que este tipo de datos tiene una naturaleza particular, ya que las agresiones discriminatorias son usualmente a personajes públicos o colectivos de personas, y se dan de manera indirecta (a través del comentario en la noticia) y no directa (es decir, como respuesta al usuario de Twitter ofendido)

\section{Trabajos previos}
\label{sec:dataset_previous}

Pocos trabajos incorporan algún tipo de contexto a los comentarios del usuario para estas tareas. \citet{gao-huang-2017-detecting} construyó un dataset de lenguaje discriminatorio sobre 1518 comentarios del sitio de Fox News. A los anotadores les fue presentado tanto el comentario como la noticia a la hora de realizar el etiquetado. Sobre este dataset, efectuó experimentos de clasificación usando modelos lineales (regresiones logísticas) y modelos neuronales. En estos experimentos observó que un clasificador (tanto lineal como neuronal) mejora su performance al consumir el título de la noticia, dando indicios de que se puede aprovechar el contexto para mejorar la detección de este fenómeno. Sin embargo, como marca \citet{pavlopoulos2020toxicity} este trabajo cuenta con algunos problemas: en primer lugar, el tamaño del dataset es pequeño, y está extraído de sólo 10 noticias, lo cual limita los posibles contextos. A su vez, la anotación fue realizada mayormente por un único anotador, lo cual hace poco confiables las etiquetas. Luego, algunos detalles menores debieran ser analizados con mayor detalle, como por ejemplo la utilización de los nombres de usuarios como features predictivas.

\citet{mubarak-etal-2017-abusive} construyó un dataset en árabe sobre comentarios con contenido abusivo del portal Al Jazeera. Sin embargo, este daaset tiene un problema: los comentarios son sólo presentados a los anotadores sobre noticias, ignorando todo el thread de la conversación. Esto hace que el contexto sea parcial.


Paralelamente a nuestro trabajo, \citet{pavlopoulos2020toxicity} analiza el impacto de agregar contexto a la tarea de detección de toxicidad. En particular, plantea dos preguntas

\begin{itemize}
    \item ¿Qué tanto afecta el contexto a la toxicidad percibida por humanos en conversaciones online?
    \item ¿Puede el contexto ayudar a mejorar la performance de clasificadores de toxicidad en comentarios?
\end{itemize}

Para responder estas preguntas, los autores construyeron dos datasets en base a Wikipedia Talk Pages\cite{hua-etal-2018-wikiconv}, un dataset de discusiones del sitio de Wikipedia. En primer lugar, armaron un dataset de 250 comentarios anotados por dos grupos disjuntos de anotadores: uno de los grupos anotó los comentarios de manera contextualizada, viendo tanto el comentario en cuestión como el título de la discusión; el otro grupo sólo vio el comentario a anotar sin contexto alguno. En dicho experimento observaron que los anotadores contextualizados percibieron 6.4\% de comentarios tóxicos versus un 4.4\% de quienes anotaron sin contexto, una diferencia significativa aplicando un test Mann-Whitney. Desagregando estos resultados, observaron que 13 de los 250 comentarios (5.2\%) tuvieron diferencias de anotación entre los dos grupos, con 9 (3.6\%) comentarios donde aumentó la toxicidad percibida y 4 comentarios donde bajó la toxicidad al ser agregado el contexto.

Para responder la segunda pregunta, anotaron un dataset de 20k comentarios, 10k anotados por un grupo que etiquetó viendo el contexto y otros 10k que no lo vio. Entre todos los comentarios del dataset original de Wikipedia Talk Pages, eligieron aquellos con profundidad entre 2 (respuestas directas) a 5, y con entre 10 y 400 caracteres de largo. Luego, entrenaron varios clasificadores usando este dataset y allí pudieron observar que el contexto no pareciera mejorar la performance. En el próximo capítulo nos extenderemos sobre las técnicas utilizadas por este trabajo.

\citet{xenos-2021-context} continúa el trabajo de \citet{pavlopoulos2020toxicity} desagregando el resultado de la segunda pregunta. Puntualmente, y observando que sólo un porcentaje pequeño de los comentarios parecen ser incididos por el contexto, construyen una nueva tarea: estimación de sensibilidad al contexto. Para ello, toman el dataset de Civil Comments\cite{borkan2019civil}, y reanotan un subconjunto de este dataset usando información de contexto a través de crowdsourcing. Las etiquetas de este dataset son de toxicidad en un estilo similar a una regresión ordinal, entendiendo las categorías no tóxico, incierto, tóxico, y muy tóxico. Ahora, teniendo las anotaciones originales del dataset (que fueron hechas sin contexto) y las nuevas anotaciones, pueden definir para cada comentario una sensibilidad al contexto, dada por

\begin{equation}
    \delta(p) = s^{oc}(p) - s^{ic}(p)
\end{equation}

donde $s^{oc}$ es la fracción de anotadores sin contexto que marcaron toxicidad, y $s^{ic}$ los que no tienen contexto.

\citet{sheth2021defining} señala algunas oportunidades para incorporar fuentes de información más ricas a la tarea de detección de toxicidad, como el historial de interacción entre los usuarios, el contexto social y otras bases de conocimiento externas.

Nuestra definición de discurso de odio contiene algunas formas implícitas de abuso y, en particular, marcamos las ``llamadas a la acción'', que se detallan en \citet{wiegand2021implicitly} entre otros tipos sutiles de comportamiento tóxico en las redes sociales.

\section{Recolección de noticias y comentarios}

\begin{table*}[t]
    \centering
    \begin{tabular}{ c|c|c }
        coronavirus  &  encierro          & síntomas \\
        covid        &  fase              & dengue   \\
        cuarentena   &  infectados        & aedes    \\
        normalidad   &  Wuhan             & mosquito \\
        aislamiento  &  distanciamiento   & cacharro \\
        padecimiento &  fiebre            &          \\
    \end{tabular}
    \caption{Words used to retrieve COVID-19 and Dengue related articles.\label{tab:article_words}}
\end{table*}


%%
%%
%%
%% https://docs.google.com/drawings/d/1IcBITgNJN-tehmvnZqcSF9cUuWIpNKJg6yHI5yjNF9c/edit
%%
%%

\begin{figure}
    \centering
    \includegraphics[width=\textwidth]{img/idea_dataset.pdf}
    \caption{Muestra de la recolección de datos}
    \label{fig:idea_dataset}
\end{figure}




Usamos la API de Twitter Stream mencionando cualquiera de estas cuentas. \todo{Explicar el proceso más detalladamente} Para cualquier tweet de uno de estos medios, reconstruimos la conversación. Para el propósito de este trabajo, solo estamos interesados en el primer nivel de respuestas al tweet original \footnote{Usamos una versión antigua de la API. La versión 2.0 parece facilitar la recopilación de conversaciones}. También eliminamos las URLs de los artículos de los enlaces.


Si bien consideramos otros medios (en particular, los ``periódicos'' electrónicos de derecha) decidimos apostar por medios más tradicionales con apoyo escrito: Clarín (@clarincom), La Nación (@LANACION), Infobae (@infobae), Perfil. (@perfilcom), Crónica (@cronicacom). Como el proceso de anotación iba a ser realizado por anotadores locales, decidimos no recopilar ningún dato de otros países de habla hispana.


El foco se hizo en artículos relacionados con COVID-19. Para ello, seleccionamos artículos buscando una cantidad de palabras en su cuerpo, por lo que seleccionamos específicamente artículos relacionados con COVID-19. La tabla \ref{tab: article_words} contiene las palabras utilizadas para recuperar estos artículos. También estábamos interesados en la pandemia del dengue; sin embargo, solo se recuperó un número muy pequeño de artículos.

La figura \ref{fig:idea_dataset} ilustra lo recolectado: en este caso, un tweet de un medio periodístico y todos los comentarios realizados sobre éste. En el caso de Twitter, los comentarios son respuestas al tweet original.


\subsection{Diarios elegidos}

Para esta tarea, elegimos 5 diarios:

\begin{itemize}
    \item Clarín
    \item La Nación
    \item Infobae
    \item Perfil
    \item Crónica
\end{itemize}

Estos diarios son los mayores generadores de contenido. Consideramos otros medios, pero nos atuvimos a medios formales tradicionales y con soporte escrito.

En un principio consideramos la posibilidad de anotar tweets de digiarios de otros países, pero teniendo en cuenta que esta tarea depende fuertemente de la jerga y de las variaciones dialectales de cada país decidimos realizar sólo anotación de estos diarios. A su vez, observamos que la mayoría de los comentarios son de la variedad dialectal rioplatense


\subsection{Datos recolectados}

\begin{table}[t]
    \centering
    \begin{tabular}{c|c|c}
    Medio      & \#Artículos recolectados & \#Comentarios \\
    \hline
    @infobae   &  45652   &  822462 \\
    @clarincom &  29050   &  672401 \\
    @perfilcom &  8764    &  61203  \\
    @LANACION  &  16040   &  506091 \\
    @cronica   &  17250   &  70872 \\
    \hline
    Total      & 116756  & 2133029 \\
    \end{tabular}
    \caption{Artículos recoletados por medio}
    \label{tab:articulos_recoletados_por_medio}
\end{table}


En la tabla \ref{tab:articulos_recoletados_por_medio} damos los números de los artículos recolectados por cada medio, luego de aplicado . Si bien recolectamos más artículos de otros medios, no son enumerados. Infobae es el medio que más producción de artículos genera, y también será finalmente sobre el que más comentarios etiquetemos.

La figura \ref{fig:fecha_articulos_por_medio_todas} muestra la distribución temporal de los artículos, sin aplicar ningún filtro por palabras, mientras que \ref{fig:fecha_articulos_por_medio_covid} muestra aquellas relacionadas al COVID-19 utilizando el filtro de palabras listado en la tabla \ref{tab:article_words}. Podemos observar dos caídas. Hay un pequeño pozo en mayo 2020 que se debió a la caída de nuestros servidores de recolección. Por otro lado, observamos que algunos medios (particularmente La Nación) parecieran mencionar menos directamente al COVID (al menos con los términos referidos anteriormente) hasta un nuevo pico cerca de fin de año, coincidente con un nuevo rebrote del virus en este país.

Sin embargo, todo esto puede ser un artefacto del método de filtrado: muchas notas contienen links a otras con sus títulos y eso puede interferir en estas estimaciones. Así y todo, decidimos mantener este método ya que consideramos que mayormente las notas en el período referido tienen relación con la pandemia.

\begin{figure}
    \centering
    \includegraphics[width=\textwidth]{img/fechas_por_medios_todas.png}
    \caption{Distribución temporal de artículos recolectados}
    \label{fig:fecha_articulos_por_medio_todas}
\end{figure}



\begin{figure}
    \centering
    \includegraphics[width=\textwidth]{img/fechas_por_medios.png}
    \caption{Distribución temporal de artículos recolectados que mencionan COVID-19 o algún término relacionado}
    \label{fig:fecha_articulos_por_medio_covid}
\end{figure}



\section{Selección de datos a anotar}


Un problema que se nos presenta antes de comenzar el etiquetado es el de seleccionar los artículos que vamos a etiquetar. Una primera posibilidad para hacer esto es realizar una selección aleatoria de artículos y comentarios; sin embargo, los comentarios discriminatorios no se distribuyen de manera uniforme entre los artículos, sino que se concentra en algunos temas. Es mucho más probable encontrar comentarios de índole discriminatoria en notas que tengan temas cercanos a alguna de las características protegidas; por ejemplo, es esperable que encontremos contenido odioso en notas sobre China y el Coronavirus o sobre una chica transgénero antes que en un artículo de fútbol o economía.

Teniendo esto en cuenta, evaluamos varias alternativas. La primera es observar los artículos e intentar seleccionar aquellos que consideremos que puedan tener contenido potencialmente discriminatorio.

Una posibilidad para esto sería usar algunas palabras ``semilla'' para seleccionar artículos interesantes. Otra sería buscar directamente comentarios que contengan algunos insultos comunes o expresiones peyorativas hacia nuestros grupos protegidos. Después de algunos experimentos, decidimos utilizar el muestreo basado en comentarios.

\subsection{Selección en base a artículos}

En primer lugar, consideramos la posibilidad de hacer una selección en base al contenido de los artículos. Luego de realizar algunos experimentos usando LDA \cite{blei2003latent} para buscar tópicos posibles de las notas, decidimos realizar una selección un poco más controlada y determinística en base a la utilización de palabras clave. Es decir, seleccionaremos artículos en base a la aparición o no de ciertas ``semillas''

Para ello, indexamos todos nuestros artículos en MongoDB \footnote{\url{https://www.mongodb.com/}}, una base de datos no relacional y desestructurada. MongoDB permite la utilización de índices en base a texto, y realizar búsquedas en base a textos, palabras, e inflexiones. Cada artículo fue indexado en base al contenido de su cuerpo (es decir, el texto en sí del artículo).

La tabla \ref{tab:palabras_articulos} muestra el conjunto utilizado para recolectar artículos. Como vemos, hay diversas palabras que recogen distintas temáticas de posibles tópicos ``calientes'', algunos muy locales respecto a eventos concretos durante la pandemia. Si algún artículo contiene una de las frases mencionadas, se selecciona el artículo para ser etiquetado.

\begin{table}[]
    \centering
    \begin{tabular}{l | l | l | l}
    China        &  piqueteros              &  mamá                & empleadas domésticas  \\
    Cuba         &  villas                  &  de género           & la modelo             \\
    cubano       &  la villa                &  aborto              & la periodista         \\
    bolivia      &  movimientos sociales    &  actriz              & la cantante           \\
    paraguayo    &  organizaciones sociales &  actrices            & travesti              \\
    judío        &  tomas de tierras        &  feminista           & trans                 \\
    camionero    &  toma de tierras         &  femicidio           & gay                   \\
    ladrón       &  sindicatos              &  enfermera           & homosexual            \\
    represión    &  Guernica                &  madre               & de la V               \\
    criminal     &  mapuches                &  personal doméstico  & Ofelia                \\
    \end{tabular}
    \caption{Palabras utilizadas para la selección de artículos}
    \label{tab:palabras_articulos}
\end{table}

\subsection{Selección en base a comentarios}

Otra posibilidad evaluada fue la de observar los comentarios de los artículos en lugar del contenido del artículo, y seleccionarlos en base a esto. En este punto, la idea es únicamente seleccionar los artículos y no los comentarios; estos últimos son sólo usados como ``pistas'' para ver comentarios con posible contenido discriminatorio, y como tal identificar a ese artículo como un posible generador de este tipo de contenido.

La idea es similar a la de la selección con artículos, sólo que aplicada a comentarios: buscamos comentarios que contengan alguna de las palabras semilla listadas en la Tabla \ref{tab:palabras_comentarios}. Estas palabras fueron recolectadas a base de experimentación y observación de los datos, y tratan de contener diversas expresiones de contenido mayormente discriminatorio.

Una idea también considerada fue la de utilizar un clasificador entrenado sobre otro dataset (por ejemplo, el de \citet{hateval2019semeval}) y con eso marcar comentarios posiblemente discriminatorios. Sin embargo, muy probablemente detectaríamos sólo comentarios para las categorías/características etiquetadas en esos datasets e ignorarían las que agregamos en nuestro trabajo; por ejemplo, la mayoría de los datasets no contienen comentarios anotados contra la comunidad LGBTI.

El procedimiento de selección consta de, dado un artículo, marcar sus comentarios que contengan una o más de las expresiones listadas. Si el artículo tiene tres o más comentarios marcados, entonces seleccionamos el artículo; caso contrario, es descartado.

\todo{Agregar comparación con métodos de recolección en base a esto de 'seed' words}

Vale remarcar que este proceso de selección es para los \emph{artículos}, no para los comentarios.

\begin{table*}[t!]
    \centering
    \begin{tabular}{l|l|l|l|l|l|l}
    bija          & urraca     & viejo puto    & trolo      & peruano  & matarlos         & negra      \\
    prostituta    & tucán      & trabuco       & sodomita   & peruca   & una bomba        & negro de   \\
    feministas    & putita     & travesti      & chinos de  & judío    & vayan a laburar  & negros     \\
    feminazis     & reventada  & trava         & bolita     & sionista & vayan a trabajar & bala       \\
    aborteras     & marica     & degenerado    & paraguayo  & villeros & gorda            & uno menos  \\
    \end{tabular}
    \caption{Palabras utilizadas para recolectar comentarios}
    \label{tab:palabras_comentarios}
\end{table*}

Luego de algunos análisis experimentales y observacionales de las dos posibles metodologías, decidimos utilizar el muestreo de artículos en base a comentarios. En base a un análisis subjetivo, los artículos seleccionados parecían tener mayor incidencia de mensajes odiosos y eso nos decantó hacia esa opción.

\section{Criterios de anotación}
\label{sec:criterios}

La definición de lenguaje discriminatorio utilizada en este trabajo está basada en trabajos de la Comisión Interamericana de Derechos Humanos (CIDH)\cite{CIDH2015}, del Centro de Estudios de Libertad de Expresión (CELE) \cite{cele2019} y en el Article 19 Hate Speech Toolkit \cite{article192015}.

Teniendo estos insumos en cuenta, entendemos que hay discurso discriminatorio en un texto social si contiene declaraciones de carácter intenso y posiblemente irracional de rechazo, enemistad y aborrecimiento contra un individuo o contra un grupo, siendo estos objetivos de estas expresiones por poseer (o aparentar poseer) una característica protegida.

Las características en cuestión son protegidas por leyes internacionales. En este trabajo consideramos las siguientes:

\begin{enumerate}
\item Sexo (Mujeres, concretamente)
\item Género o identidad sexual (Colectivo LGBTI)
\item Ser inmigrantes, extranjeros, pueblos aborígenes u otras nacionalidades (xenofobia, racismo)
\item Situación socioeconómica o barrio de residencia
\item Poseer discapacidades, problemas salud mental o de adicción al alcohol, drogas u otros estupefacientes
\item Opinión o ideología política
\item Aspecto o edad (mayormente, gordofobia/gerontofobia)
\item Antecedentes penales o estar privado de la libertad
\end{enumerate}

Si bien algunas de estas características no son consideradas en algunos tratados (por ejemplo, contra los presos), las tuvimos en cuenta por las características propias del tratamiento en medios durante la pandemia de distintos sucesos.

\todo{Linkear el apéndice}

\section{Modelo de etiquetado}



Un modelo de anotación es, según \citet{pustejovsky2012natural}, una representación práctica del objetivo de anotación. En nuestro caso, queremos marcar comentarios discriminatorios, marcar a qué grupos y/o características se está ofendiendo, y también identificar llamados a tomar alguna acción contra los objetos de esos discursos. Por lo pronto, haremos una definición que capture ese objetivo sin deternos demasiado en especificarlo formalmente (lo que llaman en ese libro ``especificación'').

\subsection{Modelo Jerárquico de Etiquetado}



\begin{figure}
    \centering
    \includegraphics[width=\textwidth]{img/modelosjerarquicos.png}
    \caption{Modelos jerárquicos de anotación. A la izquierda, tenemos el modelo jerárquico propuesto para HatEval \cite{hateval2019semeval}, a la derecha el modelo propuesto para OffensEval \cite{zampieri2019semeval2019}}
    \label{fig:modelos_offenseval_hateval}
\end{figure}

\subsection{Modelo de etiquetado jerárquico y contextualizado}

\citet{zampieri2019predicting} introdujeron un modelo jerárquico de anotación para la tarea de lenguaje ofensivo, utilizado en las competiciones OffensEval \cite{zampieri2019semeval2019} y hatEval \cite{hateval2019semeval}. La idea de la anotación jerárquica es realizar anotaciones adicionales sólo para algunos casos de anotaciones del nivel anterior.

En el caso de \emph{HatEval}, tenemos un primer nivel que consta de anotar si un tweet contiene o no lenguaje de odio (nivel 1). Si el tweet tiene lenguaje de odio, entonces anotamos si está dirigido a un individuo o a un grupo, y también anotamos si es agresivo o no (ambos nivel 2). En el caso de \emph{OffensEval}, primero anotamos si es ofensivo (nivel 1), luego si está dirigido o es un insulto no dirigido (nivel 2) y finalmente, si es dirigido y ofensivo, marcamos su objetivo (nivel 3). En la figura \ref{fig:modelos_offenseval_hateval} ilustramos ambos modelos.


%
%
% Link: https://docs.google.com/drawings/d/1ZgTmvRwMWn0B-kokfw87jfSa7eY5-OSBHwltetnNT08/edit
%



\begin{figure}
    \centering
    \includegraphics[width=0.5\textwidth]{img/Annotation Model.png}
    \caption{Modelo de anotación}
    \label{fig:annotation_model}
\end{figure}


La figura \ref{fig:annotation_model} muestra el modelo de anotación utilizado para el dataset construído en este trabajo. Seguimos un modelo jerárquico similar al propuesto por \citet{zampieri2019predicting}, aunque de sólo un nivel. Para cada comentario y su respectivo contexto (el artículo), requerimos una anotación  para decidir si el comentario es odioso o no. Si no es odioso, no se necesita más información. Si es así, el par artículo-comentario debe contener, además, una anotación por si llama o no a la acción, y al menos una categoría protegida





\section{Etiquetadores}

A diferencia de otros trabajos (como hatEval \cite{hateval2019semeval}), decidimos por un lado, garantizar que nuestros anotadores estén más cercanos culturalmente al problema en cuestión, a la vez que tener mayor control del perfil de estos. Consideramos que el discurso de odio tiene un fuerte componente cultural, muchas veces expresado a través de jerga o expresiones dialectales muy particulares, y relacionado con noticias muy propias de esta región.


\subsection{Tipos de anotación en otros trabajos}

Comentar otros trabajos acá

\begin{itemize}
    \item Davidson
    \item Waseem
    \item hatEval
    \item CONAN
    \item Gao (contextualizado)
    \item Context offensive (el de Google, y el griego)
\end{itemize}

\subsection{Perfil de etiquetadores}

Para la selección de anotadores, realizamos una búsqueda interna. Puntualmente, buscamos:

\begin{itemize}
    \item Estudiantes/graduades de carreras de Cs. Sociales, Psicología, Letras o afines
    \item Hablantes nativos (o casi) de Español Rioplatense
    \item Usuarios de redes sociales; preferentemente Twitter
\end{itemize}

Luego de una breve entrevista donde les contamos el proyecto y corroboramos que efectivamente sean hablantes nativos de Español Rioplatense y usuarios de RRSS, les mandamos una pequeña evaluación paga que constó de leer el manual de criterios de anotación (que agregamos en \ref{app:manual_criterios_anotacion}) y anotar 5 artículos. Esto lo realizamos para ver que efectivamente estén entendiendo la tarea. Estos artículos fueron luego reutilizados para el proceso de entrenamiento.

\section{Proceso de etiquetado}

\subsection{Preprocesado y filtrado de los datos}

El preprocesado de los datos es muy básico: en los hechos, efectuamos el mismo preprocesamiento que en anteriores tareas, consistente en reemplazar handles de Twitter por un token especial \verb|@usuario| para evitar cualquier sesgo. Por ejemplo, si un usuario conocido como ``odiador'' (llamemos \verb|@hater|) retwittea la noticia y otro responde a ese RT, aparece ese nombre de usuario lo cual podría condicionar al etiquetador.

Así mismo, descartamos cualquier tweet que tuviera algún link ya que pueden referir a contenido no textual


\subsection{Entrenamiento de etiquetadores}



%
% Esto quizás va después
%
\subsection{Herramienta de etiquetado}

%%
%% Link a Google Draw: https://docs.google.com/drawings/d/1E24-2l6hsNj2JSKBZOD8QvZCJR6rrGjz-cWwt8XuPRg/edit
%%

\begin{figure}
    \centering
    \includegraphics[width=\textwidth]{img/labeler.pdf}
    \caption{Pantalla del etiquetador}
    \label{fig:labeler_example}
\end{figure}

Al no utilizar ningún servicio de etiquetado, optamos por desarrollar nuestra propia aplicación para el etiquetado de tweets. En ella, a cada etiquetador les fueron asignados progresivamente los artículos a anotar, los cuales fueron agrupados en ``lotes'' para facilitar la tarea administrativa de la asignación.

La figura \ref{fig:labeler_example} muestra la interfaz presentada a los etiquetadores. Cada artículo es presentado al etiquetador junto a los comentarios asignados. Ante esto, el etiquetador puede elegir saltear el artículo o etiquetarlo. Si decide etiquetarlo, el etiquetador debe para cada comentario marcar usando un control de tipo ``switch''

\begin{enumerate}
    \item Si el comentario contiene discurso discriminatorio
    \item En caso de ser discriminatorio, marcar si llama a la acción
    \item En caso de ser discriminatorio, marcar al menos una característica ofendida
\end{enumerate}

Para el desarrollo de la aplicación usamos Django\footnote{\url{https://www.djangoproject.com/}}, un framework de python para desarrollo web, y Javascript plano. Como base de datos utilizamos SQLite ya que tenía una baja tasa de concurrencia (sólo 6 usuarios.)

\subsection{Esquema de anotación}

%Teniendo en cuenta el modelo de anotación ilustrado en la figura \ref{fig:annotation_model}, optamos por la siguiente metodología para el etiquetado de los comentarios de nuestro dataset.

%%
%%
%% Link a Google Draw:
%% https://docs.google.com/drawings/d/1esS9tAwpPVydohxd-B-xwVdAaPQRVGAo0MruBrgSKig/edit
%%
%%

\begin{figure}
    \centering
    \includegraphics[width=0.7\textwidth]{img/esquema_anotacion.pdf}
    \caption{Esquema de anotación. Caso en que ambos anotadores etiqueten los comentarios del artículo}
    \label{fig:annotation_schema}
\end{figure}

Los artículos son asignados a cada etiquetador. Cada etiquetador, al serle presentado un artículo, tiene dos opciones: etiquetarlo o saltearlo. La idea de saltear era doble: evitar contenido poco ``interesante'' en términos de comentarios discriminatorios, o evitar contenido sensible para el anotador (algo que no ocurrió afortunadamente).

Una posibilidad que barajamos en un principio fue asignar para el etiquetado el artículo completo a 3 anotadores. Sin embargo, esta modalidad sería altamente ineficiente dada la baja cantidad de contenido discriminatorio. Entonces, decidimos ir por un esquema de ``desempate'': dos anotadores anotan un artículo, y luego un tercero anota sólo aquellos donde al menos uno marcó que es discriminatorio. Esto da la posibilidad de que haya una tercera anotación incluso cuando dos previas marcaron que el comentario es discriminatorio, y lo hacemos para recolectar más información. \todo{marcar otros trabajos que hayan hecho esto}. Con este esquema de anotación, y teniendo en cuenta los números finales obtenidos del dataset, dedicamos 2.16 etiquetados por comentarios versus 3 etiquetados por comentario de anotar tres veces todo. La figura \ref{fig:annotation_schema} ilustra este flujo de anotación.

Entonces, en primer lugar cada artículo es asignado a 2 anotadores. Luego de esto, se solicita una tercera anotación pero sólo sobre los comentarios que tengan alguna de las dos etiquetadas marcando contenido discriminatorio, y no dando la posibilidad de saltear. Ahora ¿qué pasa si alguno de los dos anotadores saltea el artículo?. Tenemos dos casos. Si los dos saltean el artículo, entonces descartamos ese artículo. Ahora, puede ocurrir el caso de que uno lo saltee y el otro lo anote: en ese caso, y en pos de maximizar el contenido discriminatorio encontrado o uno lo hace y el otro anota menos de 4 comentarios odiosos, entonces no pasa a 3ra anotación y lo descartamos del dataset. Si uno salteó y el otro anotador anotó 4 o más comentarios odiosos, entonces forzamos al primer anotador a anotar el artículo, sin dar esta vez opción de saltear. La figura \ref{fig:annotation_schema_case_two} ilustra el flujo para este caso.


%%
%%
%% Link a Google Draw
%% https://docs.google.com/drawings/d/1TOlCgZggCmYHgZWV7ZrIIlXuhcFUMeYw4PcFM7XdY2k/edit
%%
%%

\begin{figure}
    \centering
    \includegraphics[width=0.6\textwidth]{img/esquema_anotacion_caso_2.pdf}
    \caption{Esquema de anotación. Caso en que un anotador saltee}
    \label{fig:annotation_schema_case_two}
\end{figure}


Como resultado de este esquema, cada comentario de nuestro dataset puede tener dos o tres anotaciones, siendo los casos posibles los siguientes:

\begin{enumerate}
    \item Dos anotaciones negativas
    \item Tres anotaciones, siendo al menos una que marque el comentario como discriminatorio
\end{enumerate}




\subsection{Asignación}

\citet{pustejovsky2012natural} denominan ``asignación'' al procedimiento de extraer las ``gold labels'' de las etiquetas. En este punto tenemos una etiqueta binaria si el contenido es discriminatorio o no (notamos HS) en el primer nivel, y luego 9 etiquetas binarias: una para la llamadas a la acción (CALLS) y otras 8 para las características ofendidas. Recordemos que una anotación negativa sólo consta de HS negativo, mientras que una positiva consta de un HS positivo, una etiqueta para CALLS y al menos una etiqueta positiva de las características restantes.

Para este dataset, tomamos las siguientes decisiones:

\begin{enumerate}
    \item Para la etiqueta de HS, realizamos la votación mayoritaria
    \item Si hay HS, CALLS es positivo sii es votación mayoritaria
    \item Si hay HS, marco como positivas todas aquellas características marcadas por los anotadores
\end{enumerate}

La primer decisión es la más obvia y razonable, pero las otras dos decisiones merecen alguna discusión. Para que sea un comentario considerado como HS, tiene que ocurrir que al menos dos etiquetadores lo marquen como tal. En ese caso, para que haya votación mayoritaria de CALLS, tiene que haber dos o más votos marcados como tal; en caso de empate, es decir, que un anotador marca que hay llamado a la acción y otro que no, marcamos que no hay llamado a la acción.

En el caso de las características, marcamos todas las que hayan marcado aquellos anotadores que hayan etiquetado HS. Esta decisión podría haberse tomado de otra manera; por ejemplo, sólo tomando aquellos casos donde haya cierto grado de coincidencia entre los comentarios. Sin embargo, al considerar que los límites entre las características son difusos (por ejemplo, apariencia y mujer tienen un grado de coincidencia, y a veces clasismo y racismo también) preferimos optar por este esquema.

\todo{Agregar algún gráfico de esto}

\subsection{Recursos utilizados}

El etiquetado constó de XXX horas. A cada etiquetador le fue pagado YYYY por hora, y luego ZZZ por hora en segunda instancia. Esto equivale a WWW USD.

\section{Dataset resultante}

\begin{table}
    \centering
    % \begin{tabular}{lrr}
    %     \toprule
    %     Total articles & 1238    \\
    %     Total comments &  56869  \\
    %     Hateful Tweets &   8715  \\
    %     Ratio          &   0.153 \\
    % \end{tabular}
    \begin{tabular}{lrr}
        \toprule
        Característica &  Count &  Calls to Action \\
        \midrule
        RACISM         &   2469 &              674 \\
        APPEARANCE     &   1803 &               34 \\
        CRIMINAL       &   1642 &              722 \\
        POLITICS       &   1428 &              136 \\
        WOMEN          &   1332 &               18 \\
        CLASS          &    823 &              135 \\
        LGBTI          &    818 &               11 \\
        DISABLED       &    580 &                4 \\
        \bottomrule
    \end{tabular}
    \caption{Figures of the annotated dataset, by total numbers and segmented by characteristic}
    \label{tab:dataset_figures}

\end{table}

El dataset resultante consta de 1238 artículos etiquetados, y 56869 comentarios respectivamente, de los cuales 8715 contienen contenido discriminatorio según los criterios de asignación antes referidos. Podemos observar que aproximadamente 1 de cada 6 comentarios es discriminatorio; esto no es representativo del universo de notas periodísticas ya que recordemos que la selección de los datos no fue aleatoria. La tabla \ref{tab:dataset_figures} contiene estos datos estadísticos.

De todos los tweets discriminatorios, tenemos en particular los llamados a la acción. La inmensa mayoría de estos está dirigido hacia la categoría CRIMINAL, muchos en la forma de llamados a matar a criminales y otros delincuentes.

La tabla \ref{tab:annotation_agreement} reporta el acuerdo entre anotadores usando la métrica alpha de Krippendorff \todo{agregar cita}. Reportamos el valor de $\alpha$ para HS sobre todas las etiquetas, y luego todas las etiquetas del segundo nivel del modelo jerárquico (características y llamado a la acción) sólo sobre aquellas que hayan marcado que el comentario contiene HS. Esto es equivalente a calcular el acuerdo con una etiqueta faltante en el segundo nivel para las características y el llamado a la acción. Si bien este acuerdo tiende a ser alto, debe leerse como el acuerdo sobre la razón detrás del hate speech; la mayor penalización queda reservada a HS, que tiene $\alpha = 0.59$, algo que podría marcarse como un buen acuerdo teniendo en cuenta los parámetros vistos en las tablas de preliminares. \todo{linkear esto}

\begin{table}
    \centering
    \begin{tabular}{lc}
        \toprule
        Categoría   & $\alpha$ de Krippendorff \\
        \midrule
        Hateful              &  0.579 \\
        Calls to Action      &  0.641 \\
        \midrule
        WOMEN                &  0.783 \\
        LGBTI                &  0.920 \\
        RACISM               &  0.929 \\
        CLASS                &  0.706 \\
        POLITICS             &  0.808 \\
        DISABLED             &  0.849 \\
        APPEARANCE           &  0.871 \\
        CRIMINAL             &  0.931 \\
        \bottomrule
    \end{tabular}
    \caption{Reported Agreements. \emph{Hateful} agreement is reported for the binary decision of a tweet assigned as hateful or not; for the other characteristics (and the calls to action) the agreement is calculated over those tweets with two or more hateful marks}
    \label{tab:annotation_agreement}
\end{table}




\begin{table}
    \centering|
    \begin{tabular}{p{0.2\textwidth} p{0.4\textwidth} p{0.5\textwidth}}
        \toprule
        Cat & & \\
        \midrule
        characteristic & text & context \\
        \midrule
                 WOMEN & @usuario Y como te quedó el ogt?. & Loly Antoniale mostró su impresionante casa en Miami: “Soy la reina de mi castillo” \\
                 WOMEN & @usuario Habla el agua viva esta vestida de verde ? & Cecilia Moreau: "No se puede permitir que la minoría le imponga a la mayoría qué temas discutir" \\
                 WOMEN & @usuario “Feminista interseccional”..:nos vamos a la mierda... & Empezó con una “relación abierta en lo sexual”, escaló al poliamor y da las claves para probar el amor libre \\
                 WOMEN & @usuario Provocador ? A mi me provoca ganas de vomitar & ¡El sensual y provocador topless de Morena Rial! \\
                 WOMEN & @usuario Que se aborte ella, vieja  bruja y degenerada!! & Martha Rosenberg: “En situación de pandemia, legalizar el aborto es más urgente que nunca” \\
                 WOMEN & @usuario Kien es la feminista resentida q está instaurando esa idea desde hace días?? & Femicidio en Catamarca: “No es gente enferma, sino que tiene una formación machista importante” \\
                 WOMEN & @usuario Pero quién puede pensar embarazar a esta mostra. & “Ya pusimos el cuerpo, ahora que los diputados se pongan las pilas”: en el Congreso, el lado verde se prepara para una fiesta \\
                 WOMEN & @usuario Pregúntaselo a Lousteau, trola de mierda & Juana Viale hizo un irónico pedido en su programa: “Que el señor Presidente me explique en qué fase de la ‘no cuarentena’ estamos” \\
                 WOMEN & @usuario Que pelotudo!!!!...lo que te va a tomar es la guita viejo decrepito...o te pensastes que se enamoro de vos..jajajaja & Eduardo Costantini y Elina Fernández mostraron la intimidad de su luna de miel: “Ella no me deja tomar una copa de vino” \\
                 WOMEN & @usuario El paskin con la trola & La pareja de Lázaro Báez contó cómo lo esperaba adentro del country: “Lo que le hicieron fue indignante” \\
                 WOMEN & @usuario Que manga de roñosas & “Ya pusimos el cuerpo, ahora que los diputados se pongan las pilas”: en el Congreso, el lado verde se prepara para una fiesta \\
                 WOMEN & @usuario A la que fue PROSTITUTA de Villa Ballester le molesta que le recuerden que fue PROSTITUTA? & Fabiola Yáñez denunció a un periodista por publicaciones agraviantes \\
                 WOMEN & @usuario Hay que comerse al termotanque de lipidos & ¿More Rial encontró el amor en un personal trainer? \\
                 WOMEN & @usuario @usuario @usuario SOS la peor mierda de argentina, junto al presidente y su titiritera de Cristina. Manga de soretes! \textbackslash nMuy sorora pero estan matando de hambre a media argentina, homicidios y terrorismo. Vayanse todos! Métete el corazón verde e... & Fuerte cruce entre Flavio Mendoza y Victoria Donda: “Es muy fácil hablar cuando uno cobra un sueldo” \\
                 WOMEN & @usuario No me sorprende, usa el pañuelito verde decisor. & Nancy Pazos reveló por qué decidió que su mamá no recibiera la donación de plasma \\
                 WOMEN & @usuario La Yañez está tratando de entender  de que habla Manes .. pobre solo conoce los tablones del teatro ... & Coronavirus. Fabiola Yáñez organizó una videollamada con Facundo Manes: "Lo importante es estar bien mentalmente" \\
                 WOMEN & @usuario Esto me hace tan feliz, jodanse aborteras de mierda. JAJAJAJAJAJAJAJAJAJAJAJAJAJAJAJAJAJAJAJAJAJAJAJAJAJAJAJAJAJAJAJAJAJAJAJAJAJAJAJAJAJAJAJAJAJAJAJAJAJAJAJAJAJAJAJAJAJAJAJJAAJAJAJAJAJAJAJAJAJAJAJAJAJAJAJAJAJAJAJAJAJAJAJAJAJAJAJAJAJAJAJAJAJAJAJ... & Aborto legal: otra promesa incumplida \\
                 WOMEN & @usuario Jaja la mina orgullosa de lo q consiguió gateando, bien ahi 🥴 & Loly Antoniale mostró su impresionante casa en Miami: “Soy la reina de mi castillo” \\
                 WOMEN & @usuario Y quien te iva a hacer un pibe...dracula o el hombre lobo.. & “Me esterilicé, pero no odio a los niños”: mi vida dentro del movimiento “libre de hijos” \\
                 WOMEN & @usuario Dirán lo que dirán de Moria y como sea trabajo toda su vida por eso tiene lo que tiene Pero Rocío  solo trabajo abriendo las piernas  en la cama  para llegar hacer figureti Y no me la cuenten que fue por amor🤣🤣 & El enojo de Moria Casán con Rocío Oliva: “Mucha agua oxigenada, le quedó media neurona para jugar a la pelota” \\
                 LGBTI & @usuario Revisen esa casa, los están envenenando. & Contó que era lesbiana, su papá le confesó que era gay y ahora su madre se enamoró de una mujer: así se inspiró para su segundo film \\
                 LGBTI & @usuario Me cruzo con 50 María Elena por día. Las feminazis son iguales & No es actriz, pero se anima al desafío: quién es Ethel Herrera, la tiktoker que se postula para ser María Elena en “Casados con hijos” \\
                 LGBTI & @usuario Biológicamente las mujeres tienen vagina y los hombres tienen pene. Lo demás es ideológico. & Tras los comentarios de J.K. Rowling, Emma Watson defendió al colectivo trans \\
                 LGBTI & @usuario por fin alguien que ponga huevos en el equipo & Histórico: Mara Gómez fue habilitada y será la primera jugadora trans en el fútbol argentino \\
                 LGBTI & @usuario pero....este no se comia la galletita? & Pablo Duggan, sobre su casamiento: "Karen es la mujer de mi vida, estoy muy enamorado" \\
                 LGBTI & @usuario Por el pirulo & Por qué Flor de la V no continuó en Mujeres de eltrece, tras la salida de Claudia Fontán \\
                 LGBTI & @usuario Pelado puto, te quedaste solo. & Horacio Rodríguez Larreta: “La convocatoria del lunes no es del partido, si alguno va es a título personal” \\
                 LGBTI & @usuario Bueno por lo menos le quedó el culo roto & El meteorólogo Matías Bertolotti se separó a ocho meses de haberse casado \\
                 LGBTI & @usuario No Clarin. Es un señor con apariencia de mujer que preñó a una mujer con apariencia de hombre. La paja no es como el trigo. & La historia de la modelo colombiana trans que besa la panza de su esposo embarazado de ocho meses \\
                 LGBTI & @usuario Che, qué pasó con el puto-no me enteré de nada. & Por qué Flor de la V no continuó en Mujeres de eltrece, tras la salida de Claudia Fontán \\
                 LGBTI & @usuario Viejo larga presos , llévalos todos a tu casa y te haces romper el orto viejo puto igual ya lo tener & Eugenio Zaffaroni le contestó a Sergio Berni tras la polémica por las domiciliarias: “Es el populacherismo vindicativo que llenó las cárceles” \\
                 LGBTI & @usuario Pero...\textbackslash n.\textbackslash n.\textbackslash n.\textbackslash n.\textbackslash nCamila es Camilo. & Camila Sosa Villada: “Con el feminismo y el psicoanálisis descubrimos que el amor es una herramienta de manipulación" \\
                 LGBTI & @usuario Vos también comes carne...\textbackslash n\textbackslash nNo te hagas la boluda, encima te comiste a un bala no asumido, ahí también comiste carne....\textbackslash n\textbackslash n@usuario \textbackslash n\textbackslash nDeja que cada uno elija qué hacer, de ultima es una elección personal como la tuya de comerte a un corrup... & Nicole Neumann apuntó contra los consumidores de carne: “Sigan comiendo asadito” \\
                 LGBTI & @usuario La mujer tiene duda siiiiiii duda la podonga 😂😂😂😂 & Pablo Duggan, sobre su casamiento: "Karen es la mujer de mi vida, estoy muy enamorado" \\
                 LGBTI & @usuario 🤮🤮🤮🤮🤮🤮 & La emotiva dedicatoria de Luis Novaresio a su pareja, Braulio Bauab, en su cumpleaños \\
                 LGBTI & @usuario Cerra el otro viejo chupa pija & Alberto Fernández: “Dejemos el tiempo del encuentro y del esparcimiento social para otro momento” \\
                 LGBTI & @usuario Quedaba feo poner "un jugador disfrazado de mujer" entendemos & Mara Gómez cumple su sueño: será la primera futbolista transgénero en el torneo profesional argentino \\
                 LGBTI & @usuario Creo que se debe referir a las feminazis,aborteras y ese circo de lgbtqia+ etc y toda esas payasadas....que son manipulados como titeres por los lobbys & Cecilia Moreau: "No se puede permitir que la minoría le imponga a la mayoría qué temas discutir" \\
                 LGBTI & @usuario Estos chetos soretes maricas nenas de mama cagones. & Coronavirus en Argentina: los “caprichos” de algunos repatriados en los hoteles porteños, entre rebeldes y cholulos \\
                 LGBTI & @usuario La discriminacion se termina despues del primer tortazo.. dejen de llorar manga de maricas!! & Viola Davis, Halle Berry y Angela Bassett, actrices negras que superaron la discriminación en Hollywood \\
        \bottomrule
        \end{tabular}
\end{table}



\subsection{Anonimización para publicación de los datos}



\section{Conclusión}

En este capítulo, describimos la construcción de un dataset contextualizado de lenguaje discriminatorio o hate speech. Para ello, recolectamos respuestas a noticias periodísticas posteadas en Twitter por los principales medios de noticias de Argentina. Exploramos distintas alternativas para la selección de artículos a etiquetar, tanto observando los tópicos de los artículos como los comentarios a este. Decidimos elegir los artículos en base a sus comentarios potencialmente discriminatorios, y luego seleccionar una muestra aleatoria y acotada de comentarios.

Para realizar la tarea de etiquetado, desarrollamos nuestra propia herramienta la cual hacemos pública. Definimos un modelo de anotación jerárquico y granular para la tarea, siendo relativamente novedoso el hecho de anotar las características ofendidas en cada texto social. Seis etiquetadores nativos de la variedad dialectal rioplatense realizaron la tarea de anotación bajo un esquema de 2 anotaciones + desempate.

Como producto, obtuvimos un dataset de cerca de 57k comentarios repartidos en 1.2k artículos, una cantidad de tamaño considerable aunque no tengamos parámetro de comparación ya que no existen muchos datasets similares. De los 57k comentarios, alrededor de 8k comentarios tienen contenido discriminatorio (una tasa de 1 cada 6). Un análisis exploratorio de los comentarios discriminatorios muestra ejemplos complejos y ricos, algunos de ellos altamente dependientes del contexto.

En el siguiente capítulo, abordaremos nuestra pregunta original: ¿puede el contexto ayudar a los algoritmos de clasificación a mejorar su performance?. Para responder esto, utilizaremos este dataset especialmente diseñado.


\chapter{Clasificadores, Análisis y Limitaciones}

En este capítulo abordaremos una de las preguntas centrales de la tesis: ¿puede ayudar el contexto a mejorar la performance de métodos automáticos de lenguaje de odio? Para intentar contestar esta pregunta, utilizaremos el dataset que construímos en el capítulo \ref{chap:dataset_creation}, y aplicaremos técnicas del estado del arte basadas en transformers. Exploraremos dos versiones de clasificadores: descontextualizadas, donde sólo observamos el comentario analizado; y contextualizadas, donde podemos consumir el título o el título y el cuerpo del artículo.

Proponemos en este capítulo dos tareas: detección ``plana'', donde sólo predecimos la característica de si hay o no discurso de odio; y la detección ``granular'', donde además predecimos todas las características ofendidas (potencialmente más de una). Analizaremos el impacto de agregar contexto para cada una de estos problemas de clasificación.


\section{Trabajos previos}

Como contamos en la anterior sección, son pocos los trabajos y datasets que poseen contexto. Ver \ref{sec:dataset_previous} para un repaso de los distintos datasets contextualizados.

\citet{gao-huang-2017-detecting} propone dos tipos de modelos: regresiones logísticas y redes neuronales recurrentes. Para los modelos de regresiones logísticas, usan como inputs bolsas de palabras, bolsas de caracteres, vectores semánticos producidos con Linguistic Inquiry and Word Count (LIWC) \cite{pennebaker2001linguistic} y features de un lexicon de emociones \cite{mohammad2013nrc}. Por otro lado, utiliza LSTM bidireccionales con mecanismo de atención de Bahdanau \cite{bahdanau2014neural} usando embeddings \emph{word2Vec} de dimensión 100.

Un punto criticable de este trabajo es que utiliza el nombre de usuario como feature; algo que a priori no suele hacerse ya que permitiría ``prejuzgar'' a un usuario antes que por el contenido de sus tweets. Si bien es cierto que la información de usuarios y sus conexiones es valiosa, introducir esta información a nuestros modelos da lugar a posibles correlaciones espurias que es preferible evitar.


\begin{figure*}[t]
    \centering
    \begin{minipage}[b]{0.49\textwidth}
        \includegraphics[width=\textwidth]{img/pavlopoulos_rnn_rnn_classifier.png}
    \end{minipage}
    \hfill
    \begin{minipage}[b]{0.49\textwidth}
        \includegraphics[width=\textwidth]{img/pavlopoulos_rnn_bert_classifier.png}
    \end{minipage}

    \begin{minipage}[b]{0.35\textwidth}
        \includegraphics[width=\textwidth]{img/pavlopoulos_bert_sep_classifier.png}
    \end{minipage}


    \caption{Clasificadores que consumen contexto propuestos por \citet{pavlopoulos2020toxicity}. Los dos primeros clasificadores proponen una arquitectura de dos encoders, uno para el texto y otro para el contexto usando bi-LSTMs y BERT como posibilidades. El tercer clasificador propuesto es un BERT usando su estructura natural para codificar dos oraciones separadas por el token $SEP$ }
    \label{fig:pavlopoulos_classifiers}
\end{figure*}


En la sección \ref{sec:dataset_previous} hemos descripto el dataset construído por \citet{pavlopoulos2020toxicity}. Nos detendremos un momento para analizar sus experimentos de clasificación  ya que guardan importantes similaridades con lo que haremos en este capítulo. En ese trabajo se obtuvieron dos datasets de entrenamiento, uno etiquetado viendo el contexto y otro sin verlo. El dataset de test fue etiquetado viendo el contexto, considerando que el etiquetado es de mejor calidad usando más información. Tenemos entonces, dos preguntas: ¿mejora la performance de la tarea usando el dataset etiquetado con contexto? ¿mejora la performance del clasificador consumiendo información contextual? Estas dos preguntas nos brinda entonces dos elecciones: dataset de entrenamiento con o sin contexto, y clasificador con o sin contexto. Tenemos 4 posibles combinaciones de experimentos, sin aún considerar posibles técnicas de clasificación.

Para cada una de estas combinaciones, se consideraron técnicas del estado del arte de clasificación. Para aquellos clasificadores que no consumen contexto, las opciones son las mismas que hemos visto en capítulos anteriores: bi-LSTM o BERT. Para aquellos que sí consumen contexto, se evaluan dos estrategias: una, concatenar con algún caracter, y otra usando dos encoders distintos para el contexto y el texto. A su vez también utilizan la API Perspective de Google \todo{Agregar alguna cita de esto, y quizás alguna explicación en capítulo 4}.

Para todas las combinaciones posibles, si bien hay una mejora en la performance medida con ROC-AUC al usar contexto en ambas formas, esta no es significativa. De esto los autores concluyen que no pueden encontrar evidencia suficiente sobre la utilidad del contexto en la detección de toxicidad.

Algo a mencionar (que retomaremos en este y en el siguiente capítulo) es que usan dos versiones de BERT: una usando los pesos del modelo de BERT, y otro haciendo un ajuste de dominio () corriendo la tarea de MLM sobre un dataset grande y no etiquetado. En el caso de el trabajo mencionado, sólo hacen un fine-tuning sobre comentarios sueltos del dataset de Civil Comments. Esto podría tener algún efecto deteriorando la performance al usar contexto; sin embargo, en el BERT a secas (sin hacer ajustes) tampoco se observa mejora significativa en la performance.

Algunas limitaciones marcadas por los autores son:

\begin{itemize}
    \item Contexto muy pequeño: sólo el título más el comentario previo
    \item Se ignora el hilo completo de comentarios
    \item Los datos fueron sampleados aleatoriamente
\end{itemize}

En \citet{xenos-2021-context}, continuación de este trabajo, reetiquetaron el dataset de Civil Comments usando contexto y --como mencionamos en la sección \ref{sec:dataset_previous}-- presentaron una nueva tarea de detección de sensibilidad al contexto. Usando la API Perspective (y la estrategia de concatenación ``básica'' con algún caracter), notaron que la performance del clasificador que consume el contexto mejora con respecto al que no lo hace a medida que restringimos nuestra atención a comentarios más ``sensibles al contexto'' (de acuerdo a la métrica definida por los autores)


\section{Tareas de clasificación propuestas}



Ahora que tenemos este corpus especialmente diseñado que contiene el contexto, ahora dirigimos nuestra atención a responder nuestra pregunta original: ¿pueden los clasificadores aprovechar el contexto para mejorar su desempeño en la tarea de detección del discurso discriminatorio? Para ello, proponemos las siguientes tareas de clasificación:

\begin{enumerate}
    \item \textbf{Detección ``plain''}: Dado un tweet y su contexto, predecir si contiene contenido discriminatorio
    \item \textbf{Detección ``fine-grained''}: Dado un tweet y su contexto, predecir las características ofendidas (si hay alguna) y si contiene un llamado a la acción
\end{enumerate}


Puede pensarse la tarea de detección plana (la que usualmente se aborda en la literatura sobre el tema) como una relajación de la tarea detallada: mientras la primera sólo nos permite detectar si hay o no contenido discriminatorio, la segunda nos pide información más precisa acerca de las características ofendidas. Estas segunda tarea es posible dado que el dataset que construímos contiene esta información, algo usualmente faltante en otros datasets.

La tarea en su versión fine-grained nos permite a su vez tener mayor entendimiento de la salida e interpretar mejor sus errores, principalmente los falsos positivos. La figura \ref{fig:hate_detection_tasks} ilustra las dos tareas propuestas. Mientras en la tarea plana sólo debemos decidir la frontera entre si el contenido es discriminatorio o no, en la tarea fine-grained necesitamos decir en cuál de todas las intersecciones está el comentario y su contexto.

Planteándolos como problemas de clasificación, la detección plana consta de predecir una sola etiqueta binaria, mientras que la tarea fine-grained consta de $n$ etiquetas binarias; es decir, $n$ problemas distintos de clasificación. Vale mencionar que, entendiendo una tarea como una relajación de la otra, si tenemos un clasificador entrenado para la tarea fine-grained podemos construir un clasificador para la tarea plana tomando la disyunción lógica de sus salidas. Retomaremos esta idea más adelante al hablar de cómo evaluamos nuestras técnicas de clasificación para cada tarea.

\begin{figure}[t]
    \centering
    \includegraphics[width=\textwidth]{img/hate_detection_tasks.pdf}
    \caption{Tareas propuestas de detección}
    \label{fig:hate_detection_tasks}
\end{figure}

To test our hypothesis, for each task we trained classifiers having as input the following comibinations:

\begin{enumerate}
    \item The comment
    \item The title and the comment
    \item The title, the body and the comment
\end{enumerate}


\subsection{Cotas a la performance}

\begin{table}
    \centering
    \begin{tabular}{lll|ll}
        \hline
                   & \multicolumn{2}{c}{Entre anotadores} & \multicolumn{2}{c}{Contra gold} \\
        {}         &  F1 mean&  F1 median  & F1 Mean  &  F1 Median \\
        \hline
        HATEFUL    &  0.6525 &   0.6751    & 0.8285   &   0.8515   \\
        CALLS      &  0.6042 &   0.7037    & 0.7741   &   0.9148   \\
        WOMEN      &  0.7258 &   0.7368    & 0.8371   &   0.8275   \\
        LGBTI      &  0.8939 &   0.9600    & 0.9660   &   0.9743   \\
        RACISM     &  0.9458 &   0.9592    & 0.9667   &   0.9731   \\
        CLASS      &  0.7310 &   0.7500    & 0.8058   &   0.8391   \\
        POLITICS   &  0.7370 &   0.7777    & 0.8920   &   0.9189   \\
        DISABLED   &  0.7973 &   0.8800    & 0.8976   &   0.9392   \\
        APPEARANCE &  0.8033 &   0.9024    & 0.9026   &   0.9493   \\
        CRIMINAL   &  0.8180 &   0.9473    & 0.9614   &   0.9788   \\
        \hline
    \end{tabular}
    \caption{Estadísticos de los cálculos de F1 entre anotadores - modo jerárquico}
    \label{tab:ia_f1_scores}
\end{table}

Como observamos en la anterior sección, la tarea de detección de lenguaje discriminatorio contiene una alta cantidad de ruido, y el acuerdo entre humanos es moderado. En este contexto, cabe preguntarse cuál es la máxima performance que puede lograr un clasificador para esta tarea. Por la misma naturaleza del problema, claramente no puede ser perfecta.

Para obtener algunas medidas de esto, calculamos en primer lugar las F1 usando todos los posibles pares de anotadores. Como la F1 es simétrica (invirtiendo roles se invierten la precisión y la sensibilidad) no necesitamos hacer ninguna asunción sobre cuál sus roles.

Algo a tener en cuenta es que nuestra métrica final será contra la etiqueta resultante del (nuestro \emph{gold standard}). Una cota que seguro está por arriba de nuestra performance es el acuerdo que haya entre los anotadores y este \emph{gold standard}; hay que también observar que cada etiqueta ``de oro'' codifica información de sus anotaciones, con lo cual éste número es una cota superior sin dudas pero también pueden ser demasiado ``gruesa''.

La tabla \ref{tab:ia_f1_scores} contiene estadísticos de estos cálculos, tanto entre anotadores como contra el \emph{gold-standard}. Como podemos observar, la mediana entre anotadores de la F1 (usada para obviar outliers) es relativamente baja para la detección de odio ($\sim 0.67$), mientras que contra el gold standard es de $0.85$. De esto entendemos que la performance máxima en la detección está entre esos dos números.

Por otro lado, para el resto de las características observamos números más elevados, pero hay que recordar que estos cálculos están hecho \textbf{solamente} entre tweets etiquetados como odiosos. Si obviamos esta restricción (lo que llamamos ``modo libre''), la performance esperada baja sustancialmente. La tabla \ref{tab:ia_f1_scores_free_mode} muestra estos números, tanto calculado entre anotadores como contra el \emph{gold standard}.

\begin{table}
    \centering
    \begin{tabular}{lll|ll}
        \hline
                   & \multicolumn{2}{c}{Entre anotadores} & \multicolumn{2}{c}{Contra gold} \\
        {}         &  F1 mean&  F1 median  & F1 Mean  &  F1 Median \\
        \hline
        CALLS      &  0.4341 &   0.4950   &  0.7042   &   0.8424  \\
        WOMEN      &  0.4896 &   0.4676   &  0.7406   &   0.7593  \\
        LGBTI      &  0.5959 &   0.5765   &  0.8462   &   0.9152  \\
        RACISM     &  0.6532 &   0.6444   &  0.8712   &   0.8789  \\
        CLASS      &  0.4431 &   0.4444   &  0.7220   &   0.7317  \\
        POLITICS   &  0.4609 &   0.4360   &  0.7951   &   0.8155  \\
        DISABLED   &  0.5502 &   0.6000   &  0.8127   &   0.8421  \\
        APPEARANCE &  0.6485 &   0.7428   &  0.8314   &   0.9146  \\
        CRIMINAL   &  0.5265 &   0.5801   &  0.8415   &   0.9292  \\
        \bottomrule
    \end{tabular}

    \caption{Estadísticos de los cálculos de F1 entre anotadores - modo libre. Cada característica es tomada como una etiqueta binaria independientemente del cálculo de odio}
    \label{tab:ia_f1_scores_free_mode}
\end{table}


\subsection{Preprocessing}

Para cada tweet, aplicamos el siguiente procesamiento previo: primero, cortamos las repeticiones de caracteres hasta tres ocurrencias; risas normalizadas; los identificadores de usuario (\emph{@user}) se reemplazan por un token especial \emph{[USER]}; convertimos emojis en una representación de texto usando la biblioteca de python \emph{emoji} \footnote {\url{https://github.com/carpedm20/emoji/}}. Los hashtags se eliminan, están rodeados por una ficha especial \emph{[HASHTAG]} y se dividen en palabras si están en mayúsculas.

Aunque no realizamos un análisis de ablación para evaluar el impacto de cada paso del preprocesamiento, el proceso general pareció mejorar el rendimiento de la clasificación en el conjunto de datos de desarrollo.

\subsection{Clasificadores}


%%
%%
%% Link a Draw
%% https://docs.google.com/drawings/d/1F8iVSIRqHhGkQ0zglxqXLGD36RHZ9OhHMZYsg_xFOS4/edit
%%
%%

\begin{figure}
    \centering
    \includegraphics[width=\textwidth]{img/bert_multioutput.pdf}
    \caption{Muestra de la recolección de datos}
    \label{fig:bert_classifier}
\end{figure}


Los clasificadores propuestos se basan en \emph{BETO}\cite{canete2020spanish}, una versión en español de \emph{BERT} \cite{devlin2018bert}. \emph{BETO} tiene un tamaño similar a \emph {BERT Base}, y tiene 12 capas de transformadores con 12 cabezas de atención cada una, sumando 110 millones de parámetros. Para obtener más referencias sobre arquitecturas BERT y Transformer, sugerimos CITA NECESARIA.

Para la tarea de detección del discurso de odio, presentamos versiones sin contexto y conscientes del contexto, utilizando el título y el cuerpo completo como posibles contextos. Usamos el token especial BERT \emph {[SEP]} para codificar el contexto y el texto analizado. Recuerde que el token \emph {[SEP]} se usa para la tarea de predicción de la siguiente oración (tarea NSP) en el preentrenamiento al estilo BERT.

En cuanto a la detección de características ofendidas, consideramos este problema como una tarea de clasificación multibinaria; es decir, dado un texto odioso y una característica protegida, lo consideramos como una tarea de clasificación binaria para predecir si el texto ofende la característica respectiva. En lugar de entrenar un clasificador diferente para cada característica, entrenamos un BERT de múltiples salidas, compartiendo todos sus pesos con la excepción de 9 capas lineales diferentes para cada salida. La pérdida utilizada es

\begin{equation*}
    J = \sum\limits_{c \in CHAR \cup \{CALLS\}} J_c
\end{equation*}

donde $CHAR$ es el conjunto de todas las características protegidas (MUJERES, LGBTI, RACISMO, CLASE, etc.) y $CALLS$ es `` llamadas a la acción ''. $ J_c $, ya que cualquier $ c $ es una pérdida de entropía cruzada binaria.

Para tener costos computacionales más amigables, limitamos nuestras secuencias a 128, 256 y 512 tokens para el modelo no contextualizado, el modelo de título y el modelo de título y cuerpo, respectivamente.


\section{Resultados}


\begin{table*}[ht!]
    \centering
    \begin{tabular}{lllll}
        \toprule
        Model &          Precision &             Recall &                 F1 &           Macro F1 \\
        \midrule
        BERT No Context &  $0.682 \pm 0.020$ &  $0.593 \pm 0.018$ &  $0.634 \pm 0.006$ &  $0.785 \pm 0.003$ \\
        BERT Title      &  $0.751 \pm 0.012$ &  $0.603 \pm 0.011$ &  $0.669 \pm 0.007$ &  $0.807 \pm 0.004$ \\
        BERT Title+Body &  $0.738 \pm 0.008$ &  $0.616 \pm 0.005$ &  $0.671 \pm 0.004$ &  $0.808 \pm 0.002$ \\
        \bottomrule
    \end{tabular}


    \caption{Task A: Hate speech detection classification results. Each row contains the performance of the models in complexity order: from the model having no context to the model having full context (title and body). }
    \label{tab:task_a_results}
\end{table*}


La tabla \ref{tab:task_a_results} contiene los resultados de la clasificación, medidos por precisión, recuperación, F1 y Macro F1 (como se acostumbra en algunas tomas compartidas de Detección de discurso de odio y lenguaje ofensivo). Podemos observar que agregar contextos parece mejorar el desempeño; en particular, el simple hecho de agregar el título parece proporcionar suficiente contexto para la tarea de detección del discurso de odio. Agregar el cuerpo mejora marginalmente el rendimiento, pero a un costo computacional más alto (recuerde que la longitud máxima con título se establece en 256, y en el otro caso es 512). La mejora en la puntuación de F1 con solo agregar el título es de aproximadamente 3 puntos; título y cuerpo suma alrededor de 3,5 puntos F1 sobre la clasificación no contextualizada. Al analizar los modelos contextuales, el cuerpo completo parece mejorar el recuerdo al tiempo que disminuye ligeramente la precisión, con una puntuación general igual de F1 que el modelo de solo título.

\begin{figure*}[t]
    \centering
    \includegraphics[width=\textwidth]{img/recall_category.png}
    \caption{Recall by characteristic for each model}
    \label{fig:recall_by_characteristic}
\end{figure*}

La figura \ref{fig:recall_by_characteristic} muestra la recuperación de nuestros algoritmos. Usar solo el contexto del título mejora la sensibilidad para las características LGBTI y CRIMINALES (Mann Whitney U, $ p \ leq 0.01 $, Bonferroni corregido); agregar el cuerpo completo también lo mejora significativamente para la característica RACISMO y lo mejora un poco más para CRIMINAL (Mann Whitney U $ p \ leq 0.01 $, Bonferroni corregido). Sin embargo, el desempeño en POLÍTICA empeora al agregar contexto en cualquier forma ($ p \ leq 0.001 $, corregido Bonferroni). Para las otras características, el contexto no mejora ni empeora el desempeño.


% \begin{table}
%     \centering
%     \begin{tabular}{ll}
%         \toprule
%         Model &            Mean F1 \\
%         \midrule
%         BERT No Context &  $0.731 \pm 0.004$ \\
%         BERT Title      &  $0.808 \pm 0.006$ \\
%         BERT Title+Body &  $0.824 \pm 0.006$ \\
%         \bottomrule
%     \end{tabular}
%     \caption{Mean F1 scores for Task B: Offended characteristic detection}
%     \label{tab:task_b_results}
% \end{table}

La figura \ref{fig:task_b_results} muestra los resultados de la característica ofendida y la tarea de detección de llamada a la acción. La tabla \ref{tab:task_b_results} muestra las puntuaciones F1 medias para tener una medida de resumen. Como se esperaba, la ganancia de tener contexto disponible es más evidente en este punto, con una diferencia media de puntuación F1 de $ 7,6 $ puntos entre no tener contexto y tener el título, y $ 1,5 $ puntos adicionales si el cuerpo está disponible. La detección de llamadas a la acción apenas se mejora al tener un contexto disponible. Nuevamente, las características LGBTI y CRIMINALES se benefician enormemente del contexto; La característica CLASS también parece tener alguna mejora. No hay grandes diferencias entre tener el título y el cuerpo completo.

\begin{figure*}[t]
    \centering
    \includegraphics[width=\textwidth]{img/task_b_scores.png}
    \caption{F1 scores for each characteristic in Task B: offended characteristic detection}
    \label{fig:task_b_results}
\end{figure*}

\begin{table*}
    \centering
    \begin{tabular}{llll}
        \toprule
        {} &    BERT No Context &         BERT Title &    BERT Title+Body \\
        \midrule
        Calls F1       &  $0.784 \pm 0.009$ &  $0.802 \pm 0.010$ &  $0.805 \pm 0.017$ \\
        Women F1       &  $0.652 \pm 0.011$ &  $0.672 \pm 0.015$ &  $0.708 \pm 0.017$ \\
        Lgbti F1       &  $0.590 \pm 0.018$ &  $0.843 \pm 0.021$ &  $0.857 \pm 0.012$ \\
        Racism F1      &  $0.863 \pm 0.005$ &  $0.943 \pm 0.004$ &  $0.939 \pm 0.007$ \\
        Class F1       &  $0.593 \pm 0.011$ &  $0.727 \pm 0.012$ &  $0.742 \pm 0.014$ \\
        Politics F1    &  $0.718 \pm 0.015$ &  $0.753 \pm 0.007$ &  $0.771 \pm 0.012$ \\
        Disabled F1    &  $0.786 \pm 0.016$ &  $0.750 \pm 0.029$ &  $0.791 \pm 0.015$ \\
        Appearance F1  &  $0.845 \pm 0.004$ &  $0.879 \pm 0.010$ &  $0.892 \pm 0.010$ \\
        Criminal F1    &  $0.744 \pm 0.009$ &  $0.901 \pm 0.008$ &  $0.907 \pm 0.008$ \\
        \hline
        Mean F1        &  $0.731 \pm 0.004$ &  $0.808 \pm 0.006$ &  $0.824 \pm 0.006$ \\
        Mean Precision &  $0.786 \pm 0.004$ &  $0.853 \pm 0.007$ &  $0.851 \pm 0.006$ \\
        Mean Recall    &  $0.687 \pm 0.006$ &  $0.770 \pm 0.008$ &  $0.800 \pm 0.007$ \\
        \bottomrule
    \end{tabular}
    \caption{Performance of models for Task B: offended characteristic detection. Mean and standard deviation for 15 runs are displayed. }
    \label{tab:task_b_results}
\end{table*}


\section{Análisis de error}

\section{Discusión}

Para esta tarea en particular, podemos observar que el contexto parece dar una mejora moderada en la tarea de detección del discurso de odio (alrededor de 3 puntos F1), ninguna mejora significativa en la detección de llamadas a la acción, y una mejora considerable en la tarea característica ofendida (alrededor de 7 puntos F1 medios).

Si bien este resultado podría estar en aparente contradicción con un trabajo reciente que no encontró ninguna mejora mediante el contexto en la detección de toxicidad \cite{pavlopoulos2020toxicity}, se puede señalar que la detección del discurso de odio es una de las formas más complejas de ``tóxico'' comportamiento y, como tal, podría permitir que los clasificadores tengan más información para predecir si el texto dado es odioso o no. Otra razón detrás de este resultado es el dominio de nuestro conjunto de datos: mientras que \citet{pavlopoulos2020toxicity} usa el contexto conversacional, nosotros usamos el título y el cuerpo del artículo como contexto para los comentarios de los usuarios.

Este trabajo tiene algunas limitaciones. Primero, una consideración práctica es que no siempre tenemos un contexto disponible para un texto dado. Incluso si podemos encontrarlo, a veces este contexto puede no ser en forma de artículo de noticias, sino como un hilo de conversación o incluso de alguna otra forma (¿una base de conocimiento?).

\section{Limitaciones y trabajo futuro}

\chapter{Adaptación de dominio}

En este capítulo exploraremos como mejorar la detección de discurso de odio desde una perspectiva más general, analizando en general la clasificación de textos sociales. Hemos visto en capítulos anteriores que las técnicas de representación utilizadas en los últimos años (desde los word-embeddings hasta ) generan representaciones ricas al ser entrenadas en dominios sociales. Así mismo, también observamos que en algunos modelos pre-entrenados (como AWD-LSTM usando la técnica de ULMFit) \todo{meter citas} realizar un ajuste de dominio más extenso sobre

Entrenar modelos de lenguaje basados en Transformers toman una cantidad de recursos importantes, algo que puede imposibilitar que

Abordaremos la pregunta ¿cómo se compara en el dominio social el ajuste de dominio de modelos pre-entrenados sobre textos formales con los modelos que fueron generados desde cero en textos sociales? Para ello, utilizamos como benchmark de este análisis las tareas que hemos tratado en esta tesis. Entrenamos desde cero un modelo de lenguaje basado en transformers (RoBERTa)\cite{liu2019roberta} sobre tweets, y comparamos su performance contra ajustes de dominio hechos sobre otros modelos pre-entrenados.

Comenzamos este capítulo haciendo una pequeña recapitulación de las técnicas de adaptación de dominio.

\section{Adaptación de dominio}

\citet{goodfellow2016deep} definen la adaptación de dominio como una situación similar a la de Transfer Learning: dado un modelo que fue entrenado sobre una distribución de datos o dominio $P_1$, lo utilizamos sobre una distribución $P_2$ relativamente similar.

En el caso de la adaptación de dominio, nos referimos a la aplicación de alguna técnica que ajuste la distribución de la entrada (de $P_1$) a la distribución de nuestro nuevo dominio. \citet{glorot2011domain} es uno de los primeros trabajos que aplica esta técnica en NLP, usando denoising auto-encoders para este fin.

Para lo que nos concierne en NLP solemos querer, dado un modelo de lenguaje (tanto causal como enmascarado) entrenado en un dominio, ajustarlo a otro dominio distinto. Por ejemplo, un modelo BERT pre-entrenado en textos formales (como Wikipedia o noticias) queremos ajustarlo a la distribución de textos sociales, que si bien ambas mantienen el idioma (inglés o español) suelen tener distribuciones notoriamente distintas.

Dentro de la última ola que sacudió NLP de modelos pre-entrenados, ULMFit \citet{howard-ruder-2018-universal} contempla una etapa de adaptación de dominio utilizando de manera no-supervisada el texto del dataset supervisado de la tarea atacada.

Recientemente, \citet{gururangan-etal-2020-dont} analizan el impacto de los ajustes de dominio. Para ello, consideran varios dominios como ser biomédico, reviews de películas, papers de cs. de la computación (CS), y noticias. Plantean dos configuraciones de adaptación de dominio:

\begin{itemize}
    \item Domain Adaptation: ajustar el modelo de lenguaje sobre un extenso conjunto de datos no etiquetado, usualmente el ``sobrante'' del proceso de recolección que no es anotado
    \item Task Adaptation: ajustar el modelo de lenguaje sobre el dataset, de la misma manera que se hace en ULM-Fit
\end{itemize}

En todos los casos, usando modelos del estado del arte como RoBERTa, que aplicar conjuntamente lo que ellos consideran Domain Adaptation y Task Adaptation mejora la performance significativamente


Algo que queda pendiente de este trabajo es analizar el dominio social, y por otro lado, hacer una comparación de los ajustes de dominios contra entrenar el modelo desde cero en dicho dominio


\chapter{Conclusiones}


\appendix

% Manual de anotación
\chapter{Manual de criterios de anotación}
\label{app:manual_criterios_anotacion}
\section{Presencia de lenguaje discriminatorio}

Entendemos que hay discurso discriminatorio en el tweet si contiene declaraciones de carácter intenso y posiblemente irracional de rechazo, enemistad y aborrecimiento contra un individuo o contra un grupo, siendo estos objetivos de estas expresiones por poseer (o aparentar poseer) una característica protegida.

Este discurso puede manifestarse de manera explícita (insultos directos), celebraciones sobre asesinatos u otros crímenes, o bien otras expresiones más veladas. Lo que queremos captar es la intención del autor del tweet. El carácter discriminatorio de un mensaje está dado tanto por el contexto (en este caso, el tweet original del medio periodístico y posiblemente la nota) y el contenido del tweet en sí mismo. Por ejemplo, un comentario que diga “excelente” sin contexto es una cosa, y decir eso mismo en una nota que relata un femicidio, o un asesinato es otra muy distinta.

Las características protegidas que vamos a tener en cuenta son las siguientes:

\begin{enumerate}
\item Sexo (Mujeres, concretamente)
\item Género o identidad sexual (Colectivo LGBTI)
\item Ser inmigrantes, extranjeros, pueblos aborígenes u otras nacionalidades (Xenofobia, racismo)
\item Situación socioeconómica o por barrio de residencia
\item Poseer discapacidades, problemas salud mental o de adicción al alcohol u otros estupefacientes
\item Opinión o ideología política
\item Aspecto o edad (mayormente, gordofobia/gerontofobia)
\item Antecedentes penales o estar privado de la libertad

\end{enumerate}

Es decir, para considerar un mensaje como discriminatorio, debe cumplir que el discurso discriminatorio está orientado hacia un individuo o grupo de al menos una (aunque posiblemente más de una) característica protegida.


Consideramos que el mensaje del tweet (a la vez que el receptor del odio) es el que determina si puede o no ser considerado discriminatorio y hacia qué grupo está dirigido. Esto puede no necesariamente coincidir con el destinatario explícito del mensaje: por ejemplo, si alguien le dice a Susana Giménez “judía sionista hdp”, a pesar de no ser Susana Giménez judía, se puede considerar esto como discurso de odio contra las minorías religiosas y/o discurso xenófobo.


\section{Llamado a la acción}

Entendemos que un tweet (que contiene discurso discriminatorio) llama a la acción si contiene alguna incitación a tomar algún tipo de medida contra el sujeto o grupo ofendido. Esta medida puede ser de carácter violento (“hay que matarlos ya” “pongámosles una bomba”) o de carácter menos violento (“hay que dejar de comprarles a estos chinos ladrones”)

Estos tweets nos interesan particularmente porque son los más peligrosos y dañinos: los que llaman a tomar algún tipo de represalia contra la persona o el grupo en cuestión.

\section{Características protegidas}

Finalmente, para cada tweet deberemos marcar qué grupo o característica protegida es atacado. En este caso, necesariamente un grupo/característica debe ser seleccionado:

Usaremos una notación abreviada en la interfaz de etiquetado, en la que algunos de los grupos o características mencionadas fueron reagrupadas de la siguiente manera:

\begin{enumerate}
    \item MUJER: por su sexo
    \item LGBTI: por género o identidad sexual
    \item RACISMO: Por ser inmigrantes, extranjeros, pueblos aborígenes u otras nacionalidades (Xenofobia, racismo)
    \item POBREZA: Por situación socioeconómica o por barrio de residencia.
    \item DISCAPAC: Por tener discapacidades, problemas salud mental o de adicciones
    \item POLITICA: Por su opinión o ideología política
    \item ASPECTO: Por su aspecto o edad
    \item CRIMINAL: Por sus antecedentes o situación penal (presos)
\end{enumerate}

A su vez, agregamos la categoría “OTROS”. Esta categoría es excepcional, y debería utilizarse sólo si algún tipo de discriminación no está contemplado en estas categorías.

Respecto a la discriminación de carácter político tiene que ser algo más que una mera opinión  sino tener una componente irracional, de descalificación y de aborrecimiento considerable sobre un individuo o una facción política.

No se contempla dentro de las categorías protegidas a las profesiones. Es decir, no tenemos en cuenta el discurso contra científicos, médicos, o periodistas; de esto último hay bastante material agresivo en los comentarios.


\subsection{Lineamientos generales}

El discurso discriminatorio no es sólo discurso ofensivo contra una persona o grupo con alguna de las características protegidas. Tiene que apelar a su condición de mujer, inmigrante, LGBTI, etc. para que lo consideremos así.

Por ejemplo: si alguien agrede a una mujer, a un inmigrante, o a alguien de la comunidad LGBTI, no necesariamente está incurriendo en un discurso discriminatorio salvo que apele a algo que remita a su característica como tal.


Expresiones de aprobación ante noticias de crímenes o acciones contra persona o grupo de las características protegidas son consideradas discriminatorias.

\subsubsection{Ejemplos}

Violan a la reconocida actriz XXXX YYYY
Asesinan a un comerciante chino por creer que tenía Coronavirus
Motín y muerte en la prisión de Marcos Paz

Comentarios de contenido discriminatorio: (emoji de aplausos) - uno menos - bravo! -


Si no queda claro que haya un mensaje discriminatorio o parece de carácter difuso o demasiado tangencial, entonces etiquetar como no discriminatorio




\subsection{MUJER}

Insultar a una mujer sin hacer ninguna referencia particular a su condición de mujer no es suficiente para ser considerado discurso discriminatorio

Como regla: si el mismo insulto o agresión aplicase contra un hombre, entonces no debiéramos considerarlo como discriminación


Insultos contra las expresiones políticas del movimiento de las mujeres son consideradas en esta categoría: si se las insulta como feminazis, aborteras, pañuelito verde, etc



Apelaciones a su apariencia o aspecto propias de una mujer son consideradas en esta categoría. En este punto consideramos comentarios cosificadores


Insultar como “vieja” a una persona no califica como misoginia. Usar para ese caso la categoría ASPECTO que contempla la gerontofobia

EJEMPLOS:

Nati Jota furiosa por los comentarios que recibe en las redes.

Comentarios sexistas:

Pero si sos de plástico nena! (opina de manera denigratoria de su apariencia)
Flor de gato!
Miauuu!
A esta sólo se la conoce por su cuerpo y ahora se hace la santa. Andá a estudiar
Le damos hasta que San Lorenzo vuelva a Boedo
Y esta rubia tarada quién es?



Comentarios ofensivos pero no sexistas:
Callate forra (ofensivo pero no particularmente sexista)
Y esta quién es? A quién le importa?
Quién?
Nati cuánto?
Esta también recibe sobres?
Otra descerebrada más (súper agresivo, pero es un comentario que podría recibir un hombre, aunque en menor cantidad…)
Andá a cagar..boluda..conductora de 4ta..
HDP
Boba



\subsection{LGBTI}

Expresiones de disgusto o rechazo del estilo “qué asco” frente a integrantes de la colectividad deben ser marcados como discurso discriminatorio


Apelaciones a la anatomía o del estilo “biologicistas” deben ser marcados como discurso discriminatorio


Ejemplos:

Anibal Pachano sobre la cuarentena: “este virus nos está destruyendo a los actores”


Comentarios discriminatorios:

Qué asco este sujeto
Y a este trolo quién le pidió su opinión?
Me desagrada


Comentarios ofensivos pero no discriminatorios:

Ajá, y?
No me importa tu opinión, gracias
Se acuerdan cuando este atacaba a Cristina? Andá a llorarle al gato, Aníbal
Pelotudazo!


Famosa actriz trans dio una entrevista a La NACION: “Como mujer, estoy en la mejor etapa de mi vida”


Comentarios discriminatorios:

Muy lindo todo, pero sos hombre
“Mujer” dice jajajaj naciste y morirás macho
Qué asco este trolo
(emoji de asco)
Una mujer con sorpresa
Trava horrendo
Perdón pero qué nombre dice tu DNI?


Comentarios ofensivos pero no discriminatorios:

Quién?
Y a quién le interesa lo que diga esta mujer?
Que se vaya a cagar
Agradecele al gobierno de Néstor que tenés derechos





\subsection{RACISMO}

Asignar a todo un pueblo con una característica o actividad particular negativa entra en esta categoría
Cualquier llamada a expulsar, matar, tirar bombas, destruir, aniquilar, etc a un pueblo, inmigrantes o país entra en esta categoría. Más aún, deben ser marcadas como “llamadas a la acción”
Teorías conspirativas (hasta cierto punto) o acusaciones a un Gobierno/Estado no deberían entrar en esta categoría
¿Acusaciones respecto al virus?



Ejemplos:

Rescatan en China a cuarenta gatitos bebé que iban a ser utilizados en restaurantes


Comentarios discriminatorios:

Chinos asquerosos
Malparidos! Chinos de mierda
Sigan desparramando pestes hijos de puta!
Por qué no se comen entre ellos?
País horrible y enfermo
Estos chinos nos diseminan su peste por todo el mundo

Comentarios discriminatorios y que además llaman a la acción:

Son una plaga estos chinos, hay que matarlos a todos
Ojalá les caigan veinte bombas
Hay que barrer ese país de la faz de la tierra

Comentarios ofensivos pero no racistas:

Son comunistas, qué querés
Lo rescataron a Mauri?




Otros comentarios no racistas

El gobierno chino está intentando controlar todo, cuidado
Mienten con que controlaron el virus, cómo no van a mentir con esto también?
De terror!


Tomas de terrenos: se profundizan las tomas en todo Buenos Aires


Comentarios discriminatorios:

Rajen a todos los peruanos de las tomas! (llama a la acción, además)
Increíble, cualquier paraguayo o bolita de mierda te saca la casa

\subsection{POBREZA}

Expresiones denigrantes sobre la condición socioeconómica son consideradas discriminatorias: referirse a las personas sin empleo como “vagos”, a otros como “(chori)planeros”
Respecto a la gente que posee planes de asistencia social, si es una opinión general sin ser ofensiva no se considera discriminatoria (del estilo “hay que reducir el gasto social y los planes”, “perdieron la cultura del trabajo”)

Presupuesto: aumentó el gasto en planes asistenciales durante la pandemia

Ejemplos discriminatorios:

Basta de mantener vagos!
Cansada de los planeros
Che laburar estos atorrantes ni en pedo no?
PARASITOS

Ejemplos no discriminatorios

La gente que trabaja y aporta impuestos es cada vez menos. Estamos al horno



\subsection{POLITICA}

Apreciaciones derogatorias sobre la posición política son consideradas discriminatorias : zurdo/a, bolchevique, peroncho, gorila, kuka, etc
Acusaciones de corrupción o de “recibir sobres” no son consideradas discriminatorias
Tampoco aquellas expresiones que traten de inútiles a funcionarios
Tratar de viejo/a, gordo/a, u otras cuestiones físicas deben ser marcados en las categorías respectivas, no acá

Aumentó el gasto en planes asistenciales durante la pandemia

Ejemplos discriminatorios:

BASTA ZURDOS DE ROBARNOS
Bolcheviques de mierda

Ejemplos no discriminatorios

La gente que trabaja y aporta impuestos es cada vez menos. Estamos al horno
Qué gobierno de inútiles
Son unos delincuentes
Hijos de mil puta!
Siguen volando los sobres para el Congreso
Siga siga la impresión




\subsection{ASPECTO}

Apreciaciones denigrantes sobre la apariencia de una persona y/o su edad
Principalmente, tenemos en mente la gordofobia y gerontofobia, pero puede referir a otras características físicas (por ejemplo, la altura)
En casos en las cuales haya solapamiento con mujer, marcar ambas


Luis Brandoni: “No convoqué el banderazo”

Ejemplos discriminatorios:

Viejo de mierda!
Qué decrépito impresentable que es este señor
Estás gagá, pelotudo

Jorge Lanata vuelve a la televisión

Ejemplos discriminatorios:

Gordo chanta otra vez volvés a vender pescado podrido?
porque no te vas vos tambien con todos bola de sebo!!!1
Estás gagá, pelotudo









\subsection{CRIMINAL}

Cualquier comentario que celebre acciones contra criminales o personas privadas de su libertad (golpizas, asesinatos, muerte en motines, etc) entra en esta categoría
En este ítem muchas veces veremos que son llamados a la acción: el de “matarlos”, llamar a reducir sus derechos, etc
Muere un delincuente tras un enfrentamiento con la policía

Ejemplos discriminatorios:

Uno menos!
Excelente!
(emoji de aplausos)
Que pena, pobrecito

Ejemplos que además llaman a la acción

MUY BIEN! Felicitaciones al policía, hay que liquidarlos sin piedad


1...100 101..200 …. 501...600

Primera fase: Et 1 => 1..100, Et 2 => 101 .. 200 … Et 6 => 501..600
Segunda fase Et2 => 1..100 Et1 => 101..200…. Et 6 => 401..500 Et 5 => 501..600

Shufflear temporalmente

Ejemplos no discriminatorios


Cómo puede ser que nuestra Ministra no haga nada?
La policía actuó correctamente.

Motín por el Coronavirus en Olmos: 3 muertos

Ejemplos discriminatorios y que llaman a la acción

Hay que rociar con nafta todas las cárceles
Soltemos 3 o 4 infectados con COVID en cada cárcel y problema solucionado
Paredón y listo



\subsection{DISCAPACIDAD}

Referencias peyorativas de adicciones a drogas, alcohol u otros estupefacientes
También referencias peyorativas a la salud mental de la persona en cuestión
Decir “está loco” no entra acá :-)
Malena Pichot sale a cruzar a Baby Etchecopar
Ejemplos discriminatorios:

Callate faloperita!
Jorge Lanata vuelve a la televisión

Ejemplos discriminatorios:

Che no probaste dejar la merca gordo?

Noticia sobre Patricia Bullrich...

Ejemplos discriminatorios:

Largá la (emoji de botella) Pato
Borracha hdp



\subsection{OTROS}

Esta categoría está reservada para cualquier otro tipo de discriminación que no esté contemplada en las categorías mencionadas
Insultos a profesiones (científicos, periodistas, por ejemplo) no entran en este apartado
ESTA CATEGORIA ES SUMAMENTE EXCEPCIONAL. NO USAR INDISCRIMINADAMENTE

% Capítulo 7 - Tablas de resultados de ajuste de dominio
\input{src/app_domain_adaptation_results.tex}





%%%% BIBLIOGRAFIA
\backmatter
\bibliography{biblio.bib}

\end{document}
