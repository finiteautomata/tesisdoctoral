 \documentclass[11pt,a4paper,twoside]{tesis}
% SI NO PENSAS IMPRIMIRLO EN FORMATO LIBRO PODES USAR
%\documentclass[11pt,a4paper]{tesis}
\usepackage[spanish]{babel}
\usepackage{emoji}
\usepackage{graphicx}
\usepackage[utf8]{inputenc}
\usepackage[left=3cm,right=3cm,bottom=3.5cm,top=3.5cm]{geometry}
\usepackage[square,sort,comma,numbers]{natbib}
\usepackage[hyphens]{url}
\PassOptionsToPackage{hyphens}{url}
\usepackage{hyperref}
\usepackage{booktabs}
\usepackage{csquotes}
\usepackage{multirow}
\usepackage{amsmath}
\usepackage{mathtools}
\usepackage{amsfonts}
\usepackage{tabularx}
\usepackage{changepage}
\usepackage{setspace}
\usepackage{rotating}
\usepackage{caption}
\usepackage{subcaption}
\usepackage{icomma}
\usepackage{siunitx}
\usepackage{makecell}
\usepackage{array}
\usepackage{xcolor}
\usepackage{soul}

\sisetup{group-separator = {.}}
\sisetup{group-minimum-digits=4}
\sisetup{output-decimal-marker = {,}}

%\newcommand{\todo}[1]{\marginpar[]{#1}}
\newcolumntype{P}[1]{>{\centering\arraybackslash}p{#1}}

\setlength{\marginparwidth}{2cm}

\bibliographystyle{plainnat}

\begin{document}

%%%% CARATULA

\def\autor{Juan Manuel Pérez}
\def\tituloTesis{Técnicas y recursos para la detección automática de lenguaje discriminatorio en redes sociales}
\def\runtitulo{Técnicas y recursos para la detección automática de lenguaje discriminatorio en redes sociales}
\def\runtitle{Techniques and resources for the automatic detection of hate speech in social networks}
\def\director{Franco Luque}
\def\codirector{Agustín Gravano}
\def\lugarTrabajo{\makecell{
Departamento de Computación \\
Facultad de Cs. Exactas y Naturales \\
Universidad de Buenos Aires}}

\def\lugar{Buenos Aires, 2021}

\newcommand{\HRule}{\rule{\linewidth}{0.2mm}}
%
\thispagestyle{empty}

\begin{center}\leavevmode

\vspace{-2cm}

\begin{tabular}{l}
\includegraphics[width=2.6cm]{img/logofcen.png}
\end{tabular}


{\large \sc Universidad de Buenos Aires

Facultad de Ciencias Exactas y Naturales

Departamento de Computaci\'on}

\vspace{6.0cm}

%\vspace{3.0cm}
%{
%\Large \color{red}
%\begin{tabular}{|p{2cm}cp{2cm}|}
%\hline
%& Pre-Final Version: \today &\\
%\hline
%\end{tabular}
%}
%\vspace{2.5cm}

\begin{huge}
\textbf{\tituloTesis}
\end{huge}

\vspace{2cm}

{\large Tesis presentada para optar al título de Doctor de la Universidad de Buenos Aires en
el área de Ciencias de la Computaci\'on}

\vspace{2cm}

{\Large \autor}

\end{center}

\vfill

{\large

\begin{tabular}{l l}
    \vspace{.3cm} Director:               & Dr. Franco Luque \\
    \vspace{.3cm} Director Asistente:     & Dr. Agustín Gravano \\
    \vspace{.3cm} Consejero de Estudios:  & Dr. Diego Fernández Slezak \\
    Lugar de Trabajo:          & Departamento de Computación \\
                               & Facultad de Cs. Exactas y Naturales \\
    \vspace{1cm}               & Universidad de Buenos Aires \\
    Buenos Aires, 2021         & \\
\end{tabular}

\vspace{.2cm}



\vspace{.2cm}


}

\newpage\thispagestyle{empty}

%
%%%% ABSTRACTS, AGRADECIMIENTOS Y DEDICATORIA
\frontmatter
\pagestyle{empty}
%\begin{center}
%\large \bf \runtitulo
%\end{center}
%\vspace{1cm}
\chapter*{\runtitulo}

\noindent El discurso discriminatorio (también conocido como discurso de odio) puede describirse como aquel discurso en clave de intenso aborrecimiento, denigración y enemistad que ataca a un individuo o un grupo de individuos por poseer –o aparentar poseer– cierta característica protegida por tratados internacionales como el sexo, el género, la etnia, etc. En los últimos años, este tipo de discurso ha tomado gran relevancia en redes sociales y otros medios virtuales debido a su intensidad y a su relación con actos violentos contra miembros de estos grupos. A raíz de esto, estados y organizaciones supranacionales como la Unión Europea han sancionado legislación que insta a las empresas de redes sociales a moderar y eliminar contenido discriminatorio, con particular foco en aquel que insta a la violencia física.

Debido a la enorme cantidad de contenido generado por usuarios en las redes sociales, es necesario contar con cierta automatización en esta tarea, bien para su análisis o para su moderación. Desde la óptica del procesamiento de lenguaje natural, la detección de discriminación puede entenderse como un problema de clasificación de texto: dado un texto generado por un usuario, predecir si es o no contenido discriminatorio. Así mismo, puede ser de interés predecir otras características: por ejemplo, si el texto contiene un llamado a la acción violenta, si está dirigido contra un individuo o un grupo, o el tipo de característica ofendida, entre otras.

Una de las limitaciones de los enfoques actuales para la detección del lenguaje discriminatorio es la falta de contexto en el mensaje. La mayoría de los estudios y recursos están hechos sobre datos fuera de contexto; es decir, mensajes aislados sin ningún tipo de contexto conversacional o del tema del cual se habla. Esto restringe la información disponible –tanto para un humano como para un sistema– para poder discernir si un texto social es discriminatorio. Otra información usualmente faltante es la característica atacada: es común que los datasets estén anotados de manera poco granular, no brindando información acerca de si la agresión es por motivos de sexo, género, clase social, etc. Por último, una limitación puntual del español es la poca disponibilidad de recursos para esta tarea.

En esta tesis pretendemos abordar algunas de las limitaciones marcadas. Por un lado, analizamos el impacto de agregar contexto a la detección de lenguaje discriminatorio en redes sociales. Para ello, construimos un conjunt de datos de tweets en base a las respuestas de los usuarios a los posteos de medios periodísticos en Twitter. Esto nos permite obtener dos tipos de contextos: uno “conversacional” al tener una respuesta a un tweet anterior, y otro más extenso al obtener el texto de la noticia en cuestión. El corpus fue recolectado sobre noticias relacionadas a la pandemia de COVID-19, en idioma español mayormente en su variedad dialectal rioplatense y anotado por hablantes nativos de ese dialecto con un nuevo modelo de etiquetado, que es granular respecto de las características ofendidas.

Sobre los comentarios de este dataset realizamos experimentos de detección de discurso de odio planteando dos tareas: detección “plana” del lenguaje discriminatorio, donde sólo predecimos una etiqueta binaria indicando presencia de lenguaje discriminatorio; y detección “granular”, donde predecimos las características ofendidas. Usando técnicas del estado del arte, obtuvimos mejoras significativas en ambas tareas al agregar contexto como entrada de cada instancia, tanto en su forma corta (sólo el titular/tweet de la noticia) como en su forma larga (titular y cuerpo de la noticia). Así mismo, observamos que un clasificador entrenado para la tarea “granular” mejora levemente su performance al ser evaluado para la tarea “plana”, obviando los posibles errores de motivos discriminatorios. Combinando la adición de contexto y granularidad, un clasificador para la detección de lenguaje discriminatorio obtiene mejoras considerables sobre un BERT en español que sólo consume el texto del comentario.

Considerando la detección de discurso de odio dentro del área más abarcativa de clasificación de documentos en dominios sociales, analizamos también algunos aspectos generales de tareas relacionadas como el análisis de sentimiento y la detección de emociones, entre otras. En particular, analizamos el desempeño de varias técnicas modernas de representación al ser entrenadas en dominios sociales. Comúnmente, los modelos de representación son entrenados a partir de textos de dominios “formales”, como pueden ser Wikipedia u otras fuentes similares. En esta tesis observamos que –desde los word embeddings hasta los modelos pre-entrenados basados en transformers– las representaciones generadas son robustas y mejoran la performance en un conjunto de tareas de clasificación en textos sociales. Sobre los modelos pre-entrenados, estudiamos el impacto de entrenarlos desde cero en textos sociales o efectuar una adaptación a este dominio.

Todos los estudios y recursos presentados en esta tesis fueron realizados en el idioma español. Como un objetivo secundario, pretendemos contribuir a mitigar la enorme asimetría de recursos existente en el área del procesamiento del lenguaje natural.





\bigskip

\noindent\textbf{Palabras claves:} Guerra, Rebelión, Wookie, Jedi, Fuerza, Imperio (no menos de 5).
%
\cleardoublepage
%\begin{center}
%\large \bf \runtitle
%\end{center}
%\vspace{1cm}
\chapter*{\runtitle}

\noindent




{
    \setstretch{1.5}
Hate speech can be described as speech containing intense hatred, denigration, and enmity that attacks an individual or a group of individuals because of possessing –or pretending to possess– any characteristic protected by international treaties such as gender, ethnicity, religion, language, among others. In recent years, this type of discourse has gained great relevance in social networks and other virtual media due to its intensity and its relationship with violent acts against members of these groups. As a result, states and supranational organizations –such as the European Union– have enacted legislation that urges social media companies to moderate and remove discriminatory content, with particular focus on that which promotes physical violence.

Due to the enormous amount of user-generated content on social media, it is necessary to have some degree of automation in this task, either for analysis or for moderation. From a natural language processing (NLP) perspective, hate speech detection can be understood as a text classification problem: given a text generated by a user, predict whether it is discriminatory content. Likewise, it may be of interest to predict other features: for example, if the text contains a call to violent action; if it is directed against an individual or a group; or the offended characteristic, among others.

One of the limitations of current approaches to hate speech detection is the lack of context. Most studies and resources are performed on data without context; that is, isolated messages without any type of conversational context or the topic being discussed. This restricts the information available –both for a human and for an automated system– to discern if a social text is hateful or not. Other information usually lacking is the offended characteristic: datasets are usually annotated with a low level of granularity, failing to provide information about whether the offending message attacks the individual or group due to their gender, social class, race, or whatsoever. Finally, a specific limitation of Spanish is the limited availability of resources for this task.

In this thesis, we intend to address some of the marked limitations. On the one hand, we analyze the impact of adding context to hate speech detection in social networks. To do this, we built a tweet dataset based on user responses to news media posts on Twitter. This provided us two types of contexts: a conversational context, given by the tweet and its answer, and another context given by the text of the news in question. This dataset was collected on news related to the COVID-19 pandemic, in the Spanish language in its Rioplatense dialectal variety. Native speakers of this dialect annotated the comments with a novel labeling model that is granular regarding the offended characteristics.

Using this dataset, we carried out hate speech detection experiments, proposing two tasks: ``binary'' detection of discriminatory language, where we only predict a binary label indicating the presence of discriminatory language; and ``granular'' detection, where we predict the attacked characteristics (n-binary classification tasks at the same time). Using state-of-the-art techniques, we obtained significant improvements in both tasks by adding context as input for each instance, both in its short form (only the headline/tweet of the news article) and in its long-form (headline and body of the news article). We also observed that a classifier trained for the ``granular'' task slightly improves its performance when being evaluated for the ``flat'' task, ignoring possible errors of discriminatory motives. Combining the addition of context and granularity, a classifier for the detection of discriminatory language obtained considerable improvements over a BERT in Spanish that only consumes the text of the comment.

Considering hate speech detection within the most comprehensive area of ​​document classification in social domains, we further explored  some general aspects of related tasks such as sentiment analysis and emotion detection, among others. In particular, we analyzed the performance of various modern representation techniques when trained in social domains. Commonly, NLP researchers train representation models on texts from ``formal'' domains, such as Wikipedia or other similar sources. We observed that –from word embeddings to pre-trained models based on transformers– the representations generated are robust and improve performance in a set of classification tasks in social texts. On the pre-trained models, we studied the impact of training them from scratch in social texts versus performing domain-adaptation on the language models.

All of the studies and resources presented in this thesis were carried out in the Spanish language. As a secondary objective, we aim to mitigate the enormous asymmetry of resources in the area of NLP.
}
\bigskip

\noindent\textbf{Keywords:} Hate Speech, Natural Language Processing, Abusive Language Detection, Domain Adaptation, Social NLP.
%
\cleardoublepage
\chapter*{Agradecimientos}


En primer lugar, quiero agradecer a mis directores Franco y Agustín, que guiaron mi trabajo y me formaron como investigador. Realizar un doctorado es una labor bastante dura, en el cual uno se encuentra muchas veces perdido en el camino. La función que ambos cumplieron guiándome en esos momentos de desorientación ---pero dándome la libertad de elección en cada momento--- ha sido fundamental para llegar hasta acá.

Quiero agradecer a todes mis compañeres del Laboratorio de Inteligencia Artificial Aplicada (LIAA) y del Departamento de Computación de Exactas UBA quienes me ayudaron a transitar este doctorado, compartiendo conocimiento, charlas --- a veces simplemente catarsis. Edgar Altszyler, Pablo Brusco, Ramiro Gálvez, Bruno Bianchi, Damián Furman, Lara Gauder, Jazmín Vidal, y todos los que me falten en esta lista. A Viviana Cotik, que me ayudó de gran manera en los momentos más críticos de este trabajo.

A todes les integrantes del Proyecto Interdisciplinario de la UBA sobre marginaciones sociales (PIUBAMAS), que fueron fundamentales en los segmentos más importantes de esta tesis.

A mis compañeres de activismo y militancia, particularmente a los compañeros de la Asociación Gremial de Docentes de la UBA (AGD-UBA) y Jóvenes Científicxs Precarizados (JCP). Luchar por nuestros derechos y reconocimiento como trabajadores ha sido sin dudas parte de mi formación.

A mis amigos que me vieron poco estos años. A Víctor, Pablo, Tamara, Silvina, Andrés, Nico, Chudi, Tomás, Pigre, Joe. A Mariela Rajngewerc, con quien atravesamos paralelamente las dificultades de la academia. A Nina Pardal, con quien compartimos caminatas y charlas en Exactas.

A mi familia. A mi hermano Fer, a Graciela, a Julio. A mis primos Nico, Héctor y Meli. A mis viejos, dondequiera que estén, por impulsar mi curiosidad desde pequeño y siempre apoyarme en el estudio. Cada uno, a su manera, me fue llevando por este camino.

Finalmente quiero agradecer a Valeria, mi compañera de vida, que me apoyó en todo momento y soportó el estado de desborde emocional permanente que atraviesa todo doctorando. Realmente hubiera sido imposible sin vos.

 % OPCIONAL: comentar si no se quiere
%
\cleardoublepage
\hfill

\textit{A Valeria}

\textit{A mis viejos}

\textit{A quienes luchan cada día por hacer este mundo más justo}
  % OPCIONAL: comentar si no se quiere
%
\cleardoublepage
\tableofcontents
%
\mainmatter
\pagestyle{headings}
%
%%%% ACA VA EL CONTENIDO DE LA TESIS
%
%

\newcommand{\tbf}[1]{\textbf{#1}}
\newcommand{\mbf}[1]{\mathbf{#1}}

\newcommand{\bert}[0]{\emph{BERT}}
\newcommand{\ulmfit}[0]{\emph{ULM-FIT}}
\newcommand{\beto}[0]{\emph{BETO}}
\newcommand{\roberta}[0]{\emph{RoBERTa}}
\newcommand{\robertuito}[0]{\emph{RoBERTuito}}
\newcommand{\bertweet}[0]{\emph{BERTweet}}
\newcommand{\mbert}[0]{mBERT}
\newcommand{\xlm}[0]{XLM-R}
\newcommand{\xlmbase}[0]{XLM-R$_{BASE}$}
\newcommand{\xlmlarge}[0]{XLM-R$_{LARGE}$}


\newcommand{\twitter}[0]{\emph{Twitter}}

\newcommand{\elmo}[0]{ELMo}
\newcommand{\hateval}[0]{\emph{HatEval}}
\newcommand{\softmax}[0]{\text{softmax}}
\newcommand{\semeval}[0]{SemEval-2019}


\newcommand{\fasttext}[0]{\emph{fastText}}

\newcommand{\mr}[2]{\multirow{#1}{*}{#2}}
\newcommand{\mc}[2]{\multicolumn{#1}{c}{#2}}
\newcommand{\thline}[1]{\Xhline{#1\arrayrulewidth}}


\newcommand{\tred}[1]{\textcolor{red}{#1}}


\part{Introducción}
\chapter{¿Por qué interesa la detección automática de discurso de odio?}

\section{Discursos de odio}


El discurso discriminatorio  puede describirse como aquel discurso en clave de intenso aborrecimiento, denigración y enemistad que ataca a un individuo o un grupo de individuos por poseer --o aparentar poseer-- cierta característica protegida por tratados internacionales como el sexo, el género, la etnia, etc. Si bien no hay un consenso generalizado sobre qué configura exactamente lenguaje de odio o discriminación\cite{article192015}, un posible punto de contacto entre las distintas definiciones apuntan hacia


En los últimos años, este tipo de discurso ha tomado gran relevancia en redes sociales y otros medios virtuales debido a su intensidad y a su relación con actos violentos contra miembros de estos grupos.

A raíz de esto, estados y organizaciones supranacionales como la Unión Europea han sancionado legislación que insta a las empresas de redes sociales a moderar y eliminar contenido discriminatorio, con particular foco de aquel que insta a la violencia física.

Debido a la enorme cantidad de contenido generado por usuarios en las redes sociales, es necesario contar con cierta automatización en esta tarea bien para su análisis o para su moderación. Desde el procesamiento de lenguaje natural, la detección de discriminación puede entenderse como una clasificación de texto: dado un texto generado por un usuario, predecir si es o no contenido discriminatorio. Así mismo, puede ser de interés predecir otras características: por ejemplo, si el texto contiene un llamado a la acción violenta, si está dirigido contra un individuo o un grupo, el tipo de característica ofendida, entre otras.

Una de las limitaciones de los enfoques actuales para la detección del lenguaje discriminatorio es la falta de contexto en el mensaje. La mayoría de los estudios y recursos están hechos sobre datos fuera de contexto; es decir, mensajes aislados sin ningún tipo de contexto conversacional o del tema del cual se habla. Esto restringe la información disponible –tanto para un humano como para un sistema– para poder discernir si un texto social es discriminatorio. Otra información usualmente faltante es la característica atacada: es común que los datasets estén anotados de manera poco granular, no brindando información acerca de si la agresión es por motivos de sexo, género, clase social, etc. Por último, una limitación puntual del español es la poca disponibilidad de recursos para esta tarea. Más aún, los datasets suelen estar anotados por anotadores que no son hablantes de las variedades dialectales de los textos utilizados, lo cual genera un déficit en su calidad al ser el lenguaje discriminatorio altamente dependiente de la jerga específica de cada región.

En esta tesis pretendemos abordar algunas de las limitaciones marcadas. Por un lado, analizamos el impacto de agregar contexto a la detección de lenguaje discriminatorio en redes sociales. Para ello, construimos un dataset de tweets en base a las respuestas de los usuarios a los posteos de medios periodísticos en Twitter. Esto nos permite obtener dos tipos de contextos: uno “conversacional” al tener una respuesta a un tweet anterior, y uno más extenso al obtener el texto de la noticia en cuestión. El corpus fue recolectado sobre noticias relacionadas a la pandemia del COVID-19, en idioma español mayormente en su variedad dialectal rioplatense y anotado por hablantes nativos de ese dialecto con un modelo de etiquetado granular respecto a las características ofendidas.

Sobre los comentarios de este dataset realizamos experimentos de detección de discurso de odio planteando dos tareas: detección “plana” del lenguaje discriminatorio, donde sólo predecimos una etiqueta binaria indicando presencia de lenguaje discriminatorio; y detección “granular”, donde predecimos las características ofendidas. Usando técnicas del estado del arte, obtuvimos mejoras significativas en ambas tareas al agregar contexto como entrada de cada instancia, tanto en su forma corta (sólo el titular/tweet de la noticia) como en su forma larga (titular + cuerpo de la noticia). Así mismo, observamos que un clasificador entrenado para la tarea “granular” mejora levemente su performance al ser evaluado para la tarea “plana”, obviando los posibles errores de motivos discriminatorios. Combinando la adición de contexto y granularidad, un clasificador para la detección de lenguaje discriminatorio obtiene mejoras considerables sobre un BERT en español que sólo consume el texto del comentario.

Considerando la detección de discurso de odio dentro del área más abarcativa de clasificación de documentos en dominios sociales, analizamos algunos aspectos generales para tareas relacionadas como el análisis de sentimiento y la detección de emociones, entre otras. En particular, analizamos el desempeño de las técnicas de representación al ser entrenadas en distintos dominios. En general, los modelos de representación son entrenados a partir de textos de dominios “formales”, como pueden ser Wikipedia u otras fuentes similares. En esta tesis analizamos el efecto de generar estas representaciones desde textos informales. Observamos que –desde los word embeddings hasta los modelos pre-entrenados basados en transformers– las representaciones generadas son robustas y mejoran la performance en un conjunto de tareas de clasificación en textos sociales. Sobre los modelos pre-entrenados, estudiamos el impacto de entrenarlos desde cero en textos sociales o efectuar una adaptación sobre este dominio

Todos los estudios y recursos de esta tesis fueron realizados en español. Como un objetivo secundario, pretendemos mitigar la enorme asimetría de recursos existente en el área del procesamiento del lenguaje natural.



\subsection{Atentados en Charlottesville}

En Agosto del 2017, una gran movilización organizada por varios movimientos de ultraderecha y supremacistas blancos tuvo lugar en la ciudad de Charlottesville, Virginia, Estados Unidos, y particularmente centrada en la Universidad de dicho Estado. Esta concentración fue llamada en el medio del intento de universitarios y el movimiento Black Lives Matter (BLM) de remover estatuas de militares conferados pro-esclavitud de la Guerra de Secesión; en el caso de la Universidad de Virginia, sobre la estatua de Robert Lee. Más aún, tuvo lugar durante los primeros meses del mandato de Donald Trump.

Numerosos grupos de ultraderecha, neonazis, neo-confederados, entre otros, convocaron a la marcha ``Unite the Right'', diseñada como una campaña militar y organizada hasta 3 meses antes de su concreción. \citet{blout2020white} describen la experiencia de Charlottesville como la de un ``terrorismo inmersivo'', ya que generaron un ámbito de terror en varios ``teatros'' (como lo llaman los autores, usando jerga militar). Principalmente, el teatro físico, con la marcha y enfrentamientos con las contra-movilizaciones, la intimidante marcha de antorchas, y el asesinato de Heather Heyer atropellada por un manifestante neo-nazi. Así mismo, el teatro ``virtual'', que sirvió para generar un clima de intimidación en la previa, durante, y luego del evento, de alguna manera. \citet{blout2020white} citan particularmente una campaña judeófoba contra el mayor de Charlottesville (de ascendencia judía) y el vicemayor (afroamericano).

En el trabajo anteriormente mencionado, los autores llegan a la conclusión de que el evento fue organizado de manera centralizada, tanto en su planificación como despliegue en un intento de ejercicio militar. También concluyen que, la propaganda y la información propagada por los organizadores sirvió para publicitar y reclutar a simpatizantes y también para aterrorizar a la población. Esta propaganda tuvo lugar tanto en medios impresos (por ejemplo, posters pegados en las calles) como por medios virtuales y redes sociales como Facebook, Twitter o Discord. \citet{klein2019twitter} analiza los intercambios en Twitter entre los dos bandos (manifestantes de ultraderecha y los contramanifestantes) y muestra que, en el caso de quienes se encontraban del lado de la marcha de UtR, se identifica como enemigos a los musulmanes, liberales o izquierdistas,a miembros de la comunidad LGBTQ, judíos, entre otros.


\subsection{Matanza en Sinagoga de Pittsburgh}

En Octubre de 2018, un hombre fuertemente armado entró a la sinagoga ``El Árbol de la Vida'' en Pittsburgh, Pensilvania, Estados Unidos. Luego de gritar ``muerte a los judíos'', abrió fuego contra la multitud matando 11 personas y dejando decenas de heridos. El tirador era usuario activo de Gab\footnote{\url{https://gab.com/}}, una red social que nació en 2016 bajo la égida de la ``libertad de expresión'' con motivo de la creciente moderación de Twitter a discursos discriminatorios. El asesino en cuestión posteaba frecuentemente contenido antisemita en dicha red social \cite{mcilroy2019welcome}, la cual ha sido descrita como el ``Twitter racista''.

A raíz de esto, Gab fue dado de baja durante cierto tiempo al serle negado alojamiento web. Desde entonces, diversos trabajos han estudiado y recopilado el contenido discriminatorio en esta red social \cite{mcilroy2019welcome,kennedy2018gab}.


\begin{figure}[htbp]
    \centering
    \includegraphics[height=6cm, keepaspectratio]{img/gab-pittsburgh-post.jpg}
    \caption{Último post de Robert Bowers, tirador en la masacre de Pittsburgh, en la red social Gab.}
    \label{fig:gab_post}
\end{figure}
















\section{Revolución de NLP en los últimos años}
\section{NLP aplicado a redes sociales}

\section{Aportes de este trabajo}

Agrego algunas referencias para ir teniendo en cuenta:

\begin{itemize}
    \item Manifesto of Computational Social Sciences \cite{conte2012manifesto}
    \item Text Analysis in Python for Social Scientists: Discovery and Exploration \cite{hovy2020text} (Leer intro nada más)
    \item Computational social science and sociology (2020): \cite{edelmann2020computational}
\end{itemize}
%
\chapter{Preliminares}
\section{Redes Neuronales}

\subsection{Multi-layer perceptron}

Una red neuronal puede pensarse simplemente como una función $f: \mathbb{R}^m \rightarrow O$ que intenta aproximar una función $f^*$ con misma aridad. Podremos notar $ y = f(x; \Theta)$, siendo $\Theta$ los parámetros de dicha función

El perceptrón, desarrollado en 1958 por Frank Rosenblatt \cite{rosenblatt1958perceptron}, intenta ajustar

\begin{equation*}
    y = H(\Theta_1 x + \Theta_0)
\end{equation*}

donde $H$ es la función de activación, y $\Theta = (\Theta_0, \Theta_1)$ son los parámetros de la función. Este modelo es el primero cuyos pesos se encontraban mediante un algoritmo, considerando que $H$ es una función derivable. Este puede considerarse como el primer modelo de una neurona, junto al modelo de McCulloch-Pitts.

\citet{minsky1969perceptrons} demostraron que este tipo de modelos sólo pueden ajustarse a datos linealmente separables, provocando el primer ``invierno'' de las redes neuronales.

Una forma de sortear estas dificultades planteadas es ``apilar'' (stack en inglés) varias de estas funciones para poder ajustar a más tipos de funciones. En términos matemáticos, esto es tan sólo una composición de funciones, tomando ahora $f = f_3 \circ f_2 \circ f_1$, donde $f_1$ es la primer ``capa'' de nuestra función correspondiente a la entrada, $f_2$ es la capa intermedia u oculta, y $f_3$ es la capa de salida. Si bien este ejemplo consta de 3 capas, se puede generalizar a arbitrarias capas ocultas. Este modelo es el que conocemos como \textbf{Perceptrón Multicapa} o \textbf{Multi-Layer Perceptron} (MLP por sus siglas en inglés)

\subsection{Redes neuronales recurrentes}

Los problemas descriptos de NLP suelen constar de procesar una secuencia de palabras o tokens $x_1, x_2, \ldots, x_k$ de longitud variable, de manera de ajustar a una función

\begin{equation*}
    y = f([x_1, \ldots, x_k])
\end{equation*}

Una manera de ajustar una función de este tipo (usando una entrada de largo fijo) es convertir este problema a ajustar una función autorregresiva

\begin{equation*}
    y_k = f(x_k, y_{k-1})
\end{equation*}

donde tenemos una salida para cada paso $k$ de tiempo. Si $f$ es una red neuronal, llamamos a este tipo de redes neuronales \textbf{recurrentes}, ya que la salida a cada paso ($y_k$) depende de la salida del paso anterior, $y_{k-1}$\footnote{No confundir con las redes neuronales recursivas}.

Una primer aproximación a este problema es la red recurrente de Elman \cite{elman1990finding} definida por las siguientes ecuaciones

\begin{align}
h_t &= \sigma(W_h x_t + U_h h_{t-1} + b_h) \\
y_t &= \sigma(W_y h_t + b_y)
\label{eq:elman}
\end{align}

$h_t$ es normalmente llamado el \textbf{estado oculto} en las redes neuronales recurrentes. Los parámetros a ajustar son $W_h, U_h$ (matrices) y $b_h, b_y$ (escalares). Podemos ver que, a grandes rasgos, este tipo de red recurrente no es nada más que un perceptrón multicapa cuya entrada consta de $x_t$, la entrada original en el tiempo actual $t$, y el estado oculto anterior, $h_{t-1}$.

Para entrenar este tipo de redes recurrentes utilizamos back-propagation through time (BPTT), que consta en desplegar la relación recurrente y aplicar back-propagation de manera normal. \todo{explicar un poco mejor esto}

Este tipo de redes recurrentes sufren de varios problemas: entre ellos, \textbf{vanishing gradient} y \textbf{exploding gradient}. Estos problemas pueden observarse ya que el cálculo del gradiente de las ecuaciones \ref{eq:elman} usando BPTT induce la potencia a la $n$ (donde $n$ es el largo de la secuencia) de las matrices $W_h$ y $U_h$. Usando la descomposición de Jordan de estas matrices, podemos ver que sus elementos en la diagonal que sean distintos de 1, o bien tienden a infinito o a cero.

El problema de \textbf{exploding gradient} puede solucionarse mediante la técnica de \textbf{gradient clipping}\todo{citation needed}, que consta de reajustar la norma del gradiente. Sin embargo, nos queda aún el problema de \textbf{vanishing gradient}. Para ello, se han propuesto otras arquitecturas recurrentes.

\citet{hochreiter1997long} propusieron las \textbf{Long Short-Term Memory} (LSTM) como solución a estos problemas. Para solucionar los problemas mencionados, proponen una arquitectura basadas en compuertas (\textbf{gates}) que regulan los cambios en el estado oculto y en la salida. Concretamente, la arquitectura de las LSTMs está regida por las siguientes ecuaciones\footnote{Para una muy buena explicación de las redes recurrentes, sugerimos este artículo: \url{https://colah.github.io/posts/2015-08-Understanding-LSTMs/}}:


\begin{align}
    h_t &= \sigma(W_h x_t + U_h h_{t-1} + b_h) \\
    y_t &= \sigma(W_y h_t + b_y)
    \label{eq:lstm}
\end{align}



\subsection{Optimización}


\section{Técnicas de representación}

Una de las necesidades que tienen las redes neuronales para poder trabajar con textos es el de tener representaciones continuas de cada token o palabra. Las representaciones utilizadas en la época previa de los modelos lineales --bolsas de palabras/caracteres ponderadas con algún esquema similar a TF/IDF-- adolecen de varios problemas: tienen una altísima dimensionalidad; no tienen representación semántica de la similaridad de las palabras; están concentradas en una o pocas dimensiones y suelen ser discretas.

Trabajo previo ha demostrado que obtener representaciones continuas y distribuidas (de manera opuesta a discretas y concentradas)

Latent Semantic Analysis (LSA) \cite{landauer1997solution} es una de las primeras técnicas de representación continua utilizadas para esta tarea. Plantean el problema de obtener representaciones continuas como el de factorizar una matriz de co-ocurrencia entre tokens y documentos (o contextos). LDA (Latent Dirichlet Allocation) \cite{blei2003latent} es otra técnica basada en modelos gráficos entrenados mediante métodos variacionales, muy utilizada aún en la actualidad ya que genera representaciones latentes de los tópicos de los textos.

\begin{figure}
    \centering
    \includegraphics[width=0.65\textwidth]{img/02/bengio_neural_language_model.pdf}
    \caption{Ilustración del modelo de lenguaje neuronal de \citet{bengio2003neural}. La entrada consta de $n-1$ palabras, que primero pasan por una lookup table (o capa de embeddings), una función de activación y son colapsadas para luego ser utilizadas como entrada de una función softmax.}
    \label{fig:bengio_neural_language_model}

\end{figure}

Dentro de los métodos neuronales, uno de los más populares ha sido el de \citet{bengio2003neural} que propone una arquitectura neuronal para un modelo de lenguaje markoviano. La arquitectura de esta red está ilustrada en la figura \ref{fig:bengio_neural_language_model}. En la capa intermedia contiene una tabla de lookup de vectores de las diferentes palabras (también conocido como capa de embeddings) donde se generan las representaciones de las palabras. Trabajo posterior (con diferentes variaciones de esta misma idea) como el de \citet{collobert2011natural} ha demostrado que la utilización de este tipo de representaciones es útil para diversas tareas de NLP como POS Tagging, NER, y otras.

Uno de los problemas de los métodos vistos hasta el momento es que sufrían problemas de eficiencia, sólo pudiéndose entrenar con pocos millones de palabras y con dimensiones reducidas. La técnica \emph{word2vec} \cite{mikolov2013efficient} permite entrenar representaciones de palabras de mayor dimensión y sobre grandes cantidades de textos de manera eficiente. Los vectores de palabras guardan cierta estructura lineal y semántica, como ilustran los autores con algunos ejemplos, como el ya clásico $v(\text{rey}) - v(\text{hombre}) + v(\text{mujer}) \approx v(\text{reina})$.

Para generar los vectores de \emph{word2vec}, los autores plantean una relajación del problema de modelado de lenguaje mediante dos alternativas: Continuous Bag of Words (CBOW) y Skip-Gram. En CBOW intentamos predecir la palabra faltante dada una bolsa de palabras del contexto, y en skip-gram intentamos predecir las palabras del contexto dada la palabra central. Para ambos problemas, se generan representaciones intermedias ricas para las distintas palabras. \citet{mikolov2013efficient} extiende la idea del anterior trabajo proponiendo plantear el problema de skip-gram como uno de distinguir ruido de palabras efectivamente del contexto, haciendo mucho más eficiente el cálculo de estas representaciones. GloVe \cite{pennington2014glove} es otra técnica de representación de palabras que combina las ideas de factorización de matrices de LSA  mediante un problema de optimización distinto y generando representaciones que superan ligeramente en algunos benchmarks de tareas a los de \emph{word2vec}.

Uno de los problemas que tienen estos métodos es que cada representación se calcula sobre las distintas palabras. En español, por ejemplo, las palabras gato, gata, gatito, gatuno, todas tienen representaciones independientes en \emph{word2vec}, a pesar de tener información morfológica en común. Esto es un problema en varios escenarios: idiomas con muchas inflexiones o aglutinantes (como el turco, alemán o finés) o --lo que es de nuestro interés-- texto altamente desnormalizado como el de redes sociales. La técnica \fasttext{} \cite{bojanowski16} extiende la idea de \emph{word2vec} mediante la asignación de vectores a secuencias de 3 caracteres (subpalabras), capturando así cierta información morfológica. La representación de una palabra se obtiene mediante una combinación lineal de los vectores de las subpalabras que la componen.

\subsection{Tweet Embeddings}
\label{sec:02_tweet_embeddings}

%%
%%
%%
%%  https://docs.google.com/drawings/d/1BU3ulBiqU0NojpW6Fkb4xFlMCDigSWwfjN7z9smO6nY/edit
%%
%%

\begin{figure}[t]
    \centering
    \includegraphics[width=0.80\textwidth]{img/tweet_embeddings.pdf}
    \caption{Representación contínua de un tweet mediante combinación lineal de las representaciones de cada palabra.}
    \label{fig:tweet_embeddings}
\end{figure}

Una forma relativamente simple de obtener una representación de un documento u oración (en nuestra tesis, esto será casi siempre un tweet) es realizar una combinación lineal de las representaciones obtenidas para cada palabra. Es decir, dada una oración $s = w_1 w_2 \ldots w_n$, y representaciones $\overline{w_1}, \overline{w_2}, \ldots, \overline{w_n} \in \mathbb{R}^m$, podemos obtener una representación

\begin{equation}
    \overline{s} = \sum\limits_{i=1}^{n} \alpha_i \overline{w_i}
\end{equation}

con $\alpha_1, \ldots, \alpha_n \in \mathbb{R}$ escalares (dependientes de la oración). De esta manera, obtenemos de $n$ representaciones independientes del contexto una representación para el tweet, sin tener en cuenta posibles interacciones entre los distintos componentes. La figura \ref{fig:tweet_embeddings} ilustra esta metodología simple para obtener representaciones de oraciones.

Tenemos entonces dos posibilidades para determinar la combinación lineal: la forma de obtener las representaciones, y la forma de calcular los coeficientes. Para las representaciones, podemos usar varias de las técnicas que ya vimos como word2vec, GloVe, o \fasttext{}. Para calcular los coeficientes, consideramos en nuestro trabajo dos formas. La primera, la forma canónica, calculando un promedio de las representaciones, es decir, tomando $\alpha_i = \frac{1}{n}$

Se utilizaron combinaciones lineales para calcular una representación de un solo tweet.
Seguimos dos enfoques simples: promedio simple y promedio ponderado. En el segundo caso, utilizamos un esquema que se asemeja a la frecuencia inversa suave (SIF) \cite{arora17}, inspirado TF-IDF.
Cada palabra $ w $ se pondera con $ \frac {a} {a + p (w)} $, donde $ p (w) $ es la palabra probabilidad unigrama y $ a $ es un hiperparámetro de suavizado.
Los valores altos de $ a $ significan más suavizado hacia el promedio simple.



\section{Transfer Learning y modelos pre-entrenados}

\subsection{ELMo y ULMFiT}
\label{subsec:elmo}

Hasta cerca de 2018, la forma canónica de abordar un problema de NLP era entrenar una red neuronal recurrente que consumiera embeddings no contextualizados de los tokens de entrada. Esta arquitectura tiene algunas limitaciones; una de ellas es que, dados dos o más problemas distintos (por ejemplo, análisis de sentimientos e inferencia de lenguaje natural --NLI--) lo único compartido por ambas arquitecturas es la capa más baja de la red -- la capa de embeddings -- teniendo que entrenar desde cero todo el resto de los parámetros del modelo. En términos coloquiales, cada red debe ``aprender a leer'' sobre cada tarea, ignorando muchas construcciones sintácticas y semánticas comunes del lenguaje.

Uno de los primeros esfuerzos exitosos en sobrepasar los embeddings no contextualizados es \elmo{} \cite{peters2018}. Este modelo aprende embeddings ya no sobre una única palabra como \emph{word2vec} y sus variantes, sino sobre todo una oración, generando representaciones contextualizadas de cada palabra.  Para aprenderlas, \elmo{} se entrena sobre modelo de lenguaje bidireccional \footnote{En realidad no es estrictamente bidireccional, sino dos modelos de lenguaje concatenados} recurrente de varias capas sobre grandes cantidades de texto. En dicho trabajo, utilizan luego una combinación lineal de la salida de cada capa para obtener representaciones contextualizadas de cada token. Esta misma idea es una continuación \citet{peters2017semi}, y también parcialmente de \citet{mccann2017learned}; en este último trabajo abordan la construcción de representaciones contextualizadas mediante la tarea de traducción automática.

\begin{figure}[t]
    \centering
    \includegraphics[width=\textwidth]{img/02/ulmfit.pdf}
    \caption{Universal Language Modeling for Text Classification (ULMFiT). Esquema del método planteado:}
    \label{fig:ulmfit}
\end{figure}

Alrededor de 2018, este paradigma comenzó a cambiar hacia un esquema donde se entrena una red neuronal sobre una tarea genérica para luego ajustarla a la tarea específica, algo muy común en el área de Computer Vision. \citet{howard-ruder-2018-universal} introdujeron la técnica de ULMFiT(Universal Language Modeling for Fine-tuning for text classification), uno de los trabajos fundamentales de este nuevo paradigma. La idea propuesta se puede resumir en entrenar un modelo de lenguaje sobre un gran dataset no etiquetado, y luego utilizar esa misma red (cambiándole la última capa) ajustándola a una tarea específica.

Los autores proponen 3 etapas: primero, el pre-entrenamiento sobre la tarea de modelado de lenguaje en un gran dataset de texto (e.g. Wikipedia o Common Crawl); segundo, un ajuste de la tarea de modelado de lenguaje sobre el texto de la tarea en cuestión (LM fine-tuning); y finalmente, el entrenamiento sobre las etiquetas de la tarea (Classifier fine-tuning). La figura \ref{fig:ulmfit} ilustra las 3 etapas para el problema de clasificación de sentimientos. Entre varias técnicas que utilizan para entrenar estos modelos y evitar el olvido catastrófico, vale destacar el uso de \emph{slanted triangular learning rates}, donde el learning rate tiene una etapa de \emph{warmup} donde sube hasta el pico y luego una etapa de \emph{annealing} donde se reduce linealmente hasta 0 por el resto del entrenamiento. Esta técnica será utilizada por \bert{} y otros modelos de lenguaje basados en transformers.

El modelo de lenguaje utilizado por los autores de ULMFiT utiliza una arquitectura AWD-LSTM \cite{merity2018regularizing}. Estas arquitecturas recurrentes fueron el estado del arte hasta el momento, pero fueron sobrepasados por las basadas en \emph{transformers}, que pasaremos a detallar.


\subsection{Modelos basados en atención}
\label{sec:02_transformers}

Una de las limitaciones de los modelos basados en redes recurrentes es que sufren un \tbf{sesgo de localidad o secuencialidad}(locality bias) \cite{battaglia2018relational}. En palabras coloquiales, las redes recurrentes tienen problemas para aprender dependencias de largo rango en las oraciones, siendo esto producto de su arquitectura autorregresiva donde se construye la salida $y_t$ en base a $y_{t-1}$. Este sesgo es particularmente dañino en tareas de transducción de sequencias con la arquitectura encoder-decoder básica ya que a esto se le suma un cuello de botella forzoso por la compresión de toda la secuencia de entrada en un vector de longitud fija. \footnote{\citet{sutskever2014sequence} de hecho en su trabajo observa que invertir la oración de entrada obtiene mejores resultados para la tarea de traducción automática}.


\begin{figure}[t]
    \centering
    \includegraphics[width=0.5\textwidth]{img/02/attention_model.pdf}
    \caption{Mecanismo general de atención. En azul, la salida del encoder recurrente de la entrada. En rojo, la salida del decoder recurrente. Fuente: \citet{luong2015effective}}
    \label{fig:attention_mechanism}
\end{figure}


Una de las formas de mitigar este sesgo es la utilización de mecanismos de \emph{atención} \cite{bahdanau2014neural}. Suponiendo una arquitectura recurrente de encoder y decoder, y siguiendo la notación de \citet{luong2015effective}, para la tarea de traducción automática de una secuencia $(x_1, \ldots , x_n)$ a $(y_1, \ldots , y_m)$, con estados ocultos $(\overline{h_1}, \ldots , \overline{h_n})$ para la entrada y $(h_1, \ldots , h_m)$ y para la salida, el mecanismo de atención \footnote{global, en \citet{luong2015effective} se menciona el mecanismo local que no consideramos} consta de calcular para cada paso $t$ un vector de contexto

\begin{equation*}
    c_t = \sum_{i=1}^n \alpha_i^{(t)} \overline{h_i}
\end{equation*}

donde $\alpha^{(t)}$ es el vector de alineamiento, calculado como

\newcommand{\score}[0]{\text{score}}

\begin{equation*}
    \alpha^{(t)} = \softmax(\score(\overline{h_1}, h_t), \score(\overline{h_2}, h_t) \ldots , \score(\overline{h_n}, h_t))
\end{equation*}

Cada $\score(\overline{h_i}, h_t)$ marca una similaridad no normalizada entre sus argumentos. Siguiendo las alternativas planteadas en \citet{luong2015effective}, tenemos algunas posibles opciones para esto:

\begin{equation}
    score(\overline{h_i}, h_t) =  \begin{cases}
        \overline{h_i}^T h_t   & \text{dot} \\
        \overline{h_i}^T W h_t & \text{general} \\
        v^T\text{tanh}(W [\overline{h_i}^T; h_t]) & concat
     \end{cases}
\end{equation}

% copypasting random stuff from the Internetz
% https://tex.stackexchange.com/questions/66537/making-hats-and-other-accents-bold
%
\newcommand{\thicktilde}[1]{\mathbf{\tilde{\text{$#1$}}}}

con $W, v$ parámetros adicionales. En el caso de la atención producto interno (dot) podemos reescribir todas las ecuaciones como:

\begin{equation}
    C = \softmax(H \widehat{H}^T ) \widehat{H}
    \label{eq:attention_product}
\end{equation}

donde $\widehat{H}, H$ son los vectores que tienen $(\overline{h_1}, \ldots , \overline{h_n})$  y $(h_1, \ldots , h_m)$ como filas respectivamente, y $\softmax$ se calcula fila a fila.

Finalmente, el vector $\widetilde{h_t}$ es calculado como una transformación del estado oculto del decoder $h_t$ y el vector contextual $c_t$:


\begin{equation*}
    \widetilde{h_t} = \tanh(W_h [h_t; c_t])
\end{equation*}


El vector $\widetilde{h_t}$ codifica información de manera global de todos los estados ocultos del codificador, mitigando este problema del sesgo de localidad. Esta técnica se convirtió en parte integral de los modelos seq2seq como ser traducción automática, sumarización, entre otras. La figura \ref{fig:attention_mechanism} ilustra esta arquitectura.

La técnica de auto-atención o intra-atención \cite{parikh-etal-2016-decomposable} consiste en aproximadamente la misma idea que la atención sólo que teniendo una única secuencia; podemos asumir ecuaciones similares con $\overline{h_i} = h_i$. La auto-atención genera representaciones de los distintos vectores de entrada observando la totalidad de la secuencia, a diferencia de las redes recurrentes que sólo construyen una representación en base al paso anterior. Esta capa se utiliza en arquitecturas para clasificación de texto encima de una capa recurrente para generar representaciones con dependencias sin distinción de la distancia entre los distintos tokens.

\section{Transformers}

\begin{figure}[t]
    \centering
    \begin{subfigure}[]{0.55\textwidth}
        \centering
        \includegraphics[width=0.55\textwidth]{img/02/transformer_architecture.png}
        \caption{Arquitectura de modelos Transformer}
        \label{fig:transformer_architecture}
    \end{subfigure}
    \begin{subfigure}[]{0.40\textwidth}
        \centering
        \includegraphics[width=0.40\textwidth]{img/02/scaled_self_attention.png}
        \caption{Arquitectura de modelos Transformer}
        \label{fig:scaled_self_attention}
    \end{subfigure}
    \caption{Modelo de transformador y su versión de auto-atención. La subfigura \ref{fig:transformer_architecture} muestra la arquitectura de los codificadores y decodificadores. Fuente: \citet{vaswani2017attention}}
    \label{fig:transformer_mechanism}
\end{figure}

Mencionamos el sesgo de la secuencialidad como uno de los problemas de los modelos recurrentes. Otro de los grandes obstáculos para las arquitecturas autorregresivas es la paralelización. El modelo de cómputo secuencial donde $h_t$ se calcula en base a $h_{t-1}$ inhibe un cálculo paralelo, donde las diferentes representaciones puedan ser generadas simultáneamente. \citet{parikh-etal-2016-decomposable} es uno de los primeros trabajos que proponen una arquitectura para el problema de inferencia (NLI) enteramente basada en modelos de atención, sin modelos recurrentes.

\citet{vaswani2017attention} introdujeron la arquitectura \emph{Transformer} para la tarea de traducción automática. Esta arquitectura no utiliza capas recurrentes ni convolucionales, basándose enteramente en el mecanismo de auto-atención. La figura \ref{fig:transformer_architecture} muestra la arquitectura de los modelos basados en Transformer, organizado en forma de encoder-decoder, con 6 capas de cada uno.

Cada capa del encoder utiliza un mecanismo de auto-atención múltiple seguido de una capa feed-forward punto a punto. Las dos capas de auto-atención o feed-forward están sucedidas por una conexión residual \cite{he2016deep} \footnote{El fin de estas conexiones residuales es facilitar el flujo del gradiente en arquitecturas profundas} y una normalización, de manera que la salida se expresa como:

\begin{equation*}
    \text{Layer}(x) = \text{Norm}(x + \text{subLayer}(x))
\end{equation*}

Las capas decodificadoras son similares, salvo que se les agrega una capa extra de auto-atención donde se combinan las salidas del encoder con las representaciones que genera el decoder. A su vez, las capas de multi-atención están enmascaradas para no poder ``ver'' las representaciones que se generan en pasos posteriores para guardar su naturaleza secuencial en la tarea.

El cálculo de atención utilizado en este trabajo es similar al visto en la ecuación \ref{eq:attention_product}, aunque normalizado por $\sqrt{d_k}$, donde $d_k$ es la dimensión de los vectores de entrada:

\begin{equation*}
    Attention(Q, K, V) = \text{softmax}(\frac{Q^T K}{\sqrt{d_k}}) V
\end{equation*}

Cada capa utiliza varias cabezas de auto-atención, las cuales concatena y proyecta en su salida en su salida. La salida de cada una de capa pasa por una regularización de tipo dropout \cite{srivastava2014dropout}.

Un punto no menor es que el modelo Transformer, siendo que no tiene ningún tipo de recurrencia y convolución, carece de cualquier ordenamiento de la secuencia de tokens. Para inyectar ese conocimiento en la red, utilizan \emph{vectores de posicionamiento} (positional embeddings) que se suman a los vectores de entrada de la capa de embeddings, como se ilustra en la figura \ref{fig:transformer_architecture}. Estos vectores no son parámetros entrenados (como sí lo son en \bert{}) sino que se calculan mediante funciones sinusoidales.


No nos extenderemos más en la explicación de esta arquitectura, y referimos para más información a los excelentes artículos \emph{Transformers from Scratch} \footnote{\url{http://peterbloem.nl/blog/transformers}}, \emph{Annotated Transformer} \footnote{\url{https://nlp.seas.harvard.edu/2018/04/03/attention.html}} y \emph{The Illustrated Transformer} \footnote{\url{https://jalammar.github.io/illustrated-transformer/}}.


\section{GPT, BERT, y modelos pre-entrenados basados en Transformers}

Combinando las ideas de ULMFit --entrenaje semi-supervisado sobre la tarea de modelado de lenguaje-- y la arquitectura Transformer --removiendo las redes recurrentes y facilitando la paralelización del cálculo-- en \citet{radford2018improving} se introduce GPT (\emph{generative pre-training}). Esta técnica consiste de un pre-entrenamiento sobre un gran corpus no etiquetado y luego un fine-tuning discriminativo para cada tarea, muy en la línea de \citet{howard-ruder-2018-universal}, introduciendo parámetros específicos únicamente para cada una de estas. El modelo que usa esta tarea es el de \tbf{modelado de lenguaje causal} -- es decir, de izquierda a derecha. Este modelo obtuvo para muchas tareas como el benchmark GLUE \cite{wang-etal-2018-glue} el estado del arte.


\bert{} \cite{devlin2018bert} (Bidirectional Encoder Representations from Transformers) plantea una modificación sobre GPT: en lugar de pre-entrenar el modelo sobre la tarea de modelado de lenguaje \tbf{causal} --de izquierda a derecha-- entrenar sobre la tarea de modelado de lenguaje \tbf{enmascarado}. Esta tarea (usualmente llamada \emph{Cloze task} \cite{taylor1953cloze}) consta de enmascarar una cierta cantidad de palabras de una frase, y luego intentar predecir las palabras faltantes. Por ejemplo, en la siguiente frase, consta de reemplazar los dos tokens \verb|[MASK]|:

\begin{center}
    El \verb|[MASK]| es celeste y el pasto \verb|[MASK]|
\end{center}


\begin{figure}
    \centering
    \includegraphics[width=\textwidth]{img/02/gpt_vs_bert.pdf}
    \caption{Comparación entre ELMo, GPT y BERT. }
\end{figure}

A diferencia de la tarea de modelado de lenguaje causal, los autores argumentan que esta tarea permite generar representaciones bidireccionales ricas. La figura


A modo de comparación entre las distintas estrategias de pre-entrenamiento, la figura




\part{Extracción de opiniones de redes sociales y detección de discurso de odio}
%
\chapter{Extracción de opiniones de textos sociales}
\label{chap:03_social_text_classification}

La extracción de opiniones en distintos espacios virtuales ha atraído mucho interés desde los comienzos de la World Wide Web. Inicialmente motivados por fines puramente comerciales, diferentes motivaciones han surgido debido al desarrollo de la técnica y la proliferación de las redes sociales: desde intereses sociológicos (como el análisis de discurso de odio o las reacciones a la pandemia) hasta políticos (como observar cuál es la opinión general sobre tal o cual candidato o sobre un tema candente). Desde principios de los años 2000, y debido a la combinación del desarrollo de métodos de aprendizaje estadístico y la cantidad creciente de datos disponibles generados por usuarios en Internet, numerosos trabajos han analizado este tipo de textos para poder extraer conocimiento \textbf{subjetivo} de estos.

Debido a la inmensa cantidad de contenido generado en diversos sitios y redes sociales (se estima que en el mundo se generan XXX tweets por segundo), hace ya muchos años esta tarea es difícil de realizar sin algún tipo de automatización. Para ello, muchísimo esfuerzo se ha volcado en utilizar técnicas de aprendizaje automático para atacarla. El avance de las técnicas de NLP --como hemos descrito en el capítulo anterior-- han permitido avanzar sobre este terreno; sin embargo, muchas de las limitaciones actuales del área \todo{citar paper Climbing towards NLU} en conjunto a las dificultades particulares de las interacciones en medios sociales hacen esta tarea difícil.

En este capítulo haremos una breve introducción a clasificación de textos sociales. Esto es, dado un texto generado por un usuario (un post en Facebook, Instagram, un tweet, etc) predecir alguna característica discreta de éste, como por ejemplo si es un texto positivo o negativo, si tiene algún tipo de emoción de ira, alegría, u otra; si contiene discurso de odio contra algún grupo o no; si es irónico; entre otras. En base a datasets en español para distintas tareas, presentaremos modelos de clasificación basados en técnicas del estado del arte.

Finalmente, analizaremos algunas cuestiones relacionadas a la adaptación de dominio y representaciones generadas sobre dominios sociales. Analizaremos para técnicas de representación no contextualizadas \footnote{Que al día de la fecha, en pocos años, han quedado obsoletas} y algunas técnicas más recientes el impacto de entrenar desde cero o realizar cierta adaptación sobre la performance de las técnicas de clasificación.


\section{Motivación}

Las motivaciones para extraer opiniones subjetivas de usuarios en Internet son múltiples, aunque intentaremos categorizarlas en algunos grupos de notable interés. Dado el aumento considerable de contenido generado por usuarios desde el comienzo de la WWW --y subsiguientemente con la explosión de las Redes Sociales-- una de las motivaciones es netamente comercial: ¿qué opinan los usuarios sobre este nuevo producto? ¿cuáles creen que son sus falencias? ¿qué tal es el servicio en el Restaurant X? Desde ya más de 20 años, numerosos sitios de e-commerce brindan la posibilidad de que los clientes vuelquen sus opiniones al respecto de los productos que consumen en sus plataformas, como así también pueden incorporarse en otras aplicaciones que brindan esta posibilidad de expresar comentarios sobre productos, servicios u otros lugares. Para citar unos ejemplos, IMDb permite agregar comentarios sobre películas, Google Maps sobre distintos sitios --tanto turísticos como locales comerciales--, o los distintos sitios de venta minorista como MercadoLibre, eBay, o Amazon.

Con la explosión de las redes sociales, otros horizontes de preguntas se abrieron\footnote{Si bien algunas preguntas de carácter sociológico tuvieron lugar con anterioridad, podemos marcar el uso intensivo de Facebook y Twitter como el comienzo de un estudio más sistemático de ellas}. Uno de estos horizontes, que es de interés particular para esta tesis, es el de las preguntas de carácter sociológico. Preguntas que pueden suscitar interés dentro de este punto pueden ser:

\begin{itemize}
    \item ¿cuál es la opinión de los usuarios acerca de la legalización del aborto?
    \item ¿cuál es el sentimiento que tienen ciertos usuarios hacia los inmigrantes subsaharianos en España?
    \item ¿cómo se ha modificado el ``humor social'' de acuerdo a crisis económicas o pandemias como la del COVID-19?
    \item ¿quiénes generan discurso de odio contra la comunidad LGBTI en Argentina?
    \item ¿qué artículos periodísticos suscitan la mayor cantidad de discurso discriminatorio en las redes sociales?
    \item ¿cuáles son las principales preocupaciones de ciertos sectores de la población?
\end{itemize}

entre otras. Estos tópicos son de gran interés para investigadores y políticos. Usualmente, la forma más estandarizada de acceder a la opinión de distintos actores sociales ha sido la de encuestas; sin embargo, la recolección y extracción automática de opiniones de medios virtuales brinda una alternativa (a veces) más económica y masiva aunque con un sesgo poblacional distinto al de otras metodologías.

\section{Cómo atacamos este problema desde Procesamiento de Lenguaje Natural}



\section{Trabajo previo}

Talleres
datasets


Describir data augmentation como otra técnica de regularización. Comentar backtranslation

Español
- Citar nuestro trabajo

\section{Tareas analizadas}
\subsection{Análisis de Sentimiento}

\subsection{Análisis de Emociones}


\section{Preprocesamiento}

\section{Técnicas de clasificación}

\subsection{Embeddings}

\section{Resultados}

\section{Discusión}

\section{Librería de análisis de sentimientos}

\newcommand{\pysentimiento}[0]{\textbf{pysentimiento}}

Algo que suele obstaculizar la utilización de herramientas de extracción de opinión (como las que acabamos de ver en este capítulo pero así mismo las que veremos más adelante) con fines de investigación es la dificultad a su acceso. O bien estos servicios están detrás de APIs pagas con precios demasiado altos para los presupuestos académicos o están disponibles pero no en español (u otro idioma de ``bajos recursos''). En otros casos, estos recursos están disponibles pero no para ser usados de forma de ``caja negra'', lo cual para alguien que no es un experto en NLP suele complicar su utilización.

Como una pequeñísima contribución de esta tesis y con el objetivo de facilitar el acceso de estos recursos para la investigación, creamos la librería \textbf{pysentimiento}\footnote{\url{https://github.com/pysentimiento/pysentimiento}}. Este paquete provee modelos pre-entrenados y herramientas de preprocesado para textos sociales. Si bien tiene soporte multilingual tanto en español como inglés, su eje original es el de proveer recursos para el español que tiene una disparidad importante en recursos.

La figura XXX muestra la arquitectura de \pysentimiento{}. Utiliza el model hub de \emph{huggingface}\footnote{\url{https://huggingface.co/models}}, un repositorio de modelos pre-entrenados basados en transformers. Allí es donde colocamos todos los modelos que entrenamos, tanto de sentimientos, emociones, y los que mostraremos más adelante como detección de discurso de odio. Cada tweet que es analizado por la librería pasa primero por una etapa de preprocesamiento (siguiendo el proceso explicado en la sección zzz), y luego procesado por el modelo, quien nos brinda un output. Dependiendo el problema, puede haber una etapa de post-procesamiento.

\todo{Completar las cosas que quedaron acá sin referencias}



%
% Pysentimiento architecture
% https://www.canva.com/design/DAEufPDskMI/Gg_phzjuXgFihF1g3x9L-A/edit#
%
%

\section{Conclusiones}

\section{Notas adicionales}

Comentar acá nuestro trabajo en TASS 2020
%
\chapter{Detección de discurso de odio}
Los Discursos de odio contra mujeres, inmigrantes y muchos otros grupos es un fenómeno generalizado en la Internet. En los primeros días de la World Wide Web, algunos académicos se aventuraron a decir a que los prejuicios y el odio sería eliminado en este espacio por la disolución de identidades \cite{levy2001cyberculture, rheingold1993virtual}. Veinte años después de esta hipótesis, podemos
decir que no ha sido el caso. La prevalencia del racismo en la ``World White Web'' se ha estudiado en una serie de trabajos \cite{adams2005white, kettrey2014staking}, como así también la misoginia en el mundo virtual \cite{filipovic2007blogging, mantilla2013gendertrolling}.

El discurso racista y sexista es una constante en las redes sociales, pero los picos se documentan después de eventos ``detonantes'', como asesinatos con motivos religiosos o políticos \cite{burnap2015cyber}. Las empresas de redes sociales están preocupadas por esto y toman acciones en su contra; sin embargo, la mayoría de los esfuerzos todavía necesitan la intervención humana, lo que hace que esta tarea sea muy costosa. Reducir la intervención humana es vital para tener herramientas efectivas para evitar la escalada del discurso de odio.


En este capítulo haremos una introducción a este problema, que a su vez trataremos en los capítulos subsiguientes. Definiremos el discurso de odio y haremos una breve reseña de este fenómeno desde un marco legal y de tratados internacionales para luego centrarnos en este problema desde una perspectiva del procesamiento de lenguaje natural. Comentaremos parte de nuestro trabajo en \citet{atalaya_tass2018} como parte de la competencia hatEval\cite{hateval2019semeval}, a la vez que marcaremos algunos problemas actuales en los enfoques actuales del discurso discriminatorio.


\section{Definición de discurso de odio}

\label{sec:hate_speech_definitions}

No existe una definición universalmente aceptada de lo que configura discurso de odio. En esta sección haremos un repaso muy breve de algunos tratados internacionales sobre la materia para intentar aproximarnos a este concepto, a la vez que también haremos un racconto de las definiciones utilizadas en trabajos dedicados a la construcción de datasets.

Un derecho que suele estar protegido por constituciones nacionales y tratados internacionales es el del derecho a la expresión. Por ejemplo, el Pacto de San José de Costa Rica (a la cual Argentina adhiere)\cite{humanos2018convencion} dice en su Artículo 13:

\begin{displayquote}[CADH, Artículo 13][]

    1. Toda persona tiene derecho a la libertad de pensamiento y de expresión.  Este derecho comprende la libertad de buscar, recibir y difundir informaciones e ideas de toda índole, sin consideración de fronteras, ya sea oralmente, por escrito o en forma impresa o artística, o por cualquier otro procedimiento de su elección.

    2. El ejercicio del derecho previsto en el inciso precedente no puede estar sujeto a previa censura sino a responsabilidades ulteriores, las que deben estar expresamente fijadas por la ley y ser necesarias para asegurar:

    a)  el respeto a los derechos o a la reputación de los demás, o

    b) la protección de la seguridad nacional, el orden público o la salud o la moral públicas.
\end{displayquote}

En Estados Unidos, la primer enmienda protege este derecho humano, mientras que en la Unión Europea, legislación similar ofrece protección a la libertad de expresión. Finalmente, la declaración universal de los derechos humanos de la ONU \todo{citation needed} menciona tanto en su preámbulo como en el artículo 19

\begin{displayquote}[Declaración Universal de los Derechos Humanos][ONU]
    Todo individuo tiene derecho a la libertad de opinión y de expresión; este derecho incluye el de no ser molestado a causa de sus opiniones, el de investigar y recibir informaciones y opiniones, y el de difundirlas, sin limitación de fronteras, por cualquier medio de expresión.
\end{displayquote}

Otro documento conocido como el Pacto Internacional de Derechos Civiles y Políticos (ICCPR por sus siglas en inglés) menciona

\begin{displayquote}[Artículo 19 de la ICCPR]
1. Nadie podrá ser molestado a causa de sus opiniones.

2. Toda persona tiene derecho a la libertad de expresión; este derecho comprende la libertad de buscar, recibir y difundir informaciones e ideas de toda índole, sin consideración de fronteras, ya sea oralmente, por escrito o en forma impresa o artística, o por cualquier otro procedimiento de su elección.

3. El ejercicio del derecho previsto en el párrafo 2 de este artículo entraña deberes y responsabilidades especiales. Por consiguiente, puede estar sujeto a ciertas restricciones, que deberán, sin embargo, estar expresamente fijadas por la ley y ser necesarias para:

a) Asegurar el respeto a los derechos o a la reputación de los demás;

b) La protección de la seguridad nacional, el orden público o la salud o la moral públicas.
\end{displayquote}

Sin embargo, y como mencionan estos dos últimos apartados, la libertad de expresión tiene un límite: el ejercicio de los derechos e igualdad ante la ley. Por ejemplo, el Artículo 1 del Pacto de San José de Costa Rica dice lo siguiente:

\begin{displayquote}[Pacto San José de Costa Rica, CADH][Artículo 1]
    1. Los Estados Partes en esta Convención se comprometen a respetar los derechos y libertades reconocidos en ella y a garantizar su libre y pleno ejercicio a toda persona que esté sujeta a su jurisdicción, sin discriminación alguna por motivos de raza, color, sexo, idioma, religión, opiniones políticas o de cualquier otra índole, origen nacional o social, posición económica, nacimiento o cualquier otra condición social.
\end{displayquote}

A su vez, la Declaración Universal de los Derechos Humanos en su Artículo 1:

\begin{displayquote}
    Todos los seres humanos nacen libres e iguales en dignidad y derechos y, dotados como están de razón y conciencia, deben comportarse fraternalmente los unos con los otros.
\end{displayquote}

Entonces, los Estados y otros organismos deben tomar medidas para poder asegurar el libre ejercicio de los derechos y la igualdad de todos sus miembros, aún cuando esto pueda significar una restricción en la libertad de expresión \cite{article192015}.


¿Qué es el discurso de odio entonces? Como hemos mencionado, no hay una definición universalmente aceptada. Repasemos algunas clasificaciones hechas en estos tratados para acercarnos un poco más a las características comunes que comparten.

Como vimos, el Pacto de San José de Costa Rica en su Artículo 1 habla del ejercicio de derechos sin discriminación alguna por varias razones, entre las que menciona raza, sexo, idioma, religión, política, nacionalidad, posición económica, entre otras. La Observación General 35 del Comité por la Eliminación de la Discriminación Racial de la ONU (CERD) considera que será discurso de odio, y debe ser tipificado penalmente:


\begin{displayquote}[Recomendación 35 del Comité por la Eliminación de la Discriminación Racial, CERD]

    a) Toda difusión de ideas basada en la superioridad o en el odio racial o étnico, por cualquier medio;

    b) La incitación al odio, el desprecio o la discriminación contra los miembros de un grupo por motivos de su raza, color, linaje, u origen nacional o étnico;

    c) Las amenazas o la incitación a la violencia contra personas o grupos por los motivos señalados en el apartado anterior;

    d) La expresión de insultos, burlas o calumnias contra personas o grupos, o la justificación del odio, el desprecio o la discriminación por los motivos señalados en el apartado b) anterior, cuando constituyan claramente incitación al odio o a la discriminación;

    e) La participación en organizaciones y actividades que promuevan e inciten a la discriminación racial.
\end{displayquote}

\citet{gagliardone2015countering} presenta un análisis de diversos organismos y sus definiciones de discurso de odio. En líneas generales, como se menciona en \citet{CIDH2015}, el concepto usualmente es referido a expresiones que incitan a tomar algún tipo de medida hostil contra una víctima o un grupo de personas, siendo esta perteneciente a un determinado grupo social definido por laguna característica. Dicho esto, podría delimitarse el discurso discriminatorio del discurso de odio por la componente de la promoción e instigación de la violencia; sin embargo, para los fines de este trabajo utilizaremos los términos indistintamente. Como se menciona también en \citet{CIDH2015}, aún cuando el discurso no contenga arengas ni incitaciones a cometer actos violentos, puede entenderse ese discurso como generador de un ambiente hostil y de intolerancia que termine promoviendo estos ataques físicos.


\citet{article192015} condensa muchas de estas definiciones de una manera succinta, desglosando esto en ``odio'' y ``discurso'':

\begin{displayquote}[Article 19: Hate Speech Toolkit]

    – Odio: emoción intensa e irracional de oprobio, enemistad y aborrecimiento hacia una persona o grupo de personas, por tener determinadas características protegidas (reconocidas en el derecho internacional), reales o percibidas. El “odio” es más que un mero prejuicio y debe ser discriminatorio. El odio es una muestra de un estado emocional u opinión y, por lo tanto, se diferencia de cualquier acto o acción que se haya llevado a cabo.
    – Discurso: cualquier expresión que vierta opiniones o ideas, que comparte una
    opinión o una idea interna con un público externo. Puede adoptar muchas
    formas: escrita, no-verbal, visual o artística y puede ser difundida en los
    medios, incluyendo Internet, material impreso, radio o televisión.
\end{displayquote}

%%
%%
%% Link
%% https://docs.google.com/drawings/d/149dpb2nrvmFgWZJYcrToAxO4M5n7JNQInfWd62kw3jc/edit
%%
%%

\begin{figure}[t]
    \centering
    \includegraphics[width=\textwidth]{img/discurso_de_odio.pdf}
    \caption{Definición de discurso de odio de acuerdo al Toolkit de Article 19}
    \label{fig:hate_speech_definition_article_19}
\end{figure}


Entonces, puede entenderse como un discurso de cierta intensidad e irracionalidad que ataca a una persona o un grupo de personas por alguna característica históricamente vulnerada: por ser mujer, por su etnia, nacionalidad, religión, idioma, etc. La clave está en la combinación: un discurso irracional e intenso contra alguien que no posea una característica protegida no configura discurso de odio; por ejemplo, ataques a ciertas personas por ser periodistas. La figura \ref{fig:hate_speech_definition_article_19} ilustra esta definición.

No todo ataque a un individuo o una persona de algún colectivo discriminado es discurso de odio. En particular, en \citet{CIDH2015} se menciona en base al informe de la UNESCO sobre discurso de odio \citet{gagliardone2015countering} que:

\begin{displayquote}[]
    (...) el discurso de odio no puede abarcar ideas amplias y abstractas, tales como las visiones e ideologías políticas, la fe o las creencias personales. Tampoco se refiere simplemente a un insulto, expresión injuriosa o provocadora respecto de una persona. Así definido, el discurso de odio puede ser manipulado fácilmente para abarcar expresiones que puedan ser consideradas ofensivas por otras personas, particularmente por quienes están en el poder, lo que conduce a la indebida aplicación de la ley para restringir las expresiones críticas y disidentes. Asimismo, el discurso de odio tiene que distinguirse de aquellos “crímenes de odio” que se basan en conductas expresivas, como las amenazas y la violencia sexual, y que se encuentran fuera de cualquier protección del derecho a la libertad de expresión
\end{displayquote}

Como vemos, no sólamente es difusa la frontera fijada la característica sobre qué es discurso de odio o insultos, sino que incluso también es difícil definir qué característica es protegida o no. En el siguiente capítulo hablaremos más de esto al relatar cuáles fueron usadas a la hora de anotar nuestro dataset.

Si bien, como mencionamos, en cierta legislación se diferencia entre discurso discriminatorio y discurso de odio, para los fines de este trabajo utilizaremos ambas acepciones indistintamente. Cuando haya un llamado o una incitación a la violencia o algún tipo de represalia se hará explícita esta cuestión.


\section{Trabajo previo}

En esta sección haremos una breve reseña de algunos trabajos destacados del área. Un análisis extensivo de esta disciplina escapa totalmente al alcance de este trabajo. Referimos a quien esté interesado a \citet{schmidt2017survey} y a \citet{fortuna2018survey}. Más recientemente, \citet{poletto2021resources} hace un análisis pormenorizado y actualizado de los recursos para la tarea de detección de discurso de odio.

La detección del discurso del odio es una tarea de clasificación de oraciones bastante relacionada con el análisis de sentimientos y ha sido estudiada para varias redes sociales \cite{thelwall2008social, pak2010twitter, saleem2017web}. Uno de los primeros trabajos al respecto es \citet{greevy2004classifying} usando bolsas de palabras y SMVs para detectar contenido racista en páginas web. Construyeron su dataset de manera semi-supervisada buscando sitios mediante keywords y sus links en motores de búsqueda. Siguiendo un enfoque similar, \citet{warner2012detecting} usó unigrams y clusters Brown con SVM para detectar mensajes antisemitas en Twitter.

\citet{waseem2016hateful} anotó un corpus y usó n-gramas de caracteres para detectar comentarios de odio, y \citet{badjatiya2017deep} usó el mismo conjunto de datos para entrenar modelos de aprendizaje profundo e incrustaciones ajustadas junto con Gradient Boosted Trees. \citet {zhang2018detecting} entrenó una red neuronal profunda que combina CNN con unidades recurrentes cerradas \cite{cho2014learning}, superando a los sistemas anteriores en varios conjuntos de datos.

\citet{anzovino2018automatic} recopiló un corpus de tweets misóginos y propuso una taxonomía para distinguirlos en diferentes categorías. Los autores propusieron una serie de técnicas diferentes para clasificarlos, mostrando que enfoques simples (como el uso de modelos lineales junto con n-gramas de token) logran un rendimiento competitivo en conjuntos de datos de pequeño tamaño.

En cuanto a las tareas compartidas, \citet{fersini2018overview} presentó un desafío en la detección de misoginia en Twitter, tanto en español como en inglés, mientras que \citet{fersini2018evalitaoverview} planteó un desafío similar pero en italiano e inglés. \citet{bosco2018overview} propuso un concurso de detección automática sobre publicaciones de Twitter y comentarios de Facebook, que incluía discursos de odio en general.



\section{Datasets}

En esta sección haremos un recuento de los conjuntos de datos para la detección de lenguaje odioso.
\citet{nobata2016abusive} fue uno de los primeros datasets => mencionar sus criterios, y su experimento (muy interesante) de Turkers vs anotadores in-house

% \begin{table*}
%     \centering
%     \begin{tabular*}

%     \end{tabular*}
% \end{table*}

\citet{gao2018detecting}

\subsection{Criterios de anotación}

\begin{table*}[t]
    \centering
    \begin{tabular}{p{0.20\linewidth}  p{0.80\linewidth}}
        \textbf{Hate Speech} & Language which attacks or demeans a group based on race, ethnic origin, religion, disability, gender, age, disability,
        or sexual orientation/gender identity. Some examples:
        \begin{itemize}
            \item The JEWZ are Market Manipulators Hate them freaking Jews !! To bad Hitler did not finish them off !!!
            \item  I hate the mall because there are too many unemployed apefricans and mudsharks.
            \item 300 missiles are cool! Love to see um launched into TelAviv! Kill all the gays there!
            \item EGYPTIANS ARE DESCENDED FROM APES AND PIGS. NUKE EGYPT NOW Ah me tu se ku sh
        \end{itemize} \\

        \hline

        \textbf{Derogatory} & Language which attacks an individual or a group, but which is not hate speech. Some examples:
        \begin{itemize}
            \item All you perverts (other than me) who posted today, needs to leave the O Board. Dfasdfdasfadfs
            \item yikes...another republiCUNT weighs in....
        \end{itemize} \\

        \hline

        \textbf{Profanity} & Language which contains sexual remarks or profanity. Some examples:

        \begin{itemize}
            \item T.Boone Pickens needs to take a minimum wage job in FL for a week. I guarantee he shuts the f up after that.
            \item Somebody told me that Little Debbie likes to take it up the A.\$.\$.
            \item So if the pre market is any indication Kind of like the bloody red tampons that you to suck on all day??
        \end{itemize}
         \\
    \end{tabular}
    \caption{Annotation guidelines used in \cite{nobata2016abusive}}

    \label{tab:nobata_guidelines}
\end{table*}


\subsubsection{}

\section{Método}

\subsection {Preprocesamiento}


\newcommand{\elmo}[0]{ELMo}
\newcommand{\elmomodel}[0]{\emph{LSTM-\elmo{}}}
\newcommand{\bow}[0]{BoW}
\newcommand{\boc}[0]{BoC}
\newcommand{\elmobowmodel}[0]{\emph{LSTM-\elmo{}+\bow{}}}
\newcommand{\svmmodel}[0]{$\mathrm{SVM}_0$}
\newcommand{\hateval}[0]{HatEval}
\newcommand{\semeval}[0]{SemEval-2019}
\newcommand{\fasttext}[0]{\emph{fastText}}

El preprocesamiento es crucial en las aplicaciones de PNL, especialmente cuando se trabaja con datos ruidosos generados por el usuario. Aquí, seguimos \citet{atalaya_tass2018}, definiendo dos niveles de preprocesamiento: preprocesamiento básico y orientado a sentimientos. Usamos uno u otro, dependiendo de la configuración.

El preprocesamiento básico de tweets incluye tokenización, reemplazo de identificadores, URL y correos electrónicos, y acortamiento de letras repetidas.

El preprocesamiento orientado a sentimientos incluye minúsculas, eliminación de puntuación, palabras vacías y números, lematización (usando TreeTagger \cite{schmid95}) y manejo de negación.
Para el manejo de la negación, seguimos un enfoque simple:
% \cite {das01, pang02}:
Buscamos palabras de negación y agregamos el prefijo 'NOT \_' a los siguientes tokens. Se niegan hasta tres tokens, o menos si se encuentra un token que no sea una palabra.

\section{Técnicas de clasificación}

Para capturar esta información, consideramos una representación de bolsa de caracteres que codifica recuentos de caracteres $n$ -gramas para algunos valores de $ n $. Estos vectores se calculan a partir de textos originales de tweets, sin ningún procesamiento previo. \boc {} s tienen las mismas variantes y parámetros que \bow {} s.


\subsection {Word-embeddings}

Usamos \fasttext {}, una biblioteca de incrustaciones consciente de subpalabras \cite{bojanowski16} para obtener representaciones de palabras independientes del contexto.
En lugar de usar vectores previamente entrenados disponibles públicamente, entrenamos nuestras propias incrustaciones en un conjunto de datos de $ \sim90 $ millones de tweets de varios países de habla hispana.
Preparamos dos versiones de los datos: una usando solo preprocesamiento básico y la otra usando preprocesamiento orientado a sentimientos (con la excepción de la lematización). Para estos dos conjuntos de datos, las incrustaciones de omisión de gramática se entrenaron utilizando diferentes configuraciones de parámetros, incluyendo una serie de dimensiones, tamaño de n-gramas de palabras y subpalabras, y tamaño de la ventana de contexto.

\subsection{Tweet Embeddings}
\label{sec:sif}

% Hay varias formas de usar incrustaciones de palabras para el análisis de sentimientos en tweets: los enfoques van desde el simple promedio de vectores para cada palabra en el tweet hasta el uso de arquitecturas más complejas como CNN o RNN. En este trabajo,
Se utilizaron combinaciones lineales para calcular una representación de un solo tweet.
Seguimos dos enfoques simples: promedio simple y promedio ponderado. En el segundo caso, utilizamos un esquema que se asemeja a la frecuencia inversa suave (SIF) \cite {arora17}, inspirado en la reponderación de TF-IDF.
Cada palabra $ w $ se pondera con $ \frac {a} {a + p (w)} $, donde $ p (w) $ es la palabra probabilidad unigrama y $ a $ es un hiperparámetro de suavizado.
Los valores altos de $ a $ significan más suavizado hacia el promedio simple.

% También consideramos dos opciones que afectan las incrustaciones de tweets: binarización, que ignora las repeticiones de tokens en los tweets; y normalización, que escalas dando como resultado que los vectores de tweets tengan una norma unitaria.


\subsection{Embeddings contextualizados}
\label{subsec:elmo}

Después del gran salto adelante que representó las incrustaciones de palabras independientes del contexto, llegó una nueva ola en los últimos años. En lugar de tener vectores entrenados para cada palabra, se generan representaciones dependientes del contexto para cada token dada una oración. Por ejemplo, \citet{mccann2017learned} usó un codificador LSTM profundo para traducción automática para generar vectores sensibles al contexto.

\elmo{} \cite{peters2018} es uno de estos enfoques dependientes del contexto y se basa en un modelo de lenguaje bidireccional profundo (biLM). La arquitectura del modelo de lenguaje consta de L capas de LSTM bidireccionales, además de una representación de token independiente del contexto. Por lo tanto, para cada token en una secuencia, obtenemos representaciones vectoriales de $ 2L + 1 $.
% Estas representaciones se consideran profundas ya que utilizan la salida de cada capa LSTM.
Para obtener un vector final para cada token, los autores sugieren colapsar las capas en vectores mediante una combinación lineal.

% Sea $ t_1, \ldots, t_n $ una secuencia de tokens, y sea $ h_ {k, j} $ el vector que representa la salida de la capa $ j $ cuando se consume el token $ t_k $. Entonces, el vector contextualizado para el token $ k $ es:
%
% \begin {ecuación}
% ELMo_k ^ {tarea} = \gamma ^ {tarea} \sum_ {j = 0} ^ {L} s_j h_ {k, j} \label {eq: elmo}
% \end {ecuación}

En este trabajo, usamos la implementación y los modelos entrenados previamente de \cite{che-EtAl:2018:K18-2}. El modelo español se entrenó con $L = 2 $ capas y 1024 dimensiones, y la combinación lineal se realizó utilizando un promedio simple.

\section{Resultados de clasificación}

\subsection{Análisis de Error}

\subsection{Intepretabilidad}

\section{Limitaciones}

\subsection{Problemas de anotación}

\subsection{Falta de contexto}

\subsection{Interpretabilidad y fragilidad de clasificadores}
%
% \chapter{Limitaciones de todo: metodológicos, de los corpus, de los modelos.}
% \section{Anotación y sus limitaciones}

\section{Falta de contexto}

Citar paper

\section{Interpretabilidad y fragilidad de clasificadores}
\part{Detección contextualizada de discurso de odio}
%
\chapter{Construcción de un dataset de discurso de odio contextualizado}
\label{chap:dataset_creation}

En este capítulo describiremos la construcción de un dataset contextualizado de discurso de odio. Describiremos en detalle el proceso de recolección, selección y anotación de datos.

Por lo marcado en anteriores secciones, consideramos interesante el problema de analizar el impacto del contexto en la detección de lenguaje discriminatorio. Para citar un ejemplo de por qué es necesario, el mensaje ``sos un hombre'' en solitario puede parecer inofensivo; ahora, si ese mismo mensaje está dirigido hacia una mujer trans, su contenido es claramente discriminatorio. Para analizar esto, nos abocamos a la decisión de crear un dataset que no sólo contenga un mensaje/comentario, sino que provea un contexto en el cual se da este mensaje. Un ámbito natural para esta tarea son las notas periodísticas, donde disponemos de una nota y comentarios realizados sobre ésta.

Muchos sitios de noticias disponen de sistemas embebidos de comentarios, pero vista la dificultad para la recolección a la vez que los limitados datos provistos por estos sitios nos llevaron a buscar otro medio: Twitter. Twitter provee una sencilla API para descargar datos, a la vez que tiene términos y condiciones amigables para poder publicar estos datos. Así mismo, esta red social opera de una manera similar a un foro de comentarios de un sitio de noticias. Este tipo de datos (comentarios sobre artículos periodísticos) tiene una naturaleza particular, ya que las agresiones discriminatorias son usualmente a personajes públicos o colectivos de personas, y se dan de manera indirecta (a través del comentario en la noticia) y no directa (es decir, como respuesta al usuario de Twitter ofendido).

El trabajo realizado en este capítulo tuvo lugar en el contexto de un Proyecto Interdisciplinario de la UBA\footnote{\url{https://cyt.rec.uba.ar/vinculacion-transferencia/piuba/}} junto a sociólogos, abogados, lingüistas, y computólogos. Particularmente, el trabajo de la construcción del manual de etiquetado fue discutido en conjunto, contemplando varias perspectivas a la hora de armar una definición propia (algunas de estas ya fueron vertidas en la discusión en la sección \ref{sec:hate_speech_definitions}). Teniendo en cuenta que un alto porcentaje de trabajos del área de detección de discurso de odio (y de manera más importante, en la construcción de sus recursos) mediante técnicas de NLP no abordan una mirada interdisciplinaria, es un aspecto a remarcar de lo realizado en la construcción de este dataset.

\section{Trabajos previos}
\label{sec:dataset_previous}

Pocos trabajos del área de detección de lenguaje abusivo o discurso de odio incorporan algún tipo de contexto a los comentarios del usuario para estas tareas. En esta sección haremos un repaso de los trabajos que han abordado esto de alguna manera. \citet{gao-huang-2017-detecting} construyó un dataset de lenguaje discriminatorio sobre 1518 comentarios del sitio de Fox News. A los anotadores les fue presentado tanto el comentario como la noticia a la hora de realizar el etiquetado. Sobre este dataset, los autores efectuaron experimentos de clasificación usando modelos lineales (regresiones logísticas) y modelos neuronales. En estos experimentos, observaron que un clasificador (tanto lineal como neuronal) mejora su performance al consumir el título de la noticia, dando indicios de que se puede aprovechar el contexto para mejorar la detección de este fenómeno. Sin embargo, como marca \citet{pavlopoulos2020toxicity} este trabajo cuenta con algunos problemas: en primer lugar, el tamaño del dataset es pequeño, y está extraído de sólo 10 noticias, lo cual limita fuertemente los posibles contextos. A su vez, la anotación fue realizada mayormente por una única persona, lo cual hace poco confiables las etiquetas obtenidas. Luego, algunos detalles menores debieran ser analizados con mayor detalle, como por ejemplo la utilización de los nombres de usuarios como features predictivas.

\citet{mubarak-etal-2017-abusive} construyó un dataset en árabe sobre comentarios con contenido abusivo del portal Al Jazeera. Sin embargo, este daaset tiene un problema: los comentarios son sólo presentados a los anotadores sobre noticias, ignorando todo el thread de la conversación. Esto hace que el contexto sea parcial.


Paralelamente a nuestro trabajo, \citet{pavlopoulos2020toxicity} analiza el impacto de agregar contexto a la tarea de detección de toxicidad. En particular, plantea dos preguntas

\begin{itemize}
    \item ¿Qué tanto afecta el contexto a la toxicidad percibida por humanos en conversaciones online?
    \item ¿Puede el contexto ayudar a mejorar la performance de clasificadores de toxicidad en comentarios?
\end{itemize}

Para responder estas preguntas, los autores construyeron dos datasets en base a Wikipedia Talk Pages\cite{hua-etal-2018-wikiconv}, un dataset de discusiones del sitio de Wikipedia. En primer lugar, armaron un dataset de 250 comentarios anotados por dos grupos disjuntos de anotadores: uno de los grupos anotó los comentarios de manera contextualizada, viendo tanto el comentario en cuestión como el título de la discusión; el otro grupo sólo vio el comentario a anotar sin contexto alguno. En dicho experimento observaron que los anotadores contextualizados percibieron 6.4\% de comentarios tóxicos versus un 4.4\% de quienes anotaron sin contexto, una diferencia significativa aplicando un test Mann-Whitney. Desagregando estos resultados, observaron que 13 de los 250 comentarios (5.2\%) tuvieron diferencias de anotación entre los dos grupos, con 9 (3.6\%) comentarios donde aumentó la toxicidad percibida y 4 comentarios donde bajó la toxicidad al ser agregado el contexto.

Para responder la segunda pregunta, anotaron un dataset de 20k comentarios, 10k anotados por un grupo que etiquetó viendo el contexto y otros 10k que no lo vio. Entre todos los comentarios del dataset original de Wikipedia Talk Pages, eligieron aquellos con profundidad entre 2 (respuestas directas) a 5, y con entre 10 y 400 caracteres de largo. Luego, entrenaron varios clasificadores usando este dataset y allí pudieron observar que el contexto no pareciera mejorar la performance. En el próximo capítulo nos extenderemos sobre las técnicas utilizadas por este trabajo.

\citet{xenos-2021-context} continúa el trabajo de \citet{pavlopoulos2020toxicity} desagregando el resultado de la segunda pregunta. Puntualmente, y observando que sólo un porcentaje pequeño de los comentarios parecen ser incididos por el contexto en el trabajo previo, construyen una nueva tarea: estimación de sensibilidad al contexto. Para ello, toman el dataset de Civil Comments\cite{borkan2019civil}, y reanotan un subconjunto de este dataset usando información de contexto a través de crowdsourcing. Las etiquetas de este dataset son de toxicidad en un estilo similar a una regresión ordinal, entendiendo las categorías no tóxico, incierto, tóxico, y muy tóxico. Ahora, teniendo las anotaciones originales del dataset (que fueron hechas sin contexto) y las nuevas anotaciones, pueden definir para cada comentario una sensibilidad al contexto, dada por

\begin{equation}
    \delta(p) = s^{oc}(p) - s^{ic}(p)
\end{equation}

donde $s^{oc}$ es la fracción de anotadores sin contexto que marcaron toxicidad, y $s^{ic}$ los que no tienen contexto.

\citet{sheth2021defining}, en un trabajo muy reciente, señala algunas oportunidades y desafíos  para incorporar fuentes de información más ricas a la tarea de detección de toxicidad. Por ejemplo, incorporar información como el background socio-cultural de los interactores puede ayudar a distinguir algunos tipos de reapropiación de términos potencialmente catalogados como tóxicos. Así mismo, el historial de interacción entre los usuarios puede ayudar a distinguir interacciones abusivas de charlas amistosas entre amigos que usan vocabulario potencialmente tóxico. Finalmente, se promueve el uso de contenido externo para acercarse lo más posible al conocimiento humano a través de conocimiento del contenido, el individuo (atacado) y la comunidad. Para ello, se promueve el uso de bases de conocimiento y knowledge-infusion learning \cite{gaur2020infusion} para combinar el cómputo neuronal y simbólico.



\citet{wiegand2021implicitly} menciona formas implícitas de abuso, mucho más complejas que las basadas solamente en palabras ofensivas. Por ejemplo, deshumanizaciones (``los judíos son una plaga que merece ser eliminada''), llamadas a la acción (``hay que tirar una bomba en ese país''), acusaciones (``los chinos inventaron el coronavirus''), entre otros tipos sutiles de comportamiento tóxico. Así mismo, menciona que la mayoría de los datasets no consiguen capturar estos fenómenos debido a la forma de recolección usualmente basada en keywords.

\section{Esquema del dataset}


%%
%%
%% Link a Draw
%% https://docs.google.com/drawings/d/1IcBITgNJN-tehmvnZqcSF9cUuWIpNKJg6yHI5yjNF9c/edit
%%
%%

\begin{figure}[ht]
    \centering
    \includegraphics[width=\textwidth]{img/idea_dataset.pdf}
    \caption{Muestra de la recolección de datos}
    \label{fig:idea_dataset}
\end{figure}

Para construir un dataset contextualizado barajamos varias opciones. Como vimos en otros datasets, se puede entender el contexto de varias maneras: un contexto ``temático'', donde sabemos que cierto comentario habla sobre un tema en particular; y un contexto conversacional, donde tenemos una secuencia de comentarios (un hilo o thread) y podemos extraer un comentario padre para cada uno salvo el raíz. La primer opción es la explorada por \cite{gao-huang-2017-detecting,mubarak-etal-2017-abusive}, donde construyen un dataset de comentarios de Fox News y Al-Jazeera respectivamente. El contexto conversacional, como hemos relatado anteriormente, es explorado en \citet{pavlopoulos2020toxicity,xenos-2021-context}; sin embargo, como es marcado en el primer trabajo, la recolección de datos es no trivial, aún en un caso más amplio como el lenguaje abusivo, ya que la incidencia es relativamente baja. Puede esperarse que en el contexto de lenguaje odioso se dificulte aún más esto.

Para analizar el contexto, decidimos entonces usar la primera opción: comentarios sobre notas periodísticas. No vamos a considerar un hilo de respuestas, sino simplemente aquellos comentarios que sean directos sobre la nota. En ese punto, el dataset que queremos construir sería similar al de \cite{gao-huang-2017-detecting}. Una diferencia respecto a este dataset sería la de incorporar dos modos de contexto: uno corto, donde sólo tengamos el título de la noticia; y uno largo, donde tengamos el texto completo de la noticia.

El dataset construido será sobre comentarios realizados en idioma español, más precisamente en la variedad dialectal del Río de la Plata. Como dice la ``Regla de Bender''\cite{bender2011achieving}

\begin{quote}
    Do state the name of the language that is being studied, even if it's English. Acknowledging that we are working on a particular language foregrounds the possibility that the techniques may in fact be language specific. Conversely, neglecting to state that the particular data used were in, say, English, gives [a] false veneer of language-independence to the work.
\end{quote}

Este punto es importante ya que, a pesar de ser el segundo idioma en hablantes nativos (por delante del inglés), los recursos suelen ser escasos y siempre a la rastra y reproducción de resultados en inglés. \todo{quizás esto lo mandaríamos a otro lado}

Algo no menor a la hora de considerar la construcción del dataset es la posibilidad de publicar los datos. Por citar un ejemplo, el dataset de \citet{gao-huang-2017-detecting}, si bien tiene sus datos de acceso público \footnote{\url{https://github.com/sjtuprog/fox-news-comments}}, no queda claro que los términos y condiciones de la fuente de donde se extrajeron permita esto. Más aún, si hubiésemos querido extraerlo de múltiples fuentes (por ejemplo, varios diarios), deberíamos chequear y/o acceder a permisos para cada sitio, a la vez que tendríamos el problema de tener fuentes diversas de los datos (diferentes longitudes, metadatos distintos, entre otras).

Para evitar muchos de estos problemas, y reutilizar muchas cuestiones con las que venimos trabajando en esta tesis, decidimos trabajar sobre comentarios hechos por usuarios en Twitter. Concretamente, sobre respuestas de comentarios de usuarios a posteos hechos por cuentas de medios. De alguna manera, esto emularía un foro de comentarios de medios, tendríamos un formato único para comentarios mientras tenemos diferentes ``audiencias''. La Figura \ref{fig:idea_dataset} ilustra esta idea. A su vez, los términos y condiciones de Twitter nos permiten publicar los datos \todo{Agregar algún link a esto}. Las notas periodísticas las descargaremos pero debido a problemas de copyright no serán publicados.



\subsection{Proceso de construcción}

Dividiremos la construcción del dataset en tres etapas:

\begin{enumerate}
    \item Recolección: Proceso de recolección de datos de Twitter y de los artículos periodísticos
    \item Selección: Dado el conjunto de artículos y comentarios recolectados, tomar una muestra de artículos y comentarios a etiquetar
    \item Anotación: Proceso de etiquetado de los artículos seleccionados
\end{enumerate}

Si bien en muchos casos las dos primeras etapas suelen ser la misma o bien la selección se limita a una muestra aleatoria de la recolección, este procedimiento sería muy ineficiente en el caso de discurso de odio. Esto se debe a que en nuestro dominio de comentarios periodísticos y discurso de odio, encontramos este tipo de discurso distribuido de manera muy poco uniforme, usualmente concentrada alrededor de ciertos tópicos. Para construir un dataset con una proporción no marginal del fenómeno estudiado, estudiamos algunas posibilidades para seleccionar los artículos y sus respectivos comentarios.

En algunos trabajos previos (como por ejemplo \citet{waseem2016hateful,hateval2019semeval}) la recolección y selección constan conjuntamente de usar ciertos keywords y, o bien recolectar tweets que usen esas palabras, o bien sirven para preseleccionar usuarios de los cuales luego extraer tweets para ser etiquetados.

En nuestro caso, la selección de artículos y comentarios presenta cierta novedad y complejidad, con lo cual separamos este procedimiento para explicarlo detalladamente en las siguientes secciones.

\section{Definición de discurso de odio}
\label{sec:our_hate_speech_definition}
\begin{table}[t]
    \centering
    \begin{tabularx}{\textwidth}{l X}
        Característica & Descripción \\
        \hline
        MUJER        & Misoginia, agresiones basadas en ser mujer  \\
        LGBTI        & Homofobia, transfobia, y ofensas a la comunidad LGBTI \\
        RACISMO      & Racismo, Xenofobia, Judeofobia, etc \\
        POBREZA      & Basado en su condición de clase \\
        POLITICA     & En base a la filiación política del agredido \\
        ASPECTO      & Gordofobia, gerontofobia \\
        CRIMINAL     & Criminales, presos, y personas en conflicto con la ley \\
        DISCAPACIDAD & Discapacidades y adictos a sustancias

    \end{tabularx}
    \label{tab:caracteristicas_protegidas}
    \caption{Características protegidas consideradas en este trabajo}
\end{table}


Teniendo en cuenta la discusión realizada en la sección \ref{sec:hate_speech_definitions} realizamos nuestra propia definición de discurso de odio. Entendemos que hay discurso de odio en un texto social si éste contiene declaraciones de carácter intenso e irracional de rechazo, enemistad y aborrecimiento contra un individuo o contra un grupo, siendo estos objetivos de estas expresiones por poseer (o aparentar poseer) una característica protegida. Esta expresión puede manifestarse de manera explícita como insultos directos, celebraciones de crímenes, incitaciones a tomar medidas contra el individuo o grupo, o también expresiones más veladas. Siempre, considerando, que no es necesario solamente un insulto o una agresión: es necesario hacer una apelación explícita o implícita a al menos una característica protegida.

A diferencia de otros trabajos, nuestra definición comprende varias características, incluso algunas que están en la frontera de ser ``protegidas''. Mientras en otros trabajos se centran mayormente en racismo y misoginia, aquí agregaremos homofobia y transfobia, odio de clase (``aporofobia''), por su aspecto físico, y otras. En particular, hay dos características no convencionales que tuvimos en cuenta. En primer lugar, el discurso de odio ``político'', que de acuerdo a XXX \todo{citation needed}, es difícil considerar como protegida ya que puede dar lugar a censuras. Por otro lado, también consideramos el discurso de odio contra criminales, presos, y otras personas en situación de conflicto con la ley. Si bien este punto ni siquiera es considerado como una característica protegida en ninguno de los trabajos mencionados en la sección \ref{sec:hate_speech_definitions}, al haber tanto contenido que incita a la violencia contra criminales en las noticias de policiales, agregamos esta característica. Así mismo, esta característica puede ser de utilidad ya que nos interesa recoger incitaciones a la violencia, y este rubro es prolífico en ello en las redes.

Tenemos entonces 8 características que agrupan tipos de discurso de odio: contra las mujeres; racismo y xenofobia; contra la comunidad LGBTI; odio de clase; gordofobia, gerontofobia y demás odio por aspecto; por su ideología política; y finalmente contra discapacitados y adictos. Las características en cuestión son listadas en la tabla \ref{tab:caracteristicas_protegidas} junto a acrónimos que usaremos en el resto del capítulo.







\section{Recolección de datos}

\begin{table*}[t]
    \centering
    \large
    \begin{tabular}{ l l r }
        Nombre     &  username          & \#Followers \\
        \hline
        La Nación  &  @LANACION         & 3.6M            \\
        Clarín     &  @clarincom        & 3.2M        \\
        Infobae    &  @infobae          & 3.0M   \\
        Perfil     &  @perfilcom        & 0.81M    \\
        Crónica    &  @cronica          & 0.80M     \\
        \hline
    \end{tabular}
    \caption{Cuentas de medios utilizadas para la recolección de datos, junto a sus nombres de usuarios y la cantidad de seguidores en Twitter (al momento de la recolección)}
    \label{tab:medios_analizados}
\end{table*}


En esta sección detallaremos el proceso de recolección de datos, cuya salida es un conjunto de artículos mencionados en Twitter y sus comentarios respectivos realizados por usuarios. Describiremos a continuación las decisiones realizadas respecto a las fuentes y a otros detalles técnicos.

En primer lugar, limitamos nuestra recolección de datos a cuentas de medios de la República Argentina y, puntualmente, nos centramos en diarios con comunidad mayormente rioplatense. Esto lo realizamos teniendo en mente que los anotadores serían nativos de esta variedad dialectal ya que, como mencionamos anteriormente, el discurso de odio contra mujeres, grupos nacionales y otros depende fuertemente de la jerga y de las variaciones dialectales de cada lugar. Esta elección, se debe además a que, habiendo buscando en otros medios de Argentina (como por ejemplo ``La voz del Interior'', diario dirigido mayormente a un público fuera de la Metrópolis de Buenos Aires) observamos que la interacción en Twitter de estos medios es muy baja, con muy pocos usuarios comentando sus notas. Centrándonos en diarios que generen interacción, seleccionamos medios periodísticos de gran llegada y tradicionales, los cuales listamos en la Tabla \ref{tab:medios_analizados}.

Si bien recolectamos notas de otros medios, no los consideraremos a partir de ahora, y los dejamos para análisis posteriores. De los cinco medios elegidos, todos son medios formales y con varios años en el medio, siendo cuatro de ellos con soporte escrito y uno sólo (Infobae) enteramente digital. Consideramos la posibilidad de elegir medios no tradicionales y más orientados a grupos de la ``derecha alternativa'', dada su alta incidencia de contenido de odio. Sin embargo, finalmente tomamos la decisión de descartarlos de la etapa de anotación.


\subsection{Método de recolección}



La API de Twitter, en su versión gratuita, nos brinda dos modos de recolectar tweets de su plataforma\footnote{Usamos la versión 1.1 de la API. La versión 2.0 parece facilitar la recopilación de conversaciones. Recomendamos investigar mejor esta versión actualizada para esquivar muchas de las dificultades técnicas que incursionamos para lo descripto en esta sección}:

\begin{enumerate}
    \item Search API: permite buscar tweets en base a términos, de hasta 15 días atrás sobre una pequeña muestra, recreando lo que vemos en la UI de Twitter
    \item Stream API: permite buscar tweets en tiempo real sobre una muestra de cerca del 1\% de todos los tweets de la red social
\end{enumerate}

La API Stream (también conocida como Spritzer), mientras por un lado limita temporalmente la recolección de datos, por el otro nos brinda la posibilidad de recolectar una mayor cantidad de información en tiempo real. Más aún, dada la naturaleza de nuestros datos (discurso de odio), se corre el riesgo de que con el tiempo sean moderados e inaccesibles para cualquier búsqueda con la API Search.

Por lo explicado, usamos la API de Twitter Stream mencionando cualquiera de estas cuentas. Si estamos entonces recolectando tweets sobre \verb|@medio|, el proceso de recolección nos da:

\begin{enumerate}
    \item Tweets de \verb|@medio|
    \item Respuestas a los tweets de \verb|@medio|
    \item Tweets de terceros que mencionan a \verb|@medio|
    \item Retweets (RT) de tweets de \verb|@medio|
    \item Citas de tweets de \verb|@medio|
\end{enumerate}

Los RTs y tweets que arroben a \verb|@medio| carecen de interés para nuestro estudio, con lo cual los descartamos. Por otro lado, también descartamos las citas, aunque podrían entenderse como ``respuestas'' a los tweets originales. Nos quedamos con tweets de \verb|@medio| y las respuestas a estos. Si bien la API nos da estos tweets desestructuradamente, reconstruimos el árbol de la discusión mediante el campo \verb|in_reply_to_status_id|\footnote{Ver la documentación y la referencia al campo en \url{https://developer.twitter.com/en/docs/twitter-api/v1/data-dictionary/object-model/tweet}}.

Algo importante a remarcar es que para el propósito de este trabajo, solo estamos interesados en el primer nivel de respuestas al tweet original, y no incorporaremos hilos de respuestas. Trabajo futuro debería explorar este nivel adicional de complejidad incorporando contexto conversacional adicional.

Accidentalmente, la recolección de datos se dio al mismo tiempo del estallido de la pandemia del COVID-19. Por ese motivo, y dadas las implicancias de la pandemia sobre el discurso discriminatorio en las redes sociales, se volcó el foco hacia artículos relacionados con el coronavirus. Para ello seleccionamos artículos buscando una cantidad de palabras en su cuerpo, por lo que seleccionamos específicamente artículos relacionados con COVID-19. Utilizamos las siguientes palabras: coronavirus, encierro, síntomas, covid, fase, fiebre, cuarentena, infectados, distanciamiento, normalidad,  Wuhan, aislamiento.

Por último, nos quedamos con aquellos tweets de los medios periodísticos que tuvieran un link a un artículo. Para ello, utilizamos la librería \emph{newspaper3k}\footnote{\url{https://newspaper.readthedocs.io/en/latest/}}, que nos permite acceder a la información relacionada a los artículos en cuestión, en particular siendo lo que más nos interesa el cuerpo del artículo. Esto vamos a utilizarlo posteriormente como el contexto ``largo'' para los comentarios. Aquellos tweets de medios periodísticos que no contengan un link a un artículo fueron descartados de las siguientes etapas.

\subsection{Datos recolectados}

\begin{table}[t]
    \centering
    \begin{tabular}{l|c|c}
    Medio      & \#Artículos & \#Comentarios \\
    \hline
    @infobae   &  45,652   &  822,462 \\
    @clarincom &  29,050   &  672,401 \\
    @perfilcom &  8,764    &  61,203  \\
    @LANACION  &  16,040   &  506,091 \\
    @cronica   &  17,250   &  70,872 \\
    \hline
    Total      & 116,756  & 2,133,029 \\
    \end{tabular}
    \caption{Artículos recoletados por medio}
    \label{tab:articulos_recoletados_por_medio}
\end{table}


La tabla \ref{tab:articulos_recoletados_por_medio} contiene los números de los artículos recolectados por cada medio, luego de aplicado el filtro de palabras mencionado en la anterior sección. Si bien recolectamos más artículos de otros medios, no son enumerados. Infobae es el medio que más producción de artículos genera, y también será finalmente sobre el que más comentarios etiquetemos.

En el apéndice \ref{app:distribucion_datos} mostramos la distribución temporal de los datos. Si bien tenemos un pequeño gap en los datos por un problema técnico en la recolección, tenemos datos desde Marzo de 2020 hasta Febrero de 2021.

En siguientes secciones realizaremos un filtrado de la mayoría de estos artículos previamente a la anotación, pero este conjunto de datos no filtrado será utilizado para efectuar ajustes de dominio, y es liberado como se recomienda en \citet{gururangan-etal-2020-dont}. Hablaremos más sobre esto en los capítulos \ref{chap:06_contextualized_hate_speech} y \ref{chap:07_domain_adaptation}.


\section{Selección de datos a anotar}


Un problema que se presenta antes de comenzar el etiquetado es el de seleccionar los artículos que vamos a etiquetar, teniendo en consideración la gran cantidad de datos recolectados y los recursos disponibles. Una primera posibilidad para esto es realizar una selección aleatoria de artículos y comentarios. Sin embargo, los comentarios discriminatorios no se distribuyen de manera uniforme entre los artículos sino que se concentran sobre algunos temas que generan este tipo de contenido. Es mucho más probable encontrar comentarios de índole discriminatoria en notas que tengan temas cercanos a alguna de las características protegidas: por ejemplo, es esperable encontrar contenido discriminatorio en notas sobre China y el Coronavirus o sobre una chica transgénero antes que en un artículo de fútbol o economía. Si bien una selección aleatoria preservaría una tasa de incidencia mucho más cercana a la observada en el universo de comentarios, es más importante poder obtener una mayor cantidad de observaciones que reflejen el fenómeno estudiado.

Teniendo esto en cuenta, evaluamos varias alternativas para realizar la selección de artículos. La primera fue intentar seleccionar aquellos artículos que consideramos como candidatos a fomentar contenido discriminatorio. Una posibilidad para esto sería usar algunas palabras semilla para seleccionar artículos interesantes en base a ciertos temas que consideramos relevantes.

Otra posibilidad evaluada fue la de buscar directamente comentarios que marquen que ese artículo suscita contenido discriminatorio. Para ello, podemos listar algunos insultos comunes o expresiones peyorativas hacia los grupos protegidos considerados. Es necesario remarcar que esto lo hacemos para seleccionar \tbf{artículos} y no los comentarios que contengan esos insultos; hacer esto último nos genera una muestra muy distorsionada y tendiente a encontrar el fenómeno más explícito de la discriminación (el insulto racista, homofóbico, etc.). Esta estrategia guarda relación con la descripta por \citet{hateval2019semeval} para seleccionar usuarios generadores de contenido discriminatorio.

Describimos a continuación las alternativas analizadas para seleccionar los artículos y sus respectivos comentarios.


\subsection{Selección en base a artículos}

\begin{table}[b!]
    \centering
    \small
    \begin{tabular}{p{0.21\textwidth}  p{0.26\textwidth} p{0.23\textwidth} p{0.19\textwidth}}
    \hline
    China        &  piqueteros              &  mamá                & domésticas            \\
    Cuba         &  villas                  &  de género           & la modelo             \\
    cubano       &  la villa                &  aborto              & la periodista         \\
    bolivia      &  movimientos sociales    &  actriz              & la cantante           \\
    paraguayo    &  organizaciones sociales &  actrices            & travesti              \\
    judío        &  tomas de tierras        &  feminista           & trans                 \\
    camionero    &  toma de tierras         &  femicidio           & gay                   \\
    ladrón       &  sindicatos              &  enfermera           & homosexual            \\
    represión    &  Guernica                &  madre               & de la V               \\
    criminal     &  mapuches                &  Ofelia              &                       \\
    \hline
    \end{tabular}
    \caption{Palabras semilla utilizadas para la selección de artículos. Cada palabra se busca sobre el cuerpo del artículo candidato a ser etiquetado}
    \label{tab:palabras_articulos}
\end{table}

En primer lugar, consideramos la posibilidad de hacer una selección en base al contenido de los artículos. Luego de hacer un análisis exploratorio de los datos usando LDA \cite{blei2003latent} para buscar tópicos posibles de las notas, decidimos realizar una selección controlada y determinística en base a la utilización de palabras y expresiones clave. Estas expresiones las recolectamos de manera subjetiva y en base a la observación de los tópicos y de nuestra percepción de la generación de discurso discriminatorio en los comentarios de los usuarios.

La Tabla \ref{tab:palabras_articulos} muestra el conjunto de expresiones utilizado para recolectar artículos. Como vemos, hay diversas palabras que recogen temáticas de posibles tópicos generadores de contenido discriminatorio, algunos muy locales respecto a eventos concretos durante la pandemia. Si algún artículo contiene una de las expresiones mencionadas, es seleccionado para ser etiquetado.

Para realizar esta búsqueda de términos en el cuerpo de los artículos, indexamos los textos en \emph{MongoDB}\footnote{\url{https://www.mongodb.com/}}, una base de datos no relacional. Este motor de bases de datos permite la utilización de índices en base a texto, permitiendo realizar búsquedas en base a expresiones, palabras, e inflexiones.



\subsection{Selección en base a comentarios}
\label{subsec:seleccion_comentarios}


\begin{table*}[h]
    \centering
    \small
    \begin{tabular}{l l l l l l l}
    \hline
    bija          & urraca     & viejo puto    & trolo      & peruano  & matarlos         & negra      \\
    prostituta    & tucán      & trabuco       & sodomita   & peruca   & una bomba        & negro de   \\
    feministas    & putita     & travesti      & chinos de  & judío    & vayan a laburar  & negros     \\
    feminazis     & reventada  & trava         & bolita     & sionista & vayan a trabajar & bala       \\
    aborteras     & marica     & degenerado    & paraguayo  & villeros & gorda            & uno menos  \\
    \hline
    \end{tabular}
    \caption{Palabras utilizadas para recolectar comentarios. Cada palabra se busca sobre el texto de un comentario para marcarlo como potencialmente discriminatorio.}
    \label{tab:palabras_comentarios}
\end{table*}

Otra posibilidad para seleccionar artículos candidatos a ser etiquetados es la de observar los comentarios de usuarios en lugar del texto completo de éste. En base a los comentarios, podemos tener alguna medida de si el artículo suscita reacciones potencialmente discriminatorias. Por ejemplo, si observamos que en un artículo hay comentaristas que usan expresiones discriminatorias contra la comunidad LGBTI, podemos pensar que el contenido de la noticia es interesante para nuestro estudio.

El procedimiento para este tipo de selección es similar al mencionado anteriormente con artículos, sólo que aplicado a comentarios: buscamos respuestas de usuarios que contengan alguna de las expresiones semilla listadas en la Tabla \ref{tab:palabras_comentarios}. Estas palabras fueron recolectadas de manera subjetiva en base a la observación y a la experimentación sobre los datos, tratando de contener diversas expresiones de contenido mayormente discriminatorio. La lista contiene expresiones ofensivas para diversas características de interés: insultos racistas, homofóbicos, misóginos; insultos dirigidos dirigidos a algún personaje particularmente atacado en las redes sociales; expresiones de odio de clase; etc.

Dado un artículo, marcamos los comentarios que contengan una o más de las expresiones listadas. Si el artículo tiene tres o más comentarios marcados, entonces el artículo es seleccionado; caso contrario, es descartado. Vale remarcar que este proceso de selección es para los \emph{artículos}, no para los comentarios. De lo contrario, sólo buscaríamos respuestas que contengan alguna de estas expresiones.

Luego de algunos análisis experimentales y observacionales de las dos posibles metodologías, decidimos utilizar el muestreo de artículos en base a comentarios. En base a un análisis subjetivo, los artículos seleccionados parecían tener mayor incidencia de mensajes discriminatorios y eso nos decantó hacia esa opción.

Una posibilidad adicional analizado fue utilizar un clasificador que nos señale posibles comentarios discriminatorios, usando esta información para seleccionar artículos candidatos a etiquetar. Para ello, aplicamos un clasificador basado en BETO \cite{canete2020spanish} entrenado sobre el dataset de \hateval{} (ver Sección \ref{chap:04_hate_speech}) sobre los comentarios de los artículos. Una evaluación subjetiva de esto nos dio pobres resultados, tanto porque no captaba algunas agresiones discriminatorias (de características no incluídas en el dataset de \citet{hateval2019semeval}) como muchos falsos positivos o errores debido al cambio de dominio (temático y también dialectal). Si bien descartamos este método, puede ser de relevancia usar algún método que no esté basado en palabras semillas o utilizar algún método semi-automático para encontrar candidatos a etiquetar.


\subsection{Muestreo de comentarios}

Una vez que seleccionamos los artículos, resta decidir qué comentarios vamos a anotar. No podemos seleccionar todos ya que muchos artículos cuentan con una cantidad importante de comentarios (en el orden de los cientos) y es deseable mantener un balance entre los comentarios anotados por artículo. Tampoco es deseable (en pos de maximizar el producto de la anotación) seleccionar comentarios de artículos escasamente discutidos. Teniendo esto en mente, conservamos sólo los comentarios de artículos que tengan al menos 20 comentarios. Luego, para cada artículo, seleccionamos aleatoriamente hasta 50 comentarios entre aquellos que no contengan URLs u otro contenido no textual.

En este punto, consideramos el muestreo aleatorio como la forma menos sesgada para seleccionar nuestros comentarios, pero mencionamos de todas formas algunas alternativas evaluadas. Una fue la de considerar todo el universo de comentarios y seleccionar la muestra de allí. Sin embargo, esto sobrerrepresentaría a aquellos temas muy comentados, siendo muchos de ellos acerca de temas políticos que se filtraron en nuestra selección. Otra consideración posible es la de utilizar información de usuarios y sus conexiones, información que Twitter nos brinda a través de los followers de cada usuario. Muchos usuarios que generan contenido discriminatorio en redes sociales se agrupan en comunidades: subgrafos de usuarios altamente conectados entre sí. Usar algún tipo de información sobre esto (por ejemplo, con algún algoritmo como el de Louvain \cite{blondel2008fast}) podría auxiliar al balance de comentarios posiblemente discriminatorios. Para un ejemplo de la utilización de esta técnica, \citet{lai2018stance} y \citet{furman2021you} usan este tipo de algoritmos como manera semi-supervisada de detectar las posturas de los usuarios respecto a distintos temas.


\section{Anotación}
\label{sec:criterios}

Una vez



\subsection{Definición de discurso de odio utilizada}

\begin{table}[t]
    \centering
    \begin{tabularx}{\textwidth}{l X}
        Característica & Descripción \\
        \hline
        MUJER        & Misoginia, agresiones basadas en ser mujer  \\
        LGBTI        & Homofobia, transfobia, y ofensas a la comunidad LGBTI \\
        RACISMO      & Racismo, Xenofobia, Judeofobia, etc \\
        POBREZA      & Basado en su condición de clase \\
        POLITICA     & En base a la filiación política del agredido \\
        ASPECTO      & Gordofobia, gerontofobia \\
        CRIMINAL     & Criminales, presos, y personas en conflicto con la ley \\
        DISCAPACIDAD & Discapacidades y adictos a sustancias

    \end{tabularx}
    \label{tab:caracteristicas_protegidas}
    \caption{Características protegidas consideradas en este trabajo}
\end{table}


Teniendo en cuenta la discusión realizada en la sección \ref{sec:hate_speech_definitions} realizamos nuestra propia definición de discurso de odio. Entendemos que hay discurso de odio en un texto social si éste contiene declaraciones de carácter intenso e irracional de rechazo, enemistad y aborrecimiento contra un individuo o contra un grupo, siendo estos objetivos de estas expresiones por poseer (o aparentar poseer) una característica protegida. Esta expresión puede manifestarse de manera explícita como insultos directos, celebraciones de crímenes, incitaciones a tomar medidas contra el individuo o grupo, o también expresiones más veladas. Siempre, considerando, que no es necesario solamente un insulto o una agresión: es necesario hacer una apelación explícita o implícita a al menos una característica protegida.

A diferencia de otros trabajos, nuestra definición comprende varias características, incluso algunas que están en la frontera de ser ``protegidas''. Mientras en otros trabajos se centran mayormente en racismo y misoginia, aquí agregaremos homofobia y transfobia, odio de clase (``aporofobia''), por su aspecto físico, y otras. En particular, hay dos características no convencionales que tuvimos en cuenta. En primer lugar, el discurso de odio ``político'', que de acuerdo a XXX \todo{citation needed}, es difícil considerar como protegida ya que puede dar lugar a censuras. Por otro lado, también consideramos el discurso de odio contra criminales, presos, y otras personas en situación de conflicto con la ley. Si bien este punto ni siquiera es considerado como una característica protegida en ninguno de los trabajos mencionados en la sección \ref{sec:hate_speech_definitions}, al haber tanto contenido que incita a la violencia contra criminales en las noticias de policiales, agregamos esta característica. Así mismo, esta característica puede ser de utilidad ya que nos interesa recoger incitaciones a la violencia, y este rubro es prolífico en ello en las redes.

Tenemos entonces 8 características que agrupan tipos de discurso de odio: contra las mujeres; racismo y xenofobia; contra la comunidad LGBTI; odio de clase; gordofobia, gerontofobia y demás odio por aspecto; por su ideología política; y finalmente contra discapacitados y adictos. Las características en cuestión son listadas en la tabla \ref{tab:caracteristicas_protegidas} junto a acrónimos que usaremos en el resto del capítulo.

En el apéndice \ref{app:manual_criterios_anotacion} puede encontrarse el manual de etiquetado, con descripciones detalladas y ejemplos de lo que consideramos que configura discurso de odio para cada característica.

\subsection{Modelo de etiquetado}

Un modelo de anotación es, según \citet{pustejovsky2012natural}, una representación práctica del objetivo de anotación. En nuestro caso, queremos marcar comentarios discriminatorios, marcar a qué grupos y/o características se está ofendiendo, y también identificar llamados a tomar alguna acción contra los objetos de esos discursos. Por lo pronto, haremos una definición que capture ese objetivo sin deternos demasiado en especificarlo formalmente (lo que llaman en ese libro ``especificación'').

\todo{Describir un poquito más lo que está mencionado en el review de Polletto}

\subsubsection{Modelo Jerárquico de Etiquetado}

\begin{figure}
    \centering
    \includegraphics[width=\textwidth]{img/modelosjerarquicos.png}
    \caption{Modelos jerárquicos de anotación. A la izquierda, tenemos el modelo jerárquico propuesto para HatEval \cite{hateval2019semeval}, a la derecha el modelo propuesto para OffensEval \cite{zampieri2019semeval2019}}
    \label{fig:modelos_offenseval_hateval}
\end{figure}


\citet{zampieri2019predicting} introdujeron un modelo jerárquico de anotación para la tarea de lenguaje ofensivo, utilizado en las competiciones OffensEval \cite{zampieri2019semeval2019} y hatEval \cite{hateval2019semeval}. La idea de la anotación jerárquica es realizar anotaciones adicionales sólo para algunos casos de anotaciones del nivel anterior.

En el caso de \emph{hatEval}, tenemos un primer nivel que consta de anotar si un tweet contiene o no lenguaje de odio (nivel 1). Si el tweet tiene lenguaje de odio, entonces anotamos si está dirigido a un individuo o a un grupo, y también anotamos si es agresivo o no (ambos nivel 2). En el caso de \emph{OffensEval}, primero anotamos si es ofensivo (nivel 1), luego si está dirigido o es un insulto no dirigido (nivel 2) y finalmente, si es dirigido y ofensivo, marcamos su objetivo (nivel 3). En la figura \ref{fig:modelos_offenseval_hateval} ilustramos ambos modelos.


%
%
% Link: https://docs.google.com/drawings/d/1ZgTmvRwMWn0B-kokfw87jfSa7eY5-OSBHwltetnNT08/edit
%



\begin{figure}
    \centering
    \includegraphics[width=0.5\textwidth]{img/Annotation Model.png}
    \caption{Modelo de anotación}
    \label{fig:annotation_model}
\end{figure}


La figura \ref{fig:annotation_model} muestra el modelo de anotación utilizado para el dataset construído en este trabajo. Seguimos un modelo jerárquico similar al propuesto por \citet{zampieri2019predicting}, aunque de sólo un nivel. Para cada comentario y su respectivo contexto (el artículo), requerimos una anotación  para decidir si el comentario es odioso o no. Si no es odioso, no se necesita más información. Si es así, el par artículo-comentario debe contener, además, una anotación por si llama o no a la acción, y al menos una categoría protegida





\subsection{Etiquetadores}

%
% Chequear https://docs.google.com/spreadsheets/d/1PaOVw_tKVRvjZIqRl2YKnaNsvX5tHJjjY0CV9PLrc6g/edit?resourcekey#gid=366330815
%

\begin{table}[t]
    \centering
    \small
    \begin{tabularx}{\textwidth}{l l l l l l l l}
        Género      & Edad   & Estudios           & Área          & Identificación    & ¿Activista?   & Experiencia\\
        \hline
        F    & 27     & Doctorado*          & Psicología    & Mujer             & No                   & Sí         \\
        NB   & 33     & Grado*              & Artes         & LGBTTIQ           & No                   & No         \\
        F    & 30     & Grado*              & Antropología  & Mujer, LGBTTIQ    & Feminista            & Sí         \\
        M    & 38     & Grado               & Sociología    & No                & No                   & No         \\
        F    & 36     & Doctorado           & Psicología    & Mujer             & No                   & No         \\
        F    & 34     & Grado               & Comunicación  & No                & Migrantes            & No         \\
        \hline
    \end{tabularx}
    \label{tab:informacion_sobre_anotadores}
    \caption{Características protegidas consideradas en este trabajo}
\end{table}

A diferencia de otros trabajos (como hatEval \cite{hateval2019semeval}), decidimos por un lado, garantizar que nuestros anotadores estén más cercanos culturalmente al problema en cuestión, a la vez que tener mayor control del perfil de estos. Consideramos que el discurso de odio tiene un fuerte componente cultural, muchas veces expresado a través de jerga o expresiones dialectales muy particulares, y relacionado con noticias muy propias de esta región.

Para ello, reclutamos etiquetadores hablantes nativos, y estudiantes o graduados/as de carreras de ciencias sociales o humanidades, como ser Psicología, Sociología, Comunicación, Antropología, etc. Algo que particularmente nos interesó fue que no tengan conocimientos de inteligencia artificial, ``ciencia de datos'' ni relacionados, de manera de no sesgar su tarea. También, que sean usuarios asiduos de redes sociales.

El proceso de reclutamiento constó en una breve entrevista donde corroboramos que sean hablantes nativos, les describimos la tarea mientras le mostrábamos la herramienta de etiquetado. Finalmente, se les solicitó hacer una prueba paga de leer el manual de etiquetado y anotar 10 artículos. Esto lo hicimos para corroborar la calidad de los etiquetadores. No rechazamos ningún etiquetador en este proceso.

La tabla \ref{tab:informacion_sobre_anotadores} brinda información desagregada sobre los 6 etiquetadores. Finalmente, nuestros etiquetadores son altamente escolarizados, con 2 etiquetadoras con experiencia previa, y siendo 2 activistas.


\subsection{Tipos de anotación en otros trabajos}

Comentar otros trabajos acá

\begin{itemize}
    \item Davidson
    \item Waseem
    \item hatEval
    \item CONAN
    \item Gao (contextualizado)
    \item Context offensive (el de Google, y el griego)
\end{itemize}

\subsection{Proceso de etiquetado}

\subsection{Preprocesado y filtrado de los datos}

El preprocesado de los datos es muy básico: en los hechos, efectuamos el mismo preprocesamiento que en anteriores tareas, consistente en reemplazar handles de Twitter por un token especial \verb|@usuario| para evitar cualquier sesgo. Por ejemplo, si un usuario conocido como ``odiador'' (llamemos \verb|@hater|) retwittea la noticia y otro responde a ese RT, aparece ese nombre de usuario lo cual podría condicionar al etiquetador.

Así mismo, descartamos cualquier tweet que tuviera algún link ya que pueden referir a contenido no textual


\subsection{Entrenamiento de etiquetadores}



%
% Esto quizás va después
%
\subsection{Herramienta de etiquetado}

%%
%% Link a Google Draw: https://docs.google.com/drawings/d/1E24-2l6hsNj2JSKBZOD8QvZCJR6rrGjz-cWwt8XuPRg/edit
%%

\begin{figure}
    \centering
    \includegraphics[width=\textwidth]{img/labeler.pdf}
    \caption{Pantalla del etiquetador}
    \label{fig:labeler_example}
\end{figure}

Al no utilizar ningún servicio de etiquetado, optamos por desarrollar nuestra propia aplicación para el etiquetado de tweets. En ella, a cada etiquetador les fueron asignados progresivamente los artículos a anotar, los cuales fueron agrupados en ``lotes'' para facilitar la tarea administrativa de la asignación.

La figura \ref{fig:labeler_example} muestra la interfaz presentada a los etiquetadores. Cada artículo es presentado al etiquetador junto a los comentarios asignados. Ante esto, el etiquetador puede elegir saltear el artículo o etiquetarlo. Si decide etiquetarlo, el etiquetador debe para cada comentario marcar usando un control de tipo ``switch''

\begin{enumerate}
    \item Si el comentario contiene discurso discriminatorio
    \item En caso de ser discriminatorio, marcar si llama a la acción
    \item En caso de ser discriminatorio, marcar al menos una característica ofendida
\end{enumerate}

Para el desarrollo de la aplicación usamos Django\footnote{\url{https://www.djangoproject.com/}}, un framework de python para desarrollo web, y Javascript plano. Como base de datos utilizamos SQLite ya que tenía una baja tasa de concurrencia (sólo 6 usuarios.)

\subsection{Esquema de anotación}

%Teniendo en cuenta el modelo de anotación ilustrado en la figura \ref{fig:annotation_model}, optamos por la siguiente metodología para el etiquetado de los comentarios de nuestro dataset.

%%
%%
%% Link a Google Draw:
%% https://docs.google.com/drawings/d/1esS9tAwpPVydohxd-B-xwVdAaPQRVGAo0MruBrgSKig/edit
%%
%%

\begin{figure}
    \centering
    \includegraphics[width=0.7\textwidth]{img/esquema_anotacion.pdf}
    \caption{Esquema de anotación. Caso en que ambos anotadores etiqueten los comentarios del artículo}
    \label{fig:annotation_schema}
\end{figure}

Los artículos son asignados a cada etiquetador. Cada etiquetador, al serle presentado un artículo, tiene dos opciones: etiquetarlo o saltearlo. La idea de saltear era doble: evitar contenido poco ``interesante'' en términos de comentarios discriminatorios, o evitar contenido sensible para el anotador (algo que no ocurrió afortunadamente).

Una posibilidad que barajamos en un principio fue asignar para el etiquetado el artículo completo a 3 anotadores. Sin embargo, esta modalidad sería altamente ineficiente dada la baja cantidad de contenido discriminatorio. Entonces, decidimos ir por un esquema de ``desempate'': dos anotadores anotan un artículo, y luego un tercero anota sólo aquellos donde al menos uno marcó que es discriminatorio. Esto da la posibilidad de que haya una tercera anotación incluso cuando dos previas marcaron que el comentario es discriminatorio, y lo hacemos para recolectar más información. \todo{marcar otros trabajos que hayan hecho esto}. Con este esquema de anotación, y teniendo en cuenta los números finales obtenidos del dataset, dedicamos 2.16 etiquetados por comentarios versus 3 etiquetados por comentario de anotar tres veces todo. La figura \ref{fig:annotation_schema} ilustra este flujo de anotación.

Entonces, en primer lugar cada artículo es asignado a 2 anotadores. Luego de esto, se solicita una tercera anotación pero sólo sobre los comentarios que tengan alguna de las dos etiquetadas marcando contenido discriminatorio, y no dando la posibilidad de saltear. Ahora ¿qué pasa si alguno de los dos anotadores saltea el artículo?. Tenemos dos casos. Si los dos saltean el artículo, entonces descartamos ese artículo. Ahora, puede ocurrir el caso de que uno lo saltee y el otro lo anote: en ese caso, y en pos de maximizar el contenido discriminatorio encontrado o uno lo hace y el otro anota menos de 4 comentarios odiosos, entonces no pasa a 3ra anotación y lo descartamos del dataset. Si uno salteó y el otro anotador anotó 4 o más comentarios odiosos, entonces forzamos al primer anotador a anotar el artículo, sin dar esta vez opción de saltear. La figura \ref{fig:annotation_schema_case_two} ilustra el flujo para este caso.


%%
%%
%% Link a Google Draw
%% https://docs.google.com/drawings/d/1TOlCgZggCmYHgZWV7ZrIIlXuhcFUMeYw4PcFM7XdY2k/edit
%%
%%

\begin{figure}
    \centering
    \includegraphics[width=0.6\textwidth]{img/esquema_anotacion_caso_2.pdf}
    \caption{Esquema de anotación. Caso en que un anotador saltee}
    \label{fig:annotation_schema_case_two}
\end{figure}


Como resultado de este esquema, cada comentario de nuestro dataset puede tener dos o tres anotaciones, siendo los casos posibles los siguientes:

\begin{enumerate}
    \item Dos anotaciones negativas
    \item Tres anotaciones, siendo al menos una que marque el comentario como discriminatorio
\end{enumerate}




\subsection{Asignación}

\citet{pustejovsky2012natural} denominan ``asignación'' al procedimiento de extraer las ``gold labels'' de las etiquetas. En este punto tenemos una etiqueta binaria si el contenido es discriminatorio o no (notamos HS) en el primer nivel, y luego 9 etiquetas binarias: una para la llamadas a la acción (CALLS) y otras 8 para las características ofendidas. Recordemos que una anotación negativa sólo consta de HS negativo, mientras que una positiva consta de un HS positivo, una etiqueta para CALLS y al menos una etiqueta positiva de las características restantes.

Para este dataset, tomamos las siguientes decisiones:

\begin{enumerate}
    \item Para la etiqueta de HS, realizamos la votación mayoritaria
    \item Si hay HS, CALLS es positivo sii es votación mayoritaria
    \item Si hay HS, marco como positivas todas aquellas características marcadas por los anotadores
\end{enumerate}

La primer decisión es la más obvia y razonable, pero las otras dos decisiones merecen alguna discusión. Para que sea un comentario considerado como HS, tiene que ocurrir que al menos dos etiquetadores lo marquen como tal. En ese caso, para que haya votación mayoritaria de CALLS, tiene que haber dos o más votos marcados como tal; en caso de empate, es decir, que un anotador marca que hay llamado a la acción y otro que no, marcamos que no hay llamado a la acción.

En el caso de las características, marcamos todas las que hayan marcado aquellos anotadores que hayan etiquetado HS. Esta decisión podría haberse tomado de otra manera; por ejemplo, sólo tomando aquellos casos donde haya cierto grado de coincidencia entre los comentarios. Sin embargo, al considerar que los límites entre las características son difusos (por ejemplo, apariencia y mujer tienen un grado de coincidencia, y a veces clasismo y racismo también) preferimos optar por este esquema.

\todo{Agregar algún gráfico de esto}

\subsection{Recursos utilizados}

El etiquetado constó de XXX horas. A cada etiquetador le fue pagado YYYY por hora, y luego ZZZ por hora en segunda instancia. Esto equivale a WWW USD.


\section{Resultados}

\begin{table}
    \centering
    % \begin{tabular}{lrr}
    %     \toprule
    %     Total articles & 1238    \\
    %     Total comments &  56869  \\
    %     Hateful Tweets &   8715  \\
    %     Ratio          &   0.153 \\
    % \end{tabular}
    \begin{tabular}{lrr}
        \toprule
        Característica &  Número &  Llamadas a acción \\
        \midrule
        RACISMO        &   2469 &              674 \\
        APARIENCIA     &   1803 &               34 \\
        CRIMINAL       &   1642 &              722 \\
        POLITICA       &   1428 &              136 \\
        MUJER          &   1332 &               18 \\
        CLASE          &    823 &              135 \\
        LGBTI          &    818 &               11 \\
        DISCAPACIDAD   &    580 &                4 \\
        \bottomrule
    \end{tabular}
    \caption{Cantidad de comentarios odiosos del dataset resultante, segmentados por característica. Se listan además la cantidad de llamados a la acción dentro de los comentarios odiosos para cada característica}
    \label{tab:dataset_figures}

\end{table}

El dataset resultante consta de 1238 artículos etiquetados, y 56869 comentarios respectivamente, de los cuales 8715 contienen contenido discriminatorio según los criterios de asignación antes referidos. Podemos observar que aproximadamente 1 de cada 6 comentarios es discriminatorio; esto no es representativo del universo de notas periodísticas ya que recordemos que la selección de los datos no fue aleatoria. La tabla \ref{tab:dataset_figures} contiene estos datos estadísticos.

De todos los tweets discriminatorios, tenemos en particular los llamados a la acción. La inmensa mayoría de estos está dirigido hacia la categoría CRIMINAL, muchos en la forma de llamados a matar a criminales y otros delincuentes.



La tabla \ref{tab:annotation_agreement} reporta el acuerdo entre anotadores usando la métrica alpha de Krippendorff \todo{agregar cita}. Reportamos el valor de $\alpha$ para HS sobre todas las etiquetas, y luego todas las etiquetas del segundo nivel del modelo jerárquico (características y llamado a la acción) sólo sobre aquellas que hayan marcado que el comentario contiene HS. Esto es equivalente a calcular el acuerdo con una etiqueta faltante en el segundo nivel para las características y el llamado a la acción. Si bien este acuerdo tiende a ser alto, debe leerse como el acuerdo sobre la razón detrás del hate speech; la mayor penalización queda reservada a HS, que tiene $\alpha = 0.59$, algo que podría marcarse como un buen acuerdo teniendo en cuenta los parámetros vistos en las tablas de preliminares. \todo{linkear esto}




\begin{table}
    \centering
    \begin{tabular}{lc}
        \toprule
        Categoría   & $\alpha$ de Krippendorff \\
        \midrule
        Hateful              &  0.579 \\
        Calls to Action      &  0.641 \\
        \midrule
        WOMEN                &  0.783 \\
        LGBTI                &  0.920 \\
        RACISM               &  0.929 \\
        CLASS                &  0.706 \\
        POLITICS             &  0.808 \\
        DISABLED             &  0.849 \\
        APPEARANCE           &  0.871 \\
        CRIMINAL             &  0.931 \\
        \bottomrule
    \end{tabular}
    \caption{Reported Agreements. \emph{Hateful} agreement is reported for the binary decision of a tweet assigned as hateful or not; for the other characteristics (and the calls to action) the agreement is calculated over those tweets with two or more hateful marks}
    \label{tab:annotation_agreement}
\end{table}

\subsection{Co-ocurrencia de características ofendidas}
%%
%%
%% Generar con
%% https://docs.google.com/drawings/d/1IcBITgNJN-tehmvnZqcSF9cUuWIpNKJg6yHI5yjNF9c/edit
%%
%%

\begin{figure}[t]
    \centering
    \includegraphics[width=0.85\textwidth]{img/heatmap_characteristics.pdf}
    \caption{Matriz de co-ocurrencia de las características ofendidas para comentarios con dos o más características marcadas. Más luminoso indica más co-ocurrencia}
    \label{fig:heatmap_characteristics}
\end{figure}


De los 8715 comentarios odiosos, el 77\% de ellos (6777) tiene una sola característica ofendida marcada. Del resto, cerca del 20\% de ellos tiene 2 características ofendidas, y 220 comentarios tienen 3 o más. En la figura \ref{fig:heatmap_characteristics} podemos observar la matriz de co-ocurrencia entre las distintas características para aquellos comentarios que tengan más de una marcada. En ella podemos ver que la máxima co-ocurrencia se da entre la característica MUJER y APARIENCIA, seguidos por RACISMO y CLASE, POLITICA y CLASE, y RACISMO y POLITICA.


La tabla \ref{tab:multi_char_examples} muestra algunos ejemplos de comentarios con más de una característica ofendida marcada. Podemos ver que algunos son ejemplos muy ``border'', justo en la frontera de las características (por ejemplo, APARIENCIA y MUJER), algunas tienen características que los anotadores marcaron implícitamente (por ejemplo, el ataque a Milagro Sala, que es un) mientras otros son directamente una clara conjunción de ofensas a las características.


\begin{table}[t]
    \small
    \begin{tabularx}{\textwidth}{XXX}
        \toprule
        Artículo        & Comentario                 & Características\\
        \midrule
        Ofelia Fernández apoyó al Gobierno en la polémica por los presos y apuntó a la Justicia que ``odia a las mujeres''  & Hijadept,, ojala pronto recibas la visita de alguno de esos gusanos. Te van a quedar. Ganas de apoyar al. Gobierno? Larva rastrera gorda. Decerebrada & MUJER, POLITICA, APARIENCIA, DISCAPACIDAD \\
        ``Es hora de ponerle límites al odio'' | Por Victoria Donda &  Justo ésta zurda mugrienta, ignorante y altanera... & MUJER, POLITICA, APARIENCIA\\
        Coronavirus en la Argentina: un video pone en evidencia la violación de la cuarentena en la Villa 1-11-14 & Cierren esa nido de negros y napalm. Hasta reducís el crimen y el gasto público. & RACISMO\\

        Fabiola Yáñez denunció a un periodista por publicaciones agraviantes & Claro si ofendel a la que se cuelga en el caño xq ahora cree ser primera dama?😂 hay que ser peruka para dar asco y ser basuras bigote enseguida ordena como se metió en Facebook y en todo que culpa te.emos que saque la mujer del cabarute? \\

        Los infectados en villas porteñas crecieron un 80\% en cuatro días & Ojalá que el virus penetre más en las villas y maten a todos esos delincuentes que viven ahi, hay paraguayos narcos, bolivianos que traen la droga de bolivia, y gente de mala vida. También hay travas que van a trabajar de noche a palermo. & RACISMO, CLASE, LGBTI  \\

        Ricky Martin: “Soy un hombre latino y homosexual viviendo en los Estados Unidos, soy una amenaza” & Ridículo perdiste tú rumbo das náuseas 🤮 famosos eternos (víctimas) 🙄🤦‍♀️ ándate a Puerto Rico entonces ahí no serás una amenaza & LGBTI, \\

        El enojo de Moria Casán contra Rocío Oliva: ``Mucha agua oxigenada, le quedó media neurona para jugar a la pelota'' & Y la vieja Moria, mucha cirugía y estiramiento. de cara que parece un travesti
    \end{tabularx}
    \label{tab:multi_char_examples}
    \caption{Ejemplos con más de una característica ofendida marcada}
\end{table}




\begin{figure}[t]
    \centering
    \includegraphics[width=0.85\textwidth]{img/heatmap_characteristics_article.pdf}
    \caption{Matriz de co-ocurrencia de las características ofendidas entre comentarios de un mismo artículo. Más luminoso indica más co-ocurrencia}
    \label{fig:heatmap_characteristics_article}
\end{figure}



Otra forma de analizar la co-ocurrencia de comentarios es agrupando por artículos, para observar como un mismo contexto puede suscitar distintos tipos de comentarios discriminatorios. La figura \ref{fig:heatmap_characteristics_article} ilustra las interacciones entre las distintas características por artículo. Podemos observar que tenemos en este mapa de calor que tenemos mayor dispersión en las co-ocurrencias que reduciendo al análisis a sólo observar comentarios. Por mencionar algunas que no aparecen en la figura agrupada únicamente por comentarios, puede verse una mayor interacción entre discurso de odio RACISMO y POLITICA, y, quizás inesperadamente entre APARIENCIA y POLITICA. Las interacciones de la característica LGBTI se mantienen muy bajas, indicando que este tema suele estar concentrado en este tipo de ataques.

Observando estas co-ocurrencias, podemos observar que el dataset anotado posee cierta diversidad en sus instancias, con comentarios conteniendo múltiples tipos de discriminación, y artículos que poseen comentarios odiosos de diversa naturaleza. Sobre esto, podemos especular que tanto el texto (el comentario en sí) como el contexto (el tweet del medio periodístico y su artículo periodístico) contienen información valiosa para poder distinguir entre las distintas categorías discriminatorias. \todo{esto es polémico, reformular: que varios comentarios sean discriminatorios y de característica distinta no implica que el contexto necesariamente ayude}




\subsection{Análisis por característica}

En la tabla XXX podemos observar algunos ejemplos seleccionados de comentarios. Algunas observaciones que pueden realizarse es que los comentarios marcados contra las mujeres tienen en algunos casos ciertas complejidades, como las acusaciones de ``mentirosa'' a una mujer que sufrió una violación (caso Thelma Fardin \todo{Agregar nota de esto}), apreciaciones a su cuerpo, entre otras cosas.

Una categoría desafiante pareciera ser los comentarios discriminatorios contra la comunidad LGBTI. Más allá de algunos insultos explícitamente ofensivos (mediante insultos del estilo trolo, trabuco, maricón, etc), hay muchos que tienen un contenido difícil de descifrar; en particular, aquellos comentarios contra personas trans. Muchos de estos mensajes hacen alusiones a su genitalidad o a su cuerpo en general, de manera metafórica o irónica, lo cual hace verdaderamente difícil su detección. A su vez, es claro que en muchos de estos comentarios es sumamente necesaria la información contextual para poder comprender el caracter abusivo de estos comentarios.

En el caso de la categoría CRIMINAL, se puede observar por un lado comentarios muy violentos (``bala'', ``mátenlos'', ``plomo'') que necesitan el contexto para entenderse como ofensivos contra esa característica (por ejemplo, si la nota fuese sobre una plaga de mosquitos no deberíamos considerarlo como ``discriminatorio''). Por otro lado, algunos comentarios son más difíciles de descifrar y dependientes del contexto, como las celebraciones ante el abatimiento de un preso o criminal (``bravo'', ``felicitaciones!'') que parecen inofensivas hasta que se lee el contexto de la noticia. De hecho, a diferencia de otros comentarios, parecen tener hasta una polaridad positiva.

En el caso de racismo (la categoría más marcada del dataset) hay una fuerte cantidad de comentarios discriminatorios contra la comunidad china. Esto es esperable por el brote racista debido a la pandemia del COVID-19, documentado en YYYY \todo{agregar cita}. Así mismo, es de las categorías que más llamados a la acción tiene, muchos del estilo de tirar bombas, aniquilar, etc a China o a la comunidad de dicho país, o llamados a tomar medidas ``blandas'', como ``no ir a comprarles a los supermercados''.

Algunas de las categorías tienen características más elementales, como política, apariencia, y discapacidad. En los comentarios ilustrados. Esto es esperable ya que

Algunas de las agresiones, a su vez, usan técnicas de camuflaje (``tafaldegaver'', falta de verga, ``docer''), que dificultan su detección por las técnicas actuales.

\begin{table}
    \centering|
    \begin{tabular}{p{0.2\textwidth} p{0.4\textwidth} p{0.5\textwidth}}
        \toprule
        Cat & & \\
        \midrule
        characteristic & text & context \\
        \midrule
                 WOMEN & @usuario Y como te quedó el ogt?. & Loly Antoniale mostró su impresionante casa en Miami: “Soy la reina de mi castillo” \\
                 WOMEN & @usuario Habla el agua viva esta vestida de verde ? & Cecilia Moreau: "No se puede permitir que la minoría le imponga a la mayoría qué temas discutir" \\
                 WOMEN & @usuario “Feminista interseccional”..:nos vamos a la mierda... & Empezó con una “relación abierta en lo sexual”, escaló al poliamor y da las claves para probar el amor libre \\
                 WOMEN & @usuario Provocador ? A mi me provoca ganas de vomitar & ¡El sensual y provocador topless de Morena Rial! \\
                 WOMEN & @usuario Que se aborte ella, vieja  bruja y degenerada!! & Martha Rosenberg: “En situación de pandemia, legalizar el aborto es más urgente que nunca” \\
                 WOMEN & @usuario Kien es la feminista resentida q está instaurando esa idea desde hace días?? & Femicidio en Catamarca: “No es gente enferma, sino que tiene una formación machista importante” \\
                 WOMEN & @usuario Pero quién puede pensar embarazar a esta mostra. & “Ya pusimos el cuerpo, ahora que los diputados se pongan las pilas”: en el Congreso, el lado verde se prepara para una fiesta \\
                 WOMEN & @usuario Pregúntaselo a Lousteau, trola de mierda & Juana Viale hizo un irónico pedido en su programa: “Que el señor Presidente me explique en qué fase de la ‘no cuarentena’ estamos” \\
                 WOMEN & @usuario Que pelotudo!!!!...lo que te va a tomar es la guita viejo decrepito...o te pensastes que se enamoro de vos..jajajaja & Eduardo Costantini y Elina Fernández mostraron la intimidad de su luna de miel: “Ella no me deja tomar una copa de vino” \\
                 WOMEN & @usuario El paskin con la trola & La pareja de Lázaro Báez contó cómo lo esperaba adentro del country: “Lo que le hicieron fue indignante” \\
                 WOMEN & @usuario Que manga de roñosas & “Ya pusimos el cuerpo, ahora que los diputados se pongan las pilas”: en el Congreso, el lado verde se prepara para una fiesta \\
                 WOMEN & @usuario A la que fue PROSTITUTA de Villa Ballester le molesta que le recuerden que fue PROSTITUTA? & Fabiola Yáñez denunció a un periodista por publicaciones agraviantes \\
                 WOMEN & @usuario Hay que comerse al termotanque de lipidos & ¿More Rial encontró el amor en un personal trainer? \\
                 WOMEN & @usuario @usuario @usuario SOS la peor mierda de argentina, junto al presidente y su titiritera de Cristina. Manga de soretes! \textbackslash nMuy sorora pero estan matando de hambre a media argentina, homicidios y terrorismo. Vayanse todos! Métete el corazón verde e... & Fuerte cruce entre Flavio Mendoza y Victoria Donda: “Es muy fácil hablar cuando uno cobra un sueldo” \\
                 WOMEN & @usuario No me sorprende, usa el pañuelito verde decisor. & Nancy Pazos reveló por qué decidió que su mamá no recibiera la donación de plasma \\
                 WOMEN & @usuario La Yañez está tratando de entender  de que habla Manes .. pobre solo conoce los tablones del teatro ... & Coronavirus. Fabiola Yáñez organizó una videollamada con Facundo Manes: "Lo importante es estar bien mentalmente" \\
                 WOMEN & @usuario Esto me hace tan feliz, jodanse aborteras de mierda. JAJAJAJAJAJAJAJAJAJAJAJAJAJAJAJAJAJAJAJAJAJAJAJAJAJAJAJAJAJAJAJAJAJAJAJAJAJAJAJAJAJAJAJAJAJAJAJAJAJAJAJAJAJAJAJAJAJAJAJJAAJAJAJAJAJAJAJAJAJAJAJAJAJAJAJAJAJAJAJAJAJAJAJAJAJAJAJAJAJAJAJAJAJAJAJ... & Aborto legal: otra promesa incumplida \\
                 WOMEN & @usuario Jaja la mina orgullosa de lo q consiguió gateando, bien ahi 🥴 & Loly Antoniale mostró su impresionante casa en Miami: “Soy la reina de mi castillo” \\
                 WOMEN & @usuario Y quien te iva a hacer un pibe...dracula o el hombre lobo.. & “Me esterilicé, pero no odio a los niños”: mi vida dentro del movimiento “libre de hijos” \\
                 WOMEN & @usuario Dirán lo que dirán de Moria y como sea trabajo toda su vida por eso tiene lo que tiene Pero Rocío  solo trabajo abriendo las piernas  en la cama  para llegar hacer figureti Y no me la cuenten que fue por amor🤣🤣 & El enojo de Moria Casán con Rocío Oliva: “Mucha agua oxigenada, le quedó media neurona para jugar a la pelota” \\
                 LGBTI & @usuario Revisen esa casa, los están envenenando. & Contó que era lesbiana, su papá le confesó que era gay y ahora su madre se enamoró de una mujer: así se inspiró para su segundo film \\
                 LGBTI & @usuario Me cruzo con 50 María Elena por día. Las feminazis son iguales & No es actriz, pero se anima al desafío: quién es Ethel Herrera, la tiktoker que se postula para ser María Elena en “Casados con hijos” \\
                 LGBTI & @usuario Biológicamente las mujeres tienen vagina y los hombres tienen pene. Lo demás es ideológico. & Tras los comentarios de J.K. Rowling, Emma Watson defendió al colectivo trans \\
                 LGBTI & @usuario por fin alguien que ponga huevos en el equipo & Histórico: Mara Gómez fue habilitada y será la primera jugadora trans en el fútbol argentino \\
                 LGBTI & @usuario pero....este no se comia la galletita? & Pablo Duggan, sobre su casamiento: "Karen es la mujer de mi vida, estoy muy enamorado" \\
                 LGBTI & @usuario Por el pirulo & Por qué Flor de la V no continuó en Mujeres de eltrece, tras la salida de Claudia Fontán \\
                 LGBTI & @usuario Pelado puto, te quedaste solo. & Horacio Rodríguez Larreta: “La convocatoria del lunes no es del partido, si alguno va es a título personal” \\
                 LGBTI & @usuario Bueno por lo menos le quedó el culo roto & El meteorólogo Matías Bertolotti se separó a ocho meses de haberse casado \\
                 LGBTI & @usuario No Clarin. Es un señor con apariencia de mujer que preñó a una mujer con apariencia de hombre. La paja no es como el trigo. & La historia de la modelo colombiana trans que besa la panza de su esposo embarazado de ocho meses \\
                 LGBTI & @usuario Che, qué pasó con el puto-no me enteré de nada. & Por qué Flor de la V no continuó en Mujeres de eltrece, tras la salida de Claudia Fontán \\
                 LGBTI & @usuario Viejo larga presos , llévalos todos a tu casa y te haces romper el orto viejo puto igual ya lo tener & Eugenio Zaffaroni le contestó a Sergio Berni tras la polémica por las domiciliarias: “Es el populacherismo vindicativo que llenó las cárceles” \\
                 LGBTI & @usuario Pero...\textbackslash n.\textbackslash n.\textbackslash n.\textbackslash n.\textbackslash nCamila es Camilo. & Camila Sosa Villada: “Con el feminismo y el psicoanálisis descubrimos que el amor es una herramienta de manipulación" \\
                 LGBTI & @usuario Vos también comes carne...\textbackslash n\textbackslash nNo te hagas la boluda, encima te comiste a un bala no asumido, ahí también comiste carne....\textbackslash n\textbackslash n@usuario \textbackslash n\textbackslash nDeja que cada uno elija qué hacer, de ultima es una elección personal como la tuya de comerte a un corrup... & Nicole Neumann apuntó contra los consumidores de carne: “Sigan comiendo asadito” \\
                 LGBTI & @usuario La mujer tiene duda siiiiiii duda la podonga 😂😂😂😂 & Pablo Duggan, sobre su casamiento: "Karen es la mujer de mi vida, estoy muy enamorado" \\
                 LGBTI & @usuario 🤮🤮🤮🤮🤮🤮 & La emotiva dedicatoria de Luis Novaresio a su pareja, Braulio Bauab, en su cumpleaños \\
                 LGBTI & @usuario Cerra el otro viejo chupa pija & Alberto Fernández: “Dejemos el tiempo del encuentro y del esparcimiento social para otro momento” \\
                 LGBTI & @usuario Quedaba feo poner "un jugador disfrazado de mujer" entendemos & Mara Gómez cumple su sueño: será la primera futbolista transgénero en el torneo profesional argentino \\
                 LGBTI & @usuario Creo que se debe referir a las feminazis,aborteras y ese circo de lgbtqia+ etc y toda esas payasadas....que son manipulados como titeres por los lobbys & Cecilia Moreau: "No se puede permitir que la minoría le imponga a la mayoría qué temas discutir" \\
                 LGBTI & @usuario Estos chetos soretes maricas nenas de mama cagones. & Coronavirus en Argentina: los “caprichos” de algunos repatriados en los hoteles porteños, entre rebeldes y cholulos \\
                 LGBTI & @usuario La discriminacion se termina despues del primer tortazo.. dejen de llorar manga de maricas!! & Viola Davis, Halle Berry y Angela Bassett, actrices negras que superaron la discriminación en Hollywood \\
        \bottomrule
        \end{tabular}
\end{table}



\section{Conclusión}

En este capítulo, describimos la construcción de un dataset contextualizado de lenguaje discriminatorio o hate speech. Para ello, recolectamos respuestas a noticias periodísticas posteadas en Twitter por los principales medios de noticias de Argentina. Exploramos distintas alternativas para la selección de artículos a etiquetar, tanto observando los tópicos de los artículos como los comentarios a este. Decidimos elegir los artículos en base a sus comentarios potencialmente discriminatorios, y luego seleccionar una muestra aleatoria y acotada de comentarios.

Para realizar la tarea de etiquetado, desarrollamos nuestra propia herramienta la cual hacemos pública. Definimos un modelo de anotación jerárquico y granular para la tarea, siendo relativamente novedoso el hecho de anotar las características ofendidas en cada texto social. Seis etiquetadores nativos de la variedad dialectal rioplatense realizaron la tarea de anotación bajo un esquema de 2 anotaciones + desempate.

Como producto, obtuvimos un dataset de cerca de 57k comentarios repartidos en 1.2k artículos, una cantidad de tamaño considerable aunque no tengamos parámetro de comparación ya que no existen muchos datasets similares. De los 57k comentarios, alrededor de 8k comentarios tienen contenido discriminatorio (una tasa de 1 cada 6). Un análisis exploratorio de los comentarios discriminatorios muestra ejemplos complejos y ricos, algunos de ellos altamente dependientes del contexto.

En el siguiente capítulo, abordaremos nuestra pregunta original: ¿puede el contexto ayudar a los algoritmos de clasificación a mejorar su performance?. Para responder esto, utilizaremos este dataset especialmente diseñado.

%
\chapter{Experimentos de detección contextualizada de discurso de odio}
\label{chap:06_contextualized_hate_speech}
En base a las discusiones previas, en este capítulo analizaremos el impacto del contexto en la tarea de detección de discurso de odio en redes sociales, algo que ha recibido poca atención en trabajos previos. Utilizaremos el dataset construído en el capítulo \ref{chap:05_dataset_creation} que está basado en comentarios de artículos periodísticos en Twitter, y que nos brinda información adicional a cada comentario tanto por el tweet del medio periodístico como el contenido del artículo. Para evaluar si la adición de contexto resulta en una mejora en la detección de discurso de odio realizaremos experimentos de clasificación con modelos que sólo consuman el comentario en cuestión, y otros que a su vez consuman algún tipo de contexto asociado. Por la naturaleza de nuestro dataset, tenemos dos tipos de contexto naturales: uno ``corto'', que contiene el tweet del medio periodístico, y otro ``largo'' que incorpora además el cuerpo del artículo asociado.

La anotación del dataset, con información detallada de las características ofendidas, nos permite salir de analizar exclusivamente la existencia de discurso de odio sino que podemos pedir más detalle sobre la ofensa cometida. Proponemos entonces dos tareas: una tarea de detección \textbf{binaria}, donde sólo predecimos si hay o no discurso de odio; y una tarea de detección \textbf{granular}, donde además predecimos todas las características ofendidas (potencialmente más de una). Para estas tareas, propondremos algoritmos de clasificación sobre modelos pre-entrenados de lenguaje, concretamente sobre \beto{} (la versión en español de \bert{}). Estos modelos tienen incorporados naturalmente la posibilidad de consumir dos entradas, con lo cual son ideales para nuestros experimentos.

Evaluaremos los resultados tanto en términos de la performance de las distintas configuraciones de nuestros clasificadores, como así también realizando análisis de error comparativos entre los modelos contextualizados y los no contextualizados. También analizaremos las dificultades en general que presenta la detección de este fenómeno sobre comentarios de notas periodísticas.

\section{Trabajos previos}
\label{sec:06_classification_previous}

Como mencionamos en la sección \ref{sec:dataset_previous}, no se ha dado demasiada atención en la literatura a la utilización de contexto en la detección de discurso de odio y otros fenómenos similares (como la detección de toxicidad, por ejemplo). En dicho capítulo se puede encontrar una descripción de algunos de los datasets que sí contienen algún tipo de contexto. Pasamos ahora a describir los algoritmos de detección que utilizaron en los trabajos correspondientes.

\citet{gao-huang-2017-detecting} propone dos tipos de modelos: regresiones logísticas y redes neuronales recurrentes. Para los modelos de regresiones logísticas, usan como inputs bolsas de palabras, bolsas de caracteres, vectores semánticos producidos con Linguistic Inquiry and Word Count (LIWC) \cite{pennebaker2001linguistic} y features de un lexicon de emociones \cite{mohammad2013nrc}. Por otro lado, utiliza LSTM bidireccionales con mecanismo de atención de Bahdanau \cite{bahdanau2014neural} usando embeddings \emph{word2Vec} de dimensión 100.

Un punto criticable de este trabajo es que utiliza el nombre de usuario como feature; algo que a priori no suele hacerse ya que permitiría ``prejuzgar'' a un usuario antes que por el contenido de sus tweets. Si bien es cierto que la información de usuarios y sus conexiones es valiosa, introducir esta información a nuestros modelos da lugar a posibles correlaciones espurias que es preferible evitar.


\begin{figure*}[t]
    \centering
    \begin{minipage}[b]{0.49\textwidth}
        \includegraphics[width=\textwidth]{img/pavlopoulos_rnn_rnn_classifier.png}
    \end{minipage}
    \hfill
    \begin{minipage}[b]{0.49\textwidth}
        \includegraphics[width=\textwidth]{img/pavlopoulos_rnn_bert_classifier.png}
    \end{minipage}

    \begin{minipage}[b]{0.35\textwidth}
        \includegraphics[width=\textwidth]{img/pavlopoulos_bert_sep_classifier.png}
    \end{minipage}


    \caption{Clasificadores que consumen contexto propuestos por \citet{pavlopoulos2020toxicity}. Los dos primeros clasificadores proponen una arquitectura de dos encoders, uno para el texto y otro para el contexto usando bi-LSTMs y BERT como posibilidades. El tercer clasificador propuesto es un BERT usando su estructura natural para codificar dos oraciones separadas por el token $SEP$ }
    \label{fig:pavlopoulos_classifiers}
\end{figure*}


En la sección \ref{sec:dataset_previous} hemos descripto el dataset construído por \citet{pavlopoulos2020toxicity}. Nos detendremos un momento para analizar sus experimentos de clasificación ya que guardan importantes similaridades con lo que haremos en este capítulo. En ese trabajo se obtuvieron dos datasets de entrenamiento: uno en el cual los etiquetadores tenían información del contexto y otro en el que no, mientras que el dataset de test fue etiquetado viendo el contexto, considerando que el etiquetado es de mejor calidad usando más información. Los autores plantearon dos preguntas sobre esta base: ¿mejora la performance de los clasificadores que son entrenados con el dataset etiquetado con contexto? ¿mejora la performance de los clasificadores consumiendo información contextual? Con estas preguntas, tenemos que tomar dos decisiones: dataset de entrenamiento con o sin contexto, y clasificador con o sin contexto. Tenemos 4 posibles combinaciones de experimentos, sin aún considerar posibles técnicas de clasificación.

Para cada una de estas combinaciones, se consideraron técnicas del estado del arte de clasificación. Para aquellos clasificadores que no consumen contexto, las opciones son las mismas que hemos visto en capítulos anteriores: bi-LSTM o BERT. Para aquellos que sí consumen contexto, se evaluan dos estrategias: una, concatenar con algún caracter, y otra usando dos encoders distintos para el contexto y el texto. A su vez también utilizan la API Perspective de Google, que finalmente obtiene los mejores resultados en términos de performance. En todas las combinaciones posibles, si bien hay una mejora en la performance medida con ROC-AUC al usar contexto en ambas formas, esta no es estadísticamente significativa.

Algo a mencionar (que retomaremos en este y en el siguiente capítulo) es que usan dos versiones de \bert{}: una usando los pesos del modelo de \bert{}, y otro haciendo un ajuste de dominio () corriendo la tarea de MLM sobre un dataset grande y no etiquetado. En el caso de el trabajo mencionado, sólo hacen un fine-tuning sobre comentarios sueltos del dataset de Civil Comments. Esto podría tener algún efecto deteriorando la performance al usar contexto; sin embargo, en el \bert{} a secas (sin hacer ajustes) tampoco se observa mejora significativa en la performance.

Algunas limitaciones marcadas por los autores son:

\begin{itemize}
    \item Contexto muy pequeño: sólo el título más el comentario previo
    \item Se ignora el hilo completo de comentarios
    \item Los datos fueron sampleados aleatoriamente
\end{itemize}

En \citet{xenos-2021-context}, continuación de este trabajo, reetiquetaron el dataset de Civil Comments usando contexto y --como mencionamos en la sección \ref{sec:dataset_previous}-- presentaron una nueva tarea de detección de sensibilidad al contexto. Usando la API Perspective (y la estrategia de concatenación ``básica'' con algún caracter), notaron que la performance del clasificador que consume el contexto mejora con respecto al que no lo hace a medida que restringimos nuestra atención a comentarios más ``sensibles al contexto'' (de acuerdo a la métrica definida por los autores)


\section{Tareas de clasificación propuestas}
\label{sec:tasks}

Para analizar el impacto del contexto en la detección de discurso de odio, y teniendo en cuenta que contamos de un dataset con anotaciones granulares sobre las características ofendidas, proponemos dos tareas de clasificación:

\begin{enumerate}
    \item \textbf{Detección binaria}: Dado un tweet y su contexto, predecir si contiene contenido discriminatorio.
    \item \textbf{Detección granular}\footnote{En inglés usamos la denominación \emph{fine-grained}, aunque no hay una traducción precisa para este término en español}: Dado un tweet y su contexto, predecir las características ofendidas (si hay alguna) y si contiene un llamado a la acción.
\end{enumerate}

%%
%%
%% Link
%% https://docs.google.com/drawings/d/11sAaOuGJlU0P61mkrPxKduFwnNOuPV31tXUFJWEwbVU/edit?usp=drive_web&ouid=117313784631536396179
%%
%%
\begin{figure}[t]
    \centering
    \includegraphics[width=\textwidth]{img/06/hate_detection_tasks.pdf}
    \caption{Tareas propuestas de detección de discurso de odio. La tarea de detección binaria consta de predecir si un tweet contiene contenido discriminatorio, discriminando la frontera conjunta. En la tarea granular, predecimos por separado cada una de las características ofendidas, pudiendo haber más de una o ninguna.}
    \label{fig:hate_detection_tasks}
\end{figure}



Puede pensarse la tarea de detección binaria (la que usualmente se aborda en la literatura sobre el tema) como una relajación de la tarea detallada: mientras la primera sólo nos permite detectar si hay o no contenido discriminatorio, la segunda nos requiere información más precisa acerca de las características ofendidas. Esta segunda tarea es posible dado que el dataset que construímos contiene esta información, algo usualmente faltante en otros trabajos. La tarea granular nos permite a su vez tener mayor entendimiento de la salida e interpretar mejor sus errores. La figura \ref{fig:hate_detection_tasks} ilustra las dos tareas propuestas. Mientras en la tarea binaria sólo debemos decidir la frontera sobre si el contenido es discriminatorio o no, en la tarea granular necesitamos decir en cuál de todas las intersecciones está el comentario y su contexto.


Planteándolos como problemas de clasificación, la detección binaria consta de predecir una sola etiqueta binaria, mientras que la tarea granular consta de $n$ etiquetas binarias; es decir, $n$ problemas distintos de clasificación. Vale mencionar que, entendiendo una tarea como una relajación de la otra, si tenemos un clasificador entrenado para la tarea granular podemos construir un clasificador para la tarea binaria tomando la disyunción lógica de sus salidas; o dicho más coloquialmente, poniendo una compuerta OR sobre la salida del clasificador granular. Retomaremos esta idea más adelante al hablar de cómo evaluamos nuestras técnicas de clasificación para cada tarea.


\section{Modelos de clasificación}
\label{sec:contextualized_classifiers}





Para las dos tareas, entrenamos varios clasificadores basados en \beto{} \cite{canete2020spanish}. Respecto a la información contextual, tenemos tres tipos de entrada por instancia: el comentario sin ningún tipo de contexto, con contexto simple (el tweet al que responde), y con contexto largo (el tweet al que responde + el texto del artículo periodístico). Usamos el token especial BERT \emph {[SEP]} para separar el contexto y el texto analizado. Este token es el que \emph {[SEP]} se usa para la tarea de predicción de la siguiente oración (tarea NSP) en el preentrenamiento al estilo BERT (ver la sección \ref{sec:02_transformers} para más información).

%%
%%
%% Link a Draw
%% https://docs.google.com/drawings/d/1F8iVSIRqHhGkQ0zglxqXLGD36RHZ9OhHMZYsg_xFOS4/edit
%%
%%

\begin{figure}
    \centering
    \includegraphics[width=1.10\textwidth]{img/06/bert_contextual_classifier.pdf}
    \caption{Modelo de multiclasificación para la tarea granular. Los modelos son entrenados de 3 maneras distintas: sin contexto (sólo el comentario), con el contexto del tweet, y con el contexto del tweet y el texto del artículo.}
    \label{fig:05_multi_bert_classifier}
\end{figure}

Para la tarea de clasificación binaria, la salida de los clasificadores es la salida estándar de clasificadores del tipo \bert{}: una capa softmax para las dos posibles clases (ver la sección \ref{sec:03_classification} o \ref{sec:04_classifiers} para más información). En cuanto a la tarea de detección granular, lo planteamos como la predicción de 9 variables distintas: llamado a la acción (CALLS) y las 8 características ofendidas: MUJER, RACISMO, CLASE, LGBTI, CRIMINAL, ASPECTO, DISCAPACIDAD, POLITICA. En lugar de entrenar un clasificador diferente para cada característica, entrenamos un BERT de múltiples salidas, compartiendo todos sus pesos con la excepción de 9 capas lineales diferentes para cada salida. La función de costo utilizada es:

\begin{equation*}
    J = \sum\limits_{c \in CHAR'} J_c
\end{equation*}

donde $CHAR'$ es el conjunto de todas las características protegidas y además la variable de llamada a la acción, y cada $ J_c $ es la función de entropía cruzada. Compartir los pesos entre todas las salidas tiene dos objetivos: primero, poder generar un modelo más compacto (de otra forma serían 9 BERT distintos) y segundo, para compartir información común entre las distintas características atacadas, ya que consideramos que guardan similaridades -- y muchas de ellas tienen cierta intersección, como hemos visto en el capítulo anterior. Para tener costos computacionales más amigables, limitamos nuestras secuencias a 128, 256 y 512 tokens para el modelo sin contexto, el modelo que consume el tweet, y el modelo de tweet + cuerpo respectivamente. La figura \ref{fig:05_multi_bert_classifier} muestra el modelo de clasificación para la tarea granular, junto a los 3 tipos de entrada descriptos.

Una práctica cada vez más extendida en trabajos del área de clasificación de documentos es realizar una adaptación de dominio sobre textos relacionados a nuestro. Esto se realiza corriendo la tarea de masked language model (ver sección \ref{sec:02_transformers}) sobre un dataset grande y no supervisado relacionado a nuestro dominio, o directamente sobre el dataset de la tarea si esto no está disponible \cite{gururangan-etal-2020-dont}. En la sección \ref{sec:domain_adaptation_previous_work} del siguiente capítulo haremos una reseña más extensa de esta técnica, pero por lo pronto podemos entender que ajusta el modelo de lenguaje a los textos disponibles; en este caso, \beto{} fue entrenado en Wikipedia y textos formales, con esta técnica lo ajustaremos a nuestro dominio particular de comentarios en Twitter a notas periodísticas.

\begin{table}[t]
    \centering
    \begin{tabular}{ll}
        \toprule
        Hiperparámetro & Valor         \\
        \midrule
        Steps          & $10K$           \\
        Batch size     & 2048            \\
        max seq length & 128, 256 y 512  \\
        $\beta_1$      & $0.9$           \\
        $\beta_2$      & $0.98$          \\
        $\epsilon$     & $10^{-6}$       \\
        decay          & $0.01$          \\
        Peak LR        & $4*10^{-4}$     \\
        warmup ratio   & 0.1             \\
        \bottomrule
    \end{tabular}
    \caption{Hiperparámetros para la adaptación de dominio de BERT}
    \label{tab:hs_ft_hyperparameter}
\end{table}

Para lo que nos concierne en esta sección, \citet{pavlopoulos2020toxicity} realizó una adaptación de dominio sobre los comentarios del corpus de \emph{Civil Comments} (a lo que denomina BERT-CCTK). Esta adaptación la realizó únicamente con los comentarios, sin utilizar ningún tipo de contexto. Proponemos a diferencia de este trabajo, 3 tipos de adaptaciones: una adaptación sin contexto, una adaptación con el contexto del tweet, y una adaptación con el contexto del tweet y el cuerpo de la noticia.

Realizamos el ajuste utilizando el sobrante de la recolección de datos del anterior capítulo: alrededor de 288k artículos y 5.1M comentarios\footnote{Utilizamos algunos datos extra recolectados a posteriori de lo mencionado en el capítulo anterior}. La tabla \ref{tab:hs_ft_hyperparameter} contiene los hiperparámetros utilizados. Estos ajustes los realizamos sobre una TPU v2-8 y una máquina de Google Colab Pro por alrededor de 10 hs (en su largo de cadena máximo).

Preprocesamos ambos tweets --contexto y texto-- utilizando las técnicas descriptas en la sección \ref{sec:03_preprocessing}: conversión de usernames a un token especial (\verb|usuario|), tratamiento de hashtags (separación e inserción de un hashtag especial), y conversión de emojis a su representación textual.


\subsection{Performance humana de la tarea}


\begin{table}
    \centering
    \begin{tabular}{l cc  cc}
                   & \multicolumn{2}{c}{Entre anotadores} & \multicolumn{2}{c}{Gold} \\
        {}         &  F1 mean&  F1 median  & F1 Mean  &  F1 Median \\
        \hline
        ODIO       &  0.653 &   0.675    & 0.829   &   0.851   \\
        CALLS      &  0.434 &   0.495   &  0.704   &   0.842  \\
        \hline
        MUJER      &  0.490 &   0.468   &  0.741   &   0.759  \\
        LGBTI      &  0.596 &   0.577   &  0.846   &   0.915  \\
        RACISMO    &  0.653 &   0.644   &  0.871   &   0.879  \\
        CLASE      &  0.443 &   0.444   &  0.722   &   0.732  \\
        POLITICA   &  0.461 &   0.436   &  0.795   &   0.815  \\
        DISCAPACIDAD& 0.550 &   0.600   &  0.813   &   0.842  \\
        APARIENCIA &  0.649 &   0.743   &  0.831   &   0.915  \\
        CRIMINAL   &  0.527 &   0.580   &  0.841   &   0.929  \\
        \hline
        Macro F1   &  0.534 &   0.554   &  0.796   &   0.848  \\
        \hline
    \end{tabular}

    \caption{Estadísticos de las cotas de performance entre anotadores. Cada característica es tomada como una etiqueta binaria independientemente del cálculo de la métrica para discurso de odio. La Macro F1 es el promedio de los F1 todas las características y la F1 de la llamada a la acción (CALLS). Las dos primeras columnas marcan las métricas medidas entre anotadores, y las dos últimas la de los anotadores contra la etiqueta gold}
    \label{tab:ia_f1_scores}
\end{table}


Como observamos en la anterior sección, la tarea de detección de lenguaje discriminatorio contiene una alta cantidad de ruido, y el acuerdo entre humanos es moderado. En este contexto, cabe preguntarse una cota a la performance que puede lograr un clasificador para esta tarea, que claramente por la misma naturaleza del problema, va a distar mucho de la perfección. Para obtener algunas medidas de esto, calculamos en primer lugar las F1 usando todos los posibles pares de anotadores. Como la F1 es simétrica no necesitamos hacer ninguna asunción sobre sus roles.

Algo a tener en cuenta es que la métrica final será contra la etiqueta resultante de la votación mayoritaria (nuestro \emph{gold standard}). Una cota que seguro está por arriba de nuestra performance es el acuerdo que haya entre los anotadores y este \emph{gold standard}; hay que también observar que cada etiqueta asignada (ver sección \ref{sec:assign}) codifica información de cada anotador, con lo cual éste número es una cota superior pero puede que sobreestimada.


La tabla \ref{tab:ia_f1_scores} contiene estadísticos para estas cotas, tanto entre anotadores como contra el \emph{gold-standard}. Como podemos observar, la mediana entre anotadores de la F1 (usada para obviar outliers) es relativamente baja para la detección de odio ($\sim 0.67$), mientras que contra el gold standard es de $0.85$. De esto entendemos que la performance una cota superior a la performance está entre esos dos números.



\section{Resultados}



\begin{table*}
    \centering
    \large
    \begin{tabular}{l P{0.08\textwidth}P{0.08\textwidth} P{0.08\textwidth}P{0.08\textwidth}  P{0.08\textwidth}P{0.08\textwidth}}
        \mr{2}{Métrica}          &\multicolumn{2}{c}{Sin Contexto}& \multicolumn{2}{c}{Tweet}          &  \multicolumn{2}{c}{Tweet + Cuerpo}    \\
                  & $\neg$FT &  FT     & $\neg$FT&    FT       &$\neg$FT &    FT \\
        \hline
        Accuracy  & $88.9$   &  $89.9$ & $90.2$  &$\mbf{91.0}$ & $90.4$  &  $90.5$ \\
        Precisión & $67.8$   &  $71.8$ & $73.1$  &$\mbf{74.8}$ & $73.9$  &  $72.8$ \\
        Recall    & $56.8$   &  $60.2$ & $60.1$  &$\mbf{65.3}$ & $61.1$  &  $64.1$ \\
        F1        & $61.8$   &  $65.5$ & $66.0$  &$\mbf{69.7}$ & $66.9$  &  $68.1$ \\
        Macro  F1 & $77.6$   &  $79.8$ & $80.1$  &$\mbf{82.2}$ & $80.6$  &  $81.3$ \\
        \hline
    \end{tabular}


    \caption{Resultados de los experimentos de clasificación para la tarea \emph{binaria} de detección de discurso de odio, expresados como la media de las distintas métricas sobre diez corridas independientes. En negrita, los mejores resultados. Cada modelo es un BERT con tres posibles entradas: sólo el comentario (\emph{Sin contexto}), el tweet de la noticia a la cual responde el comentario (\emph{Tweet}), y el tweet más el cuerpo de la noticia (\emph{Tweet + Cuerpo}). Para cada una de estas posibilidades usamos dos versiones: una sobre BETO ($\neg$FT) y otra sobre BETO ajustado al dominio (FT).}
    \label{tab:task_a_results}
\end{table*}


La Tabla \ref{tab:task_a_results} contiene los resultados para la tarea de clasificación \tbf{binaria} medidos por Accuracy, Precision, Recall, F1 de la clase positiva y Macro F1 entre las dos clases. Las métricas están expresadas como las medias de diez corridas independientes de los experimentos. Las seis columnas corresponden a la combinación de los tres posibles modelos dependiendo del contexto utilizado y de acuerdo a si ajustamos al dominio o no. Podemos observar que, en todos los casos, la adaptación de dominio (las columnas marcadas con FT) obtienen mejor rendimiento que los modelos que no están adaptados ($\neg$FT) resultando en una mejora de alrededor de 4 puntos de F1 en los casos sin contexto y con contexto de tweet. Entre los modelos sin ajustar a dominio, el que consume el contexto completo (tweet + cuerpo de la noticia) obtiene el mejor desempeño; sin embargo, esto no se replica en el caso ajustado a dominio, donde gana el contexto simple. Viendo sólo las columnas marcadas como $FT$, la mejora contra el modelo que no consume contexto es de $4.2$ puntos de F1. El modelo con el contexto completo, si bien mejora la performance general contra no tener contexto, pierde precisión al ser adaptado al dominio.


\begin{table*}
    \centering
    \Large
    \begin{tabular}{l P{0.08\textwidth}P{0.08\textwidth} P{0.08\textwidth}P{0.08\textwidth}  P{0.08\textwidth}P{0.08\textwidth}}
        Métrica        &\mc{2}{Sin Contexto}& \mc{2}{Tweet}          &  \mc{2}{Tweet + Cuerpo}    \\
                       & $\neg$FT&    FT    & $\neg$FT   &    FT     & $\neg$ FT&    FT     \\
        \hline
        LLAMA          & $64.6$ &    $65.1$   & $63.8$ &$\mbf{68.5}$  & $65.3$ &    $68.0$    \\
        MUJER          & $37.3$ &    $38.9$   & $41.1$ &$\mbf{42.1}$  & $38.1$ &$\mbf{42.1}$ \\
        LGBTI          & $35.1$ &    $36.6$   & $45.1$ &$\mbf{48.2}$  & $42.7$ &    $44.5$    \\
        RACISMO        & $63.5$ &    $65.3$   & $68.8$ &$\mbf{72.0}$  & $69.1$ &    $71.1$    \\
        CLASE          & $40.1$ &    $43.3$   & $49.1$ &$\mbf{51.1}$  & $45.1$ &    $47.6$    \\
        POLITICA       & $55.5$ &    $61.1$   & $57.9$ &$62.5$        & $59.1$ &$\mbf{64.8}$ \\
        DISCAPAC       & $55.1$ &    $58.2$   & $58.5$ &$\mbf{60.9}$  & $55.7$ &    $57.8$    \\
        APARIENCIA     & $72.6$ &    $74.2$   & $74.1$ &$\mbf{76.6}$  & $75.5$ &    $75.8$    \\
        CRIMINAL       & $51.3$ &    $52.9$   & $65.0$ &$\mbf{69.9}$  & $65.4$ &    $66.8$    \\
        \hline
        Macro F1       & $52.8$ &    $55.1$   & $58.2$ &$\mbf{0.613}$ & $57.3$ &    $59.8$    \\
        Macro Precision& $55.8$ &    $63.0$   & $64.2$ &$\mbf{0.702}$ & $67.7$ &    $67.8$    \\
        Macro Recall   & $50.6$ &    $49.9$   & $54.0$ &$\mbf{0.551}$ & $50.4$ &    $54.1$    \\
        % \hline
        % Hate Precision & 0.679 &    0.712   & 0.722 &\textbf{0.760}  & 0.748 &    0.741    \\
        % Hate Recall    & 0.621 &    0.631   & 0.642 &\textbf{0.666}  & 0.621 &    0.658    \\
        % Hate F1        & 0.648 &    0.668   & 0.679 &\textbf{0.710}  & 0.679 &    0.697    \\
        \bottomrule
        \end{tabular}
    \caption{Resultados de los experimentos de clasificación para la tarea \emph{granular} de detección de discurso de odio, expresados como la media de las distintas métricas sobre diez corridas independientes. Cada modelo es un BERT con tres posibles entradas: sólo el comentario (\emph{Sin contexto}), el tweet de la noticia a la cual responde el comentario (\emph{Tweet}), y el tweet más el cuerpo de la noticia (\emph{Tweet + Cuerpo}). Para cada una de estas posibilidades usamos dos versiones: una sobre BETO($\neg$FT) y otra sobre BETO ajustado al dominio (FT) de acuerdo a lo descripto en la Sección \ref{sec:contextualized_classifiers}}
    \label{tab:task_b_results}
\end{table*}



La Tabla \ref{tab:task_b_results} muestra los resultados de los experimentos de clasificación para la tarea de detección \tbf{granular} medidos en puntos de F1 para cada una de las características, y las métricas promediadas de precision, recall, y F1. Como era esperable, la ganancia de tener contexto disponible es más evidente en esta tarea, observándose una diferencia de aproximadamente 6 puntos de F1 entre la mejor versión sin contexto y la mejor versión con contexto ($55.1$ Macro F1 de la versión $FT$ sin contexto vs $61.3$ F1 de la versión $FT$ con el contexto del tweet). Respecto a los dos tipos de contexto, de nuevo la versión simple obtiene mejor rendimiento en prácticamente todas las características, con la única excepción de POLITICA.


\begin{figure*}[]
    \centering
    \includegraphics[width=\textwidth]{img/task_b_scores.pdf}
    \caption{Métrica F1 para cada característica de la tarea granular. Las características están ordenadas de mayor a menor de acuerdo a la diferencia de performance entre el modelo sin contexto y el modelo contextualizado. }
    \label{fig:barplot_task_b_results}
\end{figure*}

La Figura \ref{fig:barplot_task_b_results} muestra los resultados por característica ordenados de mayor a menor según la diferencia de rendimiento entre los clasificadores ajustados a dominio que consumen contexto y aquellos que no lo hacen. Para todas las características se observa una mejora estadísticamente significativa al correr un test Mann-Whitney U ($p \leq 0.005$, p valores ajustados por múltiples comparaciones con Benjamini-Hochberg \cite{benjamini1995controlling}). Las diferencias más sustanciales se dan en el caso de CRIMINAL ($+17$ puntos F1 de diferencia), LGBTI ($+12$ puntos), CLASE ($+8$ puntos), y RACISMO (casi $+7$). Del otro lado, las características con menos mejora son APARIENCIA y POLITICA, algo esperable dado que el fenómeno tiene características poco dependientes del contexto como observamos en algunos ejemplos de la Sección \ref{sec:analisis_dataset_por_caracteristica} y por la misma definición y ejemplos considerados (ver Apéndice \ref{app:05}). Finalmente, y como resumen de estas tablas, se observa que los modelos con contexto simple son los que mejor performance tienen, en general y para cada característica, con la excepción de POLITICA.


\begin{figure}[ht!]
    \centering
    \small
    \begin{subfigure}[b]{\textwidth}
        \centering
        \caption{Precisión}
        \includegraphics[width=0.75\textwidth]{img/06/precision_barplot.pdf}

    \end{subfigure}
    \begin{subfigure}[b]{\textwidth}
        \centering
        \caption{Sensibilidad exacta}
        \includegraphics[width=0.75\textwidth]{img/06/exact_recall_barplot.pdf}

        \label{subfig:exact_recall}
    \end{subfigure}
    \begin{subfigure}[b]{\textwidth}
        \centering
        \caption{Sensibilidad total}
        \includegraphics[width=0.75\textwidth]{img/06/hate_recall_barplot.pdf}

        \label{subfig:total_recall}
    \end{subfigure}
    \caption{Precisión y sensibilidad para cada característica de la tarea granular. Las diferentes barras marcan el tipo de contexto que recibe el clasificador. Sensibilidad exacta (Figura \ref{subfig:exact_recall}) cuenta la sensibilidad sobre la salida exacta de cada categoría, mientras que la sensibilidad total cuenta como recuperado un tweet si al menos alguna característica del clasificador lo marca como discurso de odio.}
    \label{fig:precision_recall_granular_classifier}
\end{figure}

Una forma distinta de evaluar el rendimiento de los clasificadores granulares es de acuerdo a su sensibilidad o capacidad de recuperación de comentarios discriminatorios, aún cuando esto ocurra por el motivo incorrecto. La Figura \ref{fig:precision_recall} ilustra esta comparación analizando la sensibilidad de dos maneras: exacta, donde consideramos recuperado un tweet sólo si el clasificador acierta a la característica analizada (es decir, si la característica es MUJER, el clasificador debe predecir MUJER); y total, donde consideramos un tweet recuperado si alguna característica es marcada por el clasificador independientemente de si es la correcta. Podemos ver que la categoría MUJER pasa de ser la de menor sensibilidad a dejar a la categoría LGBTI como la que tiene menor tasa de comentarios ofensivos recuperados. Análogamente, la categoría CLASE obtiene una mejora sustancial en su sensibilidad, algo compatible con el hecho de su solapamiento con otras características como RACISMO, POLITICA y CRIMINAL observado en la Figura \ref{fig:heatmap_characteristics}. También puede observarse que, en líneas generales, el contexto favorece una mejora de la precisión para cada característica y también un mayor recall exacto. Para el caso de la sensibilidad total, el desempeño de los clasificadores se empareja entre las versiones contextualizadas y no contextualizadas para cada característica con la excepción notoria de LGBTI y CRIMINAL.

\begin{table}[hb!]
    \centering
    \begin{tabular}{l P{0.10\textwidth}P{0.10\textwidth} P{0.10\textwidth}P{0.10\textwidth}  P{0.10\textwidth}P{0.10\textwidth}}
                  &\mc{2}{Sin Contexto}& \mc{2}{Tweet}          &  \mc{2}{Tweet + Cuerpo}    \\
                  & Bin   &    Gran         & Bin   &    Gran     & Bin  &   Gran     \\
        \hline
        Precision &  $71.8$ &  $71.1$       &  $74.8$& $\mbf{75.9}$ & $72.8$ & $74.0$ \\
        Recall    &  $60.2$ &  $63.6$       &  $65.3$& $\mbf{66.7}$ & $64.1$ & $66.0$ \\
        F1        &  $65.5$ &  $67.1$       &  $69.7$& $\mbf{71.0}$ & $68.1$ & $69.7$ \\
        Macro F1  &  $79.8$ &  $80.6$       &  $82.2$& $\mbf{83.0}$ & $81.3$ & $82.2$ \\
        \hline
        \end{tabular}
    \caption{Desempeño de los modelos para la tarea de detección binaria de discurso de odio. Los modelos considerados son modelos \beto{} ajustados a dominio que consumen tres tipos de entrada: sin contexto, tweet, y tweet+cuerpo. Dos posibles entrenamientos fueron evaluados: sobre la tarea binaria (\tbf{Bin}) o sobre la tarea granular (\tbf{Gran}).}
    \label{tab:plain_vs_granular_hate_detection}
\end{table}


Como se mencionó en la Sección \ref{sec:tasks}, un clasificador sobre la tarea granular puede convertirse fácilmente en uno para la tarea binaria tomando la disyunción lógica de sus salidas: si se detecta al menos una característica ofendida, entonces el tweet contiene discurso de odio. De esta forma, podemos evaluar el desempeño en la tarea binaria de aquellos clasificadores entrenados para la tarea granular. La Tabla \ref{tab:plain_vs_granular_hate_detection} muestra esta comparativa para las distintas métricas sobre los clasificadores ajustados a dominio. Podemos observar que, en todos los casos, entrenar el modelo sobre la tarea granular produce pequeñas mejoras en la performance de los clasificadores entrenados sobre la tarea binaria (aproximadamente $+0.8$ puntos de Macro F1 en cada de uno).



\section{Análisis de error}

En esta sección realizamos un análisis de los errores de los clasificadores y comparamos sus distintas variantes. Para ello, analizamos las diferencias entre:

\begin{itemize}
    \item las predicciones del mejor clasificador en términos de performance contra las etiquetas doradas del dataset.
    \item las predicciones del mejor clasificador contextualizado contra las predicciones del mejor clasificador no contextualizado.
    \item las predicciones entre los clasificadores entrenados sobre contra las predicciones de los entrenados sobre la tarea binaria.
\end{itemize}

Para analizar los errores de los modelos, elegimos el clasificador de mejor performance sobre la tarea granular: el clasificador que consume el contexto más el tweet de la noticia, entrenado sobre un \beto{} ajustado a dominio (ver Tabla \ref{tab:task_b_results}). De una manera similar a lo que realizamos en la Sección \ref{sec:hateval_error_analysis}, entrenamos diez clasificadores y analizamos el error sobre el ensamble de voto mayoritario. Para ver los casos más problemáticos, analizamos aquellas características donde peor desempeño muestran los clasificadores: MUJER, LGBTI, y CLASE. Por lo observado en la Figura \ref{fig:precision_recall_granular_classifier}, los clasificadores tienen una sensibilidad muy baja para estas tres características, por lo cual centraremos nuestro análisis en los casos falsos negativos.


\begin{table}[t]
    \centering
    \small
    \begin{tabular}{p{0.03\textwidth} p{0.45\textwidth} p{0.40\textwidth}}
          &Contexto & Comentario \\
        \hline
        1 & Contó que era lesbiana, su papá le confesó que era gay y ahora su madre se enamoró de una mujer, lo que inspiró su segundo film & WTF. Mucho ESI los degeneró... \\
        \rule{0pt}{3ex}2 &                        & @usuario Esta familia tiene los genes alterados \\
        \hline
        3 & Oscar González Oro ya está instalado en el Uruguay: "Recuperé mi libertad" & Ahora quedate allá, y hablá mal d Macri d nuevo pa tener rating. Y opiná q los taxi boy uruguayos son mas educados q los escorts argentinos! Ano abierto! \\
        \hline
        4 & ¿Por qué un beso entre dos hombres los vuelve tan violentos?": la vida después de haber sido víctima de ataques  homofóbicos &  Será xq va contra la naturaleza de la raza... \\
        \hline
        5 & ``¿Por qué no vemos médicos trans?'': el reclamo de un prestigioso cardiólogo para que América sea más inclusiva & Es difícil ser médico con la cabeza quemada \\
        \rule{0pt}{3ex}6  &  & porque un enfermo no cura a otro enfermo \\
        \hline
        7 & ``Te amo''. La emotiva dedicatoria de Luis Novaresio a su pareja en su cumpleaños & \emoji{face-vomiting}\emoji{face-vomiting}\emoji{face-vomiting} \\
        \hline
        8 & Elizabeth Gómez Alcorta: ``Por la pandemia, vamos a tener una suba de los femicidios y travesticidios'' &  Travesticidios... Osea asesinatos de tipos con peluca y tetas \\
        \hline
        9 & Mariana Genesio Peña pasa su cuarentena total con guantes, barbijo y desnuda: ``Mi cuerpo es el planeta Tierra'' & Coronavirus nivel pelotudo en bolas	\\
        \rule{0pt}{3ex}10 &    & Con 3 piernas cualquiera es feliz!!!	\\
        \rule{0pt}{3ex}11 &    & pasa la cuarentena rascándose las bolas \\
        \hline
        12 & Tras una ráfaga de más de 20 disparos asesinaron a una mujer trans en Rosario & Cómo no saco su escopeta y aplicó la defensa propia?! \\
        \rule{0pt}{3ex}13 & & @usuario Salió de caño... cuac!	\\
        \hline
    \end{tabular}
    \caption{Falsos negativos para la característica LGBTI. Ninguno de los diez clasificadores que consumen contexto y texto indentificaron como discriminatorios a estos comentarios. }
    \label{tab:lgbti_error_analysis}
\end{table}



La Tabla \ref{tab:lgbti_error_analysis} muestra una selección de comentarios discriminatorios para la característica LGBTI que no son detectados por los clasificadores. De estas instancias, y observando también aquellos casos donde sí puede detectar el discurso discriminatorio de esta categoría, podemos esbozar algunas posibles razones detrás de estos errores. En primer lugar, ciertos mensajes altamente ofensivos son complejos de entender: por ejemplo, los que tratan de ``enfermos'' o mencionan cuestiones de la genitalidad (ejemplos 5 y 6). Estas ofensas son realizadas por los usuarios de maneras tan sofisticadas que difícilmente un modelo de lenguaje pueda entender, como metáforas varias que hacen referencia a la genitalidad de una mujer trans (ejemplos 10 y 12) o bien refiriéndose al objetivo de la ofensa con un género distinto al autopercibido (ejemplos 8 y 9),

Algunos comentarios y sus contextos --como los ejemplos 3, 7, 9, 10 y 11-- omiten información acerca de la sexualidad o género sobre quienes versa la noticia. Esta información faltante no permite a los clasificadores (ni tampoco a un humano que carezca de esta información) entender completamente el caracter discriminatorio de los mensajes.


\begin{table}[t]
    \centering
    \small
    \begin{tabular}{p{0.03\textwidth} p{0.40\textwidth} p{0.40\textwidth}}
        & Contexto & Comentario \\
        \hline
        1 & Martha Rosenberg: ``En situación de pandemia, legalizar el aborto es más urgente que nunca'' & Quien es esta vieja?. No debería estar tejiendo? \\
        \hline
        2& Mara Gómez: la historia de la primera futbolista trans en el torneo argentino  &  Feminismo pierde de nuevo... ya le metieron un tipo... jaja punto para el patriarcado...	 \\
        \hline
        3& Tras una ráfaga de más de 20 disparos asesinaron a una mujer trans en Rosario & Las feministas en modo error 404 al no saber si celebrar o ofenderse  \\
        \hline
        4& El desesperado pedido de Actrices Argentinas ante la violencia de género en cuarentena: ``Es urgente'' & Que risa me dan las feministas!!! Ignorantes.	 \\
        \hline
        5& Leche de cucaracha, la nueva bebida nutritiva: ¿quién se anima a probarla? & No me digas q la hija de CFK está embarazada y ya sale leche por esos senos	 \\
        \hline
        6& Los fans de Florencia Kirchner le piden casamiento por Instagram & Zoofilia \\
        \rule{0pt}{3ex}7&                       & Hdp tienen que tener estomago para querer casarse con terrible adefesio \\
        \hline
        8& Rosario: para sacar una licencia de conducir habrá que hacer un curso de perspectiva de género & Te quieren adoctrinar desde cualquier ámbito, y se están metiendo en todo para que empieces a hablar como el orto, como a ellas les gusta.	\\
        \rule{0pt}{3ex}9&   & El que choca más feministas le dan más años de licencia	\\
        \hline
        10& Por qué los países liderados por mujeres parecen haber respondido mejor a la crisis del coronavirus & Son mujeres inteligentes que se dejan asesorar de sus esposos \\
        \hline
        11& Joe Biden presentó su nuevo equipo de comunicación compuesto enteramente por mujeres & Será equipe conche seque?	\\
         \hline
    \end{tabular}
    \caption{Falsos negativos para la característica MUJER. Ninguno de los diez clasificadores que consumen contexto y texto (ajustados a dominio) lograron identificar como discriminatorios a estos comentarios para ninguna otra característica. }
    \label{tab:women_error_analysis}
\end{table}




En el caso de MUJER, restringimos nuestro análisis a aquellos ejemplos que no son detectados por otras características ya que, por lo observado en la Figura \ref{fig:precision_recall_granular_classifier}, muchos ejemplos están en la frontera de otras categorías y son detectados por ellas. La Tabla \ref{tab:women_error_analysis} muestra una selección de estos falsos negativos, donde podemos apreciar que algunos ejemplos están en el borde borde de ser simplemente ofensivos (ejemplo 1 y posiblemente 6 y 7) o contienen mensajes irónicos complejos de descifrar (ejemplo 3, que las feministas celebren por la muerte de una mujer trans, o ejemplo 9, hablar de chocar mujeres en un curso de manejo). Esto es esperable dado el acuerdo relativamente bajo para la categoría MUJER en el etiquetado de este dataset reportado en la Tabla \ref{tab:annotation_agreement}.




\begin{table}[ht!]
    \centering
    \footnotesize
    \tbf{CRIMINAL}
    \begin{tabular}{p{0.03\textwidth} p{0.04\textwidth} p{0.40\textwidth} p{0.40\textwidth}}
        \hline
        & Tipo & Contexto & Comentario \\
        \hline
         1 & \multirow{2}{*}{FN} &Una policía baleó y mató a un joven de 17 años que la atacó con una tijera en Moreno url	& \emoji{clapping-hands} excelente!  \\
                             %&                                                                                          &  bellezaa \\
        \rule{0pt}{3ex}2 &                &El polémico cortejo del ladrón asesinado por el jubilado en Quilmes                       & Le hubieran puesto una bomba al cortejo \\
         \hline
         \rule{0pt}{3ex}3 & \mr{2}{FP}          & Ivana Nadal se cansó de la criticaran y sorprendió con su respuesta: "Gracias a Dios a fin de año me voy del país" & Esa si es una buena noticia. \\
                             %& Mora Godoy cierra su escuela de tango y remata su vestuario para "poder seguir adelante" & Adoro los finales felices \\
         \rule{0pt}{3ex}4 &                &¿Se va del país? Juana Viale estaría tramitando la ciudadanía uruguaya & Q bueno ? Una mierda menos . Q se quede alla \\
         \hline
    \end{tabular}
    \rule{0pt}{3ex}\tbf{LGBTI}
    \begin{tabular}{p{0.03\textwidth} p{0.05\textwidth} p{0.45\textwidth} p{0.40\textwidth}}
        \hline
        \rule{0pt}{3ex}6 & \mr{2}{FN} & ``Te amo''. La emotiva dedicatoria de Luis Novaresio a su pareja en su cumpleaños & Definitivamente no acepto esta degeneración repugnante de la humanidad. \\
        \rule{0pt}{3ex}7 &        &  Mara Gómez: la historia de la primera futbolista trans en el torneo argentino & Sigue siendo HOMBRE, que por GENÉTICA, no por una ideología u orientacion sexual, GE-NÉ-TI-CA, es más fuerte que la mujer. \\

                    %&  Contó que era lesbiana, su papá le confesó que era gay y ahora su madre se enamoró de una mujer, lo que inspiró su segundo film& Revisen esa casa, los están envenenando\\
        \hline
        \rule{0pt}{3ex}8  & \mr{2}{FP} &  URGENTE: Un hombre se incrustó con su auto en la puerta de la Embajada de China y aseguró que tenía explosivos & No es hombre . Es un boludo	\\
         \rule{0pt}{3ex}9   &      &  Detuvieron al hombre que admitió violar a su hija en audios de WhatsApp y fue tendencia en redes: un tío lo entregó & Degenerado \emoji{face-vomiting} \\
        \hline
    \end{tabular}
    \rule{0pt}{3ex}\tbf{CLASE o RACISMO}
    \begin{tabular}{p{0.03\textwidth} p{0.05\textwidth} p{0.45\textwidth} p{0.40\textwidth}}
        \hline
        10 & \mr{3}{FN} & Las organizaciones sociales salieron al cruce de la acusación de Sergio Berni por la toma de tierras:  ``La falta de vivienda no se resuelve con balas'' & Si, los podemos cagar matando a todos y listo \\
                   %&                                       & @usuario Dale alas y planes a los cuervos y veras cómo te sacan los ojos  \\
        \rule{0pt}{3ex}11 & & La temporada de verano en la Costa Atlántica empezó con un corte total en la Ruta 2: organizaciones sociales piden canastas navideñas & Hay que desparasitar urgente el país. \\
        \rule{0pt}{3ex}12 & &  La pregunta billonaria: ¿China debería pagar el costo de la pandemia? & Si obviamente y desaparecer de la faz de la tierra. Mira el quilombo que armó. Se nos están muriendo todos... \\
        \hline
        13 & \mr{4}{FP} & Javier Milei confirmó que va ``a militar en política'' junto a José Luis Espert para que ``en 35 años la Argentina sea primera potencia mundial'' & Otro parásito	 \\
                   %& Leche de cucaracha, la nueva bebida nutritiva: ¿quién se anima a probarla? & Uds pueden producir litros, son todes cucarachas y ratas!!!  \\
                   \rule{0pt}{3ex}14 &            & Martha Rosenberg: ``En situación de pandemia, legalizar el aborto es más urgente que nunca'' & La verdad que sí...así se dejan de reproducir!!!  \\
                   \hline
        15 &            & Confusión por la foto de Alberto Fernández con activistas veganos en medio de las negociaciones con China por el acuerdo porcino & Quilombo en puerta con China, son un desastre \\
        16 &            & La aberrante arenga machista que dio el preparador físico de Güemes de Santiago del Estero & Un capo. \\
        \hline
    \end{tabular}
    \caption{Ejemplos donde el clasificador contextualizado acierta y el no contextualizado falla. FN marca que el clasificador no contextualizado no detecta el comentario como discriminatorio (ni para la característica marcada ni para otras) mientras que el contextualizado sí lo hace; FP es al revés, que el clasificador no contextualizado marca erróneamente el comentario como discriminatorio  }
    \label{tab:context_vs_no_context_error}
\end{table}


En la Tabla \ref{tab:context_vs_no_context_error} podemos observar una selección de ejemplos donde el clasificador contextualizado acierta en su predicción mientras que el modelo sin contexto se equivoca, separados en falsos negativos (el modelo no contextualizado no logra detectarlos) y falsos positivos (el modelo no contextualizado predice equivocadamente que son discriminatorios). En el caso de LGBTI, la información contextual permite desambiguar casos como los 7 y 8, muy similares pero con un contexto sumamente distinto. Un problema que se puede apreciar es que el clasificador no contextualizado --al ser entrenado con datos etiquetados de manera contextualizada-- aprende correlaciones espurias como que las celebraciones son discriminatorias (ejemplo 16).

La comparativa entre los clasificadores entrenados sobre la tarea binaria y la tarea granular es más difícil de interpretar. Si bien se observa que entre los falsos negativos del clasificador binario se encuentra una proporción más alta de tweets racistas, es difícil elucidar una razón detrás de esto. En la Tabla \ref{tab:fine_vs_plain_comparison} del Apéndice \ref{app:06} se encuentra una muestra de ejemplos en los cuales el clasificador granular acertó y el binario falló.




\section{Discusión}

Para analizar el impacto del contexto en la detección de discurso de odio, planteamos dos tareas de clasificación sobre el dataset construído en el capítulo \ref{chap:05_dataset_creation}: la tarea de detección binaria, donde respondemos si un comentario contiene discurso de odio; y una tarea de detección granular, donde se debe mencionar la o las características protegidas ofendidas (si agrede a las mujeres, al colectivo LGBTI, si es racista, etc). En ambas tareas planteadas propusimos clasificadores que consumen 3 tipos de entrada: el comentario sin contexto, el comentario con el contexto del tweet al que responden, y el comentario más el tweet al que responde y también el texto del artículo. Pudimos observar que el contexto parece dar una mejora moderada pero significativamente estadística en la tarea de detección del discurso de odio (alrededor de 3 puntos F1), y una mejora considerable en la tarea característica ofendida (alrededor de 6 puntos F1 medios).

Esto, en algún punto, indicaría que el contexto puede ser aprovechado para mejorar los algoritmos de detección de discurso de odio. Si bien este resultado podría estar en aparente contradicción con trabajos recientes que no encontraron ninguna mejora en el uso del contexto en la detección de toxicidad \cite{pavlopoulos2020toxicity}, se puede señalar que la detección del discurso de odio es una de las formas más complejas de comportamiento ``tóxico'' y, como tal, podría permitir que los clasificadores tengan más información para predecir si el texto dado es odioso o no. Otra razón detrás de este resultado es el dominio de nuestro conjunto de datos: mientras que \citet{pavlopoulos2020toxicity} usa el contexto conversacional, nosotros usamos el título y el cuerpo del artículo como contexto para los comentarios de los usuarios. Más recientemente, y como marcamos en la sección de trabajo previo, en \citet{xenos-2021-context} han observado que, restringiendo el análisis a un subconjunto de comentarios sensibles al contexto (ver \ref{sec:06_classification_previous} y \ref{sec:dataset_previous} para más información), la performance de los algoritmos de detección de toxicidad mejoran de manera significativa.

La utilización de un contexto más largo como el artículo de la noticia no mejora la performance de los clasificadores en nuestros experimentos. Hay varias interpretaciones posibles de esto: en primer lugar, los humanos suelen contestar sin leer el artículo, con lo cual este resultado pareciera ser coherente con esta observación. Por otro lado, los humanos solemos tener acceso a un contexto mucho más rico, muchas veces equivalente a haber leído la nota, algo que nuestros clasificadores carecerían. Podría también esto ser atribuido al modelo pre-entrenado que usamos para codificar esto (BETO, la variante en español de BERT) suele estar pre-entrenada para textos más cortos. Teniendo esto en cuenta, realizamos el ajuste de dominio usando el contexto largo, pero aún así la performance del clasificador que consume este contexto largo se mantuvo por debajo del que usaba el contexto simple.

El análisis del error realizado demostró que, si bien el contexto pareciera mejorar la detección de discurso de odio, para muchas características protegidas se mantiene aún como una tarea difícil. Un caso ejemplificador de esto es la discriminación contra el colectivo LGBTI. En las instancias del dataset --y en muchos de los ejemplos en los que el algoritmo de detección falla--  puede verse que las agresiones contra este colectivo y sus miembros son sumamente sofisticadas, lejos de las agresiones meramente basadas en insultos u otras palabras ofensivas. Nuestros clasificadores, aún en sus mejores versiones (usando ajuste de dominio y contexto) obtuvieron una baja performance en la detección de este fenómeno (alrededor de $0.5$ F1 score) dando cuenta que es un problema no trivial y merece ser analizado más detenidamente debido a la complejidad de estos mensajes, que suelen reunir ironía, metáforas, y otros artilugios que hacen difícil su detección.

En el caso de la categoría mujer, inesperadamente, también obtuvimos una performance muy baja en la detección de agresiones misóginas. En el análisis de error, podemos observar que tenemos casos complejos de descifrar que fueron marcados por nuestros anotadores: por ejemplo, ataques velados a mujeres víctimas de violación (llamarlas mentirosas). \todo{Escribir un poco más sobre esto}

Algo que debe ser tenido en cuenta para matizar estos resultados es que utilizamos un amplio espectro de características protegidas. Incluso, la que más se beneficia del contexto es una que introdujimos ad-hoc para este experimento (discurso de odio contra criminales). En contrapeso, otras características ``no convencionales'' son poco beneficiadas por el contexto (como discurso de odio en base a la apariencia, opinión política y discapacidad).

Un resultado que también observamos es que pareciera ser que nuestros clasificadores mejoran leve pero significativamente su performance en un contexto de aprender cada característica por separado en vez de sólo aprender a distinguir la etiqueta binaria de discurso de odio. Si bien la mejora es marginal (cerca de un punto de F1) y no es observable de manera subjetiva mediante un análisis de error, una posible razón detrás de esto es que la señal más precisa acerca de la categoría ofendida puede ayudar a distinguir mejor las fronteras de este fenómeno. Aún cuando esta mejora no fuese tal, poder tener una salida más interpretable y granular es mejor que simplemente obtener una predicción binaria.

Una limitación importante de este estudio es que el entrenamiento lo realizamos sobre datos entrenados únicamente observando el contexto. Entrenar a los clasificadores no contextualizados sobre estas etiquetas induce a los clasificadores a tomar decisiones erróneas, como asumir que celebraciones son discurso de odio debido a instancias que --con contexto-- tienen esa naturaleza. Un estudio más completo del impacto del contexto en esta tarea debería incluir los datos entrenados sin contexto.

En el terreno de la aplicación, un problema práctico de este resultado es que no siempre tenemos un contexto disponible para un texto dado. Incluso si podemos encontrarlo, a veces este contexto puede no ser en forma de artículo de noticias, sino como un hilo de conversación o incluso de alguna otra representación. Teniendo en cuenta alguna de las consideraciones hechas en esta discusión, una posible línea de investigación seria la de incorporar distintos tipos de contexto, desde más mensajes en el hilo de la conversación, conocimiento estructurado (por ejemplo, la propuesta en \emph{ERNIE} \cite{zhang2019ernie} o \emph{KI-BERT} \cite{faldu2021ki}) o bien una combinación de diversas fuentes, incluso multimodales.

\section{Conclusiones}

En este capítulo hemos realizado experimentos de clasificación sobre el dataset construído en el capítulo anterior, focalizándonos en analizar el impacto de la utilización del contexto en la performance de los modelos. Planteamos dos tareas: una tarea binaria --detectar si existe discurso de odio o no-- y una tarea granular --definir qué categorías son atacadas en un tweet, si es que las hay--. Para ambas tareas, los modelos contextualizados obtienen mejoras significativas en la performance, dando indicios de que información adicional al comentario analizado puede ayudar a detectar el discurso de odio. Si bien en nuestros experimentos el contexto más pequeño (el tweet del artículo de la noticia) fue el que mejor resultados obtuvo, una línea de trabajo futuro podría explorar otras formas de incorporar el contexto más largo -- en este caso, el artículo de la noticia.

Así mismo, observamos una pequeña pero estadísticamente significativa mejora en la detección de discurso de odio al entrenar un clasificador granular al ser evaluado de manera binaria. En este caso, obtenemos una ventaja al obtener una salida más interpretable de las características ofendidas, y que además que no sólo no empeora la performance de nuestro clasificador sino que hasta incluso mejora levemente.

Del análisis de error, se observa que algunas categorías del discurso de odio se muestran elusivas para los algoritmos de detección. Uno de estos casos son los mensajes abusivos contra la comunidad LGBTI+, conteniendo mensajes semánticamente complejos, implícitos y con metáforas que son esquivas para los modelos propuestos. A pesar de estas limitaciones, esta característica fue una de las más beneficiadas por la adición de contexto, aunque su desempeño sigue siendo bajo, teniendo una puntuación de F1 de alrededor de 0.5.

Podemos concluir que los datasets de discurso de odio deberían --en la medida de lo posible-- contener \textbf{información contextual} sobre los comentarios analizados. Esta información puede darse en forma de artículos de noticias, como un hilo de conversación, o incluso como otras formas --por ejemplo, como una base de conocimiento. Sobre esto, trabajo futuro debería explorar el impacto de utilizar esta información adicional para integrarla en algoritmos de detección de discurso de odio. La evidencia de estos experimentos --por ahora preliminares, y con las limitaciones marcadas en la discusión-- indica que los modelos del estado del arte pueden utilizar esta información para mejorar la detección de discurso de odio en redes sociales. En segundo lugar, los datasets de discurso de odio deberían incluir \textbf{información granular} acerca de las características atacadas --y no sólo una etiqueta binaria-- ya que por un lado esto mejora la interpretabilidad de los algoritmos de detección, y resultados preliminares de este estudio indicarían que mejoran marginalmente la performance en la detección en general.

Finalmente, un aspecto que introdujimos en este capítulo fue el de adaptar un modelo de lenguaje pre-entrenado a su dominio, siendo en nuestro caso los comentarios sobre notas periodísticas en Twitter. Las mejoras que reportó la utilización de estas técnicas fue significativa, en consonancia con otros trabajos recientes. Pasaremos a continuación a estudiar estas técnicas en el marco más general de la clasificación de textos sociales.
%

\part{Adaptación de Dominio}

\chapter{Adaptación de dominio}
\label{chap:07_domain_adaptation}
\newcommand{\deacc}[0]{\textbf{deacc}}
\newcommand{\cased}[0]{\textbf{cased}}
\newcommand{\uncased}[0]{\textbf{uncased}}

En este capítulo analizamos cómo mejorar la detección de discurso de odio desde una perspectiva más general, centrándonos en aplicar técnicas de \textbf{adaptación de dominio} a los algoritmos de clasificación que han sido considerados hasta el momento. Como hemos mencionado en la Sección \ref{sec:03_preprocessing}, cuando hablamos de adaptación de dominio nos referimos al el conjunto de técnicas y recursos destinados a tratar de que los algoritmos tengan un correcto desempeño en un subconjunto de tareas relacionadas entre sí \cite{goodfellow2016deep}. En el caso concreto de las tareas cuyas instancias constan de analizar texto de usuarios en redes sociales u otros medios, \citet{eisenstein2013bad} describió a la adaptación de dominio en términos de la necesidad de construir herramientas propias para este tipo de texto, muy distinto al lenguaje formal proveniente de otras fuentes.

Las técnicas de representación modernas --desde los embeddings hasta los modelos de lenguaje-- suelen ser entrenadas sobre conjuntos de datos que se suponen lo suficientemente generales. Fuentes usuales son Wikipedia, que comprende textos de carácter enciclopédico, o Common Crawl, que es una recopilación de datos de distintos sitios web. El uso del lenguaje en estas fuentes suele guardar una considerable discordancia con el de muchas tareas de interés en NLP, como son aquellas basadas en textos provenientes de ciertos nichos donde el uso del lenguaje es muy específico. Ejemplos de esto son los documentos médicos, trabajos científicos, entre otros. A cada uno de estos grupos de textos con cierta relación se los denomina --de manera poco precisa-- \textbf{dominios}. Entre estos, el contenido informal de las redes sociales tiene variedades lingüísticas muy particulares, con mucha jerga, expresiones coloquiales, errores ortográficos y demás que diferencian el uso del lenguaje de los textos fuentes mencionados.

Continuamos en este Capítulo alguna de las ideas y observaciones que hemos analizado previamente acerca de la adaptación de dominio. En primer lugar, en el Capítulo \ref{chap:03_social_text_classification} se ha observado que las técnicas de representación --desde los word-embeddings hasta los modelos de lenguajes-- tienen un buen desempeño sobre tareas de textos de redes sociales son entrenadas sobre este mismo dominio. Para el idioma español, sin embargo, no existe ningún modelo de lenguaje fácilmente accesible de estas características, como BERTweet para el inglés o AlBERTo en italiano. Describimos entonces el proceso de entrenamiento desde cero de un modelo de lenguaje sobre tweets en español, al cual llamamos \robertuito{}. Evaluamos su rendimiento compilando todas las tareas analizadas en esta tesis, y mostramos que es superior a todos los modelos considerados hasta el momento.

En segundo lugar, retomamos otro enfoque para adaptar las técnicas existentes al dominio de redes sociales. En el Capítulo \ref{chap:06_contextualized_hate_speech} conseguimos mejorar la performance de los algoritmos de clasificación mediante la continuación del pre-entrenamiento del modelo de lenguaje sobre un gran conjunto de datos de noticias y comentarios no etiquetados, los cuales fueron recolectados como parte del proceso de construcción del dataset usado. Trabajo reciente muestra que esta técnica es generalizable a muchos dominios --como textos médicos, científicos, entre otros-- y resulta en mejoras consistentes del rendimiento de los algoritmos de clasificación basados en \bert{} y similares. En nuestro caso, analizamos el efecto de continuar el pre-entrenamiento ya no sólo sobre la tarea de detección contextualizada de discurso de odio sino sobre todas las tareas que hemos visto en los diferentes capítulos.

Finalmente, realizamos una comparación entre estas dos aproximaciones a adaptar nuestros algoritmos a este medio, algo de lo que no tenemos conocimiento se haya realizado hasta el momento, al menos para el idioma español. Este punto es de interés dado el gran costo que tiene entrenar modelos basados en Transformers, y sirve para verificar si el salto en el rendimiento de los modelos generados para dominios específicos puede subsanarse mediante una alternativa más económica como es la continuación del pre-entrenamiento.

Comenzamos en la Sección \ref{sec:domain_adaptation_previous_work} haciendo un racconto de las técnicas de adaptación de dominio y modelos pre-entrenados. Describimos a continuación la  construcción y el entrenamiento de \robertuito{} \cite{perez2021robertuito}. Finalmente, detallamos los experimentos de adaptación utilizando \beto{} como modelo de lenguaje a adaptar para luego comparar su desempeño contra \robertuito{}.

\section{Trabajo previo}
\label{sec:domain_adaptation_previous_work}

La definición de qué es un \textbf{dominio} en NLP suele ser relativamente amplia y poco precisa. Una posible aproximación a este concepto es la de un conjunto de textos que guardan cierta similaridad respecto al tópico o género; al medio utilizado; el público al cual apuntan, entre otras cosas. Algunos ejemplos de dominios podrían ser los artículos de noticias, las novelas u otros libros de ficción, discursos políticos, comentarios de redes sociales, entre otros \cite{gururangan-etal-2020-dont}. Subdisciplinas del procesamiento del lenguaje natural tienen su eje en tratar estas distintas categorías de documentos atendiendo sus particularidades: BioNLP, SocialNLP, entre otras nuevas denominaciones.

\citet{goodfellow2016deep} definen la adaptación de dominio como una situación similar a la de Transfer Learning: dado un modelo que fue entrenado para una tarea T y una distribución de datos $P_1$, se lo quiere utilizar sobre la misma tarea T pero esta vez con una distribución de entrada $P_2$ bajo la asunción de que ambas distribuciones son relativamente similares. Un escenario posible es el de un clasificador de polaridad entrenado sobre comentarios acerca de reviews de libros, el cual queremos utilizar para analizar reviews de productos electrónicos. \citet{glorot2011domain} es uno de los primeros trabajos que aplica la idea de adaptación de dominio sobre modelos de Deep Learning en NLP. Los autores emplean \emph{stacked denoising auto-encoders} para aprender características no supervisadas de los textos; para atacar la diferencia de distribuciones de entrada, realizan pre-entrenamiento no supervisado para los dominios analizados. Desde una óptica diferente, \citet{eisenstein2013bad} describe puntualmente la adaptación de dominio para textos de redes sociales como ``adaptar las herramientas al texto'' (social), contraponiendo esto al concepto de \emph{normalización} de la entrada, que sería intentar ``adaptar el texto a las herramientas''. Dentro de las tareas de extracción de opiniones, la adaptación resulta importante ya que expresiones en distintos ámbitos pueden tener sentidos distintos: decir ``leé el libro'' en una reseña de un libro Amazon puede ser algo positivo, mientras que en el comentario de una película puede ser considerado negativo \cite{pang2008opinion}.


Dentro de la ola de modelos pre-entrenados que sacudió el mundo de NLP, la técnica de \ulmfit{} \cite{howard-ruder-2018-universal} (descripta en la Sección \ref{subsec:elmo}) contempla tres etapas: pre-entrenamiento, ajuste de dominio, y ajuste discriminativo. En la etapa intermedia, se adaptan los pesos de la red neuronal utilizando de manera no-supervisada el texto del dataset, realizando una continuación del pre-entrenamiento de modelado de lenguaje. Los modelos basados en Transformers como BERT \cite{devlin2018bert} y subsiguientes eliminaron esta etapa intermedia, dejando sólo la adaptación de los pesos sobre las etiquetas supervisadas. Recientemente, se ha observado que reintroducir esta adaptación no supervisada es algo beneficioso.

Siguiendo esta idea, y restringiendo nuestro interés a las técnicas actuales de clasificación, puede ser beneficioso ajustar un modelo de lenguaje a un dominio distinto al que fue utilizado en su pre-entrenamiento: puntualmente, queremos adaptar \bert{} (entrenado en Wikipedia) a textos de carácter más informal. Si bien se ha observado que los modelos de lenguaje basados en Transformers son mucho más robustos frente a los cambios de dominio que otros algoritmos previos \cite{hendrycks-etal-2020-pretrained}, todavía siguen sufriendo cuando los datos analizados difieren fuertemente de los utilizados en el entrenamiento. \citet{ruder2021lmfine-tuning} hace un repaso extenso de los últimos avances en las técnicas de adaptación de dominio utilizando modelos de lenguaje del estado del arte.


\citet{gururangan-etal-2020-dont} analizan el impacto de continuar el pre-entrenamiento para los modelos de lenguaje basados en Transformers. En su estudio, consideran aplicar esta técnica sobre diversos dominios en inglés: biomédico, reviews de películas, papers de cs. de la computación (CS), y noticias. A diferencia de \ulmfit{}, que sólo plantea el ajuste de pesos de acuerdo al conjunto de datos de una tarea particular, los autores plantean dos alternativas:

\begin{itemize}
    \item \emph{Domain Adaptation}: ajustar el modelo de lenguaje sobre un extenso conjunto de datos no etiquetado, usualmente el sobrante del proceso de recolección que no es anotado.
    \item \emph{Task Adaptation}: ajustar el modelo de lenguaje sobre el dataset, de la misma manera que \citet{howard-ruder-2018-universal}.
\end{itemize}

Usando como modelo base a \roberta{}, los autores reportan que los clasificadores aumentan su rendimiento para diversas tareas de cada dominio de manera significativa, tanto en el caso de realizar adaptación de dominio como en la adaptación a la tarea. Si bien las mayores ganancias se obtienen para la segunda, algunos resultados de ese mismo trabajo indicarían que este tipo de pre-entrenamiento puede dañar la generalización, posiblemente debido a un sobreajuste a las instancias del dataset.

\begin{table}
    \centering
    \begin{tabular}{llll}
        Nombre                                 & Idioma            & Dominio                          & Familia     \\
        \hline
        SciBERT         & inglés            & Papers                           & BERT        \\
        ClinicalBERT    & inglés            & noticias médicas                 & BERT        \\
        MediBERT        & inglés            & registros médicos                & BERT         \\
        LegalBERT       & inglés            & legislación, contratos           & BERT        \\
        BERTweet        & inglés            & tweets, algunos sobre COVID      & RoBERTa     \\
        AlBERTo         & italiano          & tweets                           & RoBERTa     \\
        TwilBERT        & español           & tweets                           & $\sim$BERT        \\
        \hline
    \end{tabular}

    \caption{Modelos pre-entrenados sobre dominios no canónicos. En familia nos referimos a qué tipo de pre-entrenamiento es realizado en el modelo de lenguaje: BERT es MLM + NSP, RoBERTa sólo MLM. En el caso de TwilBERT, se usa un símil BERT ya que no usan exactamente NSP sino una tarea muy parecida.)}
    \label{tab:bert_pretrained_models}
\end{table}


Posteriormente al estallido de los modelos de lenguaje basados en Transformers, algunos trabajos se han dedicado a entrenar directamente estos modelos sobre un dominio de interés particular y no en textos genéricos como Wikipedia. Por ejemplo, \emph{SciBERT} \cite{beltagy-etal-2019-scibert}, \emph{MediBERT} \cite{rasmy2021med} y \emph{LegalBERT} \cite{chalkidis-etal-2020-legal} están entrenados sobre textos científicos, médicos y legales respectivamente. AlBERTo \cite{polignano2019alberto} es uno de los primeros modelos entrenados directamente sobre tweets, particularmente para italiano; el ya mencionado \bertweet{} \cite{dat2020bertweet} prosiguió esta línea de investigación, siendo construido mediante un pre-entrenamiento similar al de \roberta{} \cite{liu2019roberta} sobre cerca de 850M tweets en inglés, una parte de ellos relacionados a la pandemia del COVID-19. En español tenemos el modelo TwilBERT \cite{gonzalez2021twilbert}; sin embargo, tiene algunas limitaciones: en primer lugar, no queda claro cuánto tiempo de entrenamiento recibió ni si los datos fueron suficientes; en segundo, usaron un modo de entrenamiento basado en una variante de la tarea NSP (ver Subsección \ref{sec:02_bert}) cuando numerosos trabajos muestran que el tipo de entrenamiento basado en RoBERTa (sólo tarea MLM) mejora el desempeño en las tareas finales. Finalmente, su modelo no es accesible mediante el model hub de huggingface, limitando seriamente su acceso. La Tabla \ref{tab:bert_pretrained_models} lista algunos de estos modelos.


Algunas oportunidades de mejora de lo estudiado en \citet{gururangan-etal-2020-dont} son, en primer lugar y siguiendo la regla de Bender \cite{bender2011achieving}, realizar el estudio en un idioma distinto al inglés. En segundo lugar, estudiar el dominio de tareas en textos de redes sociales, algo no realizado en dicho trabajo. Finalmente, es de interés realizar una comparación de la performance de modelos adaptados al dominio contra aquellos que son entrenados desde cero, dado el enorme costo computacional, energético y ambiental que implica esto último \cite{strubell2019energy}.



\section{Modelo pre-entrenado sobre tweets}
\label{sec:robertuito_pretrained}

En esta sección describimos el proceso de construcción de \robertuito{}. Entrenamos tres versiones de este modelo: una versión que preserva las mayúsculas del texto (nombrada \tbf{cased}); una versión que convierte todo a minúsculas (\tbf{uncased}); y una versión que convierte todo a minúsculas y elimina las tildes (\tbf{deacc}). El español normativo prescribe el uso de tildes en ciertos casos para señalar la acentuación de una palabra, algo que por lo general suele pasarse por alto en los textos escritos en redes sociales. Una hipótesis de trabajo es que eliminar esta información (tildes y mayúsculas) puede ayudar al rendimiento de los modelos en las tareas finales, al ser su uso tan inconsistente en el texto informal.


\subsection{Recolección de tweets}

\label{sec:robertuito_data_collection}

A continuación describimos el proceso de recolección de tweets que utilizamos para entrenar \robertuito{}. El stream de API de acceso gratuito de Twitter (también conocida como \emph{Spritzer}) es una subconjunto generado en tiempo real de alrededor del 1\% de los tweets. Esta muestra es supuestamente aleatoria, aunque algunos estudios han mostrado algunas preocupaciones acerca de posibles formas de manipularla \cite{pfeffer2018tampering}. Muestras no representativas y sesgadas pueden afectar al modelo en tareas finales, algunos de estos errores pudiendo ser dañinos generando sesgos raciales o de género. Por ello, publicamos el conjunto de datos para ser inspeccionado, y queda como trabajo futuro un análisis más detallado de sus instancias.

Descargamos primeramente una colección de Spritzer subida a Archive.org que data de Mayo de 2019 \footnote{\url{https://archive.org/details/archiveteam-twitter-stream-2019-05}}. Filtramos aquellos tweets cuya metadata indicase que su idioma no sea el español, y nos guardamos los usuarios que los generaron. Sobre este conjunto de usuarios, usamos la API de Twitter para descargar todos sus tweets. De este proceso recolectamos alrededor de 622 millones de posteos de cerca 432 mil usuarios.

Finalmente, nos quedamos sólo con aquellos tweets que tengan 6 o más tokens, usando para esto el tokenizador entrenado en BETO \cite{canete2020spanish}, sin contar repeticiones de emojis y haciendo el preprocesado descripto en capítulos anteriores: reemplazamos los caracteres hasta un máximo de tres, convertimos los nombres de usuarios a un token especial \verb|@usuario|, convertimos los emojis a una representación textual, y partimos los hashtags en lo posible (ver Sección \ref{sec:03_preprocessing}). De este proceso obtuvimos 500 millones de tweets, ordenados en \num{1000} archivos para facilitar la lectura en procesos posteriores. El repositorio de la recolección de tweets puede encontrarse en \url{https://github.com/finiteautomata/spritzer-tweets}.

Algo a destacar es que este proceso de recolección permitió que los datos contengan texto con \emph{code-switching}\footnote{Texto o comunicación verbal que fusiona dos idiomas, por ejemplo en spanglish u otras mezclas lingüísticas} o incluso tweets de otros idiomas, ya que solo requerimos que la publicación en la muestra original esté en español. Mientras que otros trabajos como \citet{dat2020bertweet} requirieron que cada tweet estuviera en inglés, permitimos que se incluyeran otros idiomas en los datos previos al entrenamiento. Una estimación aproximada de la distribución lingüística utilizando el módulo de detección de idioma de \emph{fasttext} \cite{joulin2017bag} indicó que el 92\% de los datos están en español, el 4\% en inglés, el 3\% en portugués y el resto en otros idiomas.


\subsection{Arquitectura y entrenamiento}

\begin{table}[t]
    \centering
    \begin{tabular}{l r}
        \hline
        Hiperparámetro    & Valor \\
        \hline
        \#Cabezas         & \num{12}             \\
        \#Capas           & \num{12}             \\
        Tamaño oculto     & \num{768}            \\
        Tamaño intermedio & \num{3072}           \\
        Función de Activación & GeLU           \\
        \#Vocabulario       & \num{30000}         \\
        \hline
        \rule{0pt}{3ex}Probabilidad MLM   & \num{0.15}     \\
        Tamaño de secuencia& 128            \\
        Batch size         & \num{4096}          \\
        Learning Rate      & $3.5 * 10^{-4}$\\
        Decay              & $0.10$          \\
        $\beta_1$          & $0.90$            \\
        $\beta_2$          & $0.98$           \\
        $\epsilon$         & $10^{-6}$      \\
        Pasos de warmup    & \num{36,000} (6\%)   \\
        \hline
    \end{tabular}
    \caption{Hiperparámetros utilizados en el entrenamiento de \robertuito{}. Los valores de $\beta$ y $\epsilon$ refieren a los hiperparámetros de Adam}
    \label{tab:robertuito_architecture}
\end{table}

Para cada una de las versiones de \robertuito{} (cased, uncased, deacc) entrenamos tokenizadores usando el algoritmo \emph{SentencePiece} \cite{kudo-richardson-2018-sentencepiece} en los tweets recopilados, disponiendo de un vocabulario de \num{30000} tokens en todos los casos. Usamos la librería \emph{tokenizers} \footnote{\url{https://github.com/huggingface/tokenizers}} que proporciona implementaciones rápidas en el lenguaje de programación \emph{Rust} para muchos algoritmos modernos de tokenización.

Se utilizó una arquitectura \roberta{} base para el modelo, con 12 capas de auto atención, 12 cabezas de atención y tamaño intermedio 768, de la misma manera que \bertweet{}. Entrenamos \robertuito{} sobre la tarea de MLM, en la misma línea de \roberta{} y \bertweet{}, sin tener en cuenta la tarea de predicción de la siguiente oración usada en \bert{} u otras tareas de orden de tweets (como la usada en \citet{gonzalez2021twilbert}). Teniendo en cuenta los hiperparámetros de \roberta{} y \bertweet{}, decidimos utilizar un tamaño de batch size grande para nuestro entrenamiento. Si bien se recomienda un tamaño de \num{8192} en \citet{liu2019roberta} y \citet{dat2020bertweet}, debido a las limitaciones de recursos decidimos aumentar el número de actualizaciones utilizando un tamaño de lote de \num{4096}. La Tabla \ref{tab:robertuito_architecture}

Para comprobar la convergencia, primero entrenamos un modelo para \num{200000} pasos de optimización. Al comprobar que convergió (y obtuvo buenos resultados en las tareas del benchmark que describimos a continuación), procedimos al entrenamiento completo de los tres modelos. El proceso de pre-entrenamiento tomó aproximadamente tres semanas en una TPU \emph{v3-8} y una máquina pre-emptible \emph{e2-standard-16} en Google Cloud Platform (GCP), ambos recursos provistos por el programa Google TPU Research Cloud. Nuestro código está basado en la biblioteca \emph{huggingface's transformers} \cite{wolf-etal-2020-transformers} y su implementación de \emph{RoBERTa}. La Tabla \ref{tab:training_results} muestra los resultados del entrenamiento en términos de pérdida de entropía cruzada y perplejidad.

\begin{table}[h]
    \centering
    \begin{tabular}{l l l l}
        Model   & Train loss & Eval loss   & Eval ppl \\
        \hline
        \cased{}   & $1.864$      & $1.753$       & $5.772$    \\
        \uncased{} & $1.940$      & $1.834$       & $6.259$    \\
        \deacc{}   & $1.951$      & $1.826$       & $6.209$    \\
        \hline
    \end{tabular}
    \caption{Resultados del pre-entrenamiento para las tres versiones de \robertuito{}. La función de costo utilizada es la entropía cruzada de la tarea de MLM.}
    \label{tab:training_results}
\end{table}


\subsection{Evaluación}

Para analizar la performance de este modelo, usamos un conjunto de tareas sobre textos generados en redes sociales en español, siguiendo lo hecho en \citet{gonzalez2021twilbert} y \citet{polignano2019alberto}. Las tareas elegidas son todas las que analizamos en esta tesis hasta el momento:

\begin{enumerate}
    \item Análisis de sentimientos (Capítulo \ref{chap:03_social_text_classification})
    \item Análisis de emociones (Capítulo \ref{chap:03_social_text_classification})
    \item Detección de ironía (Capítulo \ref{chap:03_social_text_classification})
    \item Detección de discurso de odio (Capítulo \ref{chap:04_hate_speech})
    \item Detección contextualizada de discurso de odio (Capítulo \ref{chap:06_contextualized_hate_speech})
\end{enumerate}


Para más detalles sobre los conjuntos de datos y cuestiones puntuales de cada tarea, referimos a los capítulos correspondientes. Comparamos el rendimiento de \robertuito{} contra los siguientes modelos de lenguaje pre-entrenados disponibles en español:

\begin{itemize}
    \item \beto{} \cite{canete2020spanish}, tanto en versión cased como uncased.
    \item \emph{RoBERTa$_{ES}$} (o RoBERTa$_{BNE}$) \cite{gutierrezfandino2021spanish}, un modelo RoBERTa entrenado sobre una base de datos de 500GB de todos los sitios \emph{.es}
    \item \emph{BERTin}\footnote{\url{https://huggingface.co/bertin-project/bertin-roberta-base-spanish}}, otro modelo RoBERTa entrenado en el contexto de un evento de la comunidad Flax/Jax \footnote{\url{https://discuss.huggingface.co/t/open-to-the-community-community-week-using-jax-flax-for-nlp-cv/7104}}, en el cual los autores exploraron diferentes estrategias de muestreo para entrenar este modelo en relativamente poco tiempo sobre la sección en español del corpus \emph{mc4}, creado para entrenar T5 \cite{raffel2020exploringt5}.
\end{itemize}

Cada uno de estos modelos comparte una arquitectura similar a \robertuito{} y una cantidad comparable de parámetros. Seguimos las prácticas estándares para el ajuste de los modelos, descriptas en anteriores capítulos y en \citet{devlin2018bert}. Para las tareas de clasificación, ajustamos los modelos para por 5 epochs con un learning rate triangular de $5 * 10^{-5}$ y un warmup de 10 \% de los pasos de entrenamiento. Seleccionamos el modelo que mejor resultado obtuvo al final de cada epoch según la métrica de cada tarea.

\subsection{Resultados}


\begin{table}
    \centering
    \begin{tabular}{l ccccc }
        Modelo                & CONTEX             &  ODIO                 &  SENTIM        &  EMOCIÓN             &  IRONÍA               \\
        \hline
        \robertuito{}$_U$     & $\mbf{59.3 \pm 0.4}$ &  $\mbf{80.1 \pm 1.0}$ & $\mbf{70.7 \pm 0.4}$ & $\mbf{55.1 \pm 1.1}$ & $73.6 \pm 0.8$  \\
        \robertuito{}$_D$     & $\mbf{59.3 \pm 0.6}$ &  $79.8 \pm 0.8$       & $70.2 \pm 0.4$       & $54.3 \pm 1.5$       & $\mbf{74.0 \pm 0.6}$  \\
        \robertuito{}$_C$     & $59.0 \pm 0.5$       &  $79.0 \pm 1.2$       & $70.1 \pm 1.2$       & $51.9 \pm 3.2$       & $71.9 \pm 2.3$  \\
        \roberta{}$_{ES}$     & $57.7 \pm 0.4$       &  $76.6 \pm 1.5$       & $66.9 \pm 0.6$       & $53.3 \pm 1.1$       & $72.3 \pm 1.7$  \\
        BERTin                & $55.7 \pm 0.8$       &  $76.7 \pm 0.5$       & $66.5 \pm 0.3$       & $51.8 \pm 1.2$       & $71.6 \pm 0.8$  \\
        \beto{}$_U$              & $59.1 \pm 0.6$       &  $75.7 \pm 1.2$       & $64.9 \pm 0.5$       & $52.1 \pm 0.6$       & $70.2 \pm 0.8$  \\
        \beto{}$_C$              & $58.2 \pm 0.7$       &  $76.8 \pm 1.2$       & $66.5 \pm 0.4$       & $52.1 \pm 1.2$       & $70.6 \pm 0.7$  \\
        \hline
    \end{tabular}
    \caption{Resultados de los experimentos de clasificación sobre el benchmark de tareas sociales. CONTEX es la tarea de detección contextualizada de discurso de odio, ODIO es detección de discurso de odio, SENTIM, EMOCIÓN E IRONÍA son análisis de sentimiento, emociones e ironía. Resultado expresado en porcentaje de la métrica correspondiente a cada tarea y como la media $\pm$ desviación de diez corridas de los experimentos. U, C, y D significan \emph{uncased, cased y deacc} respectivamente. Más grande es mejor.}
    \label{tab:robertuito_evaluation_results}
\end{table}


La Tabla \ref{tab:robertuito_evaluation_results} muestra los resultados de la evaluación de los modelos seleccionados para las cinco tareas de clasificación propuestas, expresados como la media $\pm$ desviación de diez ejecuciones de los experimentos. Podemos observar que en la mayoría de los casos, las tres configuraciones de \robertuito{} obtienen resultados por encima de los otros modelos, en particular para las tareas de discurso de odio y análisis de sentimiento. El único caso donde esto no ocurre es en la tarea de detección de discurso de odio contextualizado, donde si bien hay una mejora, es marginal y no significativa.

Analizamos las diferencias entre los tres modelos de \robertuito{} mediante un test de Kruskal-Wallis \cite{kruskal1952use} para las performances de cada tarea. Los resultados muestran diferencias significativas entre el desempeño de los tres modelos de \robertuito{} para todas las tareas analizadas ($ H(3)=6.88, p<0.05$ para Discurso de odio, $H(3)=9.90, p<0.01$ para Análisis de sentimiento, $H(3)=11.85, p<0.01 $ para Análisis de emociones, $H(3)=11.54, p<0.01 $ para Detección de ironía), con la excepción de la tarea de discurso de odio contextualizado ($H(3)=3.59, p > 0.15$).

Para verificar las diferencias significativas entre las performances de los tres modelos para las 4 tareas mencionadas, realizamos un análisis post-hoc con un test de Dunn (con corrección de Benjamini-Hochberg). Exceptuando la tarea de análisis de sentimientos, la versión \emph{cased} muestra siempre diferencias significativas contra las versiones \emph{uncased} o \emph{deacc}. Sin embargo, no se encuentran diferencias significativas entre las versiones \emph{uncased} y \emph{deacc}.

Este resultado puede leerse de dos maneras: primero, que una normalización más fuerte (remover las tildes) del texto de entrada en español no produce una mejora significativa en el rendimiento de los modelos; también, que mantener las tildes en el texto de entrada no es beneficioso ni perjudicial para el rendimiento del modelo.

\begin{figure}
    \centering
    \includegraphics[width=\textwidth]{img/robertuito/length_tokens.pdf}
    \caption{Distribución de la cantidad de tokens por instancia para los tokenizadores de cada modelo. Las barras están agrupadas por tarea y muestran la media de la distribución junto a su intervalo de confianza a 95\%. Más chico es mejor.}
    \label{fig:length_tokens}
\end{figure}

La Figura \ref{fig:length_tokens} muestra la distribución del número de tokens en el texto de entrada agrupados por tarea. Podemos observar que los modelos de \robertuito{} tienen representaciones más compactas que \beto{} y \emph {RoBERTa-BNE}. \emph{BERTin}, a pesar de su menor rendimiento en general, tiene un tamaño medio comparable al de nuestro modelo. Entre los modelos de \robertuito{}, podemos observar que la versión \deacc{} tiene una longitud media ligeramente menor en comparación con la versión \uncased{}. En suma, esto indicaría que los modelos de \robertuito{} logran codificar de manera más eficiente los tweets de los distintos conjuntos de datos considerados.


\section{Adaptación de modelos pre-entrenados}
\label{sec:domain_adaptation_vs_robertuito}

Acabamos de observar que entrenar un modelo desde cero sobre tweets resulta en una mejor performance para un conjunto de tareas de dicho dominio. Cabe preguntarse \tbf{¿puede ser esta mejora replicada continuando el pre-entrenamiento de otro modelo de lenguaje?} Esta pregunta tiene --más allá del interés teórico de si un modelo de lenguaje entrenado en un dominio distinto puede adaptarse con éxito a un dominio diferente-- dos consideraciones prácticas:

\begin{itemize}
    \item Para lenguajes de recursos relativamente bajos o bien laboratorios menos favorecidos, entrenar un modelo desde cero como realizamos en la anterior sección puede ser prohibitivo en términos económicos.
    \item Reducir los inmensos costos computacionales de los modelos actuales de NLP puede ser de interés, no sólo en términos económicos sino también ambientales \cite{bender2021dangers}.
\end{itemize}

Para analizar si es posible replicar el rendimiento de un modelo especialmente diseñado para un dominio, continuamos el pre-entrenamiento de un modelo \beto{} sobre textos generados por usuario --tweets para nuestro caso-- y probamos su performance sobre el benchmark de tareas sociales descripto en la sección anterior.


\subsection{Metodología}

Para realizar la adaptación de dominio, seguimos las recomendaciones de \citet{gururangan-etal-2020-dont}, tomando un gran conjunto de datos no anotado de textos sociales y corriendo la tarea de Masked Language Modeling sobre estos. En términos de ese trabajo, realizamos una forma de \emph{Domain Adaptation Pre-training} (\emph{DAPT}), que consta de usar un conjunto de datos grande y relacionado a nuestra tarea final. Utilizamos para esto los mismos datos recolectados para entrenar a \robertuito{}, descriptos en la Sección \ref{sec:robertuito_data_collection}.

Usamos como modelos base las versiones \emph{cased} y \emph{uncased} de BETO, y continuamos el pre-entrenamiento sobre estos datos, descartando la tarea NSP. En lugar de correr por \num{12500} pasos de optimización como es sugerido en \citet{gururangan-etal-2020-dont}, optimizamos este hiperparámetro probando con \num{2500}, \num{5000}, \num{10000} y \num{20000} pasos de optimización. Para cada modelo nos quedamos con la configuración que obtuvo el mejor resultado en términos del benchmark analizado.

Para entrenar estos modelos, usamos una TPU v2-8, donde cada paso de optimización tomó alrededor de $2.5$ segundos. Usamos una configuración similar a la descripta para el entrenamiento de \robertuito{} (ver Tabla \ref{tab:robertuito_architecture}), con un learning rate levemente superior ($5 * 10^{-4}$) y limitando también la longitud de secuencia a 128 tokens.

\subsection{Resultados}
\label{sec:domain_adaptation_results}


\begin{figure}
    \centering
    \includegraphics[width=\textwidth]{img/robertuito/results_cased_models.pdf}
    \includegraphics[width=\textwidth]{img/robertuito/results_uncased_models.pdf}

    \caption{Resultados sobre el benchmark de los modelos BETO y \robertuito{} (en versiones cased y uncased). Las barras están agrupadas por tarea y muestran la media de la performance sobre 15 corridas, junto a su intervalo 95\%. El número al final de los modelos indica la cantidad de pasos de optimización realizados en el ajuste de dominio. En tonos azules las variantes de \robertuito{}. Más grande es mejor}
    \label{fig:robertuito_vs_domain_barplot_results}
\end{figure}


En la Figura \ref{fig:robertuito_vs_domain_barplot_results} se ilustra el desempeño sobre el benchmark de tareas para los modelos de lenguaje \beto{} y \robertuito{}, así como también para las versiones con ajuste de dominio de \beto{}. Para ambos casos, aumentar el pre-entrenamiento pareciera coincidir con una mejor performance, aunque dentro de los modelos \emph{uncased}, el que fue optimizado por \num{5000} pasos pareciera haber empeorado su rendimiento general. Esta tendencia, sin embargo, no se cumple en el caso de la tarea de detección contextualizada de discurso de odio. Una posible razón detrás de esto es que la tarea planteada tiene diferencias con el dominio sobre el cual ajustamos: utilizamos pares de tweets, uno de ellos (el contexto) proveniente de un medio periodístico. También puede argumentarse que el dominio de tweets en general es demasiado amplio \cite{eisenstein2013bad}: para el caso de nuestra tarea, el pre-entrenamiento sobre datos generados de la distribución general de tweets no pareciera ser beneficioso.



\begin{table}[ht]
    \centering
    \begin{tabular}{l ccccc}
        \toprule
        Modelo              & CONTEX             &  ODIO                 &  SENTIM        &  EMOCIÓN             &  IRONÍA               \\
        \midrule
        \robertuito{}$_U$   & $59.3 \pm 0.4$ & $80.1 \pm 1.0$ & $70.7 \pm 0.4$ & $55.1 \pm 1.1$& $73.6 \pm 0.8$ \\
        \robertuito{}$_D$   & $59.3 \pm 0.6$ & $79.8 \pm 0.8$ & $70.2 \pm 0.4$ & $54.3 \pm 1.5$& $74.0 \pm 0.6$ \\
        \robertuito{}$_C$   & $59.0 \pm 0.5$ & $79.0 \pm 1.2$ & $70.1 \pm 1.2$ & $51.9 \pm 3.2$& $71.9 \pm 2.3$ \\
        \hline
        \rule{0pt}{3ex}\beto{}$_{U}$+FT    & $58.8 \pm 0.3$ & $77.5 \pm 1.5$ & $68.0 \pm 0.4$ & $55.3 \pm 0.9$& $71.7 \pm 0.5$ \\
        \beto{}$_{C}$+FT    & $57.2 \pm 0.6$ & $77.7 \pm 0.9$ & $68.6 \pm 0.5$ & $51.7 \pm 0.9$& $73.0 \pm 0.4$ \\
        \hline
        \rule{0pt}{3ex}\beto{}$_U$         & $59.1 \pm 0.6$ & $75.7 \pm 1.2$ & $64.9 \pm 0.5$ & $52.1 \pm 0.6$& $70.2 \pm 0.8$ \\
        \beto{}$_C$         & $58.2 \pm 0.7$ & $76.8 \pm 1.2$ & $66.5 \pm 0.4$ & $52.1 \pm 1.2$& $70.6 \pm 0.7$ \\
%       bertin              & $55.7 \pm 0.8$ & $76.7 \pm 0.5$ & $66.5 \pm 0.3$ & $51.8 \pm 1.2$& $71.6 \pm 0.8$702 \\
%       roberta-bne         & $57.7 \pm 0.4$ & $76.6 \pm 1.5$ & $66.9 \pm 0.6$ & $53.3 \pm 1.1$& $72.3 \pm 1.7$565 \\
        \hline
    \end{tabular}
    \caption{Resultados de la evaluación de modelos pre-entrenados y modelos ajustados en dominio para el benchmark de tareas sociales: CONTEXT es contextualized hate speech, HATE es hate speech detection sobre el dataset de hatEval, SENTIMENT, EMOTION e IRONY son análisis de sentimiento, emociones e ironía sobre los corpus de TASS. Todos los scores son Macro F1s. beto-cased-ft y beto-uncased-ft son modelos adaptados al dominio sociall. Score es la media de cada fila. }

    \label{tab:domain_adaptation_evaluation_results}

\end{table}

La Tabla \ref{tab:domain_adaptation_evaluation_results} muestra los resultados, expresados nuevamente como medias y desviaciones estándar, para los modelos considerados. Seleccionamos como modelos ajustados a dominio (indicados con $+FT$) a los que obtuvieron mejores resultados entre aquellos que entrenamos con distinta cantidad de pasos de optimización: \num{10000} pasos de optimización para \beto{} versión uncased, y \num{20000} para versión cased \footnote{Incluimos en el Apéndice \ref{app:07} la tabla completa de resultados para cada uno de los pasos}. Haciendo una comparación entre los modelos \beto{} y \robertuito{} en versiones \emph{uncased}, vemos para la tarea de detección de discurso de odio que la brecha es de alrededor de $4.4$ puntos de F1, mientras que la versión FT achica esa diferencia a $2.26$ puntos F1, una reducción del 48\% de la diferencia de performance. En el caso de análisis de sentimiento, pasamos de un gap de $5.8$ puntos de F1 a uno de $2.7$, una reducción del 53\%; para análisis de emociones, esta diferencia pasa de $3$ puntos de F1 a $0$, logrando de hecho mejores resultados. Finalmente, en el caso de detección de ironía, el gap de $3.4$ puntos de F1 pasa a $1.9$, una reducción de 44\%.

\section{Adaptación de dominio para detección contextualizada de discurso de odio}

\begin{table}
    \centering
    \begin{tabular}{l cc cc cc}
                        & \mc{6}{Modelos} \\
                        & \mc{2}{BETO}              & \mc{2}{\robertuito{}$_U$}       & \mc{2}{\robertuito{}$_C$} \\
        Métrica         &  $\neg$FT&   FT           &  $\neg$FT      & FT             & $\neg$FT &   FT \\
        \thline{2}
        LLAMA           &  $63.80$ &  $68.47$       &  $67.19$       & $69.74$        &  $66.63$ &   $\mbf{70.12}$ \\
        MUJER           &  $41.07$ &  $42.06$       &  $42.99$       & $\mbf{44.29}$  &  $41.08$ &   $43.68$ \\
        LGBTI           &  $45.13$ &  $48.23$       &  $48.45$       & $47.94$        &  $45.56$ &   $\mbf{51.36}$ \\
        RACISMO         &  $68.79$ &  $\mbf{72.05}$ &  $67.70$       & $69.78$        &  $66.60$ &   $68.94$ \\
        CLASE           &  $49.13$ &  $\mbf{51.15}$ &  $46.96$       & $47.06$        &  $47.01$ &   $48.65$ \\
        POLITICA        &  $57.90$ &  $62.48$       &  $63.00$       & $62.87$        &  $62.19$ &   $\mbf{63.16}$ \\
        DISCAPACIDAD    &  $58.49$ &  $\mbf{60.89}$ &  $57.43$       & $60.64$        &  $58.00$ &   $59.01$ \\
        APARIENCIA      &  $74.13$ &  $76.63$       &  $73.83$       & $74.74$        &  $74.71$ &   $\mbf{76.68}$ \\
        CRIMINAL        &  $65.03$ &  $\mbf{69.94}$ &  $62.61$       & $66.63$        &  $61.28$ &   $65.65$ \\
        \hline
        Macro F1        &  $58.16$ &  $\mbf{61.32}$ &  $58.91$       & $60.41$        &  $58.12$ &    $60.80$ \\
        Macro Precision &  $64.16$ &  $\mbf{70.21}$ &  $60.11$       & $61.63$        &  $60.15$ &    $62.30$ \\
        Macro Recall    &  $53.97$ &  $55.09$       &  $58.38$       & $59.62$        &  $56.91$ &   $\mbf{59.80}$ \\
        \thline{2}
    \end{tabular}
    \caption{Resultados de los experimentos de clasificación para la tarea \emph{granular} de detección de discurso de odio, expresados en porcentajes de F1 para las características. Cada modelo consume el comentario analizado y el tweet de la noticia. Consideramos tres modelos: \beto{}, \robertuito{}$_U$ (uncased) y \robertuito{}$_C$ (cased). Para cada uno, tenemos dos versiones: una sin ajustar al dominio ($\neg$FT) y ajustada a dominio (FT) de acuerdo a lo descripto en la Sección \ref{sec:contextualized_classifiers}. En negrita, los mejores resultados.}
    \label{tab:domain_adaptation_context_results}
\end{table}

Para la tarea de detección contextualizada de discurso de odio, el rendimiento de los clasificadores basados en \beto{} se deteriora al realizar la adaptación de dominio sobre tweets en solitario. Sin embargo, los resultados de la Sección \ref{sec:06_results} indicaron una mejora al continuar el pre-entrenamiento sobre pares de tweets que contengan texto y contexto. Una posible conclusión de ello es que el dominio de nuestro conjunto de datos construido en el Capítulo \ref{chap:05_dataset_creation} es suficientemente distinto en sus instancias al de las demás tareas consideradas.

Teniendo eso en cuenta, efectuamos experimentos de clasificación complementarias realizando una adaptación a dominio de \robertuito{}. Efectuamos el mismo procedimiento descripto en la Sección \ref{sec:06_domain_adaptation} sobre las versiones \emph{cased} y \emph{uncased} de nuestro modelo, y corrimos el experimento sobre la tarea granular considerando únicamente el contexto simple (sólo el tweet de la noticia).

La Tabla \ref{tab:domain_adaptation_context_results} contiene los resultados para cada característica y modelo, tanto en sus versiones no ajustadas a dominio ($\neg$FT) como aquellas que sí fueron ajustadas (FT). Mientras que \robertuito{} obtiene mejores resultados en la columna $\neg$FT, al atravesar el proceso de adaptación de dominio, \beto{} obtiene mejores resultados que nuestro modelo. Para la versión uncased, algunas características empeoran su rendimiento luego de la adaptación, aunque esto no ocurre para la versión cased.


\section{Discusión}


En primer lugar, el modelo pre-entrenado sobre tweets construido en este capítulo (\robertuito{}) presenta mejoras significativas para casi todas las tareas analizadas en español, con picos de hasta casi 4 puntos de F1 para la tarea de Análisis de Sentimientos contra la mejor versión de \beto{}. De los distintos tipos de normalización de texto utilizados en \robertuito{} (\emph{cased}, \emph{uncased} y \emph{deacc}), podemos observar que los modelos \emph{uncased} y \emph{deacc} obtienen mejores performances en general. Si bien el modelo \emph{uncased} muestra una ligera performance superior al modelo \emph{deacc}, esta mejora no es significativa, y esto indicaría que remover tildes no reporta una degradación de la performance. Esto es esperable ya que usualmente no se utiliza de manera consistente esta marcación en el español ``vulgar'' de las redes sociales, aunque, de todas formas, pruebas más extensivas son necesarias sobre otras tareas y modelos para verificar esta hipótesis.


Una de las limitaciones de esta comparación es que \robertuito{} fue entrenado sólo por 600 mil pasos de optimización, contra los casi 900 mil pasos de \beto{}, y el millón de pasos de \bertweet{}. Hay que observar que la optimización de \beto{} se da con un batch size menor (512 vs \num{4096}) y la de \bertweet{} se hace con un batch size mayor ($\sim$\num{7000}). En términos de cantidad de tokens procesados en el pre-entrenamiento, la comparación puede no ser del todo justa, aunque de todas formas ilustra que el pre-entrenamiento sobre este dominio particular es efectivo.

Con respecto a los experimentos de adaptación de dominio, podemos observar que adaptando \beto{} sobre un conjunto de tweets en español obtenemos una mejora en todos las tareas contra la versión base. Comparado con \robertuito{}, las versiones adaptadas logran recortar alrededor del 50\% de la brecha de rendimiento entre ambos, incluso reduciéndolo a cero en algunos casos. Esta comparación, sin embargo, no es del todo justa ya que \beto{} fue pre-entrenado de una manera distinta que \robertuito{}. Por cuestiones de tiempo no pudieron ser realizados sobre la versión de \roberta{} en español\footnote{Principalmente, ya que éste modelo y \emph{bertin} fueron lanzados mientras realizábamos estos experimentos} pero trabajo futuro debería usar como base este modelo. Otra opción que no tuvimos en cuenta en este trabajo y que podría reducir más la diferencia es la de agregar vocabulario en el ajuste de dominio, algo que \citet{howard-ruder-2018-universal} realizan en su implementación de ULM-FIT.

Una consideración práctica de ajustar los modelos de lenguaje es que esta técnica permite mejorar el desempeño de una manera relativamente económica, sin tener que efectuar un costoso pre-entrenamiento desde cero. En términos concretos, un ajuste de dominio puede realizarse utilizando una placa de GPU en uno o dos días, mientras que pre-entrenar un modelo desde cero requiere acceso a un hardware más oneroso. Algunos trabajos recientes \cite{izsak2021train} muestran alternativas para construir desde cero modelos de lenguaje basados en Transformers bajo escenarios de recursos reducidos, ajustando varios hiperparámetros y usando algunas técnicas de optimización como LAMB \cite{you2019large}. Los recursos mencionados en esos trabajos, lamentablemente, están lejos del alcance de los disponibles de laboratorios no tan favorecidos. En este contexto, continuar el pre-entrenamiento para un dominio particular aparece como una alternativa mucho más factible.

Una pequeña discusión aparte merece la tarea de detección contextualizada introducida en el Capítulo \ref{chap:06_contextualized_hate_speech}. La mejora al utilizar \robertuito{} en esta tarea no pareciera ser significativa, a diferencia de los demás casos. Esto puede deberse a que las instancias de este conjunto de datos tienen una estructura bastante diferente de la de las demás: cada una consta de dos tweets, donde el contexto suele ser un titular de diarios formulado como un tweet, y el comentario en cuestión. El titular de un diario tiene una forma mucho más cercana al dominio del pre-entrenamiento de \beto{}, mitigando una de las posibles mejoras de \robertuito{}. Más aún, observamos que realizar un ajuste de dominio sobre tweets aislados empeora el rendimiento sobre esta tarea; haciendo este ajuste de la misma manera que en el capítulo anterior, tampoco se logró obtener mejores resultados que con \beto{}.


\section{Conclusiones}

En este capítulo hemos abordado la tarea de mejorar el rendimiento de la detección de discurso de odio en el contexto más general de tareas de clasificación sobre textos sociales en español. Para ello, utilizamos como benchmark varias de las tareas que vimos en esta tesis: detección de discurso de odio (en sus dos versiones, no contextualizada y contextualizada), análisis de sentimiento, análisis de emociones, y detección de ironía.

En primer lugar, y en la corriente de modelos pre-entrenados sobre distintos dominios, generamos un nuevo y valioso recurso para la clasificación de textos sociales: \robertuito{}, un modelo de lenguaje basado en \roberta{} sobre tweets en español. Para ello, recolectamos un gran corpus de tweets en español, y utilizando las TPU provistas por Google realizamos el pre-entrenamiento de este modelo. Los experimentos de clasificación sobre el conjunto de tareas arrojaron que \robertuito{} obtiene mejores significativas sobre otros modelos en español. Así mismo, observamos que remover tildes en el preprocesado no reporta una degradación significativa en la performance.

Por otro lado, exploramos una técnica de ajuste de dominio sobre modelos actuales para comparar la ganancia de rendimiento y compararla contra \robertuito{}. Para ello, tomamos los modelos de \beto{} (en sus versiones cased y uncased) y corrimos la tarea de MLM sobre los tweets recolectados para entrenar nuestro modelo anterior. Si bien la performance de estos modelos ajustados mejora con respecto de \beto{}, se mantiene por debajo de \robertuito{}, aunque recortando considerablemente la brecha de performance entre ambos modelos. De todas formas, este análisis puede ser de consideración para aquellos lenguajes con menos recursos que no pueden pre-entrenar modelos de lenguaje desde cero.

Resta como trabajo futuro realizar estos experimentos sobre tareas más desafiantes, y también realizar los ajustes sobre los modelos \roberta{} en español para hacer una comparación más justa. Así mismo, explorar otras alternativas de mejora sobre la tarea de detección contextualizada de discurso de odio, elusiva para ambas técnicas consideradas.

Todos estos experimentos han sido realizados en español, y sus recursos publicados. El modelo puede ser encontrado en el hub de Huggingface \footnote{\url{https://huggingface.co/pysentimiento/robertuito-base-uncased}}, como así también el código para entrenarlo y para correr el benchmark con otros modelos pre-entrenados \footnote{Ambos en  \footnote{\url{https://github.com/pysentimiento/robertuito}}}, y en un futuro la base de datos de tweets en español.

\section{Notas}

En \citet{perez2021robertuito} puede encontrarse la descripción de la construcción de \robertuito{}. En el Apéndice \ref{app:07} puede encontrarse la tabla completa de resultados para las distintas cantidades de pasos de optimización, junto a experimentos adicionales sobre las habilidades multilinguales de \robertuito{}.


%
\part*{Conclusiones}
\chapter{Conclusiones}
%

Debemos ser, sin embargo, cautelosos acerca de los resultados obtenidos en este trabajo y, en líneas generales, de la mayoría de los avances en la detección de discurso discriminatorio. El estado del arte actual en NLP está basado en modelos de lenguajes neuronales pre-entrenados. En términos de lo mencionado en \emph{The Book of Why} de Judea Pearl \cite{pearl2018book}, estos sistemas están en una etapa meramente ``asociacional''. Es decir, nuestros actuales sistemas tan sólo detectan regularidades en los datos, como si fueran sólamente el ajuste de una curva que un estadístico realiza hace más de un siglo, sin realizar ningún tipo de razonamiento causal o simbólico.

En nuestro problema concreto, un clasificador puede detectar que decirle ``sos hombre'' a un artículo relacionado a una mujer (quizás trans) conlleva discurso de odio contra la comunidad LGBTI. Sin embargo, este mismo mensaje ofuscado de alguna manera (por ejemplo, preguntándole el nombre, o alguna otra forma que no hayamos observado en los datos) logra burlar a nuestros sistemas.

Ligado a esta reflexión, Bender y Gebru \todo{citation needed} han realizado una serie de trabajos ilustrando este punto: nuestros actuales sistemas, basados en modelos de lenguaje, aún en sus formas más complejas y sobreparametrizadas con miles de millones de parámetros, no son más que ``loros estadísticos'', muy hábiles en detectar regularidades y hacernos creer que llevan adentro algún tipo de razonamiento. Sin embargo, la realidad es que no lo tienen.

¿Significa esto que los sistemas actuales no sirven para nada? En absoluto. Los actuales sistemas, aún con sus defectos y siendo bastante rudimentarios, logran detectar parte del lenguaje discriminatorio que observamos en redes sociales. Sin embargo, es necesario entender sus limitaciones: a medida que estos sistemas puedan encontrar regularidades con más detalle, muchos usuarios ocultarán este discurso de manera más sofisticada para lograr burlarlos (en caso de que estemos hablando de sistemas que se usen con fines de moderación).

- Agregar algo de foundation models
- Why AI is harder than we think, by
%
\appendix
%
\small
% Manual de anotación
\chapter{Manual de criterios de anotación}
\label{app:manual_criterios_anotacion}
\section{Presencia de lenguaje discriminatorio}

Entendemos que hay discurso discriminatorio en el tweet si contiene declaraciones de carácter intenso y posiblemente irracional de rechazo, enemistad y aborrecimiento contra un individuo o contra un grupo, siendo estos objetivos de estas expresiones por poseer (o aparentar poseer) una característica protegida.

Este discurso puede manifestarse de manera explícita (insultos directos), celebraciones sobre asesinatos u otros crímenes, o bien otras expresiones más veladas. Lo que queremos captar es la intención del autor del tweet. El carácter discriminatorio de un mensaje está dado tanto por el contexto (en este caso, el tweet original del medio periodístico y posiblemente la nota) y el contenido del tweet en sí mismo. Por ejemplo, un comentario que diga “excelente” sin contexto es una cosa, y decir eso mismo en una nota que relata un femicidio, o un asesinato es otra muy distinta.

Las características protegidas que vamos a tener en cuenta son las siguientes:

\begin{enumerate}
\item Sexo (Mujeres, concretamente)
\item Género o identidad sexual (Colectivo LGBTI)
\item Ser inmigrantes, extranjeros, pueblos aborígenes u otras nacionalidades (Xenofobia, racismo)
\item Situación socioeconómica o por barrio de residencia
\item Poseer discapacidades, problemas salud mental o de adicción al alcohol u otros estupefacientes
\item Opinión o ideología política
\item Aspecto o edad (mayormente, gordofobia/gerontofobia)
\item Antecedentes penales o estar privado de la libertad

\end{enumerate}

Es decir, para considerar un mensaje como discriminatorio, debe cumplir que el discurso discriminatorio está orientado hacia un individuo o grupo de al menos una (aunque posiblemente más de una) característica protegida.


Consideramos que el mensaje del tweet (a la vez que el receptor del odio) es el que determina si puede o no ser considerado discriminatorio y hacia qué grupo está dirigido. Esto puede no necesariamente coincidir con el destinatario explícito del mensaje: por ejemplo, si alguien le dice a Susana Giménez “judía sionista hdp”, a pesar de no ser Susana Giménez judía, se puede considerar esto como discurso de odio contra las minorías religiosas y/o discurso xenófobo.


\section{Llamado a la acción}

Entendemos que un tweet (que contiene discurso discriminatorio) llama a la acción si contiene alguna incitación a tomar algún tipo de medida contra el sujeto o grupo ofendido. Esta medida puede ser de carácter violento (“hay que matarlos ya” “pongámosles una bomba”) o de carácter menos violento (“hay que dejar de comprarles a estos chinos ladrones”)

Estos tweets nos interesan particularmente porque son los más peligrosos y dañinos: los que llaman a tomar algún tipo de represalia contra la persona o el grupo en cuestión.

\section{Características protegidas}

Finalmente, para cada tweet deberemos marcar qué grupo o característica protegida es atacado. En este caso, necesariamente un grupo/característica debe ser seleccionado:

Usaremos una notación abreviada en la interfaz de etiquetado, en la que algunos de los grupos o características mencionadas fueron reagrupadas de la siguiente manera:

\begin{enumerate}
    \item MUJER: por su sexo
    \item LGBTI: por género o identidad sexual
    \item RACISMO: Por ser inmigrantes, extranjeros, pueblos aborígenes u otras nacionalidades (Xenofobia, racismo)
    \item POBREZA: Por situación socioeconómica o por barrio de residencia.
    \item DISCAPAC: Por tener discapacidades, problemas salud mental o de adicciones
    \item POLITICA: Por su opinión o ideología política
    \item ASPECTO: Por su aspecto o edad
    \item CRIMINAL: Por sus antecedentes o situación penal (presos)
\end{enumerate}

A su vez, agregamos la categoría “OTROS”. Esta categoría es excepcional, y debería utilizarse sólo si algún tipo de discriminación no está contemplado en estas categorías.

Respecto a la discriminación de carácter político tiene que ser algo más que una mera opinión  sino tener una componente irracional, de descalificación y de aborrecimiento considerable sobre un individuo o una facción política.

No se contempla dentro de las categorías protegidas a las profesiones. Es decir, no tenemos en cuenta el discurso contra científicos, médicos, o periodistas; de esto último hay bastante material agresivo en los comentarios.


\subsection{Lineamientos generales}

El discurso discriminatorio no es sólo discurso ofensivo contra una persona o grupo con alguna de las características protegidas. Tiene que apelar a su condición de mujer, inmigrante, LGBTI, etc. para que lo consideremos así.

Por ejemplo: si alguien agrede a una mujer, a un inmigrante, o a alguien de la comunidad LGBTI, no necesariamente está incurriendo en un discurso discriminatorio salvo que apele a algo que remita a su característica como tal.


Expresiones de aprobación ante noticias de crímenes o acciones contra persona o grupo de las características protegidas son consideradas discriminatorias.

\subsubsection{Ejemplos}

Violan a la reconocida actriz XXXX YYYY
Asesinan a un comerciante chino por creer que tenía Coronavirus
Motín y muerte en la prisión de Marcos Paz

Comentarios de contenido discriminatorio: (emoji de aplausos) - uno menos - bravo! -


Si no queda claro que haya un mensaje discriminatorio o parece de carácter difuso o demasiado tangencial, entonces etiquetar como no discriminatorio




\subsection{MUJER}

Insultar a una mujer sin hacer ninguna referencia particular a su condición de mujer no es suficiente para ser considerado discurso discriminatorio

Como regla: si el mismo insulto o agresión aplicase contra un hombre, entonces no debiéramos considerarlo como discriminación


Insultos contra las expresiones políticas del movimiento de las mujeres son consideradas en esta categoría: si se las insulta como feminazis, aborteras, pañuelito verde, etc



Apelaciones a su apariencia o aspecto propias de una mujer son consideradas en esta categoría. En este punto consideramos comentarios cosificadores


Insultar como “vieja” a una persona no califica como misoginia. Usar para ese caso la categoría ASPECTO que contempla la gerontofobia

EJEMPLOS:

Nati Jota furiosa por los comentarios que recibe en las redes.

Comentarios sexistas:

Pero si sos de plástico nena! (opina de manera denigratoria de su apariencia)
Flor de gato!
Miauuu!
A esta sólo se la conoce por su cuerpo y ahora se hace la santa. Andá a estudiar
Le damos hasta que San Lorenzo vuelva a Boedo
Y esta rubia tarada quién es?



Comentarios ofensivos pero no sexistas:
Callate forra (ofensivo pero no particularmente sexista)
Y esta quién es? A quién le importa?
Quién?
Nati cuánto?
Esta también recibe sobres?
Otra descerebrada más (súper agresivo, pero es un comentario que podría recibir un hombre, aunque en menor cantidad…)
Andá a cagar..boluda..conductora de 4ta..
HDP
Boba



\subsection{LGBTI}

Expresiones de disgusto o rechazo del estilo “qué asco” frente a integrantes de la colectividad deben ser marcados como discurso discriminatorio


Apelaciones a la anatomía o del estilo “biologicistas” deben ser marcados como discurso discriminatorio


Ejemplos:

Anibal Pachano sobre la cuarentena: “este virus nos está destruyendo a los actores”


Comentarios discriminatorios:

Qué asco este sujeto
Y a este trolo quién le pidió su opinión?
Me desagrada


Comentarios ofensivos pero no discriminatorios:

Ajá, y?
No me importa tu opinión, gracias
Se acuerdan cuando este atacaba a Cristina? Andá a llorarle al gato, Aníbal
Pelotudazo!


Famosa actriz trans dio una entrevista a La NACION: “Como mujer, estoy en la mejor etapa de mi vida”


Comentarios discriminatorios:

Muy lindo todo, pero sos hombre
“Mujer” dice jajajaj naciste y morirás macho
Qué asco este trolo
(emoji de asco)
Una mujer con sorpresa
Trava horrendo
Perdón pero qué nombre dice tu DNI?


Comentarios ofensivos pero no discriminatorios:

Quién?
Y a quién le interesa lo que diga esta mujer?
Que se vaya a cagar
Agradecele al gobierno de Néstor que tenés derechos





\subsection{RACISMO}

Asignar a todo un pueblo con una característica o actividad particular negativa entra en esta categoría
Cualquier llamada a expulsar, matar, tirar bombas, destruir, aniquilar, etc a un pueblo, inmigrantes o país entra en esta categoría. Más aún, deben ser marcadas como “llamadas a la acción”
Teorías conspirativas (hasta cierto punto) o acusaciones a un Gobierno/Estado no deberían entrar en esta categoría
¿Acusaciones respecto al virus?



Ejemplos:

Rescatan en China a cuarenta gatitos bebé que iban a ser utilizados en restaurantes


Comentarios discriminatorios:

Chinos asquerosos
Malparidos! Chinos de mierda
Sigan desparramando pestes hijos de puta!
Por qué no se comen entre ellos?
País horrible y enfermo
Estos chinos nos diseminan su peste por todo el mundo

Comentarios discriminatorios y que además llaman a la acción:

Son una plaga estos chinos, hay que matarlos a todos
Ojalá les caigan veinte bombas
Hay que barrer ese país de la faz de la tierra

Comentarios ofensivos pero no racistas:

Son comunistas, qué querés
Lo rescataron a Mauri?




Otros comentarios no racistas

El gobierno chino está intentando controlar todo, cuidado
Mienten con que controlaron el virus, cómo no van a mentir con esto también?
De terror!


Tomas de terrenos: se profundizan las tomas en todo Buenos Aires


Comentarios discriminatorios:

Rajen a todos los peruanos de las tomas! (llama a la acción, además)
Increíble, cualquier paraguayo o bolita de mierda te saca la casa

\subsection{POBREZA}

Expresiones denigrantes sobre la condición socioeconómica son consideradas discriminatorias: referirse a las personas sin empleo como “vagos”, a otros como “(chori)planeros”
Respecto a la gente que posee planes de asistencia social, si es una opinión general sin ser ofensiva no se considera discriminatoria (del estilo “hay que reducir el gasto social y los planes”, “perdieron la cultura del trabajo”)

Presupuesto: aumentó el gasto en planes asistenciales durante la pandemia

Ejemplos discriminatorios:

Basta de mantener vagos!
Cansada de los planeros
Che laburar estos atorrantes ni en pedo no?
PARASITOS

Ejemplos no discriminatorios

La gente que trabaja y aporta impuestos es cada vez menos. Estamos al horno



\subsection{POLITICA}

Apreciaciones derogatorias sobre la posición política son consideradas discriminatorias : zurdo/a, bolchevique, peroncho, gorila, kuka, etc
Acusaciones de corrupción o de “recibir sobres” no son consideradas discriminatorias
Tampoco aquellas expresiones que traten de inútiles a funcionarios
Tratar de viejo/a, gordo/a, u otras cuestiones físicas deben ser marcados en las categorías respectivas, no acá

Aumentó el gasto en planes asistenciales durante la pandemia

Ejemplos discriminatorios:

BASTA ZURDOS DE ROBARNOS
Bolcheviques de mierda

Ejemplos no discriminatorios

La gente que trabaja y aporta impuestos es cada vez menos. Estamos al horno
Qué gobierno de inútiles
Son unos delincuentes
Hijos de mil puta!
Siguen volando los sobres para el Congreso
Siga siga la impresión




\subsection{ASPECTO}

Apreciaciones denigrantes sobre la apariencia de una persona y/o su edad
Principalmente, tenemos en mente la gordofobia y gerontofobia, pero puede referir a otras características físicas (por ejemplo, la altura)
En casos en las cuales haya solapamiento con mujer, marcar ambas


Luis Brandoni: “No convoqué el banderazo”

Ejemplos discriminatorios:

Viejo de mierda!
Qué decrépito impresentable que es este señor
Estás gagá, pelotudo

Jorge Lanata vuelve a la televisión

Ejemplos discriminatorios:

Gordo chanta otra vez volvés a vender pescado podrido?
porque no te vas vos tambien con todos bola de sebo!!!1
Estás gagá, pelotudo









\subsection{CRIMINAL}

Cualquier comentario que celebre acciones contra criminales o personas privadas de su libertad (golpizas, asesinatos, muerte en motines, etc) entra en esta categoría
En este ítem muchas veces veremos que son llamados a la acción: el de “matarlos”, llamar a reducir sus derechos, etc
Muere un delincuente tras un enfrentamiento con la policía

Ejemplos discriminatorios:

Uno menos!
Excelente!
(emoji de aplausos)
Que pena, pobrecito

Ejemplos que además llaman a la acción

MUY BIEN! Felicitaciones al policía, hay que liquidarlos sin piedad


1...100 101..200 …. 501...600

Primera fase: Et 1 => 1..100, Et 2 => 101 .. 200 … Et 6 => 501..600
Segunda fase Et2 => 1..100 Et1 => 101..200…. Et 6 => 401..500 Et 5 => 501..600

Shufflear temporalmente

Ejemplos no discriminatorios


Cómo puede ser que nuestra Ministra no haga nada?
La policía actuó correctamente.

Motín por el Coronavirus en Olmos: 3 muertos

Ejemplos discriminatorios y que llaman a la acción

Hay que rociar con nafta todas las cárceles
Soltemos 3 o 4 infectados con COVID en cada cárcel y problema solucionado
Paredón y listo



\subsection{DISCAPACIDAD}

Referencias peyorativas de adicciones a drogas, alcohol u otros estupefacientes
También referencias peyorativas a la salud mental de la persona en cuestión
Decir “está loco” no entra acá :-)
Malena Pichot sale a cruzar a Baby Etchecopar
Ejemplos discriminatorios:

Callate faloperita!
Jorge Lanata vuelve a la televisión

Ejemplos discriminatorios:

Che no probaste dejar la merca gordo?

Noticia sobre Patricia Bullrich...

Ejemplos discriminatorios:

Largá la (emoji de botella) Pato
Borracha hdp



\subsection{OTROS}

Esta categoría está reservada para cualquier otro tipo de discriminación que no esté contemplada en las categorías mencionadas
Insultos a profesiones (científicos, periodistas, por ejemplo) no entran en este apartado
ESTA CATEGORIA ES SUMAMENTE EXCEPCIONAL. NO USAR INDISCRIMINADAMENTE

% Capítulo 7 - Tablas de resultados de ajuste de dominio
\input{src/app_domain_adaptation_results.tex}




%%%% BIBLIOGRAFIA
\backmatter
\bibliography{biblio.bib}

\end{document}
