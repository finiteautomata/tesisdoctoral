
Para analizar el error de nuestros clasificadores, elegimos el que mejor performance tuvo en la tarea B: el clasificador que toma el tweet más el tweet de la noticia, ajustado a dominio. De una manera similar a la que realizamos en la sección \ref{sec:hateval_error_analysis}, entrenamos 10 clasificadores y analizamos el error sobre el ensamble de voto mayoritario para centrarnos en los errores más frecuentes. Para ver los casos más problemáticos, analizamos aquellas clases donde peor desempeño muestran los clasificadores: MUJER, LGBTI, y CLASE. Observando las matrices de confusión \todo{Mandar esto a un apéndice}, vemos que los resultados de los clasificadores son bajos en términos de recall. Observaremos entonces los casos de falsos negativos, que revisten el mayor problema.


\begin{table}[t]
    \centering
    \small
    \begin{tabular}{p{0.03\textwidth} p{0.45\textwidth} p{0.50\textwidth}}
          &Contexto & Comentario \\
        \hline
        1 & Contó que era lesbiana, su papá le confesó que era gay y ahora su madre se enamoró de una mujer, lo que inspiró su segundo film & WTF. Mucho ESI los degeneró... \\
        2 &                        & @usuario Esta familia tiene los genes alterados \\
        \hline
        3 & Oscar González Oro ya está instalado en el Uruguay: "Recuperé mi libertad" & Ahora quedate allá, y hablá mal d Macri d nuevo pa tener rating. Y opiná q los taxi boy uruguayos son mas educados q los escorts argentinos! Ano abierto! \\
        \hline
        4 & ¿Por qué un beso entre dos hombres los vuelve tan violentos?": la vida después de haber sido víctima de ataques  homofóbicos &  Será xq va contra la naturaleza de la raza... \\
        \hline
        5 & ``¿Por qué no vemos médicos trans?'': el reclamo de un prestigioso cardiólogo para que América sea más inclusiva & Es difícil ser médico con la cabeza quemada \\
        6  &  & porque un enfermo no cura a otro enfermo \\
        \hline
        7 & ``Te amo''. La emotiva dedicatoria de Luis Novaresio a su pareja en su cumpleaños & \emoji{face-vomiting}\emoji{face-vomiting}\emoji{face-vomiting} \\
        \hline
        8 & Elizabeth Gómez Alcorta: ``Por la pandemia, vamos a tener una suba de los femicidios y travesticidios'' &  Travesticidios... Osea asesinatos de tipos con peluca y tetas \\
        \hline
        9 & Mariana Genesio Peña pasa su cuarentena total con guantes, barbijo y desnuda: ``Mi cuerpo es el planeta Tierra'' & Coronavirus nivel pelotudo en bolas	\\
        10 &    & Con 3 piernas cualquiera es feliz!!!	\\
        11 &    & pasa la cuarentena rascándose las bolas \\
        \hline
        12 & Tras una ráfaga de más de 20 disparos asesinaron a una mujer trans en Rosario & Cómo no saco su escopeta y aplicó la defensa propia?! \\
        13 & & @usuario Salió de caño... cuac!	\\
        \hline
    \end{tabular}
    \caption{Falsos negativos para la característica LGBTI. Ninguno de los 10 clasificadores que consumen contexto y texto indentificaron como discriminatorios a estos comentarios }
    \label{tab:lgbti_error_analysis}
\end{table}

La tabla \ref{tab:lgbti_error_analysis} muestra una selección de comentarios falsos negativos para la característica LGBTI junto a sus contextos. De estas instancias, y observando también aquellos casos donde sí puede detectar el discurso discriminatorio contra LGBTI, podemos esbozar algunas posibles razones. En primer lugar, ciertos mensajes altamente ofensivos son complejos de entender o con una alta variabilidad léxica: por ejemplo, los que tratan de ``enfermos'' o mencionan cuestiones de la genitalidad (ejemplos 5 y 6). Esto puede hacerse de numerosas e ``ingeniosas'' maneras, y haciendo alusiones que difícilmente pueda un modelo de lenguaje entender (pensar todas las maneras que se puede hacer referencia a la genitalidad de una mujer trans, como las de los ejemplos 10 y 12).

Otros contextos --como los ejemplos 3, 7, 9, 10 y 11-- omiten información acerca de la sexualidad o género sobre quienes versa la nota. Esta información faltante no permite a los clasificadores (y tampoco a un humano que carezca de esta información) entender completamente el caracter discriminatorio de los mensajes.


\begin{table}[t]
    \centering
    \small
    \begin{tabular}{p{0.03\textwidth} p{0.45\textwidth} p{0.50\textwidth}}
        & Contexto & Comentario \\
        \hline
        1 & Martha Rosenberg: ``En situación de pandemia, legalizar el aborto es más urgente que nunca'' & Quien es esta vieja?. No debería estar tejiendo? \\
        \hline
        2& Mara Gómez: la historia de la primera futbolista trans en el torneo argentino  &  Feminismo pierde de nuevo... ya le metieron un tipo... jaja punto para el patriarcado...	 \\
        \hline|
        3& Tras una ráfaga de más de 20 disparos asesinaron a una mujer trans en Rosario & Las feministas en modo error 404 al no saber si celebrar o ofenderse  \\
        \hline
        4& El desesperado pedido de Actrices Argentinas ante la violencia de género en cuarentena: ``Es urgente'' & Que risa me dan las feministas!!! Ignorantes.	 \\
        \hline
        5& Leche de cucaracha, la nueva bebida nutritiva: ¿quién se anima a probarla? & No me digas q la hija de CFK está embarazada y ya sale leche por esos senos	 \\
        \hline
        6& Los fans de Florencia Kirchner le piden casamiento por Instagram & Zoofilia \\
        7&                       & Hdp tienen que tener estomago para querer casarse con terrible adefesio \\
        \hline
        8& Rosario: para sacar una licencia de conducir habrá que hacer un curso de perspectiva de género & Te quieren adoctrinar desde cualquier ámbito, y se están metiendo en todo para que empieces a hablar como el orto, como a ellas les gusta.	\\
        9&   & El que choca más feministas le dan más años de licencia	\\
        \hline
        10& Por qué los países liderados por mujeres parecen haber respondido mejor a la crisis del coronavirus & Son mujeres inteligentes que se dejan asesorar de sus esposos \\
        11& Joe Biden presentó su nuevo equipo de comunicación compuesto enteramente por mujeres & Será equipe conche seque?	\\
         \hline
    \end{tabular}
    \caption{Falsos negativos para la característica MUJER. Ninguno de los 10 clasificadores que consumen contexto y texto (ajustados a dominio) lograron identificar como discriminatorios a estos comentarios }
    \label{tab:women_error_analysis}
\end{table}




En el caso de la clase MUJER, podemos observar que de nuevo el principal problema es un bajo recall. Sin embargo, observando que hay muchos casos que pueden mezclarse con otras características como APARIENCIA, buscamos aquellos ejemplos que no son detectados por otras características. La tabla \ref{tab:women_error_analysis} muestra una selección de estos falsos negativos. Podemos observar que algunos ejemplos están muy en el borde de ser simplemente ofensivos, o algunos mensajes irónicos complejos de descifrar (que las feministas celebren por la muerte de una mujer trans, o hablar de chocar mujeres en un curso de manejo).

