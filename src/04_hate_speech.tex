
\label{chap:04_hate_speech}

El discurso de odio contra mujeres, inmigrantes y otros grupos protegidos es un fenómeno generalizado en la Internet y que resulta importante monitorear dada su potencial relación con actos violentos, como hemos comentado en la introducción de esta tesis. En los primeros días de la World Wide Web, algunos académicos se aventuraron a decir a que los prejuicios y el odio serían removidos en este espacio mediante la disolución de identidades en el ámbito virtual \cite{levy2001cyberculture, rheingold1993virtual}. Veinte años después de esta hipótesis, podemos decir que no ha sido el caso. La prevalencia del racismo en la ``World White Web''  y  en las redes sociales ha sido estudiada en numerosos trabajos \cite{adams2005white, kettrey2014staking}, como así también la misoginia en el mundo virtual \cite{filipovic2007blogging, mantilla2013gendertrolling}, entre otros ataques discriminatorios.

Si bien el discurso racista y sexista es una constante en las redes sociales, muchos picos se documentan luego de eventos detonantes, como pueden ser asesinatos con motivos religiosos o políticos \cite{burnap2015cyber}. Debido a esto, algunos estados y organizaciones supranacionales han tomado cartas en el asunto instando a las empresas de redes sociales a que tomen medidas para bajar la incidencia del discurso de odio. Debido a la enorme cantidad de contenido generado por usuarios en estos medios, es necesario desarrollar herramientas que faciliten la labor humana en la detección y prevención de este fenómeno, con particular foco de aquel que incita a la violencia física.


En este capítulo haremos una introducción a este problema desde varias ópticas. Analizaremos las diversas definiciones de discurso de odio y haremos una breve reseña desde un marco legal y de tratados internacionales para luego centrarnos en este problema desde una perspectiva del procesamiento de lenguaje natural. En base al dataset de la competencia \hateval{} \cite{hateval2019semeval}, analizaremos de técnicas de detección de discurso de odio, algunas de ellas presentadas en \citet{perez-2019-atalaya}. Finalmente, marcaremos algunos problemas en los enfoques actuales de la detección de discurso discriminatorio y algunas oportunidades de mejora que abordaremos en capítulos subsiguientes.


\section{¿Qué es el discurso de odio?}
\label{sec:hate_speech_definitions}

No existe una definición universalmente aceptada de lo que configura discurso de odio. Para intentar acercarnos lo más posible a este concepto, en esta sección haremos un repaso muy general de algunos tratados internacionales sobre la materia. Antes de continuar, hacemos \tbf{una aclaración:} en la normativa sobre derechos humanos muchas veces se encuentra delimitado el discurso \tbf{discriminatorio} del discurso de \tbf{odio}, siendo este último una subcategoría del primero de mayor intensidad y con incitaciones a la violencia contra grupos protegidos o individuos miembros de estos grupos. En la literatura de NLP sobre el tema se utiliza la expresión discurso de odio (\emph{hate speech}) para referirse indistintamente a ambos fenómenos.

Aún cuando entendemos que la acepción general del discurso de odio puede entenderse como incorrecta desde la perspectiva de tratados internacionales, teniendo en cuenta que esta tesis está centrada en técnicas para su detección automática usaremos esta terminología para plegarnos a los usos y costumbres de la comunidad de NLP.

\subsection{Abordaje desde una perspectiva legal y de los Derechos Humanos}

Un principio general que hace a los derechos más elementales del hombre y a la vida en sociedad es la posibilidad de expresarse libremente, el \tbf{derecho a la libre expresión}. Este derecho está protegido por constituciones nacionales y numerosos tratados internacionales. Uno de estos tratados, el Pacto Internacional de Derechos Civiles y Políticos (\emph{ICCPR} por sus siglas en inglés)\footnote{Este pacto desarrolla los derechos civiles y políticos establecidos por la Declaración Universal de los Derechos Humanos de la ONU}, sancionado en 1966 en la Asamblea de las Naciones Unidas y ratificado por 166 países, incluye en su artículo 19:

\begin{displayquote}[Artículo 19 de la ICCPR]
1. Nadie podrá ser molestado a causa de sus opiniones.

2. Toda persona tiene derecho a la libertad de expresión; este derecho comprende la libertad de buscar, recibir y difundir informaciones e ideas de toda índole, sin consideración de fronteras, ya sea oralmente, por escrito o en forma impresa o artística, o por cualquier otro procedimiento de su elección.

3. El ejercicio del derecho previsto en el párrafo 2 de este artículo entraña deberes y responsabilidades especiales. Por consiguiente, puede estar sujeto a ciertas restricciones, que deberán, sin embargo, estar expresamente fijadas por la ley y ser necesarias para:

a) Asegurar el respeto a los derechos o a la reputación de los demás;

b) La protección de la seguridad nacional, el orden público o la salud o la moral públicas.
\end{displayquote}

Este artículo que garantiza el derecho a la libertad de expresión da cuenta también de que esta libertad no es completamente irrestricta. El ejercicio de los derechos e igualdad ante la ley de otros marca este límite, no pudiéndose invocarse este derecho para avasallar los de terceros. La Convención Internacional sobre toda forma de Discriminación Racial (ICERD) \footnote{\url{http://servicios.infoleg.gob.ar/infolegInternet/anexos/120000-124999/122553/norma.htm}} dice en su artículo 4 al respecto:

\begin{displayquote}[Artículo 4, ICERD]

Los Estados partes condenan toda la propaganda y todas las organizaciones que se inspiren en ideas o teorías basadas en la superioridad de una raza o de un grupo de personas de un determinado color u origen étnico, o que pretendan justificar o promover el odio racial y la discriminación racial, cualquiera que sea su forma, y se comprometen a tomar medidas inmediatas y positivas destinadas a eliminar toda incitación a tal discriminación o actos de tal discriminación y, con ese fin, teniendo debidamente en cuenta los principios incorporados en la Declaración Universal de Derechos Humanos, así como los derechos expresamente enunciados en el artículo 5 de la presente Convención, tomarán, entre otras, las siguientes medidas:

a) Declararán como acto punible conforme a la ley, toda difusión de ideas basadas en la superioridad o en el odio racial, toda incitación a la discriminación racial así como todo acto de violencia o toda incitación a cometer tal efecto, contra cualquier raza o grupo de personas de otro color u origen étnico, y toda asistencia a las actividades racistas, incluida su financiación;

b) Declararán ilegales y prohibirán las organizaciones, así como las actividades organizadas de propaganda y toda otra actividad de propaganda, que promuevan la discriminación racial e inciten a ella y reconocerán que la participación en tales organizaciones o en tales actividades constituye un delito penado por la ley;

c) No permitirán que las autoridades ni las instituciones públicas nacionales o locales, promuevan la discriminación racial o inciten a ella.
\end{displayquote}


Los Estados y otros organismos deben entonces tomar medidas para poder asegurar el libre ejercicio de los derechos y la igualdad de todos sus miembros, aún cuando esto pueda significar una restricción en la libertad de expresión \cite{article192015}. Entendiendo entonces que este derecho tiene sus límites, podemos pensar que el discurso de odio es una de esas fronteras. Si bien este fenómeno es algo que no está completamente delimitado, repasaremos algunas definiciones de este fenómeno hechas en tratados para acercarnos un poco más a las características comunes que comparten las diferentes definiciones. La Observación General 35 del Comité por la Eliminación de la Discriminación Racial de la ONU (CERD) considera que será discurso de odio, y debe ser tipificado penalmente:

\begin{displayquote}[Recomendación 35 del Comité por la Eliminación de la Discriminación Racial, CERD]

    a) Toda difusión de ideas basada en la superioridad o en el odio racial o étnico, por cualquier medio;

    b) La incitación al odio, el desprecio o la discriminación contra los miembros de un grupo por motivos de su raza, color, linaje, u origen nacional o étnico;

    c) Las amenazas o la incitación a la violencia contra personas o grupos por los motivos señalados en el apartado anterior;

    d) La expresión de insultos, burlas o calumnias contra personas o grupos, o la justificación del odio, el desprecio o la discriminación por los motivos señalados en el apartado b) anterior, cuando constituyan claramente incitación al odio o a la discriminación;

    e) La participación en organizaciones y actividades que promuevan e inciten a la discriminación racial.
\end{displayquote}

%\citet{gagliardone2015countering} presenta un análisis de diversos organismos y sus definiciones de discurso de odio.

En líneas generales, como se menciona en el reporte de la CIDH sobre discurso de odio contra lesbianas, gay, trans e intersex en Latinoamérica \cite{CIDH2015}, el concepto usualmente es referido a expresiones que incitan a tomar algún tipo de medida hostil contra una víctima o un grupo de personas, siendo esta perteneciente a un determinado grupo social definido por alguna característica particular como ser la etnia, lenguaje, género, entre otras. Dicho esto, podría delimitarse el discurso discriminatorio del discurso de odio por la componente de la promoción e instigación de la violencia; sin embargo, para los fines de este trabajo utilizaremos los términos indistintamente. Aún cuando el discurso no contenga arengas ni incitaciones a cometer actos violentos, puede entenderse ese discurso como generador de un ambiente hostil y de intolerancia que termine promoviendo estos ataques físicos \cite{CIDH2015}.

\citet{article192015} condensa muchas de estas definiciones de una manera sucinta, desglosando esto en \textbf{odio} y \textbf{discurso}:

\begin{displayquote}[Article 19: Hate Speech Toolkit]
    1. Odio: emoción intensa e irracional de oprobio, enemistad y aborrecimiento hacia una persona o grupo de personas, por tener determinadas características protegidas (reconocidas en el derecho internacional), reales o percibidas. El “odio” es más que un mero prejuicio y debe ser discriminatorio. El odio es una muestra de un estado emocional u opinión y, por lo tanto, se diferencia de cualquier acto o acción que se haya llevado a cabo.
    2. Discurso: cualquier expresión que vierta opiniones o ideas, que comparte una opinión o una idea interna con un público externo. Puede adoptar muchas formas: escrita, no-verbal, visual o artística y puede ser difundida en los medios, incluyendo Internet, material impreso, radio o televisión.
\end{displayquote}

%%
%%
%% Link
%% https://docs.google.com/drawings/d/149dpb2nrvmFgWZJYcrToAxO4M5n7JNQInfWd62kw3jc/edit
%%
%%

\begin{figure}[t]
    \centering
    \includegraphics[width=\textwidth]{img/discurso_de_odio.pdf}
    \caption{Definición de discurso de odio de acuerdo al Toolkit de Article 19}
    \label{fig:hate_speech_definition_article_19}
\end{figure}


En base a esta definición, puede entenderse al discurso de odio como un discurso de cierta intensidad e irracionalidad que ataca a una persona o un grupo de personas por alguna característica históricamente vulnerada: por ser mujer, por su género, por su etnia, nacionalidad, religión, idioma, etc. La clave está en la combinación: un discurso irracional e intenso contra alguien que no posea una característica protegida no configura discurso de odio; por ejemplo, ataques a ciertas personas por ser periodistas. La Figura \ref{fig:hate_speech_definition_article_19} ilustra esta definición.

No todo ataque a un individuo o una persona de algún colectivo discriminado es discurso de odio. En particular, la CIDH \cite{CIDH2015} menciona en base al informe de la UNESCO sobre discurso de odio \cite{gagliardone2015countering} que:

\begin{displayquote}[]
    (...) el discurso de odio no puede abarcar ideas amplias y abstractas, tales como las visiones e ideologías políticas, la fe o las creencias personales. Tampoco se refiere simplemente a un insulto, expresión injuriosa o provocadora respecto de una persona. Así definido, el discurso de odio puede ser manipulado fácilmente para abarcar expresiones que puedan ser consideradas ofensivas por otras personas, particularmente por quienes están en el poder, lo que conduce a la indebida aplicación de la ley para restringir las expresiones críticas y disidentes. Asimismo, el discurso de odio tiene que distinguirse de aquellos ``crímenes de odio'' que se basan en conductas expresivas, como las amenazas y la violencia sexual, y que se encuentran fuera de cualquier protección del derecho a la libertad de expresión
\end{displayquote}

Como vemos, no solamente es difusa la frontera fijada sobre qué es discurso de odio o insultos, sino que incluso también es difícil definir qué característica es protegida o no. En el siguiente capítulo hablaremos más de esto al describir los criterios utilizados a la hora de anotar un conjunto de datos sobre comentarios en Twitter.


\subsection{Definiciones utilizadas desde NLP}

\begin{figure}
    \centering
    \includegraphics[width=0.75\textwidth]{img/04/poletto_hs_vs_offensive.pdf}
    \caption{Discurso de odio y conceptos relacionados. Fuente: \citet{poletto2021resources}}
    \label{fig:hate_speech_definition_and_other_concepts}
\end{figure}



Debido a las razones comentadas en el Capítulo \ref{chap:01_intro}, la comunidad de NLP ha observado en los últimos años un creciente interés en la investigación de técnicas automáticas para combatir el discurso de odio en redes sociales. Una de las primeras dificultades de esta tarea es la mencionada dificultad para definir unívocamente este discurso, algo que ha provocado que los investigadores del área aún no tengan un marco teórico común acerca de su definición. Esta falta de una teoría unificada, a su vez, se debe en parte a que la investigación de técnicas automáticas de reconocimiento se encuentra aún en una etapa relativamente prematura \footnote{la gran mayoría de los trabajos y recursos son de los últimos 5 años}, y a la relación de este discurso con otros fenómenos de las redes, como ser el lenguaje tóxico, grosero, entre otras distinciones, que también es de interés estudiar. La Figura \ref{fig:hate_speech_definition_and_other_concepts} ilustra el marco teórico utilizado por \citet{poletto2021resources} en relación al discurso de odio: mientras éste es un subconjunto del lenguaje abusivo y tóxico, algunos discursos racistas o misóginos \footnote{Los autores utilizan el término \emph{microagresiones}} no configurarían discurso de odio.



\begin{table}
    \centering
    \small
    \begin{tabular}{p{0.65\textwidth} p{0.30\textwidth}}
        Definición & Fuente \\
        \Xhline{2\arrayrulewidth}
        Cualquier comunicación que menosprecia a una persona o grupo en base a su raza, etnia, género, orientación sexual, nacionalidad, religión u otra característica & \citet{warner2012detecting}, \citet{hateval2019semeval} \\
        \hline
        Uso de insultos racistas o sexistas; ataques a minorías; promoción de discurso de odio o crímenes violentos; distorsiones o mentiras acerca de minorías; apoyo a hashtags racistas o sexistas; defender xenofobia o sexismo; contener nombre de usuario ofensivo & \citet{waseem2016hateful} \\
        \hline
        Lenguaje que es usado para expresar odio hacia un grupo objetivo o que pretende ser despectivo, o humillar o insultar sus miembros & \citet{Davidson2017AutomatedHS} \\
        \hline
        Acto de ofender, insultar o amenazar a una persona o grupo de acuerdo a su religión, raza, casta, orientación sexual, género, o pertenencia a alguna comunidad estereotipada & \citet{schmidt2017survey} \\
        \hline
        Contenido definido por: primero, su intención de diseminar odio, incitar a la violencia, o amenazar la libertad, dignidad o integridad de las personas; segundo, su objetivo, que debe ser un grupo protegido o un miembro de tal grupo & \citet{sanguinetti2018italian} \\
        \Xhline{2\arrayrulewidth}
    \end{tabular}
    \caption{Definiciones de discurso de odio para diferentes trabajos del área. Fuente: \citet{poletto2021resources}}
    \label{tab:hate_speech_definitions}
\end{table}

La Tabla \ref{tab:hate_speech_definitions} muestra las definiciones de discurso de odio utilizadas en algunos trabajos del área. Podemos ver que las definiciones tienen diferencias sustanciales,  algunas siendo directamente operacionales, y algunas otras con matices respecto a la intención o su intensidad.

\section{Trabajo previo}

Haremos una reseña de la literatura de la detección de discurso de odio y otros fenómenos similares. Un análisis exhaustivo de esta subdisciplina sería inviable debido a la enorme cantidad de trabajo del área, con un ritmo meteórico en los últimos años. Referimos para repasos más extensivos a \citet{schmidt2017survey} y \citet{fortuna2018survey}. Más recientemente, \citet{poletto2021resources} hacen un análisis pormenorizado y actualizado de los recursos existentes para esta tarea.

La detección del discurso del odio es una tarea de clasificación de textos relacionada con el análisis de sentimientos y ha sido estudiada para varias redes sociales \cite{thelwall2008social, pak2010twitter, saleem2017web}. Uno de los primeros trabajos al respecto es el de \citet{greevy2004classifying}, quienes utilizan bolsas de palabras y Support Vector Machines para detectar contenido racista en páginas web, utilizando un dataset construido de manera semi-supervisada buscando sitios mediante keywords y sus links en motores de búsqueda. Siguiendo un enfoque similar, \citet{warner2012detecting} usaron unigramas y Brown clusters \cite{brown1992class} con SVMs para detectar mensajes antisemitas en Twitter.

\citet{waseem2016hateful} anotaron un corpus y usaron técnicas basadas en n-gramas de caracteres para detectar discurso de odio en comentarios de Twitter. \citet{badjatiya2017deep} usaron el mismo conjunto de datos para entrenar modelos de aprendizaje profundo con embeddings ajustados a los datos, obteniendo mejoras sustanciales en el rendimiento para la tarea en cuestión aunque sujeto a algunos problemas de entrenamiento observado por otros trabajos \cite{arango2019hate}. \citet{zhang2018detecting} entrenaron una red neuronal profunda que combina CNNs con Gated Recurrent Units \cite{cho2014learning}, superando a los sistemas anteriores en varios conjuntos de datos de detección de discurso de odio. \citet{anzovino2018automatic} recopilaron un corpus de tweets misóginos y propusieron una taxonomía para distinguirlos en diferentes categorías. A su vez, los autores mostraron que enfoques simples (como el uso de modelos lineales junto con n-gramas) logran un rendimiento competitivo en conjuntos de datos de pequeño tamaño.

En cuanto a las tareas compartidas, \citet{fersini2018overview} presentaron un dataset para la detección de misoginia en Twitter, tanto en español como en inglés, mientras que \citet{fersini2018evalitaoverview} planteó un desafío similar pero en italiano e inglés. \citet{bosco2018overview} propuso un concurso de detección automática sobre publicaciones de Twitter y comentarios de Facebook, que incluía discursos de odio en general.

Una de las herramientas más utilizadas, no sólo para la detección de discurso de odio sino para la detección de contenido tóxico en general es Perspective API de Google, desarrollada originalmente por Jigsaw \footnote{\url{https://developers.perspectiveapi.com/s/}}. Esta API de acceso libre brinda un analizador muy potente para la detección de lenguaje tóxico, con información granular sobre los tipos de ataques. Algunos trabajos lo utilizan como algoritmo de detección en modalidad zero-shot, obteniendo mejores resultados que modelos entrenados sobre los propios datos \cite{pavlopoulos2020toxicity}. Sin embargo, algunas de sus debilidades han sido marcadas mediante ejemplos adversariales, algo que obviamente es propio de las actuales limitaciones de las técnicas de NLP \cite{hosseini2017deceiving,jain2018adversarial}. Más aún, la información de grano fino --e.g. si es un ataque a un grupo protegido y a cuál se ataca-- sólo está disponible para el inglés.

Dentro de los trabajos en español, \citet{plaza2021pretrained} evalúan distintos modelos pre-entrenados de lenguaje sobre la tarea de detección de discriminación usando dos datasets: el primero, \citet{pereira2019detecting} que consta de 6000 tweets, recolectado por el Estado Español para monitorear el discurso de odio en redes sociales; y el segundo, el dataset de SemEval 2019 Task 5 (\hateval{}) \cite{hateval2019semeval}, presentado en contexto de una shared-task y que comprende ataques contra inmigrantes y mujeres.


\section{Descripción del dataset utilizado}
\label{sec:hateval_dataset}

\begin{table}[t]
    \centering
    \begin{tabular}{l c c c  c c c}
        Categoría  &    \mc{3}{Español}                          & \mc{3}{Inglés}                                \\
                   &Train   & Dev    & Test   &Train   & Dev   & Test           \\
        \thline{2.5}
        No HS      &$2643$  & $278$  & $940$  &$5217$  & $573$ & $1740$  \\
        HS         &$1857$  & $222$  & $660$  &$3783$  & $427$ & $1260$  \\
        TR         &$1129$  & $137$  & $423$  &$1341$  & $219$ & $529$    \\
        AG         &$1502$  & $176$  & $474$  &$1559$  & $204$ & $594$    \\
        \hline
        Total      &$4500$  & $500$  & $1600$ &$9000$  & $1000$& $3000$  \\
        \thline{2.5}
    \end{tabular}
    \caption{Números del dataset de \citet{hateval2019semeval}, por idioma y por partición. No HS representa los tweets que no tienen contenido odioso, HS aquellos que sí, TR aquellos que son individualizados, y AG aquellos que son agresivos. Entre paréntesis encontramos los porcentajes de incidencia, considerando TR y AG dentro de aquellos que son discriminatorios}
    \label{tab:hateval_dataset}
\end{table}

Utilizamos en este capítulo el dataset provisto por \citet{hateval2019semeval}, presentado en SemEval 2019 y orientado a la detección de discurso de odio contra mujeres e inmigrantes en Twitter. Los autores recopilaron comentarios en inglés y en español de dicha red social mediante tres estrategias combinadas: monitoreando a las posibles víctimas de cuentas de odio; chequeando el historial de usuarios creadores de contenido discriminatorio; y filtrando contenido mediante palabras clave. A su vez, este trabajo distingue entre el discurso de odio dirigido a individuos y el discurso de odio genérico, y entre mensajes agresivos y no agresivos. En el Capítulo \ref{chap:05_dataset_creation} construiremos un conjunto de datos contextualizado de discurso de odio en base a algunas de las limitaciones observadas sobre en este capítulo.

Las instancias del dataset poseen las siguientes etiquetas:

\begin{itemize}
    \item \textbf{HS}: una etiqueta binaria que marca si el tweet tiene contenido discriminatorio contra mujeres o inmigrantes (0 si no lo tiene, 1 si hay discurso de odio)
    \item \textbf{TR}: Si hay HS, una etiqueta binaria que marca si el objetivo del discurso de odio es un objetivo genérico (0) o si se refiere a un individuo específico (1)
    \item \textbf{AG}: Si hay HS, una etiqueta binaria que marca si el tweet es agresivo
\end{itemize}

\begin{table}
    \centering
    \small
    \begin{tabularx}{\textwidth}{X c c c}
        Texto & HS & TR & AG \\
        \hline
        Los tomas asi puro como si fuera jugo y cuando te querés rescatar estas hablando en árabe URL & 0 & 0 & 0 \\
        \rule{0pt}{4ex}Como son españoles nada... sin fueran refugiados...GLORIA \#migrates \#refugiados \#EspañaLoPrimero URL & 1 & 0 & 0 \\
        \rule{0pt}{4ex}@OmarPrietoGob ``Extranjero sin identificación será puesto en la frontera'' ENVÍA AL EJERCITO A TOMAR CONTROL DE LAS PULGAS PLATANEROS Y CURVA DE AHÍ PARA QUE VEAS COMO HAY COLOMBIANOS INDOCUMENTADOS COMO MONTE AHÍ DE BUHONERS PORQUE LA POLICÍA & 1 & 0 & 1 \\
        \rule{0pt}{4ex}Inmigrante da una brutal paliza a una joven por no dejarse besar en Ciudad Real.\#stopinvasion \#YoSiTeCreo \#NoesNo lo peor que no han salido a la calle las feminas del Twitter que tanto se indignaron con la salida de La Manda a la calle. & 1 & 1 & 0 \\
        \rule{0pt}{4ex}@elisacarrio Callate hija de puta gorda falopera & 1 & 1 & 1 \\
        \hline
    \end{tabularx}
    \caption{Ejemplos del dataset de SemEval 2019 Task 5: \hateval{}. HS indice la presencia de discurso de odio, TR la presencia de discriminación individualizada, y AG la presencia de discriminación agresiva}
    \label{tab:hateval_dataset_examples}
\end{table}



La Tabla \ref{tab:hateval_dataset} muestra los números para cada partición, cada idioma, y cada una de las etiquetas. Podemos observar entre los dos idiomas que, si bien la proporción de discurso de odio se mantiene muy similar (58\% vs 42\% aproximadamente), la proporción de discurso de odio individualizado (TR) y agresivo (AG) es notoriamente más alto para el español que para el inglés. Esto puede deberse, entre otras cosas, a distintas estrategias de recolección de los tweets. La tabla \ref{tab:hateval_dataset_examples} posee algunos ejemplos para cada una de las características en cuestión para la porción en español, que es la de nuestro interés.

\section{Tareas de clasificación}

Sobre los datos mencionados en la anterior sección, los autores propusieron dos tareas de clasificación:

\newcommand{\subtaska}[0]{\textbf{Tarea A}}
\newcommand{\subtaskb}[0]{\textbf{Tarea B}}

\begin{itemize}
    \item \subtaska{}: Dado un tweet predecir si contiene discurso de odio contra mujeres o inmigrantes (HS)
    \item \subtaskb{}: Dado un tweet, predecir si contiene discurso de odio (HS), si está dirigido contra un individuo o un grupo (TR), y si es agresivo o no (AG)
\end{itemize}


La primer tarea es la versión más básica de la detección de discurso de odio, donde predecimos una etiqueta binaria que marca la presencia de contenido de esta índole. La segunda es una versión más rica, de grano fino, donde predecimos varias características de particular interés para distinguir algunas formas potencialmente más peligrosas de este fenómeno: por ejemplo, si es agresivo y si es individualizado, lo que puede indicar alguna incitación a un ataque de un individuo o miembros de algún grupo protegido.

\citet{hateval2019semeval} propusieron para medir el desempeño en la \subtaska{} la Macro F1 de las clases positiva y negativa. Para el caso de la \subtaskb{}, utilizaron dos métricas: Macro F1 de las 3 clases (HS, TR, AG) y también la medida Exact Match Ratio:

\begin{equation*}
    EMR = \frac{1}{n} \sum\limits_{i=1}^{n} I(Y_i, Y_i^*)
\end{equation*}

\noindent siendo $Y_i$ las etiquetas respectivas $(HS, TR, AG)$, $Y_i^*$ las etiquetas que predice nuestro sistema, e  $I$ la función indicadora ($I(x, x) = 1$; $0$ en cualquier otro caso). Observado más de cerca, esto puede entenderse la accuracy sobre la 3-upla de la salida de los clasificadores, pero para evitar confusiones usamos el nombre de Exact Match Ratio (EMR). \cite{zhang-2014-multilabel}

\section{Método}

En esta sección describimos los distintos modelos planteados para abordar las dos tareas de clasificación de discurso de odio, así como detalles de preprocesamiento introducidos en \citet{perez-2019-atalaya} para optimizar las representaciones de modelos lineales de clasificación.

\subsection{Preprocesamiento}
\label{sec:04_preprocessing}

Definimos dos niveles de preprocesamiento: básico y orientado a sentimientos, dependiendo del modelo a utilizar. El preprocesamiento básico de tweets es el mismo que describimos en la Sección \ref{sec:03_preprocessing}, y es el usado con los modelos pre-entrenados o modelos neuronales.

El preprocesamiento orientado a sentimientos incluye además lematización (usando TreeTagger \cite{schmid95}) y manejo de negación. Para el manejo de la negación, seguimos un enfoque simple:
% \cite {das01, pang02}:
Buscamos palabras de negación y agregamos el prefijo 'NOT \_' a los siguientes tokens. Se niegan hasta tres tokens, o menos si se encuentra un token que no sea una palabra.

\subsection{Modelos de clasificación}
\label{sec:04_classifiers}

Para las tareas propuestas, analizamos el desempeño de diversos modelos de clasificación. Algunos de ellos son los presentados para la shared-task \hateval{} en \citet{perez-2019-atalaya}, a las cuales agregamos modelos basados en transformers. \footnote{Estos modelos no estaban disponibles al momento de presentar dicho trabajo. El trabajo de \bert{} \cite{devlin2018bert} es de finales de 2018, y hasta finales de 2019 no fue publicada una versión entrenada en español, \beto{}} Para la tarea de detección binaria (\subtaska{}) planteamos 3 tipos de clasificadores:

\begin{enumerate}
    \item Modelos lineales: regresiones logísticas y SVM con kernel lineales, consumiendo como entrada bolsas de palabras, bolsas de caracteres, y tweet embeddings
    \item Redes neuronales recurrentes: usando como entrada representaciones no contextualizadas (\fasttext{}) y  contextualizadas (\elmo{})
    \item Modelos pre-entrenados de lenguaje.
\end{enumerate}

Para los modelos lineales, utilizamos representaciones de cada tweet calculadas con Smooth Inverse Frequency (ver Sección \ref{sec:02_tweet_embeddings} para más detalles), usando como base los vectores de \fasttext{} entrenados sobre tweets con preprocesamiento orientado a sentimientos. Los modelos recurrentes consumieron como entrada la concatenación de representaciones de \fasttext{} que describimos en la Sección \ref{sec:03_classification} a las cuales adosamos vectores contextualizados basados en \elmo{}. \cite{peters2018} Para esta última técnica, usamos la versión en español entrenada por \citet{che-EtAl:2018:K18-2}. Finalmente, consideramos los siguientes modelos pre-entrenados: para el español \beto{} \cite{canete2020spanish}, y para el inglés \bert{} \cite{devlin2018bert}, \roberta{} \cite{liu2019roberta} y \bertweet{}. \cite{dat2020bertweet}

La tarea de multidetección de discurso de odio (\subtaskb{}) podemos pensarla de dos maneras:

\begin{enumerate}
    \item Un problema de clasificación múltiple
    \item Un problema de clasificación de 5 clases
\end{enumerate}

En el primer caso, el enfoque es el de predecir por separado cada una de las variables HS, AG, y TR. La segunda formulación se basa en observar que no tenemos 8 combinaciones permitidas sino sólo cinco: si no hay HS no nos interesa observar las otras dos variables. Con esta observación, convertimos cada combinación en una clase de un problema de clasificación estándar. En \citet{perez-2019-atalaya} propusimos un modelo basado en Support Vector Machines que consume la misma entrada que detallamos anteriormente y con una salida de cinco clases. No evaluamos en dicho trabajo un modelo recurrente con este esquema de clasificación ni tampoco lo haremos aquí, considerando que evaluamos opciones que han demostrado tener mejor desempeño para numerosas tareas de clasificación de texto.

\begin{figure}
    \centering
    \includegraphics[width=0.8\textwidth]{img/05/bert_model_hateval.pdf}
    \caption{Modelo basado en BERT para la tarea de clasificación múltiple. Cada variable (HS, TR, AG) representa un problema de clasificación en sí mismo}
    \label{fig:bert_hateval_classifier}
\end{figure}


Asimismo, considerando la opción de multiclasificación, proponemos para la \subtaskb{} modelos de lenguaje pre-entrenados con tres salidas. Recordemos que la arquitectura usual de clasificación basada en \bert{} consta de poner como última capa una softmax que consume como entrada la salida del token \verb|[CLS]|. Esto, además, agregando como parámetro una matriz de proyección $W \in \mathbb{R}^{m \times 768}$ donde $m$ es la cantidad de clases de nuestro problema y 768 corresponde a la dimensión de cada vector del modelo de transformers.

Para construir un modelo de multiclasificación, mantenemos la misma arquitectura pero, en lugar de usar como activación la función softmax, utilizamos la función sigmoidea elemento a elemento. En el caso de clasificación de $n$ clases, interpretamos a $\text{softmax}(W x + b)_i$ como la probabilidad\footnote{Estrictamente hablando, más bien sería un puntaje entre 0 y 1} instancia pertenezca a la clase $i$. Por otro lado, en el caso de multiclasificación de $n$ variables, $\sigma(W x + b)_i$ \footnote{Esta expresión es elemento a elemento} nos da la probabilidad de predecir la etiqueta positiva para la variable i-ésima: en nuestro caso, $\sigma(W x + b)_1$ nos da $P(HS = 1\mid x)$,  $\sigma(W x + b)_2$ nos da  $P(TR = 1 \mid x)$ y finalmente el tercer subíndice nos da  $P(AG = 1\mid x)$. La Figura \ref{fig:bert_hateval_classifier} ilustra el modelo utilizado.

Para entrenar el modelo de clasificación, evaluamos dos tipos de funciones de costo. En primer lugar, utilizamos la suma de las entropías cruzadas binarias. Concretamente, si $y = (y_{HS}, y_{TR}, y_{AG})$ son las etiquetas de una instancia e $\widehat{y}$ la predicción del modelo, la función de costo es:

\begin{equation}
\label{eq:multi_loss}
L(y, \widehat{y}) = \sum\limits_{\mathclap{k \in \{HS, TR, AG\}}} J(y_k, \widehat{y_k})
\end{equation}

\noindent donde $J$ es la entropía cruzada binaria. Esta función de costo, sin embargo, ignora cualquier tipo de jerarquía entre las variables; por ejemplo, si para una instancia tenemos $HS = 0$, calcula el costo también de las variables $TR$ y $AG$. Contemplamos entonces una variante de la función descripta en la ecuación \ref{eq:multi_loss} para tener en cuenta esto:

\begin{equation}
    \label{eq:hierarchical_loss}
    L(y, \widehat{y}) =  J(y_{HS}, \widehat{y_{HS}}) + \beta(y_{HS})\sum\limits_{\mathclap{k \in \{TR, AG\}}} J(y_k, \widehat{y_k})
\end{equation}

\noindent donde $\beta(y_{HS})$ pondera la pérdida de las variables del segundo nivel de nuestra jerarquía. Una opción puede ser considerar $\beta(1) = 1, \beta(0) = 0$, donde ignoramos las pérdidas de las variables $TR$ y $AG$ cuando no hay discurso discriminatorio. Análogamente, $\beta(y) = 1$ sería el caso descripto en la ecuación \ref{eq:multi_loss}. Una forma de generalizar esto es agregando un hiperparámetro $\gamma \in [0, 1]$ para escribir $\beta(y) = (1-y) \gamma + y$. Optimizamos este hiperparámetro realizando una búsqueda lineal entre $0$ y $1$ utilizando pasos de $0.1$.

Al momento de realizar inferencia, realizamos las evaluaciones de los modelos realizamos poniendo una máscara por encima de estos modelos de clasificación múltiple de manera de evitar salidas incoherentes (por ejemplo, $HS = 0, TR = 1, AG= 0$).


\section{Resultados}

\newcommand{\esrow}[1]{\multirow{#1}{*}{es}}
\newcommand{\enrow}[1]{\multirow{#1}{*}{en}}

\begin{table}[ht!]
    \centering
    \begin{tabular}{l c c c c c}
        Modelo       & Idioma      & Precision    & Recall       & F1            & Macro F1 \\
        \hline
        SVM$^*$      & \mr{3}{es}  & $63.9$       & $80.0$       & $71.1$       & $73.0$  \\
        ELMO-RNN     &             & $66.1$       & $75.3$       & $70.4$       & $73.5$  \\
        BETO         &             & $\mbf{67.4}$ & $\mbf{83.9}$ & $\mbf{74.7}$ & $\mbf{76.4}$ \\
\rule{0pt}{4ex}BERT  & \mr{3}{en}  & $47.4$       & $\mbf{96.8}$ & $63.7$       & $49.6$ \\
        RoBERTa      &             & $47.0$       & $96.7$       & $63.2$       & $48.6$ \\
        BERTweet     &             & $\mbf{49.5}$ & $95.9$       & $\mbf{65.3}$& $\mbf{54.6}$ \\
        \hline
    \end{tabular}
    \caption{Resultados de la evaluación para la detección de discurso de odio en el dataset de test, medidas por \% de precisión, sensitividad y F1 sobre la clase positiva (discurso de odio) y por la métrica Macro F1. Con $^*$ están marcados los resultados presentados en \citet{perez-2019-atalaya}. En negrita, el mejor resultado.}
    \label{tab:hateval_task_a}
\end{table}



La Tabla \ref{tab:hateval_task_a} muestra los resultados de la evaluación para la detección de discurso de odio binaria (\subtaska{}), marcando con un asterisco aquellos modelos presentados en \citet{perez-2019-atalaya}. Respecto a los resultados en español, el clasificador basado en SVMs obtiene una buena performance, aún comparado con aquel basado en embeddings contextualizados. Este algoritmo basado en SVMs obtuvo el mejor desempeño en la competencia con $0.730$ de Macro F1 \cite{hateval2019semeval}. El pobre desempeño de \elmo{} contra un algoritmo mucho más simple puede deberse a un mal pre-entrenamiento del modelo base en español\footnote{No queda claro que en entrenar este modelo sobre 20M palabras sea suficiente, ni que sea un dataset suficientemente general} y también debido al cambio de dominio, a los cuales los modelos pre-entrenados previos a BERT son sumamente sensibles \cite{hendrycks-etal-2020-pretrained}.

Para ambos idiomas, los modelos basados en Transformers \cite{vaswani2017attention} obtienen la mejor performance, con considerables mejoras respecto a los modelos basados en ELMo y a los SVMs \footnote{Un modelo que no evaluamos en el presente trabajo es la versión en español de RoBERTa, recientemente entrenada. En el capítulo 7 evaluaremos su rendimiento en esta tarea}. Particularmente, en el caso del inglés, \bertweet{} \cite{dat2020bertweet} obtiene la mejor Macro F1, algo esperable considerando que está particularmente diseñado para Twitter.


\begin{table}[ht!]
    \centering
    \begin{tabular}{lll ccc cc}
        Modelo            &        & Idioma      &  HS F1       & TR F1        &  AG F1      &  Macro F1    &   EMR           \\
        \thline{2.5}
        \mr{3}{BETO}      & multi  & \mr{3}{es}  & $74.1$       & $\mbf{76.5}$&$\mbf{68.8}$  & $\mbf{73.1}$ & $68.5$          \\
                          & hier   &             & $73.5$       & $75.8$       & $67.4$      & $72.2$       & $\mbf{70.3}$    \\
                          & combi  &             & $\mbf{74.2}$ & $76.3$       & $66.8$      & $72.4$       & $69.8$          \\

        \hline
              \mr{3}{BERT}& multi & \mr{3}{en}  & $63.8$       & $60.0$       & $44.3$      & $56.0$       & $38.0$          \\
                          & hier   &             & $64.2$       & $59.2$       & $45.1$      & $56.2$       & $38.8$          \\
                          & combi  &             & $64.4$       & $59.3$       & $44.2$      & $56.0$       & $39.8$          \\
        \hline
        \mr{3}{RoBERTa}   & multi  & \mr{3}{en}  & $63.4$       & $57.8$       & $45.4$      & $55.5$       & $36.5$          \\
                          & hier   &             & $63.7$       & $57.2$       & $45.6$      & $55.5$       & $37.0$          \\
                          & combi  &             & $63.6$       & $57.6$       & $44.2$      & $55.1$       & $37.7$          \\
        \hline
        \mr{3}{BERTweet}  & multi  & \mr{3}{en}  & $65.8$       & $62.9$       &$\mbf{46.2}$ &$\mbf{58.3}$  & $42.6$          \\
                          & hier   &             & $65.6$       & $61.7$       & $45.0$      & $57.4$       & $42.3$          \\
                          & combi  &             & $\mbf{66.6}$ &$\mbf{63.7}$   & $44.4$     & $58.2$       &$\mbf{44.9}$     \\
        \thline{2.5}
    \end{tabular}

    \caption{Resultados de la evaluación para para \subtaskb{} en términos de las F1 de las clases HS (Hate Speech), TR (Targeted), AG (Aggressive), el Exact Match Ratio (EMR), las Macro F1 de las clases en cuestión, y la Macro F1 de la clase HS. Las 3 variaciones de los modelos son: \emph{multi} es la salida de multiclasificación estándar, \emph{hier} es la salida de multiclasificación con una jerarquía de clasificación, y \emph{combi} es la salida de multiclasificación con una combinación de clasificaciones. Los resultados están expresados como las medias de 10 corridas independientes.}
    \label{tab:hateval_task_b}
\end{table}


La tabla \ref{tab:hateval_task_b} muestra los resultados de la \subtaskb{}, reportado por las F1 de cada variable predicha (HS, TR, AG), así como por la Macro F1 de las 3 variables mencionadas y el Exact Match Ratio. Los resultados están expresados como la media de 10 corridas independientes del experimento para cada configuración distinta. Consideramos las 3 versiones: \emph{multi} refiere a clasificación múltiple, \emph{hier} a clasificación múltiple con la función de costo jerárquica, y \emph{combi} a la conversión del problema en una clasificación de cinco clases.

Podemos observar que para español, la mejor performance en términos de EMR (la métrica más estricta) es el clasificador entrenado con la función de costo definida en \ref{eq:hierarchical_loss} (con el hiperparámetro $\gamma = 0.1$); sin embargo, la diferencia entre las performances no es significativa al correr un test de Kruskal-Wallis ($H(9) = 3.492, p = 0.174$). En términos de Macro F1, la mejor performance es de \beto{} con la salida múltiple y sin la función de costo jerárquica (\emph{multi}) pero de nuevo esta diferencia no es significativa ($H(9) = 3.656, p=0.16$).

Respecto al inglés, los mejores resultados pueden observarse en el modelo entrenado con \bertweet{} con la salida de cinco clases en el caso del EMR, y con la salida múltiple (sin pérdida jerárquica) para la Macro-F1. Este resultado, sin embargo, queda en términos de EMR por debajo del baseline propuesto por los autores del dataset \cite{hateval2019semeval}, aunque cercano en términos de Macro F1 a los mejores resultados de la competencia. En \citet{gertner-etal-2019-mitre}, se basaron en un ensemble de modelos entrenados con BERT y usando también un ajuste de dominio sobre tweets. Esta baja performance de nuestros modelos (y de los modelos en general sobre ese dataset) puede deberse a problemas de anotación y a que las particiones de train y test no son idénticamente distribuidas. \footnote{En \citet{gertner-etal-2019-mitre} dan evidencia de esto, algo que perjudica el desempeño de estos modelos}



\begin{table}[t]
    \centering
    \begin{tabular}{l  l l  c c c c}
        Modelo              & Idioma        &  Tarea  &     Precision  &   Recall        &          F1    &  Macro F1       \\
        \thline{2}
        \mr{2}{BERTweet}    & \mr{2}{en}    &  A      & $49.5 \pm 1.2$ &  $95.9 \pm 1.2$ & $65.3 \pm 0.9$ &  $54.6 \pm 2.7$ \\
                            &               &  B      & $50.5 \pm 1.1$ &  $94.8 \pm 1.8$ & $65.8 \pm 0.5$ &  $56.7 \pm 2.2$ \\
        \hline
        \mr{2}{BETO}        & \mr{2}{es}    &  A      & $67.4 \pm 2.1$ &  $83.9 \pm 2.6$ & $74.7 \pm 0.7$ &  $76.4 \pm 1.1$ \\
                            &               &  B      & $71.3 \pm 4.2$ &  $77.8 \pm 5.4$ & $74.1 \pm 1.3$ &  $77.1 \pm 1.5$ \\
        \thline{2}
    \end{tabular}
    \caption{Comparación de la performance sobre la detección de discurso de odio para los clasificadores entrenados sobre \subtaska{} y \subtaskb{}. Resultados expresados como la media de 10 corridas independientes del experimento junto a sus desviaciones estándar. Ambos clasificadores de la \subtaskb{} están entrenados sobre el problema de multi-clasificación}
    \label{tab:hateval_task_a_vs_b}
\end{table}

La Tabla \ref{tab:hateval_task_a_vs_b} muestra la comparativa para la detección de discurso de odio (HS) para aquellos clasificadores que obtuvieron mejores resultados para \subtaska{} y \subtaskb{} (\beto{} y \bertweet{}). Consideramos para la \subtaskb{} al clasificador \emph{multi} de cada modelo de lenguaje. Lejos de dañarse la performance de la detección de lenguaje discriminatorio (lo que analizamos en la \subtaska{}), predecir más de una variable pareciera mantener el desempeño general; más aún, podemos observar que en términos de Macro F1, incluso parecieran tener una ligera mejora al ser entrenados sobre una tarea más compleja.






\subsection{Análisis de Error}
\label{sec:hateval_error_analysis}

\begin{figure}[t]
    \centering
    \includegraphics[width=0.6\textwidth]{img/hateval_confusion_matrix.pdf}
    \caption{Error analysis of the model}
    \label{fig:hateval_error_analysis}
\end{figure}


Para tener una mejor idea de lo ocurrido con nuestros clasificadores, realizamos un análisis de error. Para ello, tomamos el clasificador BETO con clasificación múltiple, y lo aplicamos sobre el dataset de test. Para intentar mitigar los efectos de la varianza de los clasificadores, tomamos las salidas de 10 clasificadores BETO entrenados con una distinta semilla y analizamos el error sobre un clasificador de ensamble por voto mayoritario; de esta manera, tratamos de buscar aquellos errores frecuentes, aquellos que la mayoría de los 10 clasificadores erran. Nos centraremos en observar los errores de la etiqueta HS, y particularmente veremos qué pasa con los falsos positivos. La figura \ref{fig:hateval_error_analysis} muestra la matriz de confusión de nuestros clasificadores. Como era de esperar (dado que la precisión es más baja que la sensitividad) tenemos muchos falsos positivos, por lo cual haremos eje en observar estos.


\begin{table}
    \begin{tabularx}{\columnwidth}{l l X}
        Etiqueta & Motivo   &  Tweet \\
        \hline
         \mr{7}{No HS} &\mr{7}{Uso irónico o apropiación cultural} & Una cosa que le pasó a una amiga mexicana en Barcelona. Una chica repartía flyers en castellano y catalán, le dio uno a mí amiga en catalán y al segundo se lo quitó y se lo dio en castellano. -No, déjalo que entiendo catalán -Ah, es que como te he visto cara de sudaca Racismo 0 URL \\
                    &        & @saludhernandezm Esas personas que extrañan dictaduras, cualquier dictadura, aún más si son las ibéricas que ponen como eje a Madrid; su cultura, política e idioma sobre la idiosincrasia de los demás. Esos mismos que añoran a Franco y viven en un país 'Sudaca' creyéndose dueños de la verdad.	 \\
                    &        & @caciena después sigo siendo un sudaca pero bueno \\
                    &        & @ArturoMonteduro Pues tienes toda la razón del mundo. Y lo peor es que seguro que tenía papeles y DNI español, pero queda mejor decir "argelino" o "uno d'estos del top manta", "puto inmigrante", o "moro mierda" y ya pues matas dos pájaros de un tiro. \\
                    &        & Obvio me ofendo cuando algn Sudaca hace algn comentario presuponiendo que los mexicanos somos feos, o que el pas est de la verga. Entre mexicanos podemos hablar mierda de Mxico, pero que a ningn pinche extranjero se le ocurra, porque va a haber pedo! \\
                    &        & TODOS LOS INMIGRANTES Y GITANOS FUERA!!! Menos: el colombiano que me vende coca, el negro que me consigue putas, el moro que me pasa costo y el gitano que me vende maría. \\
                    &        & Ayer nos fuimos a tomar algo con los cumpas: Dos españoles, un ponja, un africano y un sudaca. Estamos para campaña de United Colours of Benneton. \\
                    \hline
                    %& Overfitting a keywords & Los fascistas salen de la ratonera para atacar a los 100 inmigrantes que han logrado saltar la valla de Ceuta. Asco de prensa y partidos políticos que se pelean por lograr los votos de los fascistas. Ningún ser humano es ilegal. \#Ceuta \#FelizMiércoles https://t.co/iAIqTj9qmi
                    %&                       & @saludhernandezm Esas personas que extrañan dictaduras, cualquier dictadura, aún más si son las ibéricas que ponen como eje a Madrid; su cultura, política e idioma sobre la idiosincrasia de los demás. Esos mismos que añoran a Franco y viven en un país 'Sudaca' creyéndose dueños de la verdad. \\
            \hline
    \end{tabularx}

    \caption{Errores de etiquetado. En la columna Etiqueta mostramos la etiqueta asignada en el dataset}
    \label{tab:hateval_error_analysis}
\end{table}


En la tabla \ref{tab:hateval_error_analysis} podemos observar algunos de los errores que cometen nuestros clasificadores. Por un lado, podemos observar que algunos errores se deben a cierto overfitting a ciertas palabras ``clave'', muchas de las cuales son producto del proceso de recolección que está fuertemente basada en keywords (inmigrante, sudaca, por ejemplo). Haciendo un poco de probing en los clasificadores, podemos ver que ciertas palabras como ``inmigrante'' automáticamente disparan una salida odiosa.

Otros de los casos que nuestros clasificadores no parecieran detectar son los reportes de discriminación (por ejemplo...) o casos de apropiación cultural o contenido irónico (\emph{ Ayer nos fuimos a tomar algo con los cumpas: Dos españoles, un ponja, un africano y un sudaca. Estamos para campaña de United Colours of Benneton.}). Estos casos dan cuenta de la dificultad de la detección de esta tarea.

Así mismo, en el proceso del análisis de error descubrimos algunos tweets a los cuales pareciera faltar contexto para poder tomar una decisión. Si bien esto a veces puede ser inferido por un humano (por ejemplo, infiriendo que un comentario habla de tal o cual suceso), un clasificador necesitaría esa información. Para el dataset en concreto, muchos de estos comentarios tienen un contexto implícito: muchos tweets hablan de incidentes en la frontera de Ceuta (España) entre inmigrantes y la policía, debido al proceso de recolección realizado. La tabla \ref{tab:hateval_lack_of_context} muestra algunos de estos ejemplos.


Por otro lado, en la tabla YYY podemos observar que el dataset contiene



\begin{table}
    \begin{tabularx}{\textwidth}{l X}
        Etiqueta      & Tweet \\
        \hline
        \mr{13}{No HS} & - @ArturoMonteduro Pues tienes toda la razón del mundo. Y lo peor es que seguro que tenía papeles y DNI español, pero queda mejor decir ``argelino'' o ``uno d'estos del top manta'', ``puto inmigrante'', o ``moro mierda'' y ya pues matas dos pájaros de un tiro.      \\
                      & @UbedaJoseA @WalkingDead\_AMC Perro cállate ala puta       \\
                      & @iancandado Es sudaca como.tu       \\
                      & @Stopfalsasdenun Pero que tienes en la cabesaaaa?(con voz de sudaca de mierda)       \\
                      & @\_Lassana\_ Me la has puesto como un tanque apuntando a un campanario negrata		       \\
                      & @odiomistweets Es una analogía pendejo sudaca	       \\
                      & @MifasolMaite Sudaca? K racista eres \\
                      & @ElQuemaMarias te parece que un sudaca tatuado excelentemente bien dotado y carismático no es buen ejemplo?       \\
                      & @HQGatica @AnglicaVillarN1 Este sueco con cara de bolita .xq eso es este aspiracional desclasado . x dignidad debería irse . no lo hará xq este sudaca no la tiene !!!       \\
                      & @SOFIAGAVIRIAC Ambas cosas y Sudaca (lo que eso signifique)       \\
                      & @SleeplessXenn @envdiaN Este finde te recojo y vamos al barrio negrata de envdian       \\
                      & @elmundoes El q quería quitar concertinas y luego quitarse a los inmigrantes !!       \\
                      & @AlexTor1975 @sanchezcastejon Disfràcese de moro o negro, y al momento tiene resuelto el problema de espera, de verdad, mano de santo!. Nada como ser invasor, refugiado, o inmigrante en este paraíso terrenal. Ahora, si eres nativo del paraíso!!!!, Se te puede complicar la existencia!!!!. \\
        \hline
        \mr{2}{HS}    & CALLATE YA Y HAZME LA PUTA CENA JODERRR URL \\
                      & Basta ya de buenismo, si te no respetas a la polica ni a la guardia civil en la frontera nada de concertinas, simplemente 4 tiros. Fin del problema \\
                      & @benemeritosGC La solución es fácil, muro de hormigón de 3 metros de alto, garitas cerradas para vigilancia y por delante la valla actual con concertinas. No es caro, es cuestión de quererlo hacer. \\
                      & Por desgracia, no queda otra, aportan poco y nos cuestan mucho. Incluido nuestra seguridad. \#Inmigración \#Immigration URL \\
        \hline
    \end{tabularx}
    \caption{Algunos ejemplos observados en el análisis de error que carecen de contexto conversacional.}
    \label{tab:hateval_lack_of_context}
\end{table}


Algo que observamos también es que hay una fuerte cantidad de errores de etiquetado. La tabla \ref{tab:hateval_label_errors} muestra algunos de estos ejemplos. Si bien es difícil establecer una causa específica para cada uno de estos errores, es posible que sean causados por una combinación de: confusión entre lenguaje ofensivo y discurso de odio; un mismatch cultural entre los anotadores; falta de contexto; algunos tweets que fueron mal recolectados

\begin{table}
    \begin{tabularx}{\textwidth}{l X}
        Etiqueta      & Tweet \\
        \hline
        \mr{13}{HS} &Callate la puta boca pesada @Lauu\_tb \\
           & me cansada tía cállate ya la puta boca lo que te queda de vid \\
           & FLOR SALTANDO AHÍ LA RE PUTA MADRE TE LO MERECES PEDAZO DE FORRA TE LO MERECES, EN LA CARA DE TODOS HIJA ASÍ SE HACE \\
           & Callate! cerra el orto un ratooo! es mas, no te queres morir mejor? la puta que te pario  \\
           & @csdb530 @carvasar Claro, porque la culpa la tienen las niñas embarazadas y las víctimas de acoso sexual, violación… https://t.co/f16xSqT37G \\
           & @nlopezi\_ Imbécil tu puta prima! Gilipollas! Estúpida \\
           & Esta mujer es lo más la puta madre https://t.co/8SWPhKbXQe \\
           & \#Pendejos Don't call me gringo You fuckin beaner Stay on your side Of that goddamn river Don't call me gringo You beaner No me digas beaner Mr. puñetero Te sacaré un susto Por racista y culero No me llames frijolero Pinche gringo puñetero \\
           & Mónica que te calles la puta boca \#OTGala7 \\
           & @bastadetopos Callate que me camine dosmillonesquinientas cuadras para llegar el semi. La puta que los parió a todos. \\
           & @pablocasado\_ @imbrodamelilla @TeoGarciaEgea pablo no quieren concertinas , un muro de 12 metros o mas de altura de dos plantas de pisos ,se acabo los problemas @guardiacivil @policia @EMADmde @MonarquiaEspana nadie se quejaría de nada \\
           & \#OTGala7 Noemí JAJAJAJAJAJA ESTA MUJER ES LA PUTA AMA \\
           & No puedo creer lo que le hicieron a Cersei, yo se que es una hija de puta pero ni así se merecia lo que le hicieron… https://t.co/uftkVl5ene \\
        \hline
    \end{tabularx}
    \caption{Ejemplos mal etiquetados}
\end{table}

\section{Discusión}
\label{sec:04_discussion}

Respecto a la performance de los modelos presentados, los modelos basados en Transformers son notoriamente superiores a los demás modelos, en ambas tareas e idiomas. Particularmente, en inglés podemos observar que aquellos pre-entrenados sobre tweets como \bertweet{} tienen mejor performance que aquellos que son entrenados sobre Wikipedia como \bert{} o \roberta{}. El discurso de odio está muchas veces basado en la utilización de jergas e insultos raciales o misóginos, con lo cual es esperable el mejor desempeño de un modelo que tiene en sus datos de entrenamiento este tipo de expresiones.

Sobre la tarea más difícil de detección múltiple de discurso de odio (\subtaskb{}), propusimos varios enfoques: uno basado en predecir cada variable por separado y otro en predecir una variable que indique la combinación en cuestión. El modelo de predicción múltiple entrenado con la función de costo jerárquica obtuvo la mejor performance en términos de EMR, y la de multi-clasificación obtuvo la mejor en términos de Macro F1. En el caso de inglés, el modelo entrenado sobre cinco clases obtuvo la mejor performance en EMR y de nuevo el de multi-clasificación sobre Macro F1; sin embargo, esta queda por debajo de la mejor performance de la competencia (obtenida por el equipo MITRE \cite{gertner-etal-2019-mitre}) que usa una compleja combinación de técnicas, algunas de las cuales veremos en el Capítulo \ref{chap:07_domain_adaptation}. De estos dos casos, el modelo de multi-clasificación corre con la ventaja de calcular cada variable de manera independiente y tener un hiperparámetro menos.

Algo que merece cierta atención es que, lejos de empeorar el desempeño de nuestros modelos, agregar nuevas variables a predecir (además de la existencia de discurso de odio) pareciera mejorar levemente la performance de la detección de este fenómeno, a la vez que obteniendo salidas más ricas e interpretables. Más aún, observamos que otros trabajos \cite{gertner-etal-2019-mitre} utilizando el mismo conjunto de datos mejoraron la performance con una capa adaptadora que modela las dos variables latentes codificadas conjuntamente en la salida binaria: la misoginia y el racismo. Teniendo esto en cuenta, una pregunta a explorar es si contar con esta información (las características agredidas) puede mejorar la performance de los clasificadores o tener salidas más interpretables que sólo una etiqueta binaria.

Hacemos a continuación una disquisición no sólo sobre este trabajo y el dataset en el que se basa sino en líneas generales sobre los recursos y enfoques actuales en el área de detección de discurso de odio. Continuando con la idea del párrafo anterior, una limitación que puede verse es que la mayoría de los trabajos atacan una, dos, o a como mucho tres características protegidas. Por ejemplo, los trabajos de \citet{waseem2016hateful} y \citet{hateval2019semeval} sólo consideran racismo y sexismo, mientras que el de \citet{Davidson2017AutomatedHS} agrega homofobia a esta consideración. Sería deseable poder contar con un dataset que como mínimo cuente con estas tres características en conjunto a otras quizás menos utilizadas: odio de clase (a veces conocida como \emph{aporofobia}), discriminación por aspecto físico, por discapacidad, entre otras. A su vez, contar con la información de la característica atacada (algo que no ocurre en los datos utilizados en este capítulo) puede ser interesante para tener una mejor interpretabilidad de las salidas de nuestros algoritmos, y posiblemente para mejorar su rendimiento.

Un problema particular que se puede observar en los datos de \citet{hateval2019semeval} (pero que atraviesa a muchos otros) es el proceso de recolección de los datos: los tweets son recolectados mayormente a través de keywords. Como mencionamos en la Sección \ref{sec:hateval_dataset}, el proceso de recolección consta de varias estrategias combinadas; sin embargo (ver Apéndice \ref{app:04}), hay una altísima incidencia de algunas palabras (como \emph{sudaca} o \emph{inmigrante}) que sesgan fuertemente el dataset. Esto (entre otras cuestiones) puede ser un problema para los modelos que se entrenan sobre estos datos, haciendo que aprendan correlaciones espurias generadas por estas distribuciones. De todas formas, esto es una limitación general para estos tipos de aprendizaje sobre datos crudos y etiquetas, donde es difícil establecer e interpretar cómo un clasificador termina encontrando patrones para detectar el fenómeno medido.

La \textbf{anotación}, la etapa subsiguiente a la recolección de datos, pareciera presentar en este dataset algunos problemas. Hemos visto en la anterior sección una lista no extensiva de varios errores de etiquetado, aún cuando este dataset fue realizado con un complejo sistema combinando crowdsourcing,  etiquetado con expertos y desempate. Si bien es difícil trazar las razones detrás de estos problemas, observando las instancias incorrectamente etiquetadas puede hipotetizarse que esto es producto de un no entendimiento de las expresiones en los distintos dialectos del español y diferentes realidades socioculturales. \citet{waseem-2016-racist} mostró que las anotaciones ``amateurs'' (producto del uso de crowdsourcing) tienden a tener mayores instancias de Hate Speech (algo que daría la impresión de ocurrir aquí) y que datasets anotados por expertos mejoran la performance de los modelos. Este problema podría profundizarse dado que no queda claro si los anotadores son hablantes nativos de español, al no tener información detallada de quienes realizaron la tarea de etiquetado.

%%
%% Falta de contexto
%%

Un problema del dataset estudiado en este capítulo (pero que aplica a muchos otros también) es la \textbf{falta de contexto}: los mensajes carecen de información adicional sobre la noticia o el tema del que se está hablando. Cuando leemos un mensaje de un tweet, casi siempre lo leemos en el contexto de una noticia, o un trending topic. Muy rara vez leemos un mensaje en total aislamiento. De hecho, gran parte de los comentarios de este dataset tiene un contexto implícito: la noticia de conflicto migratorio en Ceuta. Otros comentarios, por otro lado, no se entienden bien ya que son respuestas a un tweet y que según el hilo de conversación pueden entenderse o no como discriminatorios.

Sobre esta falta de contexto, hay muchos mensajes aislados que pueden requerir información adicional para entender su significado. Por ejemplo, un comentario que dice ``hay que matarlos'' puede o no entenderse como discurso de odio. Si el objeto del mensaje se refiere a mosquitos, ese mensaje no es odioso; si, por otro lado, está hablando sobre migrantes chinos en el contexto del COVID-19, entonces ese mensaje es discriminatorio (y además llama a tomar una medida violenta). Podemos preguntarnos sobre este punto si el acceso a información contextual nos puede auxiliar en la detección de discurso de odio, siendo este contexto un hilo de conversación, una noticia a la que se refiere, o alguna otra forma de de información adicional.

Finalmente, otro problema que suele ocurrir relacionado al anterior es que no tenemos \textbf{información granular} de los datos anotados. Si bien algunos trabajos agregan información de la característica vulnerada, la mayoría simplemente agrega una etiqueta binaria sobre la existencia o no de discurso de odio (o bien algún nivel intermedio como si hay o no discurso ofensivo, como el caso de \citet{Davidson2017AutomatedHS}). Teniendo en cuenta lo observado en este capítulo, agregar información más detallada sobre cada caso puede ayudar a mejorar la detección del discurso de odio mediante una señal más rica a nuestros clasificadores sobre las diferentes fronteras de cada característica ofendida.


\section{Conclusiones}

En este capítulo hemos hecho un primer acercamiento a la tarea de detección de lenguaje discriminatorio, repasando de su definición desde un marco legal y desde el usado en la literatura de Procesamiento de Lenguaje Natural. Analizamos técnicas de clasificación del estado del arte sobre el dataset presentado en la shared task multilingual \cite{hateval2019semeval}. En base a este dataset, analizamos dos tareas: detección binaria de discurso de odio, y detección de múltiples variables (si es discurso de odio, si es dirigido, si es agresivo).

Para estas tareas, presentamos técnicas de clasificación basadas en modelos lineales que consumen distintos tipos de entrada como ser tweet embeddings y bolsas de caracteres; modelos basados en redes recurrentes que consumen embeddings contextualizados; y finalmente, utilizamos modelos de lenguaje pre-entrenados usando la arquitectura de Transformers. Para ambas, los modelos de Transformers obtuvieron el mejor desempeño, superando ampliamente a las demás técnicas.

En el caso de la tarea de detección múltiple, propusimos dos formas de atacar el problema: como clasificación múltiple (prediciendo simultáneamente las tres variables), y convirtiendo a un problema de clasificación simple sobre cinco clases posibles. Observamos, a su vez, que lejos de dañar la performance de la detección de discurso de odio, predecir más de una variable mejora la performance de nuestros clasificadores.

Analizando este dataset y algunos otros de la bibliografía, marcamos algunas oportunidades de mejora y observaciones en la detección de discurso de odio. En primer lugar, la posibilidad de agregar información contextual a los mensajes a analizar, sea sobre el tópico del que se está hablando o contexto conversacional previo. En segundo lugar, agregar \tbf{información granular} sobre las características ofendidas. Y finalmente, un punto no menor a la hora de la creación de recursos para un fenómeno tan complejo y social es indispensable tener muchos recaudos a la hora de la anotación --algo que ya ha sido observado en otros trabajos-- teniendo particular cuidado sobre el trasfondo sociocultural de quienes tomen esa tarea.

En los siguientes capítulos, exploraremos algunas de estas oportunidades de mejora. Particularmente, nos centraremos en la incorporación de contexto en la detección de discurso discriminatorio, construyendo un conjunto de datos que incorpore esta información a los mensajes anotados, y explorando cómo mejorar los algoritmos del estado del arte que aprovechen esa información.
