\section{Selección de datos a anotar}


Un problema que se nos presenta antes de comenzar el etiquetado es el de seleccionar los artículos que vamos a etiquetar. Una primera posibilidad para hacer esto es realizar una selección aleatoria de artículos y comentarios; sin embargo, los comentarios discriminatorios no se distribuyen de manera uniforme entre los artículos, sino que se concentran sobre algunos temas. Es mucho más probable encontrar comentarios de índole discriminatoria en notas que tengan temas cercanos a alguna de las características protegidas; por ejemplo, es esperable que encontremos contenido odioso en notas sobre China y el Coronavirus o sobre una chica transgénero antes que en un artículo de fútbol o economía. Si bien una selección aleatoria preservaría una tasa de incidencia mucho más cercana a la observada en el universo de comentarios, es más importante poder obtener una mayor cantidad de observaciones que reflejen el fenómeno estudiado.

Teniendo esto en cuenta, evaluamos varias alternativas. La primera fue intentar seleccionar aquellos artículos que consideremos que puedan tener contenido potencialmente discriminatorio.

Una posibilidad para esto sería usar algunas palabras ``semilla'' para seleccionar artículos interesantes. Otra sería buscar directamente comentarios que contengan algunos insultos comunes o expresiones peyorativas hacia nuestros grupos protegidos. Después de algunos experimentos, decidimos utilizar el muestreo basado en comentarios.

\subsection{Selección en base a artículos}

En primer lugar, consideramos la posibilidad de hacer una selección en base al contenido de los artículos. Luego de realizar algunos experimentos usando LDA \cite{blei2003latent} para buscar tópicos posibles de las notas, decidimos realizar una selección un poco más controlada y determinística en base a la utilización de palabras clave. Es decir, seleccionaremos artículos en base a la aparición o no de ciertas ``semillas''

Para ello, indexamos todos nuestros artículos en MongoDB \footnote{\url{https://www.mongodb.com/}}, una base de datos no relacional y desestructurada. MongoDB permite la utilización de índices en base a texto, y realizar búsquedas en base a textos, palabras, e inflexiones. Cada artículo fue indexado en base al contenido de su cuerpo (es decir, el texto en sí del artículo).

La tabla \ref{tab:palabras_articulos} muestra el conjunto utilizado para recolectar artículos. Como vemos, hay diversas palabras que recogen distintas temáticas de posibles tópicos ``calientes'', algunos muy locales respecto a eventos concretos durante la pandemia. Si algún artículo contiene una de las frases mencionadas, se selecciona el artículo para ser etiquetado.

\begin{table}[]
    \centering
    \begin{tabular}{l | l | l | l}
    China        &  piqueteros              &  mamá                & empleadas domésticas  \\
    Cuba         &  villas                  &  de género           & la modelo             \\
    cubano       &  la villa                &  aborto              & la periodista         \\
    bolivia      &  movimientos sociales    &  actriz              & la cantante           \\
    paraguayo    &  organizaciones sociales &  actrices            & travesti              \\
    judío        &  tomas de tierras        &  feminista           & trans                 \\
    camionero    &  toma de tierras         &  femicidio           & gay                   \\
    ladrón       &  sindicatos              &  enfermera           & homosexual            \\
    represión    &  Guernica                &  madre               & de la V               \\
    criminal     &  mapuches                &  personal doméstico  & Ofelia                \\
    \end{tabular}
    \caption{Palabras utilizadas para la selección de artículos}
    \label{tab:palabras_articulos}
\end{table}

\subsection{Selección en base a comentarios}
\label{subsec:seleccion_comentarios}


\begin{table*}[t!]
    \centering
    \begin{tabular}{l|l|l|l|l|l|l}
    bija          & urraca     & viejo puto    & trolo      & peruano  & matarlos         & negra      \\
    prostituta    & tucán      & trabuco       & sodomita   & peruca   & una bomba        & negro de   \\
    feministas    & putita     & travesti      & chinos de  & judío    & vayan a laburar  & negros     \\
    feminazis     & reventada  & trava         & bolita     & sionista & vayan a trabajar & bala       \\
    aborteras     & marica     & degenerado    & paraguayo  & villeros & gorda            & uno menos  \\
    \end{tabular}
    \caption{Palabras utilizadas para recolectar comentarios}
    \label{tab:palabras_comentarios}
\end{table*}

Otra posibilidad evaluada fue la de observar los comentarios de los artículos en lugar del contenido del artículo, y seleccionarlos en base a esto. En este punto, la idea es únicamente seleccionar los artículos y no los comentarios; estos últimos son sólo usados como ``pistas'' para ver comentarios con posible contenido discriminatorio, y como tal identificar a ese artículo como un posible generador de este tipo de contenido.

La idea es similar a la de la selección con artículos, sólo que aplicada a comentarios: buscamos comentarios que contengan alguna de las palabras semilla listadas en la Tabla \ref{tab:palabras_comentarios}. Estas palabras fueron recolectadas a base de experimentación y observación de los datos, y tratan de contener diversas expresiones de contenido mayormente discriminatorio.

Una idea también considerada fue la de utilizar un clasificador entrenado sobre otro dataset (por ejemplo, el de \citet{hateval2019semeval}) y con eso marcar comentarios posiblemente discriminatorios. Sin embargo, muy probablemente detectaríamos sólo comentarios para las categorías/características etiquetadas en esos datasets e ignorarían las que agregamos en nuestro trabajo; por ejemplo, la mayoría de los datasets no contienen comentarios anotados contra la comunidad LGBTI.

El procedimiento de selección consta de, dado un artículo, marcar sus comentarios que contengan una o más de las expresiones listadas. Si el artículo tiene tres o más comentarios marcados, entonces seleccionamos el artículo; caso contrario, es descartado.

\todo{Agregar comparación con métodos de recolección en base a esto de 'seed' words}

Vale remarcar que este proceso de selección es para los \emph{artículos}, no para los comentarios.


Luego de algunos análisis experimentales y observacionales de las dos posibles metodologías, decidimos utilizar el muestreo de artículos en base a comentarios. En base a un análisis subjetivo, los artículos seleccionados parecían tener mayor incidencia de mensajes odiosos y eso nos decantó hacia esa opción.

\subsection{Selección en base a clasificadores}

Una posibilidad que tuvimos en cuenta fue la de, en vez de ``marcar'' comentarios usando las palabras clave mencionadas en la sección \ref{subsec:seleccion_comentarios}, usar un clasificador pre-entrenado que nos señale posibles comentarios odiosos y luego usar eso para seleccionar artículos candidatos a etiquetar.

Para ello, aplicamos un clasificador basado en BETO \citet{canete2020spanish} (ver sección XXX) sobre los comentarios de los artículos. Una evaluación subjetiva de esto nos dio pobres resultados, tanto porque no captaba algunas agresiones discriminatorias (de características no incluídas en el dataset de \citet{hateval2019semeval}) como muchos falsos positivos o errores debido al cambio de dominio (temático y también dialectal).

Si bien descartamos este método, puede ser de relevancia usar algún método que no esté basado en palabras semillas.

