\label{chap:dataset_creation}

En este capítulo describiremos la construcción de un dataset contextualizado de discurso de odio. Describiremos en detalle el proceso de recolección, selección y anotación de datos.

Por lo marcado en anteriores secciones, consideramos interesante el problema de analizar el impacto del contexto en la detección de lenguaje discriminatorio. Para citar un ejemplo de por qué es necesario, el mensaje ``sos un hombre'' en solitario puede parecer inofensivo; ahora, si ese mismo mensaje está dirigido hacia una mujer trans, su contenido es claramente discriminatorio. Para analizar esto, nos abocamos a la decisión de crear un dataset que no sólo contenga un mensaje/comentario, sino que provea un contexto en el cual se da este mensaje. Un ámbito natural para esta tarea son las notas periodísticas, donde disponemos de una nota y comentarios realizados sobre ésta.

Muchos sitios de noticias disponen de sistemas embebidos de comentarios, pero vista la dificultad para la recolección a la vez que los limitados datos provistos por estos sitios nos llevaron a buscar otro medio: Twitter. Twitter provee una sencilla API para descargar datos, a la vez que tiene términos y condiciones amigables para poder publicar estos datos. Así mismo, esta red social opera de una manera similar a un foro de comentarios de un sitio de noticias. Este tipo de datos (comentarios sobre artículos periodísticos) tiene una naturaleza particular, ya que las agresiones discriminatorias son usualmente a personajes públicos o colectivos de personas, y se dan de manera indirecta (a través del comentario en la noticia) y no directa (es decir, como respuesta al usuario de Twitter ofendido).

El trabajo realizado en este capítulo tuvo lugar en el contexto de un Proyecto Interdisciplinario de la UBA\footnote{\url{https://cyt.rec.uba.ar/vinculacion-transferencia/piuba/}} junto a sociólogos, abogados, lingüistas, y computólogos. Particularmente, el trabajo de la construcción del manual de etiquetado fue discutido en conjunto, contemplando varias perspectivas a la hora de armar una definición propia (algunas de estas ya fueron vertidas en la discusión en la sección \ref{sec:hate_speech_definitions}). Teniendo en cuenta que un alto porcentaje de trabajos del área de detección de discurso de odio (y de manera más importante, en la construcción de sus recursos) mediante técnicas de NLP no abordan una mirada interdisciplinaria, es un aspecto a remarcar de lo realizado en la construcción de este dataset.

\section{Trabajos previos}
\label{sec:dataset_previous}

Pocos trabajos del área de detección de lenguaje abusivo o discurso de odio incorporan algún tipo de contexto a los comentarios del usuario para estas tareas. En esta sección haremos un repaso de los trabajos que han abordado esto de alguna manera. \citet{gao-huang-2017-detecting} construyó un dataset de lenguaje discriminatorio sobre 1518 comentarios del sitio de Fox News. A los anotadores les fue presentado tanto el comentario como la noticia a la hora de realizar el etiquetado. Sobre este dataset, los autores efectuaron experimentos de clasificación usando modelos lineales (regresiones logísticas) y modelos neuronales. En estos experimentos, observaron que un clasificador (tanto lineal como neuronal) mejora su performance al consumir el título de la noticia, dando indicios de que se puede aprovechar el contexto para mejorar la detección de este fenómeno. Sin embargo, como marca \citet{pavlopoulos2020toxicity} este trabajo cuenta con algunos problemas: en primer lugar, el tamaño del dataset es pequeño, y está extraído de sólo 10 noticias, lo cual limita fuertemente los posibles contextos. A su vez, la anotación fue realizada mayormente por una única persona, lo cual hace poco confiables las etiquetas obtenidas. Luego, algunos detalles menores debieran ser analizados con mayor detalle, como por ejemplo la utilización de los nombres de usuarios como features predictivas.

\citet{mubarak-etal-2017-abusive} construyó un dataset en árabe sobre comentarios con contenido abusivo del portal Al Jazeera. Sin embargo, este daaset tiene un problema: los comentarios son sólo presentados a los anotadores sobre noticias, ignorando todo el thread de la conversación. Esto hace que el contexto sea parcial.


Paralelamente a nuestro trabajo, \citet{pavlopoulos2020toxicity} analiza el impacto de agregar contexto a la tarea de detección de toxicidad. En particular, plantea dos preguntas

\begin{itemize}
    \item ¿Qué tanto afecta el contexto a la toxicidad percibida por humanos en conversaciones online?
    \item ¿Puede el contexto ayudar a mejorar la performance de clasificadores de toxicidad en comentarios?
\end{itemize}

Para responder estas preguntas, los autores construyeron dos datasets en base a Wikipedia Talk Pages\cite{hua-etal-2018-wikiconv}, un dataset de discusiones del sitio de Wikipedia. En primer lugar, armaron un dataset de 250 comentarios anotados por dos grupos disjuntos de anotadores: uno de los grupos anotó los comentarios de manera contextualizada, viendo tanto el comentario en cuestión como el título de la discusión; el otro grupo sólo vio el comentario a anotar sin contexto alguno. En dicho experimento observaron que los anotadores contextualizados percibieron 6.4\% de comentarios tóxicos versus un 4.4\% de quienes anotaron sin contexto, una diferencia significativa aplicando un test Mann-Whitney. Desagregando estos resultados, observaron que 13 de los 250 comentarios (5.2\%) tuvieron diferencias de anotación entre los dos grupos, con 9 (3.6\%) comentarios donde aumentó la toxicidad percibida y 4 comentarios donde bajó la toxicidad al ser agregado el contexto.

Para responder la segunda pregunta, anotaron un dataset de 20k comentarios, 10k anotados por un grupo que etiquetó viendo el contexto y otros 10k que no lo vio. Entre todos los comentarios del dataset original de Wikipedia Talk Pages, eligieron aquellos con profundidad entre 2 (respuestas directas) a 5, y con entre 10 y 400 caracteres de largo. Luego, entrenaron varios clasificadores usando este dataset y allí pudieron observar que el contexto no pareciera mejorar la performance. En el próximo capítulo nos extenderemos sobre las técnicas utilizadas por este trabajo.

\citet{xenos-2021-context} continúa el trabajo de \citet{pavlopoulos2020toxicity} desagregando el resultado de la segunda pregunta. Puntualmente, y observando que sólo un porcentaje pequeño de los comentarios parecen ser incididos por el contexto en el trabajo previo, construyen una nueva tarea: estimación de sensibilidad al contexto. Para ello, toman el dataset de Civil Comments\cite{borkan2019civil}, y reanotan un subconjunto de este dataset usando información de contexto a través de crowdsourcing. Las etiquetas de este dataset son de toxicidad en un estilo similar a una regresión ordinal, entendiendo las categorías no tóxico, incierto, tóxico, y muy tóxico. Ahora, teniendo las anotaciones originales del dataset (que fueron hechas sin contexto) y las nuevas anotaciones, pueden definir para cada comentario una sensibilidad al contexto, dada por

\begin{equation}
    \delta(p) = s^{oc}(p) - s^{ic}(p)
\end{equation}

donde $s^{oc}$ es la fracción de anotadores sin contexto que marcaron toxicidad, y $s^{ic}$ los que no tienen contexto.

\citet{sheth2021defining}, en un trabajo muy reciente, señala algunas oportunidades y desafíos  para incorporar fuentes de información más ricas a la tarea de detección de toxicidad. Por ejemplo, incorporar información como el background socio-cultural de los interactores puede ayudar a distinguir algunos tipos de reapropiación de términos potencialmente catalogados como tóxicos. Así mismo, el historial de interacción entre los usuarios puede ayudar a distinguir interacciones abusivas de charlas amistosas entre amigos que usan vocabulario potencialmente tóxico. Finalmente, se promueve el uso de contenido externo para acercarse lo más posible al conocimiento humano a través de conocimiento del contenido, el individuo (atacado) y la comunidad. Para ello, se promueve el uso de bases de conocimiento y knowledge-infusion learning \cite{gaur2020infusion} para combinar el cómputo neuronal y simbólico.



\citet{wiegand2021implicitly} menciona formas implícitas de abuso, mucho más complejas que las basadas solamente en palabras ofensivas. Por ejemplo, deshumanizaciones (``los judíos son una plaga que merece ser eliminada''), llamadas a la acción (``hay que tirar una bomba en ese país''), acusaciones (``los chinos inventaron el coronavirus''), entre otros tipos sutiles de comportamiento tóxico. Así mismo, menciona que la mayoría de los datasets no consiguen capturar estos fenómenos debido a la forma de recolección usualmente basada en keywords.

\section{Esquema del dataset}


%%
%%
%% Link a Draw
%% https://docs.google.com/drawings/d/1IcBITgNJN-tehmvnZqcSF9cUuWIpNKJg6yHI5yjNF9c/edit
%%
%%

\begin{figure}[ht]
    \centering
    \includegraphics[width=\textwidth]{img/idea_dataset.pdf}
    \caption{Muestra de la recolección de datos}
    \label{fig:idea_dataset}
\end{figure}

Para construir un dataset contextualizado barajamos varias opciones. Como vimos en otros datasets, se puede entender el contexto de varias maneras: un contexto ``temático'', donde sabemos que cierto comentario habla sobre un tema en particular; y un contexto conversacional, donde tenemos una secuencia de comentarios (un hilo o thread) y podemos extraer un comentario padre para cada uno salvo el raíz. La primer opción es la explorada por \cite{gao-huang-2017-detecting,mubarak-etal-2017-abusive}, donde construyen un dataset de comentarios de Fox News y Al-Jazeera respectivamente. El contexto conversacional, como hemos relatado anteriormente, es explorado en \citet{pavlopoulos2020toxicity,xenos-2021-context}; sin embargo, como es marcado en el primer trabajo, la recolección de datos es no trivial, aún en un caso más amplio como el lenguaje abusivo, ya que la incidencia es relativamente baja. Puede esperarse que en el contexto de lenguaje odioso se dificulte aún más esto.

Para analizar el contexto, decidimos entonces usar la primera opción: comentarios sobre notas periodísticas. No vamos a considerar un hilo de respuestas, sino simplemente aquellos comentarios que sean directos sobre la nota. En ese punto, el dataset que queremos construir sería similar al de \cite{gao-huang-2017-detecting}. Una diferencia respecto a este dataset sería la de incorporar dos modos de contexto: uno corto, donde sólo tengamos el título de la noticia; y uno largo, donde tengamos el texto completo de la noticia.

El dataset construido será sobre comentarios realizados en idioma español, más precisamente en la variedad dialectal del Río de la Plata. Como dice la ``Regla de Bender''\cite{bender2011achieving}

\begin{quote}
    Do state the name of the language that is being studied, even if it's English. Acknowledging that we are working on a particular language foregrounds the possibility that the techniques may in fact be language specific. Conversely, neglecting to state that the particular data used were in, say, English, gives [a] false veneer of language-independence to the work.
\end{quote}

Este punto es importante ya que, a pesar de ser el segundo idioma en hablantes nativos (por delante del inglés), los recursos suelen ser escasos y siempre a la rastra y reproducción de resultados en inglés. \todo{quizás esto lo mandaríamos a otro lado}

Algo no menor a la hora de considerar la construcción del dataset es la posibilidad de publicar los datos. Por citar un ejemplo, el dataset de \citet{gao-huang-2017-detecting}, si bien tiene sus datos de acceso público \footnote{\url{https://github.com/sjtuprog/fox-news-comments}}, no queda claro que los términos y condiciones de la fuente de donde se extrajeron permita esto. Más aún, si hubiésemos querido extraerlo de múltiples fuentes (por ejemplo, varios diarios), deberíamos chequear y/o acceder a permisos para cada sitio, a la vez que tendríamos el problema de tener fuentes diversas de los datos (diferentes longitudes, metadatos distintos, entre otras).

Para evitar muchos de estos problemas, y reutilizar muchas cuestiones con las que venimos trabajando en esta tesis, decidimos trabajar sobre comentarios hechos por usuarios en Twitter. Concretamente, sobre respuestas de comentarios de usuarios a posteos hechos por cuentas de medios. De alguna manera, esto emularía un foro de comentarios de medios, tendríamos un formato único para comentarios mientras tenemos diferentes ``audiencias''. La Figura \ref{fig:idea_dataset} ilustra esta idea. A su vez, los términos y condiciones de Twitter nos permiten publicar los datos \todo{Agregar algún link a esto}. Las notas periodísticas las descargaremos pero debido a problemas de copyright no serán publicados.



\subsection{Proceso de construcción}

Dividiremos la construcción del dataset en tres etapas:

\begin{enumerate}
    \item Recolección: Proceso de recolección de datos de Twitter y de los artículos periodísticos
    \item Selección: Dado el conjunto de artículos y comentarios recolectados, tomar una muestra de artículos y comentarios a etiquetar
    \item Anotación: Proceso de etiquetado de los artículos seleccionados
\end{enumerate}

Si bien en muchos casos las dos primeras etapas suelen ser la misma o bien la selección se limita a una muestra aleatoria de la recolección, este procedimiento sería muy ineficiente en el caso de discurso de odio. Esto se debe a que en nuestro dominio de comentarios periodísticos y discurso de odio, encontramos este tipo de discurso distribuido de manera muy poco uniforme, usualmente concentrada alrededor de ciertos tópicos. Para construir un dataset con una proporción no marginal del fenómeno estudiado, estudiamos algunas posibilidades para seleccionar los artículos y sus respectivos comentarios.

En algunos trabajos previos (como por ejemplo \citet{waseem2016hateful,hateval2019semeval}) la recolección y selección constan conjuntamente de usar ciertos keywords y, o bien recolectar tweets que usen esas palabras, o bien sirven para preseleccionar usuarios de los cuales luego extraer tweets para ser etiquetados.

En nuestro caso, la selección de artículos y comentarios presenta cierta novedad y complejidad, con lo cual separamos este procedimiento para explicarlo detalladamente en las siguientes secciones.

\section{Definición de discurso de odio}
\label{sec:our_hate_speech_definition}
\begin{table}[t]
    \centering
    \begin{tabularx}{\textwidth}{l X}
        Característica & Descripción \\
        \hline
        MUJER        & Misoginia, agresiones basadas en ser mujer  \\
        LGBTI        & Homofobia, transfobia, y ofensas a la comunidad LGBTI \\
        RACISMO      & Racismo, Xenofobia, Judeofobia, etc \\
        POBREZA      & Basado en su condición de clase \\
        POLITICA     & En base a la filiación política del agredido \\
        ASPECTO      & Gordofobia, gerontofobia \\
        CRIMINAL     & Criminales, presos, y personas en conflicto con la ley \\
        DISCAPACIDAD & Discapacidades y adictos a sustancias

    \end{tabularx}
    \label{tab:caracteristicas_protegidas}
    \caption{Características protegidas consideradas en este trabajo}
\end{table}


Teniendo en cuenta la discusión realizada en la sección \ref{sec:hate_speech_definitions} realizamos nuestra propia definición de discurso de odio. Entendemos que hay discurso de odio en un texto social si éste contiene declaraciones de carácter intenso e irracional de rechazo, enemistad y aborrecimiento contra un individuo o contra un grupo, siendo estos objetivos de estas expresiones por poseer (o aparentar poseer) una característica protegida. Esta expresión puede manifestarse de manera explícita como insultos directos, celebraciones de crímenes, incitaciones a tomar medidas contra el individuo o grupo, o también expresiones más veladas. Siempre, considerando, que no es necesario solamente un insulto o una agresión: es necesario hacer una apelación explícita o implícita a al menos una característica protegida.

A diferencia de otros trabajos, nuestra definición comprende varias características, incluso algunas que están en la frontera de ser ``protegidas''. Mientras en otros trabajos se centran mayormente en racismo y misoginia, aquí agregaremos homofobia y transfobia, odio de clase (``aporofobia''), por su aspecto físico, y otras. En particular, hay dos características no convencionales que tuvimos en cuenta. En primer lugar, el discurso de odio ``político'', que de acuerdo a XXX \todo{citation needed}, es difícil considerar como protegida ya que puede dar lugar a censuras. Por otro lado, también consideramos el discurso de odio contra criminales, presos, y otras personas en situación de conflicto con la ley. Si bien este punto ni siquiera es considerado como una característica protegida en ninguno de los trabajos mencionados en la sección \ref{sec:hate_speech_definitions}, al haber tanto contenido que incita a la violencia contra criminales en las noticias de policiales, agregamos esta característica. Así mismo, esta característica puede ser de utilidad ya que nos interesa recoger incitaciones a la violencia, y este rubro es prolífico en ello en las redes.

Tenemos entonces 8 características que agrupan tipos de discurso de odio: contra las mujeres; racismo y xenofobia; contra la comunidad LGBTI; odio de clase; gordofobia, gerontofobia y demás odio por aspecto; por su ideología política; y finalmente contra discapacitados y adictos. Las características en cuestión son listadas en la tabla \ref{tab:caracteristicas_protegidas} junto a acrónimos que usaremos en el resto del capítulo.







\section{Recolección de datos}

\begin{table*}[t]
    \centering
    \large
    \begin{tabular}{ l l r }
        Nombre     &  username          & \#Followers \\
        \hline
        La Nación  &  @LANACION         & 3.6M            \\
        Clarín     &  @clarincom        & 3.2M        \\
        Infobae    &  @infobae          & 3.0M   \\
        Perfil     &  @perfilcom        & 0.81M    \\
        Crónica    &  @cronica          & 0.80M     \\
        \hline
    \end{tabular}
    \caption{Cuentas de medios utilizadas para la recolección de datos, junto a sus nombres de usuarios y la cantidad de seguidores en Twitter (al momento de la recolección)}
    \label{tab:medios_analizados}
\end{table*}


En esta sección detallaremos el proceso de recolección de datos, cuya salida es un conjunto de artículos mencionados en Twitter y sus comentarios respectivos realizados por usuarios. Describiremos a continuación las decisiones realizadas respecto a las fuentes y a otros detalles técnicos.

En primer lugar, limitamos nuestra recolección de datos a cuentas de medios de la República Argentina y, puntualmente, nos centramos en diarios con comunidad mayormente rioplatense. Esto lo realizamos teniendo en mente que los anotadores serían nativos de esta variedad dialectal ya que, como mencionamos anteriormente, el discurso de odio contra mujeres, grupos nacionales y otros depende fuertemente de la jerga y de las variaciones dialectales de cada lugar. Esta elección, se debe además a que, habiendo buscando en otros medios de Argentina (como por ejemplo ``La voz del Interior'', diario dirigido mayormente a un público fuera de la Metrópolis de Buenos Aires) observamos que la interacción en Twitter de estos medios es muy baja, con muy pocos usuarios comentando sus notas. Centrándonos en diarios que generen interacción, seleccionamos medios periodísticos de gran llegada y tradicionales, los cuales listamos en la Tabla \ref{tab:medios_analizados}.

Si bien recolectamos notas de otros medios, no los consideraremos a partir de ahora, y los dejamos para análisis posteriores. De los cinco medios elegidos, todos son medios formales y con varios años en el medio, siendo cuatro de ellos con soporte escrito y uno sólo (Infobae) enteramente digital. Consideramos la posibilidad de elegir medios no tradicionales y más orientados a grupos de la ``derecha alternativa'', dada su alta incidencia de contenido de odio. Sin embargo, finalmente tomamos la decisión de descartarlos de la etapa de anotación.


\subsection{Método de recolección}



La API de Twitter, en su versión gratuita, nos brinda dos modos de recolectar tweets de su plataforma\footnote{Usamos la versión 1.1 de la API. La versión 2.0 parece facilitar la recopilación de conversaciones. Recomendamos investigar mejor esta versión actualizada para esquivar muchas de las dificultades técnicas que incursionamos para lo descripto en esta sección}:

\begin{enumerate}
    \item Search API: permite buscar tweets en base a términos, de hasta 15 días atrás sobre una pequeña muestra, recreando lo que vemos en la UI de Twitter
    \item Stream API: permite buscar tweets en tiempo real sobre una muestra de cerca del 1\% de todos los tweets de la red social
\end{enumerate}

La API Stream (también conocida como Spritzer), mientras por un lado limita temporalmente la recolección de datos, por el otro nos brinda la posibilidad de recolectar una mayor cantidad de información en tiempo real. Más aún, dada la naturaleza de nuestros datos (discurso de odio), se corre el riesgo de que con el tiempo sean moderados e inaccesibles para cualquier búsqueda con la API Search.

Por lo explicado, usamos la API de Twitter Stream mencionando cualquiera de estas cuentas. Si estamos entonces recolectando tweets sobre \verb|@medio|, el proceso de recolección nos da:

\begin{enumerate}
    \item Tweets de \verb|@medio|
    \item Respuestas a los tweets de \verb|@medio|
    \item Tweets de terceros que mencionan a \verb|@medio|
    \item Retweets (RT) de tweets de \verb|@medio|
    \item Citas de tweets de \verb|@medio|
\end{enumerate}

Los RTs y tweets que arroben a \verb|@medio| carecen de interés para nuestro estudio, con lo cual los descartamos. Por otro lado, también descartamos las citas, aunque podrían entenderse como ``respuestas'' a los tweets originales. Nos quedamos con tweets de \verb|@medio| y las respuestas a estos. Si bien la API nos da estos tweets desestructuradamente, reconstruimos el árbol de la discusión mediante el campo \verb|in_reply_to_status_id|\footnote{Ver la documentación y la referencia al campo en \url{https://developer.twitter.com/en/docs/twitter-api/v1/data-dictionary/object-model/tweet}}.

Algo importante a remarcar es que para el propósito de este trabajo, solo estamos interesados en el primer nivel de respuestas al tweet original, y no incorporaremos hilos de respuestas. Trabajo futuro debería explorar este nivel adicional de complejidad incorporando contexto conversacional adicional.

Accidentalmente, la recolección de datos se dio al mismo tiempo del estallido de la pandemia del COVID-19. Por ese motivo, y dadas las implicancias de la pandemia sobre el discurso discriminatorio en las redes sociales, se volcó el foco hacia artículos relacionados con el coronavirus. Para ello seleccionamos artículos buscando una cantidad de palabras en su cuerpo, por lo que seleccionamos específicamente artículos relacionados con COVID-19. Utilizamos las siguientes palabras: coronavirus, encierro, síntomas, covid, fase, fiebre, cuarentena, infectados, distanciamiento, normalidad,  Wuhan, aislamiento.

Por último, nos quedamos con aquellos tweets de los medios periodísticos que tuvieran un link a un artículo. Para ello, utilizamos la librería \emph{newspaper3k}\footnote{\url{https://newspaper.readthedocs.io/en/latest/}}, que nos permite acceder a la información relacionada a los artículos en cuestión, en particular siendo lo que más nos interesa el cuerpo del artículo. Esto vamos a utilizarlo posteriormente como el contexto ``largo'' para los comentarios. Aquellos tweets de medios periodísticos que no contengan un link a un artículo fueron descartados de las siguientes etapas.

\subsection{Datos recolectados}

\begin{table}[t]
    \centering
    \begin{tabular}{l|c|c}
    Medio      & \#Artículos & \#Comentarios \\
    \hline
    @infobae   &  45,652   &  822,462 \\
    @clarincom &  29,050   &  672,401 \\
    @perfilcom &  8,764    &  61,203  \\
    @LANACION  &  16,040   &  506,091 \\
    @cronica   &  17,250   &  70,872 \\
    \hline
    Total      & 116,756  & 2,133,029 \\
    \end{tabular}
    \caption{Artículos recoletados por medio}
    \label{tab:articulos_recoletados_por_medio}
\end{table}


La tabla \ref{tab:articulos_recoletados_por_medio} contiene los números de los artículos recolectados por cada medio, luego de aplicado el filtro de palabras mencionado en la anterior sección. Si bien recolectamos más artículos de otros medios, no son enumerados. Infobae es el medio que más producción de artículos genera, y también será finalmente sobre el que más comentarios etiquetemos.

En el apéndice \ref{app:distribucion_datos} mostramos la distribución temporal de los datos. Si bien tenemos un pequeño gap en los datos por un problema técnico en la recolección, tenemos datos desde Marzo de 2020 hasta Febrero de 2021.

En siguientes secciones realizaremos un filtrado de la mayoría de estos artículos previamente a la anotación, pero este conjunto de datos no filtrado será utilizado para efectuar ajustes de dominio, y es liberado como se recomienda en \citet{gururangan-etal-2020-dont}. Hablaremos más sobre esto en los capítulos \ref{chap:06_contextualized_hate_speech} y \ref{chap:07_domain_adaptation}.


\section{Selección de datos a anotar}


Un problema que se presenta antes de comenzar el etiquetado es el de seleccionar los artículos que vamos a etiquetar, teniendo en consideración la gran cantidad de datos recolectados y los recursos disponibles. Una primera posibilidad para esto es realizar una selección aleatoria de artículos y comentarios. Sin embargo, los comentarios discriminatorios no se distribuyen de manera uniforme entre los artículos sino que se concentran sobre algunos temas que generan este tipo de contenido. Es mucho más probable encontrar comentarios de índole discriminatoria en notas que tengan temas cercanos a alguna de las características protegidas: por ejemplo, es esperable encontrar contenido discriminatorio en notas sobre China y el Coronavirus o sobre una chica transgénero antes que en un artículo de fútbol o economía. Si bien una selección aleatoria preservaría una tasa de incidencia mucho más cercana a la observada en el universo de comentarios, es más importante poder obtener una mayor cantidad de observaciones que reflejen el fenómeno estudiado.

Teniendo esto en cuenta, evaluamos varias alternativas para realizar la selección de artículos. La primera fue intentar seleccionar aquellos artículos que consideramos como candidatos a fomentar contenido discriminatorio. Una posibilidad para esto sería usar algunas palabras semilla para seleccionar artículos interesantes en base a ciertos temas que consideramos relevantes.

Otra posibilidad evaluada fue la de buscar directamente comentarios que marquen que ese artículo suscita contenido discriminatorio. Para ello, podemos listar algunos insultos comunes o expresiones peyorativas hacia los grupos protegidos considerados. Es necesario remarcar que esto lo hacemos para seleccionar \tbf{artículos} y no los comentarios que contengan esos insultos; hacer esto último nos genera una muestra muy distorsionada y tendiente a encontrar el fenómeno más explícito de la discriminación (el insulto racista, homofóbico, etc.). Esta estrategia guarda relación con la descripta por \citet{hateval2019semeval} para seleccionar usuarios generadores de contenido discriminatorio.

Describimos a continuación las alternativas analizadas para seleccionar los artículos y sus respectivos comentarios.


\subsection{Selección en base a artículos}

\begin{table}[b!]
    \centering
    \small
    \begin{tabular}{p{0.21\textwidth}  p{0.26\textwidth} p{0.23\textwidth} p{0.19\textwidth}}
    \hline
    China        &  piqueteros              &  mamá                & domésticas            \\
    Cuba         &  villas                  &  de género           & la modelo             \\
    cubano       &  la villa                &  aborto              & la periodista         \\
    bolivia      &  movimientos sociales    &  actriz              & la cantante           \\
    paraguayo    &  organizaciones sociales &  actrices            & travesti              \\
    judío        &  tomas de tierras        &  feminista           & trans                 \\
    camionero    &  toma de tierras         &  femicidio           & gay                   \\
    ladrón       &  sindicatos              &  enfermera           & homosexual            \\
    represión    &  Guernica                &  madre               & de la V               \\
    criminal     &  mapuches                &  Ofelia              &                       \\
    \hline
    \end{tabular}
    \caption{Palabras semilla utilizadas para la selección de artículos. Cada palabra se busca sobre el cuerpo del artículo candidato a ser etiquetado}
    \label{tab:palabras_articulos}
\end{table}

En primer lugar, consideramos la posibilidad de hacer una selección en base al contenido de los artículos. Luego de hacer un análisis exploratorio de los datos usando LDA \cite{blei2003latent} para buscar tópicos posibles de las notas, decidimos realizar una selección controlada y determinística en base a la utilización de palabras y expresiones clave. Estas expresiones las recolectamos de manera subjetiva y en base a la observación de los tópicos y de nuestra percepción de la generación de discurso discriminatorio en los comentarios de los usuarios.

La Tabla \ref{tab:palabras_articulos} muestra el conjunto de expresiones utilizado para recolectar artículos. Como vemos, hay diversas palabras que recogen temáticas de posibles tópicos generadores de contenido discriminatorio, algunos muy locales respecto a eventos concretos durante la pandemia. Si algún artículo contiene una de las expresiones mencionadas, es seleccionado para ser etiquetado.

Para realizar esta búsqueda de términos en el cuerpo de los artículos, indexamos los textos en \emph{MongoDB}\footnote{\url{https://www.mongodb.com/}}, una base de datos no relacional. Este motor de bases de datos permite la utilización de índices en base a texto, permitiendo realizar búsquedas en base a expresiones, palabras, e inflexiones.



\subsection{Selección en base a comentarios}
\label{subsec:seleccion_comentarios}


\begin{table*}[h]
    \centering
    \small
    \begin{tabular}{l l l l l l l}
    \hline
    bija          & urraca     & viejo puto    & trolo      & peruano  & matarlos         & negra      \\
    prostituta    & tucán      & trabuco       & sodomita   & peruca   & una bomba        & negro de   \\
    feministas    & putita     & travesti      & chinos de  & judío    & vayan a laburar  & negros     \\
    feminazis     & reventada  & trava         & bolita     & sionista & vayan a trabajar & bala       \\
    aborteras     & marica     & degenerado    & paraguayo  & villeros & gorda            & uno menos  \\
    \hline
    \end{tabular}
    \caption{Palabras utilizadas para recolectar comentarios. Cada palabra se busca sobre el texto de un comentario para marcarlo como potencialmente discriminatorio.}
    \label{tab:palabras_comentarios}
\end{table*}

Otra posibilidad para seleccionar artículos candidatos a ser etiquetados es la de observar los comentarios de usuarios en lugar del texto completo de éste. En base a los comentarios, podemos tener alguna medida de si el artículo suscita reacciones potencialmente discriminatorias. Por ejemplo, si observamos que en un artículo hay comentaristas que usan expresiones discriminatorias contra la comunidad LGBTI, podemos pensar que el contenido de la noticia es interesante para nuestro estudio.

El procedimiento para este tipo de selección es similar al mencionado anteriormente con artículos, sólo que aplicado a comentarios: buscamos respuestas de usuarios que contengan alguna de las expresiones semilla listadas en la Tabla \ref{tab:palabras_comentarios}. Estas palabras fueron recolectadas de manera subjetiva en base a la observación y a la experimentación sobre los datos, tratando de contener diversas expresiones de contenido mayormente discriminatorio. La lista contiene expresiones ofensivas para diversas características de interés: insultos racistas, homofóbicos, misóginos; insultos dirigidos dirigidos a algún personaje particularmente atacado en las redes sociales; expresiones de odio de clase; etc.

Dado un artículo, marcamos los comentarios que contengan una o más de las expresiones listadas. Si el artículo tiene tres o más comentarios marcados, entonces el artículo es seleccionado; caso contrario, es descartado. Vale remarcar que este proceso de selección es para los \emph{artículos}, no para los comentarios. De lo contrario, sólo buscaríamos respuestas que contengan alguna de estas expresiones.

Luego de algunos análisis experimentales y observacionales de las dos posibles metodologías, decidimos utilizar el muestreo de artículos en base a comentarios. En base a un análisis subjetivo, los artículos seleccionados parecían tener mayor incidencia de mensajes discriminatorios y eso nos decantó hacia esa opción.

Una posibilidad adicional analizado fue utilizar un clasificador que nos señale posibles comentarios discriminatorios, usando esta información para seleccionar artículos candidatos a etiquetar. Para ello, aplicamos un clasificador basado en BETO \cite{canete2020spanish} entrenado sobre el dataset de \hateval{} (ver Sección \ref{chap:04_hate_speech}) sobre los comentarios de los artículos. Una evaluación subjetiva de esto nos dio pobres resultados, tanto porque no captaba algunas agresiones discriminatorias (de características no incluídas en el dataset de \citet{hateval2019semeval}) como muchos falsos positivos o errores debido al cambio de dominio (temático y también dialectal). Si bien descartamos este método, puede ser de relevancia usar algún método que no esté basado en palabras semillas o utilizar algún método semi-automático para encontrar candidatos a etiquetar.


\subsection{Muestreo de comentarios}

Una vez que seleccionamos los artículos, resta decidir qué comentarios vamos a anotar. No podemos seleccionar todos ya que muchos artículos cuentan con una cantidad importante de comentarios (en el orden de los cientos) y es deseable mantener un balance entre los comentarios anotados por artículo. Tampoco es deseable (en pos de maximizar el producto de la anotación) seleccionar comentarios de artículos escasamente discutidos. Teniendo esto en mente, conservamos sólo los comentarios de artículos que tengan al menos 20 comentarios. Luego, para cada artículo, seleccionamos aleatoriamente hasta 50 comentarios entre aquellos que no contengan URLs u otro contenido no textual.

En este punto, consideramos el muestreo aleatorio como la forma menos sesgada para seleccionar nuestros comentarios, pero mencionamos de todas formas algunas alternativas evaluadas. Una fue la de considerar todo el universo de comentarios y seleccionar la muestra de allí. Sin embargo, esto sobrerrepresentaría a aquellos temas muy comentados, siendo muchos de ellos acerca de temas políticos que se filtraron en nuestra selección. Otra consideración posible es la de utilizar información de usuarios y sus conexiones, información que Twitter nos brinda a través de los followers de cada usuario. Muchos usuarios que generan contenido discriminatorio en redes sociales se agrupan en comunidades: subgrafos de usuarios altamente conectados entre sí. Usar algún tipo de información sobre esto (por ejemplo, con algún algoritmo como el de Louvain \cite{blondel2008fast}) podría auxiliar al balance de comentarios posiblemente discriminatorios. Para un ejemplo de la utilización de esta técnica, \citet{lai2018stance} y \citet{furman2021you} usan este tipo de algoritmos como manera semi-supervisada de detectar las posturas de los usuarios respecto a distintos temas.


\section{Anotación}
\label{sec:criterios}

Una vez



\subsection{Definición de discurso de odio utilizada}

\begin{table}[t]
    \centering
    \begin{tabularx}{\textwidth}{l X}
        Característica & Descripción \\
        \hline
        MUJER        & Misoginia, agresiones basadas en ser mujer  \\
        LGBTI        & Homofobia, transfobia, y ofensas a la comunidad LGBTI \\
        RACISMO      & Racismo, Xenofobia, Judeofobia, etc \\
        POBREZA      & Basado en su condición de clase \\
        POLITICA     & En base a la filiación política del agredido \\
        ASPECTO      & Gordofobia, gerontofobia \\
        CRIMINAL     & Criminales, presos, y personas en conflicto con la ley \\
        DISCAPACIDAD & Discapacidades y adictos a sustancias

    \end{tabularx}
    \label{tab:caracteristicas_protegidas}
    \caption{Características protegidas consideradas en este trabajo}
\end{table}


Teniendo en cuenta la discusión realizada en la sección \ref{sec:hate_speech_definitions} realizamos nuestra propia definición de discurso de odio. Entendemos que hay discurso de odio en un texto social si éste contiene declaraciones de carácter intenso e irracional de rechazo, enemistad y aborrecimiento contra un individuo o contra un grupo, siendo estos objetivos de estas expresiones por poseer (o aparentar poseer) una característica protegida. Esta expresión puede manifestarse de manera explícita como insultos directos, celebraciones de crímenes, incitaciones a tomar medidas contra el individuo o grupo, o también expresiones más veladas. Siempre, considerando, que no es necesario solamente un insulto o una agresión: es necesario hacer una apelación explícita o implícita a al menos una característica protegida.

A diferencia de otros trabajos, nuestra definición comprende varias características, incluso algunas que están en la frontera de ser ``protegidas''. Mientras en otros trabajos se centran mayormente en racismo y misoginia, aquí agregaremos homofobia y transfobia, odio de clase (``aporofobia''), por su aspecto físico, y otras. En particular, hay dos características no convencionales que tuvimos en cuenta. En primer lugar, el discurso de odio ``político'', que de acuerdo a XXX \todo{citation needed}, es difícil considerar como protegida ya que puede dar lugar a censuras. Por otro lado, también consideramos el discurso de odio contra criminales, presos, y otras personas en situación de conflicto con la ley. Si bien este punto ni siquiera es considerado como una característica protegida en ninguno de los trabajos mencionados en la sección \ref{sec:hate_speech_definitions}, al haber tanto contenido que incita a la violencia contra criminales en las noticias de policiales, agregamos esta característica. Así mismo, esta característica puede ser de utilidad ya que nos interesa recoger incitaciones a la violencia, y este rubro es prolífico en ello en las redes.

Tenemos entonces 8 características que agrupan tipos de discurso de odio: contra las mujeres; racismo y xenofobia; contra la comunidad LGBTI; odio de clase; gordofobia, gerontofobia y demás odio por aspecto; por su ideología política; y finalmente contra discapacitados y adictos. Las características en cuestión son listadas en la tabla \ref{tab:caracteristicas_protegidas} junto a acrónimos que usaremos en el resto del capítulo.

En el apéndice \ref{app:manual_criterios_anotacion} puede encontrarse el manual de etiquetado, con descripciones detalladas y ejemplos de lo que consideramos que configura discurso de odio para cada característica.

\subsection{Modelo de etiquetado}

Un modelo de anotación es, según \citet{pustejovsky2012natural}, una representación práctica del objetivo de anotación. En nuestro caso, queremos marcar comentarios discriminatorios, marcar a qué grupos y/o características se está ofendiendo, y también identificar llamados a tomar alguna acción contra los objetos de esos discursos. Por lo pronto, haremos una definición que capture ese objetivo sin deternos demasiado en especificarlo formalmente (lo que llaman en ese libro ``especificación'').

\todo{Describir un poquito más lo que está mencionado en el review de Polletto}

\subsubsection{Modelo Jerárquico de Etiquetado}

\begin{figure}
    \centering
    \includegraphics[width=\textwidth]{img/modelosjerarquicos.png}
    \caption{Modelos jerárquicos de anotación. A la izquierda, tenemos el modelo jerárquico propuesto para HatEval \cite{hateval2019semeval}, a la derecha el modelo propuesto para OffensEval \cite{zampieri2019semeval2019}}
    \label{fig:modelos_offenseval_hateval}
\end{figure}


\citet{zampieri2019predicting} introdujeron un modelo jerárquico de anotación para la tarea de lenguaje ofensivo, utilizado en las competiciones OffensEval \cite{zampieri2019semeval2019} y hatEval \cite{hateval2019semeval}. La idea de la anotación jerárquica es realizar anotaciones adicionales sólo para algunos casos de anotaciones del nivel anterior.

En el caso de \emph{hatEval}, tenemos un primer nivel que consta de anotar si un tweet contiene o no lenguaje de odio (nivel 1). Si el tweet tiene lenguaje de odio, entonces anotamos si está dirigido a un individuo o a un grupo, y también anotamos si es agresivo o no (ambos nivel 2). En el caso de \emph{OffensEval}, primero anotamos si es ofensivo (nivel 1), luego si está dirigido o es un insulto no dirigido (nivel 2) y finalmente, si es dirigido y ofensivo, marcamos su objetivo (nivel 3). En la figura \ref{fig:modelos_offenseval_hateval} ilustramos ambos modelos.


%
%
% Link: https://docs.google.com/drawings/d/1ZgTmvRwMWn0B-kokfw87jfSa7eY5-OSBHwltetnNT08/edit
%



\begin{figure}
    \centering
    \includegraphics[width=0.5\textwidth]{img/Annotation Model.png}
    \caption{Modelo de anotación}
    \label{fig:annotation_model}
\end{figure}


La figura \ref{fig:annotation_model} muestra el modelo de anotación utilizado para el dataset construído en este trabajo. Seguimos un modelo jerárquico similar al propuesto por \citet{zampieri2019predicting}, aunque de sólo un nivel. Para cada comentario y su respectivo contexto (el artículo), requerimos una anotación  para decidir si el comentario es odioso o no. Si no es odioso, no se necesita más información. Si es así, el par artículo-comentario debe contener, además, una anotación por si llama o no a la acción, y al menos una categoría protegida





\subsection{Etiquetadores}

%
% Chequear https://docs.google.com/spreadsheets/d/1PaOVw_tKVRvjZIqRl2YKnaNsvX5tHJjjY0CV9PLrc6g/edit?resourcekey#gid=366330815
%

\begin{table}[t]
    \centering
    \small
    \begin{tabularx}{\textwidth}{l l l l l l l l}
        Género      & Edad   & Estudios           & Área          & Identificación    & ¿Activista?   & Experiencia\\
        \hline
        F    & 27     & Doctorado*          & Psicología    & Mujer             & No                   & Sí         \\
        NB   & 33     & Grado*              & Artes         & LGBTTIQ           & No                   & No         \\
        F    & 30     & Grado*              & Antropología  & Mujer, LGBTTIQ    & Feminista            & Sí         \\
        M    & 38     & Grado               & Sociología    & No                & No                   & No         \\
        F    & 36     & Doctorado           & Psicología    & Mujer             & No                   & No         \\
        F    & 34     & Grado               & Comunicación  & No                & Migrantes            & No         \\
        \hline
    \end{tabularx}
    \label{tab:informacion_sobre_anotadores}
    \caption{Características protegidas consideradas en este trabajo}
\end{table}

A diferencia de otros trabajos (como hatEval \cite{hateval2019semeval}), decidimos por un lado, garantizar que nuestros anotadores estén más cercanos culturalmente al problema en cuestión, a la vez que tener mayor control del perfil de estos. Consideramos que el discurso de odio tiene un fuerte componente cultural, muchas veces expresado a través de jerga o expresiones dialectales muy particulares, y relacionado con noticias muy propias de esta región.

Para ello, reclutamos etiquetadores hablantes nativos, y estudiantes o graduados/as de carreras de ciencias sociales o humanidades, como ser Psicología, Sociología, Comunicación, Antropología, etc. Algo que particularmente nos interesó fue que no tengan conocimientos de inteligencia artificial, ``ciencia de datos'' ni relacionados, de manera de no sesgar su tarea. También, que sean usuarios asiduos de redes sociales.

El proceso de reclutamiento constó en una breve entrevista donde corroboramos que sean hablantes nativos, les describimos la tarea mientras le mostrábamos la herramienta de etiquetado. Finalmente, se les solicitó hacer una prueba paga de leer el manual de etiquetado y anotar 10 artículos. Esto lo hicimos para corroborar la calidad de los etiquetadores. No rechazamos ningún etiquetador en este proceso.

La tabla \ref{tab:informacion_sobre_anotadores} brinda información desagregada sobre los 6 etiquetadores. Finalmente, nuestros etiquetadores son altamente escolarizados, con 2 etiquetadoras con experiencia previa, y siendo 2 activistas.


\subsection{Tipos de anotación en otros trabajos}

Comentar otros trabajos acá

\begin{itemize}
    \item Davidson
    \item Waseem
    \item hatEval
    \item CONAN
    \item Gao (contextualizado)
    \item Context offensive (el de Google, y el griego)
\end{itemize}

\subsection{Proceso de etiquetado}

\subsection{Preprocesado y filtrado de los datos}

El preprocesado de los datos es muy básico: en los hechos, efectuamos el mismo preprocesamiento que en anteriores tareas, consistente en reemplazar handles de Twitter por un token especial \verb|@usuario| para evitar cualquier sesgo. Por ejemplo, si un usuario conocido como ``odiador'' (llamemos \verb|@hater|) retwittea la noticia y otro responde a ese RT, aparece ese nombre de usuario lo cual podría condicionar al etiquetador.

Así mismo, descartamos cualquier tweet que tuviera algún link ya que pueden referir a contenido no textual


\subsection{Entrenamiento de etiquetadores}



%
% Esto quizás va después
%
\subsection{Herramienta de etiquetado}

%%
%% Link a Google Draw: https://docs.google.com/drawings/d/1E24-2l6hsNj2JSKBZOD8QvZCJR6rrGjz-cWwt8XuPRg/edit
%%

\begin{figure}
    \centering
    \includegraphics[width=\textwidth]{img/labeler.pdf}
    \caption{Pantalla del etiquetador}
    \label{fig:labeler_example}
\end{figure}

Al no utilizar ningún servicio de etiquetado, optamos por desarrollar nuestra propia aplicación para el etiquetado de tweets. En ella, a cada etiquetador les fueron asignados progresivamente los artículos a anotar, los cuales fueron agrupados en ``lotes'' para facilitar la tarea administrativa de la asignación.

La figura \ref{fig:labeler_example} muestra la interfaz presentada a los etiquetadores. Cada artículo es presentado al etiquetador junto a los comentarios asignados. Ante esto, el etiquetador puede elegir saltear el artículo o etiquetarlo. Si decide etiquetarlo, el etiquetador debe para cada comentario marcar usando un control de tipo ``switch''

\begin{enumerate}
    \item Si el comentario contiene discurso discriminatorio
    \item En caso de ser discriminatorio, marcar si llama a la acción
    \item En caso de ser discriminatorio, marcar al menos una característica ofendida
\end{enumerate}

Para el desarrollo de la aplicación usamos Django\footnote{\url{https://www.djangoproject.com/}}, un framework de python para desarrollo web, y Javascript plano. Como base de datos utilizamos SQLite ya que tenía una baja tasa de concurrencia (sólo 6 usuarios.)

\subsection{Esquema de anotación}

%Teniendo en cuenta el modelo de anotación ilustrado en la figura \ref{fig:annotation_model}, optamos por la siguiente metodología para el etiquetado de los comentarios de nuestro dataset.

%%
%%
%% Link a Google Draw:
%% https://docs.google.com/drawings/d/1esS9tAwpPVydohxd-B-xwVdAaPQRVGAo0MruBrgSKig/edit
%%
%%

\begin{figure}
    \centering
    \includegraphics[width=0.7\textwidth]{img/esquema_anotacion.pdf}
    \caption{Esquema de anotación. Caso en que ambos anotadores etiqueten los comentarios del artículo}
    \label{fig:annotation_schema}
\end{figure}

Los artículos son asignados a cada etiquetador. Cada etiquetador, al serle presentado un artículo, tiene dos opciones: etiquetarlo o saltearlo. La idea de saltear era doble: evitar contenido poco ``interesante'' en términos de comentarios discriminatorios, o evitar contenido sensible para el anotador (algo que no ocurrió afortunadamente).

Una posibilidad que barajamos en un principio fue asignar para el etiquetado el artículo completo a 3 anotadores. Sin embargo, esta modalidad sería altamente ineficiente dada la baja cantidad de contenido discriminatorio. Entonces, decidimos ir por un esquema de ``desempate'': dos anotadores anotan un artículo, y luego un tercero anota sólo aquellos donde al menos uno marcó que es discriminatorio. Esto da la posibilidad de que haya una tercera anotación incluso cuando dos previas marcaron que el comentario es discriminatorio, y lo hacemos para recolectar más información. \todo{marcar otros trabajos que hayan hecho esto}. Con este esquema de anotación, y teniendo en cuenta los números finales obtenidos del dataset, dedicamos 2.16 etiquetados por comentarios versus 3 etiquetados por comentario de anotar tres veces todo. La figura \ref{fig:annotation_schema} ilustra este flujo de anotación.

Entonces, en primer lugar cada artículo es asignado a 2 anotadores. Luego de esto, se solicita una tercera anotación pero sólo sobre los comentarios que tengan alguna de las dos etiquetadas marcando contenido discriminatorio, y no dando la posibilidad de saltear. Ahora ¿qué pasa si alguno de los dos anotadores saltea el artículo?. Tenemos dos casos. Si los dos saltean el artículo, entonces descartamos ese artículo. Ahora, puede ocurrir el caso de que uno lo saltee y el otro lo anote: en ese caso, y en pos de maximizar el contenido discriminatorio encontrado o uno lo hace y el otro anota menos de 4 comentarios odiosos, entonces no pasa a 3ra anotación y lo descartamos del dataset. Si uno salteó y el otro anotador anotó 4 o más comentarios odiosos, entonces forzamos al primer anotador a anotar el artículo, sin dar esta vez opción de saltear. La figura \ref{fig:annotation_schema_case_two} ilustra el flujo para este caso.


%%
%%
%% Link a Google Draw
%% https://docs.google.com/drawings/d/1TOlCgZggCmYHgZWV7ZrIIlXuhcFUMeYw4PcFM7XdY2k/edit
%%
%%

\begin{figure}
    \centering
    \includegraphics[width=0.6\textwidth]{img/esquema_anotacion_caso_2.pdf}
    \caption{Esquema de anotación. Caso en que un anotador saltee}
    \label{fig:annotation_schema_case_two}
\end{figure}


Como resultado de este esquema, cada comentario de nuestro dataset puede tener dos o tres anotaciones, siendo los casos posibles los siguientes:

\begin{enumerate}
    \item Dos anotaciones negativas
    \item Tres anotaciones, siendo al menos una que marque el comentario como discriminatorio
\end{enumerate}




\subsection{Asignación}

\citet{pustejovsky2012natural} denominan ``asignación'' al procedimiento de extraer las ``gold labels'' de las etiquetas. En este punto tenemos una etiqueta binaria si el contenido es discriminatorio o no (notamos HS) en el primer nivel, y luego 9 etiquetas binarias: una para la llamadas a la acción (CALLS) y otras 8 para las características ofendidas. Recordemos que una anotación negativa sólo consta de HS negativo, mientras que una positiva consta de un HS positivo, una etiqueta para CALLS y al menos una etiqueta positiva de las características restantes.

Para este dataset, tomamos las siguientes decisiones:

\begin{enumerate}
    \item Para la etiqueta de HS, realizamos la votación mayoritaria
    \item Si hay HS, CALLS es positivo sii es votación mayoritaria
    \item Si hay HS, marco como positivas todas aquellas características marcadas por los anotadores
\end{enumerate}

La primer decisión es la más obvia y razonable, pero las otras dos decisiones merecen alguna discusión. Para que sea un comentario considerado como HS, tiene que ocurrir que al menos dos etiquetadores lo marquen como tal. En ese caso, para que haya votación mayoritaria de CALLS, tiene que haber dos o más votos marcados como tal; en caso de empate, es decir, que un anotador marca que hay llamado a la acción y otro que no, marcamos que no hay llamado a la acción.

En el caso de las características, marcamos todas las que hayan marcado aquellos anotadores que hayan etiquetado HS. Esta decisión podría haberse tomado de otra manera; por ejemplo, sólo tomando aquellos casos donde haya cierto grado de coincidencia entre los comentarios. Sin embargo, al considerar que los límites entre las características son difusos (por ejemplo, apariencia y mujer tienen un grado de coincidencia, y a veces clasismo y racismo también) preferimos optar por este esquema.

\todo{Agregar algún gráfico de esto}

\subsection{Recursos utilizados}

El etiquetado constó de XXX horas. A cada etiquetador le fue pagado YYYY por hora, y luego ZZZ por hora en segunda instancia. Esto equivale a WWW USD.


\section{Resultados}

\begin{table}
    \centering
    % \begin{tabular}{lrr}
    %     \toprule
    %     Total articles & 1238    \\
    %     Total comments &  56869  \\
    %     Hateful Tweets &   8715  \\
    %     Ratio          &   0.153 \\
    % \end{tabular}
    \begin{tabular}{lrr}
        \toprule
        Característica &  Número &  Llamadas a acción \\
        \midrule
        RACISMO        &   2469 &              674 \\
        APARIENCIA     &   1803 &               34 \\
        CRIMINAL       &   1642 &              722 \\
        POLITICA       &   1428 &              136 \\
        MUJER          &   1332 &               18 \\
        CLASE          &    823 &              135 \\
        LGBTI          &    818 &               11 \\
        DISCAPACIDAD   &    580 &                4 \\
        \bottomrule
    \end{tabular}
    \caption{Cantidad de comentarios odiosos del dataset resultante, segmentados por característica. Se listan además la cantidad de llamados a la acción dentro de los comentarios odiosos para cada característica}
    \label{tab:dataset_figures}

\end{table}

El dataset resultante consta de 1238 artículos etiquetados, y 56869 comentarios respectivamente, de los cuales 8715 contienen contenido discriminatorio según los criterios de asignación antes referidos. Podemos observar que aproximadamente 1 de cada 6 comentarios es discriminatorio; esto no es representativo del universo de notas periodísticas ya que recordemos que la selección de los datos no fue aleatoria. La tabla \ref{tab:dataset_figures} contiene estos datos estadísticos.

De todos los tweets discriminatorios, tenemos en particular los llamados a la acción. La inmensa mayoría de estos está dirigido hacia la categoría CRIMINAL, muchos en la forma de llamados a matar a criminales y otros delincuentes.



La tabla \ref{tab:annotation_agreement} reporta el acuerdo entre anotadores usando la métrica alpha de Krippendorff \todo{agregar cita}. Reportamos el valor de $\alpha$ para HS sobre todas las etiquetas, y luego todas las etiquetas del segundo nivel del modelo jerárquico (características y llamado a la acción) sólo sobre aquellas que hayan marcado que el comentario contiene HS. Esto es equivalente a calcular el acuerdo con una etiqueta faltante en el segundo nivel para las características y el llamado a la acción. Si bien este acuerdo tiende a ser alto, debe leerse como el acuerdo sobre la razón detrás del hate speech; la mayor penalización queda reservada a HS, que tiene $\alpha = 0.59$, algo que podría marcarse como un buen acuerdo teniendo en cuenta los parámetros vistos en las tablas de preliminares. \todo{linkear esto}




\begin{table}
    \centering
    \begin{tabular}{lc}
        \toprule
        Categoría   & $\alpha$ de Krippendorff \\
        \midrule
        Hateful              &  0.579 \\
        Calls to Action      &  0.641 \\
        \midrule
        WOMEN                &  0.783 \\
        LGBTI                &  0.920 \\
        RACISM               &  0.929 \\
        CLASS                &  0.706 \\
        POLITICS             &  0.808 \\
        DISABLED             &  0.849 \\
        APPEARANCE           &  0.871 \\
        CRIMINAL             &  0.931 \\
        \bottomrule
    \end{tabular}
    \caption{Reported Agreements. \emph{Hateful} agreement is reported for the binary decision of a tweet assigned as hateful or not; for the other characteristics (and the calls to action) the agreement is calculated over those tweets with two or more hateful marks}
    \label{tab:annotation_agreement}
\end{table}

\subsection{Co-ocurrencia de características ofendidas}
%%
%%
%% Generar con
%% https://docs.google.com/drawings/d/1IcBITgNJN-tehmvnZqcSF9cUuWIpNKJg6yHI5yjNF9c/edit
%%
%%

\begin{figure}[t]
    \centering
    \includegraphics[width=0.85\textwidth]{img/heatmap_characteristics.pdf}
    \caption{Matriz de co-ocurrencia de las características ofendidas para comentarios con dos o más características marcadas. Más luminoso indica más co-ocurrencia}
    \label{fig:heatmap_characteristics}
\end{figure}


De los 8715 comentarios odiosos, el 77\% de ellos (6777) tiene una sola característica ofendida marcada. Del resto, cerca del 20\% de ellos tiene 2 características ofendidas, y 220 comentarios tienen 3 o más. En la figura \ref{fig:heatmap_characteristics} podemos observar la matriz de co-ocurrencia entre las distintas características para aquellos comentarios que tengan más de una marcada. En ella podemos ver que la máxima co-ocurrencia se da entre la característica MUJER y APARIENCIA, seguidos por RACISMO y CLASE, POLITICA y CLASE, y RACISMO y POLITICA.


La tabla \ref{tab:multi_char_examples} muestra algunos ejemplos de comentarios con más de una característica ofendida marcada. Podemos ver que algunos son ejemplos muy ``border'', justo en la frontera de las características (por ejemplo, APARIENCIA y MUJER), algunas tienen características que los anotadores marcaron implícitamente (por ejemplo, el ataque a Milagro Sala, que es un) mientras otros son directamente una clara conjunción de ofensas a las características.


\begin{table}[t]
    \small
    \begin{tabularx}{\textwidth}{XXX}
        \toprule
        Artículo        & Comentario                 & Características\\
        \midrule
        Ofelia Fernández apoyó al Gobierno en la polémica por los presos y apuntó a la Justicia que ``odia a las mujeres''  & Hijadept,, ojala pronto recibas la visita de alguno de esos gusanos. Te van a quedar. Ganas de apoyar al. Gobierno? Larva rastrera gorda. Decerebrada & MUJER, POLITICA, APARIENCIA, DISCAPACIDAD \\
        ``Es hora de ponerle límites al odio'' | Por Victoria Donda &  Justo ésta zurda mugrienta, ignorante y altanera... & MUJER, POLITICA, APARIENCIA\\
        Coronavirus en la Argentina: un video pone en evidencia la violación de la cuarentena en la Villa 1-11-14 & Cierren esa nido de negros y napalm. Hasta reducís el crimen y el gasto público. & RACISMO\\

        Fabiola Yáñez denunció a un periodista por publicaciones agraviantes & Claro si ofendel a la que se cuelga en el caño xq ahora cree ser primera dama?😂 hay que ser peruka para dar asco y ser basuras bigote enseguida ordena como se metió en Facebook y en todo que culpa te.emos que saque la mujer del cabarute? \\

        Los infectados en villas porteñas crecieron un 80\% en cuatro días & Ojalá que el virus penetre más en las villas y maten a todos esos delincuentes que viven ahi, hay paraguayos narcos, bolivianos que traen la droga de bolivia, y gente de mala vida. También hay travas que van a trabajar de noche a palermo. & RACISMO, CLASE, LGBTI  \\

        Ricky Martin: “Soy un hombre latino y homosexual viviendo en los Estados Unidos, soy una amenaza” & Ridículo perdiste tú rumbo das náuseas 🤮 famosos eternos (víctimas) 🙄🤦‍♀️ ándate a Puerto Rico entonces ahí no serás una amenaza & LGBTI, \\

        El enojo de Moria Casán contra Rocío Oliva: ``Mucha agua oxigenada, le quedó media neurona para jugar a la pelota'' & Y la vieja Moria, mucha cirugía y estiramiento. de cara que parece un travesti
    \end{tabularx}
    \label{tab:multi_char_examples}
    \caption{Ejemplos con más de una característica ofendida marcada}
\end{table}




\begin{figure}[t]
    \centering
    \includegraphics[width=0.85\textwidth]{img/heatmap_characteristics_article.pdf}
    \caption{Matriz de co-ocurrencia de las características ofendidas entre comentarios de un mismo artículo. Más luminoso indica más co-ocurrencia}
    \label{fig:heatmap_characteristics_article}
\end{figure}



Otra forma de analizar la co-ocurrencia de comentarios es agrupando por artículos, para observar como un mismo contexto puede suscitar distintos tipos de comentarios discriminatorios. La figura \ref{fig:heatmap_characteristics_article} ilustra las interacciones entre las distintas características por artículo. Podemos observar que tenemos en este mapa de calor que tenemos mayor dispersión en las co-ocurrencias que reduciendo al análisis a sólo observar comentarios. Por mencionar algunas que no aparecen en la figura agrupada únicamente por comentarios, puede verse una mayor interacción entre discurso de odio RACISMO y POLITICA, y, quizás inesperadamente entre APARIENCIA y POLITICA. Las interacciones de la característica LGBTI se mantienen muy bajas, indicando que este tema suele estar concentrado en este tipo de ataques.

Observando estas co-ocurrencias, podemos observar que el dataset anotado posee cierta diversidad en sus instancias, con comentarios conteniendo múltiples tipos de discriminación, y artículos que poseen comentarios odiosos de diversa naturaleza. Sobre esto, podemos especular que tanto el texto (el comentario en sí) como el contexto (el tweet del medio periodístico y su artículo periodístico) contienen información valiosa para poder distinguir entre las distintas categorías discriminatorias. \todo{esto es polémico, reformular: que varios comentarios sean discriminatorios y de característica distinta no implica que el contexto necesariamente ayude}




\subsection{Análisis por característica}

En la tabla XXX podemos observar algunos ejemplos seleccionados de comentarios. Algunas observaciones que pueden realizarse es que los comentarios marcados contra las mujeres tienen en algunos casos ciertas complejidades, como las acusaciones de ``mentirosa'' a una mujer que sufrió una violación (caso Thelma Fardin \todo{Agregar nota de esto}), apreciaciones a su cuerpo, entre otras cosas.

Una categoría desafiante pareciera ser los comentarios discriminatorios contra la comunidad LGBTI. Más allá de algunos insultos explícitamente ofensivos (mediante insultos del estilo trolo, trabuco, maricón, etc), hay muchos que tienen un contenido difícil de descifrar; en particular, aquellos comentarios contra personas trans. Muchos de estos mensajes hacen alusiones a su genitalidad o a su cuerpo en general, de manera metafórica o irónica, lo cual hace verdaderamente difícil su detección. A su vez, es claro que en muchos de estos comentarios es sumamente necesaria la información contextual para poder comprender el caracter abusivo de estos comentarios.

En el caso de la categoría CRIMINAL, se puede observar por un lado comentarios muy violentos (``bala'', ``mátenlos'', ``plomo'') que necesitan el contexto para entenderse como ofensivos contra esa característica (por ejemplo, si la nota fuese sobre una plaga de mosquitos no deberíamos considerarlo como ``discriminatorio''). Por otro lado, algunos comentarios son más difíciles de descifrar y dependientes del contexto, como las celebraciones ante el abatimiento de un preso o criminal (``bravo'', ``felicitaciones!'') que parecen inofensivas hasta que se lee el contexto de la noticia. De hecho, a diferencia de otros comentarios, parecen tener hasta una polaridad positiva.

En el caso de racismo (la categoría más marcada del dataset) hay una fuerte cantidad de comentarios discriminatorios contra la comunidad china. Esto es esperable por el brote racista debido a la pandemia del COVID-19, documentado en YYYY \todo{agregar cita}. Así mismo, es de las categorías que más llamados a la acción tiene, muchos del estilo de tirar bombas, aniquilar, etc a China o a la comunidad de dicho país, o llamados a tomar medidas ``blandas'', como ``no ir a comprarles a los supermercados''.

Algunas de las categorías tienen características más elementales, como política, apariencia, y discapacidad. En los comentarios ilustrados. Esto es esperable ya que

Algunas de las agresiones, a su vez, usan técnicas de camuflaje (``tafaldegaver'', falta de verga, ``docer''), que dificultan su detección por las técnicas actuales.

\begin{table}
    \centering|
    \begin{tabular}{p{0.2\textwidth} p{0.4\textwidth} p{0.5\textwidth}}
        \toprule
        Cat & & \\
        \midrule
        characteristic & text & context \\
        \midrule
                 WOMEN & @usuario Y como te quedó el ogt?. & Loly Antoniale mostró su impresionante casa en Miami: “Soy la reina de mi castillo” \\
                 WOMEN & @usuario Habla el agua viva esta vestida de verde ? & Cecilia Moreau: "No se puede permitir que la minoría le imponga a la mayoría qué temas discutir" \\
                 WOMEN & @usuario “Feminista interseccional”..:nos vamos a la mierda... & Empezó con una “relación abierta en lo sexual”, escaló al poliamor y da las claves para probar el amor libre \\
                 WOMEN & @usuario Provocador ? A mi me provoca ganas de vomitar & ¡El sensual y provocador topless de Morena Rial! \\
                 WOMEN & @usuario Que se aborte ella, vieja  bruja y degenerada!! & Martha Rosenberg: “En situación de pandemia, legalizar el aborto es más urgente que nunca” \\
                 WOMEN & @usuario Kien es la feminista resentida q está instaurando esa idea desde hace días?? & Femicidio en Catamarca: “No es gente enferma, sino que tiene una formación machista importante” \\
                 WOMEN & @usuario Pero quién puede pensar embarazar a esta mostra. & “Ya pusimos el cuerpo, ahora que los diputados se pongan las pilas”: en el Congreso, el lado verde se prepara para una fiesta \\
                 WOMEN & @usuario Pregúntaselo a Lousteau, trola de mierda & Juana Viale hizo un irónico pedido en su programa: “Que el señor Presidente me explique en qué fase de la ‘no cuarentena’ estamos” \\
                 WOMEN & @usuario Que pelotudo!!!!...lo que te va a tomar es la guita viejo decrepito...o te pensastes que se enamoro de vos..jajajaja & Eduardo Costantini y Elina Fernández mostraron la intimidad de su luna de miel: “Ella no me deja tomar una copa de vino” \\
                 WOMEN & @usuario El paskin con la trola & La pareja de Lázaro Báez contó cómo lo esperaba adentro del country: “Lo que le hicieron fue indignante” \\
                 WOMEN & @usuario Que manga de roñosas & “Ya pusimos el cuerpo, ahora que los diputados se pongan las pilas”: en el Congreso, el lado verde se prepara para una fiesta \\
                 WOMEN & @usuario A la que fue PROSTITUTA de Villa Ballester le molesta que le recuerden que fue PROSTITUTA? & Fabiola Yáñez denunció a un periodista por publicaciones agraviantes \\
                 WOMEN & @usuario Hay que comerse al termotanque de lipidos & ¿More Rial encontró el amor en un personal trainer? \\
                 WOMEN & @usuario @usuario @usuario SOS la peor mierda de argentina, junto al presidente y su titiritera de Cristina. Manga de soretes! \textbackslash nMuy sorora pero estan matando de hambre a media argentina, homicidios y terrorismo. Vayanse todos! Métete el corazón verde e... & Fuerte cruce entre Flavio Mendoza y Victoria Donda: “Es muy fácil hablar cuando uno cobra un sueldo” \\
                 WOMEN & @usuario No me sorprende, usa el pañuelito verde decisor. & Nancy Pazos reveló por qué decidió que su mamá no recibiera la donación de plasma \\
                 WOMEN & @usuario La Yañez está tratando de entender  de que habla Manes .. pobre solo conoce los tablones del teatro ... & Coronavirus. Fabiola Yáñez organizó una videollamada con Facundo Manes: "Lo importante es estar bien mentalmente" \\
                 WOMEN & @usuario Esto me hace tan feliz, jodanse aborteras de mierda. JAJAJAJAJAJAJAJAJAJAJAJAJAJAJAJAJAJAJAJAJAJAJAJAJAJAJAJAJAJAJAJAJAJAJAJAJAJAJAJAJAJAJAJAJAJAJAJAJAJAJAJAJAJAJAJAJAJAJAJJAAJAJAJAJAJAJAJAJAJAJAJAJAJAJAJAJAJAJAJAJAJAJAJAJAJAJAJAJAJAJAJAJAJAJAJ... & Aborto legal: otra promesa incumplida \\
                 WOMEN & @usuario Jaja la mina orgullosa de lo q consiguió gateando, bien ahi 🥴 & Loly Antoniale mostró su impresionante casa en Miami: “Soy la reina de mi castillo” \\
                 WOMEN & @usuario Y quien te iva a hacer un pibe...dracula o el hombre lobo.. & “Me esterilicé, pero no odio a los niños”: mi vida dentro del movimiento “libre de hijos” \\
                 WOMEN & @usuario Dirán lo que dirán de Moria y como sea trabajo toda su vida por eso tiene lo que tiene Pero Rocío  solo trabajo abriendo las piernas  en la cama  para llegar hacer figureti Y no me la cuenten que fue por amor🤣🤣 & El enojo de Moria Casán con Rocío Oliva: “Mucha agua oxigenada, le quedó media neurona para jugar a la pelota” \\
                 LGBTI & @usuario Revisen esa casa, los están envenenando. & Contó que era lesbiana, su papá le confesó que era gay y ahora su madre se enamoró de una mujer: así se inspiró para su segundo film \\
                 LGBTI & @usuario Me cruzo con 50 María Elena por día. Las feminazis son iguales & No es actriz, pero se anima al desafío: quién es Ethel Herrera, la tiktoker que se postula para ser María Elena en “Casados con hijos” \\
                 LGBTI & @usuario Biológicamente las mujeres tienen vagina y los hombres tienen pene. Lo demás es ideológico. & Tras los comentarios de J.K. Rowling, Emma Watson defendió al colectivo trans \\
                 LGBTI & @usuario por fin alguien que ponga huevos en el equipo & Histórico: Mara Gómez fue habilitada y será la primera jugadora trans en el fútbol argentino \\
                 LGBTI & @usuario pero....este no se comia la galletita? & Pablo Duggan, sobre su casamiento: "Karen es la mujer de mi vida, estoy muy enamorado" \\
                 LGBTI & @usuario Por el pirulo & Por qué Flor de la V no continuó en Mujeres de eltrece, tras la salida de Claudia Fontán \\
                 LGBTI & @usuario Pelado puto, te quedaste solo. & Horacio Rodríguez Larreta: “La convocatoria del lunes no es del partido, si alguno va es a título personal” \\
                 LGBTI & @usuario Bueno por lo menos le quedó el culo roto & El meteorólogo Matías Bertolotti se separó a ocho meses de haberse casado \\
                 LGBTI & @usuario No Clarin. Es un señor con apariencia de mujer que preñó a una mujer con apariencia de hombre. La paja no es como el trigo. & La historia de la modelo colombiana trans que besa la panza de su esposo embarazado de ocho meses \\
                 LGBTI & @usuario Che, qué pasó con el puto-no me enteré de nada. & Por qué Flor de la V no continuó en Mujeres de eltrece, tras la salida de Claudia Fontán \\
                 LGBTI & @usuario Viejo larga presos , llévalos todos a tu casa y te haces romper el orto viejo puto igual ya lo tener & Eugenio Zaffaroni le contestó a Sergio Berni tras la polémica por las domiciliarias: “Es el populacherismo vindicativo que llenó las cárceles” \\
                 LGBTI & @usuario Pero...\textbackslash n.\textbackslash n.\textbackslash n.\textbackslash n.\textbackslash nCamila es Camilo. & Camila Sosa Villada: “Con el feminismo y el psicoanálisis descubrimos que el amor es una herramienta de manipulación" \\
                 LGBTI & @usuario Vos también comes carne...\textbackslash n\textbackslash nNo te hagas la boluda, encima te comiste a un bala no asumido, ahí también comiste carne....\textbackslash n\textbackslash n@usuario \textbackslash n\textbackslash nDeja que cada uno elija qué hacer, de ultima es una elección personal como la tuya de comerte a un corrup... & Nicole Neumann apuntó contra los consumidores de carne: “Sigan comiendo asadito” \\
                 LGBTI & @usuario La mujer tiene duda siiiiiii duda la podonga 😂😂😂😂 & Pablo Duggan, sobre su casamiento: "Karen es la mujer de mi vida, estoy muy enamorado" \\
                 LGBTI & @usuario 🤮🤮🤮🤮🤮🤮 & La emotiva dedicatoria de Luis Novaresio a su pareja, Braulio Bauab, en su cumpleaños \\
                 LGBTI & @usuario Cerra el otro viejo chupa pija & Alberto Fernández: “Dejemos el tiempo del encuentro y del esparcimiento social para otro momento” \\
                 LGBTI & @usuario Quedaba feo poner "un jugador disfrazado de mujer" entendemos & Mara Gómez cumple su sueño: será la primera futbolista transgénero en el torneo profesional argentino \\
                 LGBTI & @usuario Creo que se debe referir a las feminazis,aborteras y ese circo de lgbtqia+ etc y toda esas payasadas....que son manipulados como titeres por los lobbys & Cecilia Moreau: "No se puede permitir que la minoría le imponga a la mayoría qué temas discutir" \\
                 LGBTI & @usuario Estos chetos soretes maricas nenas de mama cagones. & Coronavirus en Argentina: los “caprichos” de algunos repatriados en los hoteles porteños, entre rebeldes y cholulos \\
                 LGBTI & @usuario La discriminacion se termina despues del primer tortazo.. dejen de llorar manga de maricas!! & Viola Davis, Halle Berry y Angela Bassett, actrices negras que superaron la discriminación en Hollywood \\
        \bottomrule
        \end{tabular}
\end{table}



\section{Conclusión}

En este capítulo, describimos la construcción de un dataset contextualizado de lenguaje discriminatorio o hate speech. Para ello, recolectamos respuestas a noticias periodísticas posteadas en Twitter por los principales medios de noticias de Argentina. Exploramos distintas alternativas para la selección de artículos a etiquetar, tanto observando los tópicos de los artículos como los comentarios a este. Decidimos elegir los artículos en base a sus comentarios potencialmente discriminatorios, y luego seleccionar una muestra aleatoria y acotada de comentarios.

Para realizar la tarea de etiquetado, desarrollamos nuestra propia herramienta la cual hacemos pública. Definimos un modelo de anotación jerárquico y granular para la tarea, siendo relativamente novedoso el hecho de anotar las características ofendidas en cada texto social. Seis etiquetadores nativos de la variedad dialectal rioplatense realizaron la tarea de anotación bajo un esquema de 2 anotaciones + desempate.

Como producto, obtuvimos un dataset de cerca de 57k comentarios repartidos en 1.2k artículos, una cantidad de tamaño considerable aunque no tengamos parámetro de comparación ya que no existen muchos datasets similares. De los 57k comentarios, alrededor de 8k comentarios tienen contenido discriminatorio (una tasa de 1 cada 6). Un análisis exploratorio de los comentarios discriminatorios muestra ejemplos complejos y ricos, algunos de ellos altamente dependientes del contexto.

En el siguiente capítulo, abordaremos nuestra pregunta original: ¿puede el contexto ayudar a los algoritmos de clasificación a mejorar su performance?. Para responder esto, utilizaremos este dataset especialmente diseñado.
