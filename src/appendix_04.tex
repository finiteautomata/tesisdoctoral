
\section{Tratados internacional sobre libertad de expresión y discurso de odio}
\label{app:tratados-internacionales}

\subsection{Libertad de expresión}

La Convención Americana de Derechos Humanos (CADH) establece que:

\begin{displayquote}[CADH, Artículo 13][]

    1. Toda persona tiene derecho a la libertad de pensamiento y de expresión.  Este derecho comprende la libertad de buscar, recibir y difundir informaciones e ideas de toda índole, sin consideración de fronteras, ya sea oralmente, por escrito o en forma impresa o artística, o por cualquier otro procedimiento de su elección.

    2. El ejercicio del derecho previsto en el inciso precedente no puede estar sujeto a previa censura sino a responsabilidades ulteriores, las que deben estar expresamente fijadas por la ley y ser necesarias para asegurar:

    a)  el respeto a los derechos o a la reputación de los demás, o

    b) la protección de la seguridad nacional, el orden público o la salud o la moral públicas.
\end{displayquote}

En Estados Unidos, la primer enmienda protege este derecho humano, mientras que en la Unión Europea, legislación similar ofrece protección a la libertad de expresión. Finalmente, la declaración universal de los derechos humanos de la ONU \footnote{\url{https://www.un.org/es/about-us/universal-declaration-of-human-rights}} menciona tanto en su preámbulo como en el artículo 19

\begin{displayquote}[Declaración Universal de los Derechos Humanos][ONU]
    Todo individuo tiene derecho a la libertad de opinión y de expresión; este derecho incluye el de no ser molestado a causa de sus opiniones, el de investigar y recibir informaciones y opiniones, y el de difundirlas, sin limitación de fronteras, por cualquier medio de expresión.
\end{displayquote}


Otro documento conocido como el Pacto Internacional de Derechos Civiles y Políticos (ICCPR por sus siglas en inglés), sancionado en 1966 en la Asamblea de las Naciones Unidas y ratificado por 166 países, incluye en su artículo 19:

\begin{displayquote}[Artículo 19 de la ICCPR]
1. Nadie podrá ser molestado a causa de sus opiniones.

2. Toda persona tiene derecho a la libertad de expresión; este derecho comprende la libertad de buscar, recibir y difundir informaciones e ideas de toda índole, sin consideración de fronteras, ya sea oralmente, por escrito o en forma impresa o artística, o por cualquier otro procedimiento de su elección.

3. El ejercicio del derecho previsto en el párrafo 2 de este artículo entraña deberes y responsabilidades especiales. Por consiguiente, puede estar sujeto a ciertas restricciones, que deberán, sin embargo, estar expresamente fijadas por la ley y ser necesarias para:

a) Asegurar el respeto a los derechos o a la reputación de los demás;

b) La protección de la seguridad nacional, el orden público o la salud o la moral públicas.
\end{displayquote}


Este último apartado ilustra que la libertad de expresión no es un derecho completamente irrestricto. El ejercicio de los derechos e igualdad ante la ley de otros marca este límite. Citando nuevamente al Pacto de San José de Costa Rica:

\begin{displayquote}[Pacto San José de Costa Rica, CADH][Artículo 1]
    1. Los Estados Partes en esta Convención se comprometen a respetar los derechos y libertades reconocidos en ella y a garantizar su libre y pleno ejercicio a toda persona que esté sujeta a su jurisdicción, sin discriminación alguna por motivos de raza, color, sexo, idioma, religión, opiniones políticas o de cualquier otra índole, origen nacional o social, posición económica, nacimiento o cualquier otra condición social.
\end{displayquote}

y a la Declaración Universal de los Derechos Humanos de la ONU:

\begin{displayquote}
    Todos los seres humanos nacen libres e iguales en dignidad y derechos y, dotados como están de razón y conciencia, deben comportarse fraternalmente los unos con los otros.
\end{displayquote}


\subsection{Discurso de odio}

Una recopilación de definiciones realizada por un informe de la UNESCO

\begin{displayquote}[Countering Hate Speech Online, \citet{gagliardone2015countering}]
    Mientras que el sistema interamericano de derechos humanos ha desarollado determinados estándares, no existe una definición universalmente aceptada de “discurso de odio” en el derecho internacional. Según un reciente informe emitido por la UNESCO que estudió las distintas definiciones de discurso de odio en el derecho internacional, el concepto con frecuencia se refiere a “expresiones a favor de la incitación a hacer daño (particularmente a la discriminación, hostilidad o violencia) con base en la identificación de la víctima como perteneciente a determinado grupo social o demográfico. Puede incluir, entre otros, discursos que incitan, amenazan o motivan a cometer actos de violencia. No obstante, para algunos el concepto se extiende también a las expresiones que alimentan un ambiente de prejuicio e intolerancia en el entendido de que tal ambiente puede incentivar la discriminación, hostilidad y ataques violentos dirigidos a ciertas personas
\end{displayquote}


\section{Incidencia de keywords en el dataset}

\begin{table}
    \centering
    \small
    \begin{tabular}{l c}
    Palabra   &  \#tweets\\
    \thline{2}
    puta      &  $2226$ \\
    callate   &  $1223$ \\
    migra     &  $ 802$ \\
    perra     &  $ 640$ \\
    arabe     &  $ 483$ \\
    zorra     &  $ 367$ \\
    sudac     &  $ 355$ \\
    ceuta     &  $ 310$ \\
    acoso     &  $ 284$ \\
    polla     &  $ 272$ \\
    \thline{2}
    \end{tabular}
    \caption{Palabras del dataset y su cantidad de incidencia en tweets del dataset de HatEval.}
    \label{tab:keywords}
\end{table}


La Tabla \ref{tab:keywords} muestra un listado de palabras obtenido por observación de instancias del dataset de \citet{hateval2019semeval}. Podemos observar que algunas palabras (como \emph{puta, callate, migrante, árabe}) tienen una altísima tasa de incidencia en tweets, dando cuenta de un posible sesgo de recolección de los datos.