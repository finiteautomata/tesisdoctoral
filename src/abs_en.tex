%\begin{center}
%\large \bf \runtitle
%\end{center}
%\vspace{1cm}
\chapter*{\runtitle}

\noindent




{
    \setstretch{1.5}
Hate speech can be described as speech containing intense hatred, denigration, and enmity that attacks an individual or a group of individuals because of possessing –or pretending to possess– any characteristic protected by international treaties such as gender, ethnicity, religion, language, among others. In recent years, this type of discourse has gained great relevance in social networks and other virtual media due to its intensity and its relationship with violent acts against members of these groups. As a result, states and supranational organizations –such as the European Union– have enacted legislation that urges social media companies to moderate and remove discriminatory content, with particular focus on that which promotes physical violence.

Due to the enormous amount of user-generated content on social media, it is necessary to have some degree of automation in this task, either for analysis or for moderation. From a natural language processing (NLP) perspective, hate speech detection can be understood as a text classification problem: given a text generated by a user, predict whether it is discriminatory content. Likewise, it may be of interest to predict other features: for example, if the text contains a call to violent action; if it is directed against an individual or a group; or the offended characteristic, among others.

One of the limitations of current approaches to hate speech detection is the lack of context. Most studies and resources are performed on data without context; that is, isolated messages without any type of conversational context or the topic being discussed. This restricts the information available –both for a human and for an automated system– to discern if a social text is hateful or not. Other information usually lacking is the offended characteristic: datasets are usually annotated with a low level of granularity, failing to provide information about whether the offending message attacks the individual or group due to their gender, social class, race, or whatsoever. Finally, a specific limitation of Spanish is the limited availability of resources for this task.

In this thesis, we intend to address some of the marked limitations. On the one hand, we analyze the impact of adding context to hate speech detection in social networks. To do this, we built a tweet dataset based on user responses to news media posts on Twitter. This provided us two types of contexts: a conversational context, given by the tweet and its answer, and another context given by the text of the news in question. This dataset was collected on news related to the COVID-19 pandemic, in the Spanish language in its Rioplatense dialectal variety. Native speakers of this dialect annotated the comments with a novel labeling model that is granular regarding the offended characteristics.

Using this dataset, we carried out hate speech detection experiments, proposing two tasks: "flat" detection of discriminatory language, where we only predict a binary label indicating the presence of discriminatory language; and "granular" detection, where we predict the attacked characteristics (n-binary classification tasks at the same time). Using state-of-the-art techniques, we obtained significant improvements in both tasks by adding context as input for each instance, both in its short form (only the headline/tweet of the news article) and in its long-form (headline and body of the news article). We also observed that a classifier trained for the "granular" task slightly improves its performance when being evaluated for the "flat" task, ignoring possible errors of discriminatory motives. Combining the addition of context and granularity, a classifier for the detection of discriminatory language obtained considerable improvements over a BERT in Spanish that only consumes the text of the comment.

Considering hate speech detection within the most comprehensive area of ​​document classification in social domains, we further explored  some general aspects of related tasks such as sentiment analysis and emotion detection, among others. In particular, we analyzed the performance of various modern representation techniques when trained in social domains. Commonly, NLP researchers train representation models on texts from "formal" domains, such as Wikipedia or other similar sources. We observed that –from word embeddings to pre-trained models based on transformers– the representations generated are robust and improve performance in a set of classification tasks in social texts. On the pre-trained models, we studied the impact of training them from scratch in social texts versus performing domain-adaptation on the language models.

All of the studies and resources presented in this thesis were carried out in the Spanish language. As a secondary objective, we aim to mitigate the enormous asymmetry of resources in the area of NLP.
}
\bigskip

\noindent\textbf{Keywords:} Hate Speech, Natural Language Processing, Abusive Language Detection, Domain Adaptation, Social NLP.