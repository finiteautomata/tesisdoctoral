La extracción de opiniones (opinion mining) de usuarios en redes sociales ha atraído mucho interés desde los inicios de estos espacios virtuales. Con fines comerciales, políticos, sociológicos (como el análisis de discurso de odio o las reacciones a la pandemia), numerosos trabajos han analizado . Debido a la cantidad de contenido generado (se estima que en el mundo se generan XXX tweets por segundo) esta tarea es difícil de realizar sin algún tipo de automatización. Para ello, muchísimo esfuerzo se ha volcado en utilizar técnicas de inteligencia artificial para este fin.

En este capítulo haremos una breve introducción a la clasificación de textos sociales. Esto es, dado un texto generado por un usuario (un post en Facebook, Instagram, un tweet, etc) predecir alguna característica discreta de éste, como por ejemplo si es un texto positivo o negativo, si tiene algún tipo de emoción de ira, alegría, u otra; si contiene discurso de odio contra algún grupo o no; si es irónico; entre otras. En base a datasets en español para distintas tareas, presentaremos modelos de clasificación basados en técnicas del estado del arte.

Finalmente, analizaremos algunas cuestiones relacionadas a la adaptación de dominio y representaciones generadas sobre dominios sociales. Analizaremos para técnicas de representación no contextualizadas \footnote{Que al día de la fecha, en pocos años, han quedado obsoletas} y algunas técnicas más recientes el impacto de entrenar desde cero o realizar cierta adaptación sobre la performance de las técnicas de clasificación.

\section{Trabajo previo}

\section{Adaptación de dominio}

\subsection{Embeddings}
\subsection{Entrenar desde cero vs. adaptar}

\section{Data augmentation}

Describir data augmentation como otra técnica de regularización. Comentar backtranslation

\section{Análisis de polaridad dirigido}

\section{Librería de análisis de sentimientos}
