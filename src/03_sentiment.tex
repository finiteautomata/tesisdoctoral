La extracción de opiniones (opinion mining) de usuarios en redes sociales ha atraído mucho interés desde los inicios de estos espacios virtuales. Con fines comerciales, políticos, sociológicos (como el análisis de discurso de odio o las reacciones a la pandemia), numerosos trabajos han analizado . Debido a la cantidad de contenido generado (se estima que en el mundo se generan XXX tweets por segundo) esta tarea es difícil de realizar sin algún tipo de automatización. Para ello, muchísimo esfuerzo se ha volcado en utilizar técnicas de inteligencia artificial para este fin.

En este capítulo haremos una breve introducción a la clasificación de textos sociales. Esto es, dado un texto generado por un usuario (un post en Facebook, Instagram, un tweet, etc) predecir alguna característica discreta de éste, como por ejemplo si es un texto positivo o negativo, si tiene algún tipo de emoción de ira, alegría, u otra; si contiene discurso de odio contra algún grupo o no; si es irónico; entre otras. En base a datasets en español para distintas tareas, presentaremos modelos de clasificación basados en técnicas del estado del arte.

Finalmente, analizaremos algunas cuestiones relacionadas a la adaptación de dominio y representaciones generadas sobre dominios sociales. Analizaremos para técnicas de representación no contextualizadas \footnote{Que al día de la fecha, en pocos años, han quedado obsoletas} y algunas técnicas más recientes el impacto de entrenar desde cero o realizar cierta adaptación sobre la performance de las técnicas de clasificación.

\section{Trabajo previo}

Describir data augmentation como otra técnica de regularización. Comentar backtranslation

\section{Análisis de Sentimiento}

\section{Análisis de Emociones}

\section{Modelos de clasificación}
\subsection{Preprocesamiento}
\subsection{Embeddings}


\section{Librería de análisis de sentimientos}

\newcommand{\pysentimiento}[0]{\textbf{pysentimiento}}

Algo que suele obstaculizar la utilización de herramientas de extracción de opinión (como las que acabamos de ver en este capítulo pero así mismo las que veremos más adelante) con fines de investigación es la dificultad a su acceso. O bien estos servicios están detrás de APIs pagas con precios demasiado altos para los presupuestos académicos o están disponibles pero no en español (u otro idioma de ``bajos recursos''). En otros casos, estos recursos están disponibles pero no para ser usados de forma de ``caja negra'', lo cual para alguien que no es un experto en NLP suele complicar su utilización.

Como una pequeñísima contribución de esta tesis y con el objetivo de facilitar el acceso de estos recursos para la investigación, creamos la librería \textbf{pysentimiento}\footnote{\url{https://github.com/pysentimiento/pysentimiento}}. Este paquete provee modelos pre-entrenados y herramientas de preprocesado para textos sociales. Si bien tiene soporte multilingual tanto en español como inglés, su eje original es el de proveer recursos para el español que tiene una disparidad importante en recursos.

La figura XXX muestra la arquitectura de \pysentimiento{}. Utiliza el model hub de \emph{huggingface}\footnote{\url{https://huggingface.co/models}}, un repositorio de modelos pre-entrenados basados en transformers. Allí es donde colocamos todos los modelos que entrenamos, tanto de sentimientos, emociones, y los que mostraremos más adelante como detección de discurso de odio. Cada tweet que es analizado por la librería pasa primero por una etapa de preprocesamiento (siguiendo el proceso explicado en la sección zzz), y luego procesado por el modelo, quien nos brinda un output. Dependiendo el problema, puede haber una etapa de post-procesamiento.

\todo{Completar las cosas que quedaron acá sin referencias}

%
% Pysentimiento architecture
% https://www.canva.com/design/DAEufPDskMI/Gg_phzjuXgFihF1g3x9L-A/edit#
%
%

\section{Conclusiones}