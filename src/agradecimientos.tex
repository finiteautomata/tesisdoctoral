\chapter*{Agradecimientos}


En primer lugar, quiero agradecer a mis directores Franco y Agustín, que guiaron mi trabajo y me formaron como investigador. Realizar un doctorado es una labor bastante dura, en el cual uno se encuentra muchas veces perdido en el camino. La función que ambos cumplieron guiándome en esos momentos de desorientación ---pero dándome la libertad de elección en cada momento--- ha sido fundamental para llegar hasta acá.

Quiero agradecer a todes mis compañeres del Laboratorio de Inteligencia Artificial Aplicada (LIAA) y del Departamento de Computación de Exactas UBA quienes me ayudaron a transitar este doctorado, compartiendo conocimiento, charlas --- a veces simplemente catarsis. Edgar Altszyler, Pablo Brusco, Ramiro Gálvez, Bruno Bianchi, Damián Furman, Lara Gauder, Jazmín Vidal, y todos los que me falten en esta lista. A Viviana Cotik, que me ayudó de gran manera en los momentos más críticos de este trabajo.

A todes les integrantes del Proyecto Interdisciplinario de la UBA sobre marginaciones sociales (PIUBAMAS), que fueron fundamentales en los segmentos más importantes de esta tesis.

A mis compañeres de activismo y militancia, particularmente a los compañeros de la Asociación Gremial de Docentes de la UBA (AGD-UBA) y Jóvenes Científicxs Precarizados (JCP). Luchar por nuestros derechos y reconocimiento como trabajadores ha sido sin dudas parte de mi formación.

A mis amigos que me vieron poco estos años. A Víctor, Pablo, Tamara, Silvina, Andrés, Nico, Chudi, Tomás, Pigre, Joe. A Mariela Rajngewerc, con quien atravesamos paralelamente las dificultades de la academia. A Nina Pardal, con quien compartimos caminatas y charlas en Exactas.

A mi familia. A mi hermano Fer, a Graciela, a Julio. A mis primos Nico, Héctor y Meli. A mis viejos, dondequiera que estén, por impulsar mi curiosidad desde pequeño y siempre apoyarme en el estudio. Cada uno, a su manera, me fue llevando por este camino.

Finalmente quiero agradecer a Valeria, mi compañera de vida, que me apoyó en todo momento y soportó el estado de desborde emocional permanente que atraviesa todo doctorando. Realmente hubiera sido imposible sin vos.

